%% LyX 2.3.6.1 created this file.  For more info, see http://www.lyx.org/.
%% Do not edit unless you really know what you are doing.
\documentclass[12pt,a4paper,oneside,reqno,english,russian]{scrbook}
\usepackage{ccfonts}
\renewcommand{\sfdefault}{cmss}
\renewcommand{\ttdefault}{cmtt}
\usepackage{eulervm}
\usepackage[T1,T2A]{fontenc}
\usepackage[utf8]{inputenc}
\usepackage{fancyhdr}
\pagestyle{fancy}
\setcounter{secnumdepth}{5}
\setcounter{tocdepth}{5}
\setlength{\parindent}{4ex}
\usepackage{color}
\definecolor{page_backgroundcolor}{rgb}{0, 0.171875, 0.214844}
\pagecolor{page_backgroundcolor}
\definecolor{document_fontcolor}{rgb}{0.578125, 0.632812, 0.632812}
\color{document_fontcolor}
\usepackage{babel}
\usepackage{array}
\usepackage{longtable}
\usepackage{varioref}
\usepackage{textcomp}
\usepackage{footnotehyper}
\usepackage{amsmath}
\PassOptionsToPackage{normalem}{ulem}
\usepackage{ulem}
\usepackage[unicode=true,
 bookmarks=true,bookmarksnumbered=true,bookmarksopen=true,bookmarksopenlevel=10,
 breaklinks=true,pdfborder={0 0 0},pdfborderstyle={},backref=false,colorlinks=true]
 {hyperref}
\hypersetup{pdftitle={RICS Valuation Global Standards 2020 RUS},
 pdfauthor={Kirill A. Murashev},
 pdfsubject={RICS Valuation Global Standards russian unofficial translation},
 pdfkeywords={RICS, RICS Valuation, RICS Global Standards, RICS Russia},
 linkcolor=yellow}

\makeatletter

%%%%%%%%%%%%%%%%%%%%%%%%%%%%%% LyX specific LaTeX commands.

\makesavenoteenv{description}

\pdfpageheight\paperheight
\pdfpagewidth\paperwidth

\providecommand{\LyX}{\texorpdfstring%
  {L\kern-.1667em\lower.25em\hbox{Y}\kern-.125emX\@}
  {LyX}}
\DeclareRobustCommand{\cyrtext}{%
  \fontencoding{T2A}\selectfont\def\encodingdefault{T2A}}
\DeclareRobustCommand{\textcyr}[1]{\leavevmode{\cyrtext #1}}

%% Because html converters don't know tabularnewline
\providecommand{\tabularnewline}{\\}

%%%%%%%%%%%%%%%%%%%%%%%%%%%%%% Textclass specific LaTeX commands.
\usepackage{paralist}

%%%%%%%%%%%%%%%%%%%%%%%%%%%%%% User specified LaTeX commands.
%\renewcommand{\labelenumi}{\alph{enumi})}
\renewcommand{\theenumi}{\alph{enumi}} % изменение нумерации списков на буквенную
\usepackage{lineno} % подключение пакета для нумерации строк
\linenumbers % включение нумерации строк
%\setcounter{chapter}{-1}
%\newcounter{chapter}[part]
\newcounter{ChapCounter}[chapter] % Создание счётчика, подчинение его Главе
\newcounter{SecCounter}[section] % Создание счётчика, подчинение его Надразделу
\newcounter{SubSecCounter}[subsection] % Создание счётчика, подчинение его Разделу
\newcounter{SubSubSecCounter}[subsubsection] % Создание счётчика, подчинение его Подразделу
\newcounter{ParCounter}[paragraph] % Создание счётчика, подчинение его Параграфу
\newcounter{SubParCounter}[subparagraph] % Создание счётчика, подчинение его Подпараграфу
\usepackage{indentfirst}            % красная строка
\usepackage{misccorr}               % доработки для babel
\frenchspacing                      % французский стиль пробелов
\usepackage{graphicx}
%\usepackage[backend=biber]{biblatex}
\usepackage{draftwatermark}% Загрузка пакета
\SetWatermarkScale{2.3}%Масштабирование базового шрифта
\SetWatermarkLightness{0.25}%Изменение насыщенности
\SetWatermarkText{\textbf{Рабочая версия}}%Пользовательская строка
%\usepackage[english,main=russian]{babel}
%\usepackage{fancyhdr}
%\fancyhf{}% Clear header footer

\makeatother

\usepackage[style=numeric]{biblatex}
\addbibresource{Basic_principles.bib}
\addbibresource{LaTeX.bib}
\addbibresource{Murashev.bib}
\addbibresource{RussianLaws.bib}
\addbibresource{Sci&Tech.bib}
\addbibresource{ValuationStandards.bib}
\begin{document}
\titlehead{Профессиональные стандарты и~руководства RICS}
\title{{\huge{}RICS оценка "---~Всемирные Стандарты 2020.}}
\title{Версия 0.1911.513}
\subtitle{вступают в~силу с~31~января 2020~г. Текущая версия ориг. англ.
текста от~28.11.2019~г.}
\author{пер.~с~англ. К.$\,$А.~Мурашев}
\maketitle

\lhead{\thepart}

\chead{\thechapter}

\rhead{\thesection}

\lfoot{0.1911.513}

\cfoot{\thepage/\pageref{End-of-all.}}

\rfoot{\today}

\href{https://teacode.com/online/udc/33/338.5.html}{УДК 338.5}~\cite{UDK:338.5}

\href{https://www.triumph.ru/html/serv/bbk-tablicy-onlayn.html?category_id=20789&parent_id=20717}{ББК 65.25}~\cite{BBK:65.25}

\href{https://www.triumph.ru/html/serv/avtorskij-znak.html}{М91}~\cite{Avt_znak}

RICS\foreignlanguage{english}{ professional standards and~guidance,
global. RICS Valuation --- Global Standards. Effective from 31~January
2020} / пер.~с~англ. К.\,А.~Мурашев "--- Inkeri, Санкт-Петербург,
22~января 2021~г. -- \today, \pageref{End-of-all.}~с.

\rule[0.5ex]{1\paperwidth}{1pt}

Данный материал является \emph{неофициальным} переводом \href{https://www.rics.org/globalassets/rics-website/media/upholding-professional-standards/sector-standards/valuation/rics-valuation--global-standards-jan.pdf}{документа}~\cite{RICS:RVGS-2020},
размещённого на~\href{https://www.rics.org/eu/}{официальном сайте}~\cite{RICS:site}
и~имеющего статус действующего с~31~января~2020~г. Предназначен
для~ознакомления интересующимся вопросами международного регулирования
и~международных стандартов в~области \emph{оценочной деятельности}.
Сам~RICS, равно и~его структуры и~представительства не~имеют никакого
отношения к~созданию данного материала. Позиция автора перевода по~тем~или~иным
аспектам содержания, а~также вопросам \emph{оценочной деятельности}
не~является позицией RICS.

Следует отметить, что~с~июня 2015~г. RICS публикует две версии
стандартов: \href{https://www.rics.org/globalassets/rics-website/media/upholding-professional-standards/sector-standards/valuation/red-book-uk-supplement-rics.pdf}{национальную}~\cite{RICS:RVNS-2017},
предназначенную для~рынка Соединённого Королевства Великобритании
и~Северной Ирландии, и~\href{https://www.rics.org/globalassets/rics-website/media/upholding-professional-standards/sector-standards/valuation/rics-valuation--global-standards-jan.pdf}{международную}~\cite{RICS:RVGS-2020},
предназначенную для~остальных рынков. По~очевидным причинам предметом
интереса настоящей работы является \href{https://www.rics.org/globalassets/rics-website/media/upholding-professional-standards/sector-standards/valuation/rics-valuation--global-standards-jan.pdf}{международная}~\cite{RICS:RVGS-2020}
версия стандартов RICS. Автор не~гарантирует точность перевода и~не~несёт
ответственности за~последствия его~использования. Поскольку основной
задачей является передача смысла, перевод не~является дословным.

Данное произведение является неофициальным переводом и~распространяется
на условиях лицензии \href{https://creativecommons.org/licenses/by-sa/4.0/}{Creative Commons Attribution-ShareAlike 4.0 International (CC BY-SA 4.0)}~\cite{cc-by-sa-4.0},
оригинальный текст которой доступен по~\href{https://creativecommons.org/licenses/by-sa/4.0/legalcode}{ссылке}~\cite{CC_BY-SA_4.0},
перевод которого на~русский язык доступен по~\href{https://creativecommons.org/licenses/by-sa/4.0/legalcode.ru}{ссылке}~\cite{CC_BY-SA_4.0-RUS}.

Для~подготовки данной материала использовался \href{https://www.ctan.org/}{язык}~\TeX ~\cite{TeX:site}
с~\href{https://www.latex-project.org/}{набором макрорасширений}~\LaTeXe ~\cite{LaTeX:site}.
Конкретная техническая реализация заключается в~использовании дистрибутива
\href{https://www.tug.org/texlive/}{TexLive}~\cite{TeXLive:site},
\href{https://www.lyx.org/}{редактора} \LyX ~\cite{Lyx:site},
компилятора PdfLaTeX и~системы цитирования BibLaTeX/Biber. Исходный
код и~дополнительные файлы, необходимые для~его~компиляции, доступны
по~\href{https://web.tresorit.com/l/oFpJF\#xr3UGoxLvszsn4vAaHtjqw}{ссылке}~\cite{RVGS-Rus-TeX-source},
а~также по~\href{https://github.com/Kirill-Murashev/RVGS-2020-RUS.git}{запасной ссылке}~\cite{RVGS-Rus-TeX-source-res}.
Материал подготовлен в~форме гипертекста: ссылки на~ресурсы, размещённые
в~\href{https://normativ.kontur.ru/document?moduleId=1&documentId=376603&cwi=22898}{информационно-телекоммуникационной сети «Интернет»}~\cite{FZ-149},
выделены \textcolor{magenta}{пурпурным}\textcolor{blue}{{} }\textcolor{magenta}{(}\foreignlanguage{english}{\textcolor{magenta}{magenta}}\textcolor{magenta}{)}\textcolor{blue}{{}
}\textcolor{magenta}{цветом}, внутренние перекрёстные ссылки выделены\textcolor{magenta}{{}
}\textcolor{yellow}{жёлтым (}\foreignlanguage{english}{\textcolor{yellow}{yellow}}\textcolor{yellow}{)
цветом}, библиографические ссылки выделены \textcolor{green}{зелёным
(}\foreignlanguage{english}{\textcolor{green}{green}}\textcolor{green}{)
цветом}. При~подготовке данного материала использовался шаблон \foreignlanguage{english}{\href{https://ctan.org/pkg/koma-script}{KOMA-Script Book}~\cite{KOMA-Script}}.
Для~облегчения понимания согласования слов в~сложноподчинённых предложениях
в~тексте реализована графическая \uline{разметка}, \uline{позволяющая}
понять структуру предложения: \uuline{слова}, \uuline{согласованные
между собой} внутри предложения, подчёркнуты одинаковыми линиями,
данное решение применяется только в~тех \uwave{предложениях},
в~\uwave{которых}, по~мнению автора перевода, возможно неоднозначное
толкование в~части согласования слов внутри него.

Данный материал выпускается в~соответствии с~философией \emph{\href{https://ru.wikipedia.org/wiki/Rolling_release}{Rolling Release}~\cite{Rolling_release_rus},
}что~означает что~материал будет непрерывно дорабатываться по~мере
обнаружения в~нём~ошибок и~неточностей, в~целях улучшения внешнего
вида, а~также по~мере обновления оригинального английского текста.
Идентификатором, предназначенным для~определения версии материала,
служат номер и~дата, указанные на~титульном листе, а~также в~колонтитулах.
История версий приводится в~таблице~0.1 \vpageref{tab:version_history}--\pageref{tab:version_history-End}.
Актуальная версия перевода в~формате PDF доступна по~\href{https://web.tresorit.com/l/oFpJF\#xr3UGoxLvszsn4vAaHtjqw}{ссылке}~\cite{RVGS-Rus-TeX-source},
а~также по~\href{https://github.com/Kirill-Murashev/RVGS-2020-RUS.git}{запасной ссылке}~\cite{RVGS-Rus-TeX-source-res}.

Данный материал существует в~двух формах, отличие которых заключается
в~цветовых решениях и~иных свойствах отображения:
\begin{itemize}
\item для~чтения при~стандартном освещении~(светлая тема) "--- белый
фон (\#ffffff), чёрный основной текст~(\#000000), шрифт \foreignlanguage{english}{\href{https://ru.wikipedia.org/wiki/Computer_Modern}{Computer Modern}~\cite{CM_font}}
12 кегль;
\item для~чтения при~недостаточном освещении~(тёмная тема) "--- фон
\#002b36, основной текст \#93a1a1 (цвета соответствуют цветовой схеме
\href{https://lonewolfonline.net/uploads/downloads/Solarized-Dark.xml}{Solarized Dark}~\cite{Solarized_dark_theme}),
шрифт \foreignlanguage{english}{\href{https://ru.wikipedia.org/wiki/Concrete_Roman}{Concrete}~\cite{CR_font}}
12 кегль.
\end{itemize}
В~целях соответствия принципам \href{https://www.investopedia.com/terms/s/sustainability.asp}{устойчивого развития}~\cite{Investo:sustainability,Wiki:sustainability},
установленным в~частности Стратегией \foreignlanguage{english}{\textbf{\href{https://www.europarl.europa.eu/doceo/document/TA-9-2020-0005_EN.html}{The European Green Deal}~\cite{EU_Green_Deal}}}
и~являющимся приоритетными для~\href{https://lisbon-vladivostok.pro/}{Единой Европы}~\cite{Lisbon-Vladivostok,Greater_Europe_Institute,Wiki:Greater_Europe},
а~также содействия достижению \href{https://en.wikipedia.org/wiki/Carbon_neutrality\#Companies_and_organizations}{углеродной нейтральности}~\cite{Wiki:carbon_neutral}
рекомендуется использовать материал, реализованный в~тёмной цветовой
схеме и~исключительно в~электронной форме без~распечатывания на~бумаге. 

Данный материал повторяет структуру оригинального английского текста,
однако в~нём~используется собственная сквозная система нумерации,
в~которой: I~и~II "--- номера частей, X "--- глав, XX "--- разделов,
XXX "--- подразделов, XXXX "--- секций, ХХХХХ "--- подсекций, XXXXXX
"--- параграфов.

Для~связи с~автором данного перевода можно использовать
\begin{itemize}
\item любой клиент, совместимый с~протоколом \href{https://tox.chat/}{Tox}~\cite{Tox:site,Wiki:Tox},
Tox~ID~=~2E71 CA29 AF96 DEF6 ABC0 55BA 4314 BCB4 072A 60EC C2B1
0299 04F8 5B26 6673 C31D 8C90 7E19 3B35;
\item адрес электронной почты: kirill.murashev@tutanota.de;
\item \href{https://www.facebook.com/murashev.kirill/}{https://www.facebook.com/murashev.kirill/}~\cite{Facebook:Murashev_Kirill};
\end{itemize}
Реквизиты для~оказания помощи проекту.

Тинькоф: +79219597644

BTC: bc1qjzwtk3hc7ft9cf2a3u77cxfklgnw93jktyjfsl?time=1627474534\&exp=86400

ETH: 

Monero: 45ho 6Na3 dzoW DwYp 4ebD BXBr 6CuC F9L5 NGCD ccpa w2W4 W15a
fiMM dGmf dhnp e6hP JSXk 9Mwm o9Up kh3a ek96 LFEa BZYX zGQ

USDT: 0x885e0b0E0bDCFE48750Be534f284EFfbEf6d247C

EURT: 0x885e0b0E0bDCFE48750Be534f284EFfbEf6d247C

CNHT: 0x885e0b0E0bDCFE48750Be534f284EFfbEf6d247C

\newpage{}

\chapter*{История версий}

\label{tab:version_history}

\begin{longtable}[c]{|ccccc|}
\caption{История версия перевода}
\tabularnewline
\hline 
№ & Номер версии & Дата & Автор & Описание\tabularnewline
\hline 
\hline
\endfirsthead
\caption{История версия перевода}
\tabularnewline
\hline 
№ & Номер версии & Дата & Автор & Описание\tabularnewline
\hline 
\hline
\endhead
0 & 1 & 2 & 3 & 4\tabularnewline
\hline 
001 & 0.1911.001 & 2021-01-22 & KAM & Создание\tabularnewline
\hline 
002 & 0.1911.002 & 2021-01-31 & KAM & Решение технических проблем\tabularnewline
\hline 
003 & 0.1911.021 & 2021-02-09 & KAM & Завершён перевод главы 1\tabularnewline
\hline 
004 & 0.1911.022 & 2021-02-17 & KAM & Завершён перевод главы 2\tabularnewline
\hline 
005 & 0.1911.061 & 2021-03-24 & KAM & Завершён перевод главы 3\tabularnewline
\hline 
006 & 0.1911.081 & 2021-04-10 & KAM & Завершён перевод раздела 1 главы 4\tabularnewline
\hline 
007 & 0.1911.091 & 2021-04-25 & KAM & Завершён перевод раздела 2 главы 4\tabularnewline
\hline 
008 & 0.1911.111 & 2021-05-09 & KAM & Завершён перевод раздела 3 главы 4\tabularnewline
\hline 
009 & 0.1911.131 & 2021-05-17 & KAM & Завершён перевод раздела 4 главы 4\tabularnewline
\hline 
010 & 0.1911.132 & 2021-05-28 & KAM & Завершён перевод раздела 5 главы 4\tabularnewline
\hline 
011 & 0.1911.132 & 2021-05-28 & KAM & Завершён перевод раздела 5 главы 4\tabularnewline
\hline 
012 & 0.1911.167 & 2021-06-10 & KAM & Завершён перевод раздела 3 главы 5\tabularnewline
\hline 
013 & 0.1911.180 & 2021-06-20 & KAM & Завершён перевод раздела 4 главы 5\tabularnewline
\hline 
014 & 0.1911.189 & 2021-06-25 & KAM & Завершён перевод раздела 5 главы 5\tabularnewline
\hline 
015 & 0.1911.205 & 2021-07-12 & KAM & Завершён перевод раздела 6 главы 5\tabularnewline
\hline 
016 & 0.1911.215 & 2021-07-19 & KAM & Завершён перевод раздела 7 главы 5\tabularnewline
\hline 
017 & 0.1911.225 & 2021-07-21 & KAM & Завершён перевод раздела 8 главы 5\tabularnewline
\hline 
018 & 0.1911.228 & 2021-07-21 & KAM & Завершён перевод раздела 9 главы 5\tabularnewline
\hline 
019 & 0.1911.233 & 2021-07-22 & KAM & Завершён перевод раздела 10 главы 5\tabularnewline
\hline 
020 & 0.1911.246 & 2021-07-22 & KAM & Сформирована глава 6\tabularnewline
\hline 
021 & 0.1911.252 & 2021-07-22 & KAM & Сформирована глава 7\tabularnewline
\hline 
022 & 0.1911.258 & 2021-07-22 & KAM & Сформирована глава 8\tabularnewline
\hline 
023 & 0.1911.339 & 2021-07-28 & KAM & Сформирована глава 9\tabularnewline
\hline 
024 & 0.1911.500 & 2021-07-28 & KAM & Сформирована глава 10\tabularnewline
\hline 
025 & 0.1911.501 & 2021-07-28 & KAM & Технические правки\tabularnewline
\hline 
026 & 0.1911.510 & 2021-08-06 & KAM & Первая проверка всего текста\tabularnewline
\hline 
027 & 0.1911.511 & 2021-08-10 & KAM & Исправление нумерации СПО~1\tabularnewline
\hline 
028 & 0.1911.512 & 2021-08-10 & KAM & Стандартизация императивности\tabularnewline
\hline 
029 & 0.1911.513 & 2021-08-13 & KAM & Исправление ошибок\tabularnewline
\hline 
 &  &  &  & \tabularnewline
\hline 
 &  &  &  & \tabularnewline
\hline 
 &  &  &  & \tabularnewline
\hline 
 &  &  &  & \tabularnewline
\hline 
\end{longtable}

\label{tab:version_history-End}

Номер версии состоит из~8 знаков и~двух десятичных разделителей.
Цифра до~первого десятичного разделителя означает текущий статус
документа: 0 "--- драфт, 1 "--- готовая версия, четыре знака после
первого десятичного разделителя означают год и~месяц выхода актуальной
версии оригинального английского текста, два знака после второго десятичного
разделителя означают процент готовности, определяемый согласно таблице~\vref{tab:version_description},
третья цифра после десятичного знака означает версию внутри определённой
стадии готовности.

Три буквы в~столбце~3 означают автора, внёсшего вклад в~те~или~иные
этапы создания данного перевода. Перечень участников создания и~доработки
данного материала приводится в~таблице~\vref{tab:authors}.

\begin{table}
\caption{Перечень авторов, принявших участие в~создании данного материала}
\label{tab:authors}

\begin{tabular}{|c|c|c|c|}
\hline 
№ & Имя & Сокращение & Контакты\tabularnewline
\hline 
\hline 
0 & 1 & 2 & \tabularnewline
\hline 
1 & Мурашев К.\,А. & KAM & \href{https://www.facebook.com/murashev.kirill/}{https://www.facebook.com/murashev.kirill/}~\cite{Facebook:Murashev_Kirill}\tabularnewline
\hline 
2 & ... & ... & \tabularnewline
\hline 
3 & ... & ... & \tabularnewline
\hline 
\end{tabular}

\end{table}

\newpage{}

\begin{table}

\caption{Описание стадий готовности}
\label{tab:version_description}

\begin{tabular*}{1\textwidth}{@{\extracolsep{\fill}}|c|c|>{\raggedright}m{0.8\textwidth}|}
\hline 
№ & \selectlanguage{english}%
Stage\selectlanguage{russian}%
 & \foreignlanguage{english}{\centering{}Description}\tabularnewline
\hline 
\hline 
0 & 1 & \centering{}2\tabularnewline
\hline 
01 & \textless 0 & \foreignlanguage{english}{Means the~same as~above~0, but~indicate that the~author is~stuck
and~needs consultation with co-author(s).}\tabularnewline
\hline 
02 & 0--0.50 & \foreignlanguage{english}{Unfinished\foreignlanguage{russian}{.}}\tabularnewline
\hline 
03 & 0.51--0.70 & \foreignlanguage{english}{1st draft --- can~have ,,rough edges”. Focus: main form, contents,
major points.}\tabularnewline
\hline 
04 & 0.71--0.80 & \foreignlanguage{english}{2nd draft. Focus: section consistency internally within the~chapter,
errors, misunderstandings, disagreements, missing points, missing
references, additions, readability.}\tabularnewline
\hline 
05 & 0.81--0.90 & \foreignlanguage{english}{3rd draft. Focus: chapter consistency externally within the~report,
agreement with content, form, last check if~points have been left
out, readability.}\tabularnewline
\hline 
06 & 0.91--0.99 & \foreignlanguage{english}{Deliverable\foreignlanguage{russian}{.}}\tabularnewline
\hline 
07 & 0.95 & \foreignlanguage{english}{Deliverable, all quotes checked.}\tabularnewline
\hline 
08 & 0.96 & \foreignlanguage{english}{Deliverable, \foreignlanguage{russian}{BibTeX} references checked.}\tabularnewline
\hline 
09 & 0.97 & \foreignlanguage{english}{Deliverable, punctuation checked.}\tabularnewline
\hline 
10 & 0.98 & \foreignlanguage{english}{Deliverable, thesaurus consulted.}\tabularnewline
\hline 
11 & 0.99 & \foreignlanguage{english}{Deliverable, whole document spelling checked.}\tabularnewline
\hline 
12 & 1.00 & \foreignlanguage{english}{Finished}\tabularnewline
\hline 
\end{tabular*}

\end{table}

Описание взято из~документации пакета \href{https://ctan.asis.ai/macros/latex/contrib/progress/progress.pdf}{progress}~\cite{Latex:package:progress}.\newpage{}

\chapter*{Предисловие переводчика}

По~мнению переводчика, основными аспектами новых \emph{Всемирных
стандартов оценки RICS}, определяющими направление развития передовых
практик \textsl{оценки}, являются:
\begin{itemize}
\item приоритет данных, непосредственно наблюдаемых на~рынке над~данными,
полученными из~любых других источников;
\item признание результатов \textsl{оценки}, полученных посредством использования
автоматизированных моделей и~компьютерных программ; 
\item высокое, но~пока~что неопределённое значение вопросов \href{https://en.wikipedia.org/wiki/Sustainability}{устойчивого развития}~\cite{Wiki:sustainability,RICS:Cont_Env_Sust,Investo:sustainability};
\item выведение \emph{метода чистых активов} из~перечня методов \textsl{затратного
подхода} в~отдельный метод, существующий вне рамок трёх классических
подходов.
\end{itemize}
В~данной работе преимущественно используется терминология, принятая
в~русской оценочной традиции.

В~оригинальном английском тексте определены 3~(три) категории императивности
конкретных требований, приведённые ниже в~порядке убывания строгости
требования:
\begin{itemize}
\item \guillemotleft\foreignlanguage{english}{valuer must}\guillemotright{}
"--- \guillemotleft\emph{оценщик} обязан\guillemotright , \guillemotleft\emph{оценщику}
необходимо\guillemotright ;
\item \guillemotleft\foreignlanguage{english}{valuer should}\guillemotright{}
"--- \guillemotleft\emph{оценщику} следует\guillemotright ;
\selectlanguage{english}%
\item \guillemotleft valuer may\foreignlanguage{russian}{\guillemotright{}
"--- \guillemotleft\emph{оценщик} может\guillemotright .}
\end{itemize}
Автор заранее выражает благодарность всем тем, кто~найдёт ошибки
и~неточности в~данном материале либо предложит его~улучшения.\tableofcontents{}

\listoftables

\listoffigures

\printbibliography[heading=bibintoc]


\chapter*{Предисловие}

Данное обновлённое издание Всемирных стандартов оценки RICS (RICS\foreignlanguage{english}{
Valuation~---~Global Standards}), также широко известных как~«Красная
Книга RICS» (\foreignlanguage{english}{RICS ,,Red Book Global”}) (далее
--- RVGS/RBG), отражает, среди прочего, недавние изменения, принятые
и~внесённые в~\href{https://www.rics.org/globalassets/rics-website/media/upholding-professional-standards/sector-standards/valuation/international-valuation-standards-rics2.pdf}{Международные стандарты оценки}
(далее~---~\textbf{МСО})~\cite{IVS-2020}, равно как~и~текущий
прогресс в~разработке Международных стандартов \href{https://www.rics.org/globalassets/rics-website/media/upholding-professional-standards/standards-of-conduct/international-ethics-standards-ies-rics.pdf}{этики}~\cite{IES-2016}
и~\href{https://ipmsc.org/standards/}{самой оценки}~\cite{IPMS}.
К~числу иных усовершенствований относятся:
\begin{enumerate}
\item акцент на~применимости стандартов при~проведении оценки для~целей
\textsl{финансовой отчётности}, потенциально являющейся обязательной
частью системы оценки результатов деятельности в~ряде юрисдикций;
\item включение практического руководства по~оценке: \nameref{sec:5.7_VPGA-7_Valuation_of_personal_propetry}
(см.~\vref{sec:5.7_VPGA-7_Valuation_of_personal_propetry}--\pageref{sec:5.7_VPGA-7_Valuation_of_personal_propetry-End})
с~целью лучшего освещения вопросов \textsl{оценки} предметов искусства
и~антиквариата;
\item совершенствование и~(или) уточнение некоторых существующих положений
RVGS/RBG в~свете накопленного опыта и~меняющихся потребностей.
\end{enumerate}
RVGS/RBG отражает растущее значение успешного сочетания профессиональных
стандартов, стандартов проведения \textsl{оценки} и~стандартов подготовки
\emph{отчёта об~оценке} для~целей предоставления оценочных консультаций
высокого качества, отвечающих ожиданиям и~требованиям клиентов, государства
как~института, регуляторов отрасли, иных органов, устанавливающих
требования в~сфере оценочной деятельности и~разрабатывающих стандарты
\textsl{оценки}, а~также общества в~целом.

Прозрачность, последовательность и~недопущение \href{https://en.wikipedia.org/wiki/Conflict_of_interest}{конфликта интересов}~\cite{COI}
ещё~никогда не~были столь важны. Равно как~и~технические знания
и~практические навыки, включающие в~себя опыт и~понимание рынков,
необходимые для~интерпретации исходной информации, анализа их~динамики
и~тенденций, а~"---~в случае активов, относящихся к~недвижимости,
"--- признание растущего влияния вопросов \href{https://en.wikipedia.org/wiki/Sustainability}{устойчивого развития}~\cite{Wiki:sustainability,Investo:sustainability}
в~качестве факторов рыночного окружения. \emph{Оценщики}, сертифицированные
RICS, находятся на~переднем крае профессии, а~RVGS/RBG является
их~определяющим руководством по~внедрению лучших практик \textsl{оценки}.

Изменения в~данном обновлённом издании RVGS/RBG по-прежнему направлены
на~повышение его~ясности и~простоты использования, в~т.\,ч. доработаны
\textcolor{yellow}{перекрёстные ссылки} на~другие исходные документы.
Как~и~раньше, присутствуют перекрёстные ссылки на~\href{https://www.rics.org/globalassets/rics-website/media/upholding-professional-standards/sector-standards/valuation/international-valuation-standards-rics2.pdf}{Международные стандарты оценки}~\cite{IVS-2020},
полный текст которых приводится в~Части~\vref{part:II_IVS}--\pageref{End-of-all.}
данного перевода.

Все~\textsl{члены RICS} в~процессе проведения оценки, результатом
которой является \emph{отчёт об~оценке}, обязаны соблюдать \uline{стандарты},
приведённые в~данном обновлённом издании, "--- иными словами, если
не~указано иное, \uline{они}~являются обязательными к~применению.
Практические руководства по~оценке (\foreignlanguage{english}{The~Valuation
Practice Guidances}) (ПР/VPGA), приведённые в~Главе~\ref{chap:5_Valuation_applications}~\nameref{chap:5_Valuation_applications}
на~с.~\pageref{chap:5_Valuation_applications}--\pageref{chap:5_Valuation-applications-End},
также включённые в~данное издание, предназначены для~углублённой
проработки вопросов применения стандартов в~конкретных ситуациях:
для~определённых целей \textsl{оценки} либо для~\textsl{оценки}
определённых типов активов. ПР/VPGA носят \emph{рекомендательный}
характер. Статус отдельных элементов RVGS/RBG подробнее описан в~Главе~\ref{chap:1_Introduction}.
\nameref{chap:1_Introduction} на~с.~\pageref{chap:1_Introduction}--\pageref{chap:1_Introduction-End},
а~также в~разделе~\ref{sec:2.2_Standards_naming}. \nameref{sec:2.2_Standards_naming}
на~с.~\pageref{sec:2.2_Standards_naming}--\pageref{sec:2.2_Standards_naming-End}.

\part{Всемирные стандарты оценки RICS\label{part:I_RVGS/RBG}}

\chapter{Введение\label{chap:1_Introduction}}

\section{Основные цели\label{sec:1.1_Overall_purpose}}

\stepcounter{SecCounter}

\thesection.\theSecCounter.\label{1.1.1} Принципы последовательности,
объективности и~прозрачности являются основополагающими для~создания
и~поддержания доверия общества к~институту \textsl{оценки}. В~свою
очередь, возможность их~соблюдения в~решающей степени зависит от~того,
обладают~ли субъекты оценочной деятельности достаточным профессионализмом,
знаниями, опытом и~этическими качествами, применяют~ли их~должным
образом в~своей деятельности как~при~формировании \emph{суждения
о~стоимости}, так~и~для~ясного не~вводящего в~заблуждение изложения
такого суждения заказчику \textsl{оценки} либо иным пользователям
результатов оказания услуг по~\textsl{оценке} в~соответствии с~общепризнанными
нормами.\label{1.1.1-End}

\stepcounter{SecCounter}

\thesection.\theSecCounter.\label{1.1.2} По~мере того как~усиливаются
регулирование и~надзор со~стороны государства, а~ожидания пользователей
оценочных услуг непрерывно растут, Всемирные стандарты оценки продолжают
развиваться и~в~настоящее время реализованы в~трёх различных, но
вместе с~тем тесно взаимосвязанных формах:
\begin{enumerate}
\item \textbf{Стандарты профессиональной деятельности} "--- в~центре внимания
которых находятся этика и~поведение Оценщика, базирующиеся на~его~знаниях
и~компетенции;
\item \textbf{Технические стандарты} "--- в~центре внимания которых находятся
общие определения и~договорённости, основанные на~последовательном
применении общепризнанных подходов к~\textsl{оценке};
\item \textbf{Стандарты исполнения и~изложения} "--- в~центре внимания
которых находятся строгость анализа и~объективность суждения, основанные
на~соответствующих документах и~чёткости повествования.\label{1.1.2-End}
\end{enumerate}
\stepcounter{SecCounter}

\thesection.\theSecCounter. \label{1.1.3} Данное обновлённое издание
Всемирных стандартов оценки RICS, обычно именуемых «Красная книга
RICS», акцентировано на~их~практическом применении, использует новейшие
международные стандарты и~расширяет их~дополнительными требованиями
и~практическими руководствами которые в~сочетании между собой гарантируют
высочайший уровень гарантии профессионализма и~качества.\label{1.1.3-End}

\stepcounter{SecCounter}

\thesection.\theSecCounter. \label{1.1.4} В~основе данного издания
лежат \href{https://www.rics.org/globalassets/rics-website/media/upholding-professional-standards/sector-standards/valuation/international-valuation-standards-rics2.pdf}{Международные стандарты оценки}
(\href{https://www.rics.org/globalassets/rics-website/media/upholding-professional-standards/sector-standards/valuation/international-valuation-standards-rics2.pdf}{МСО})~\cite{IVS-2020},
издаваемые \href{https://en.wikipedia.org/wiki/International_Valuation_Standards_Council}{Международным Советом по стандартам оценки}~\cite{IVSC:site,IVSC:Wiki}.
На~протяжении долгого времени RICS поддерживает их~развитие и~вносит
свой вклад в~разработку этих стандартов, являющихся универсальными
для~\textsl{оценки} всех видов активов. RICS не~только признаёт
данные стандарты, требуя от~своих членов их~соблюдения, но~и~активно
поддерживает их~дальнейшее развитие и~распространение среди \emph{оценщиков}
по~всему миру.\label{1.1.4_End}

\stepcounter{SecCounter}

\thesection.\theSecCounter. \label{1.1.5} Данные технические стандарты
существуют в~рамках широкого набора стандартов RICS, называемых по~отдельности
«положениями о~профессиональной деятельности», охватывающих также
вопросы этики, навыков и~профессионального поведения, включая чёткие
требования в~части обеспечения конфиденциальности и~предотвращения
\emph{\href{https://en.wikipedia.org/wiki/Conflict_of_interest}{конфликта интересов}}~\cite{COI}.
Данные стандарты также учитывают положения \href{https://www.rics.org/globalassets/rics-website/media/upholding-professional-standards/standards-of-conduct/international-ethics-standards-ies-rics.pdf}{Международных Стандартов Этики}~\cite{IES-2016},
впервые опубликованных в~декабре 2016~г. Наконец, при~проведении
\textsl{оценки}, связанной с~активами, относящимися к~недвижимости,
\textsl{члены RICS} обязаны, там~где~это~возможно, также применять
\href{https://ipmsc.org/standards/}{Международные стандарты оценки собственности}~\cite{IPMS},
разработка которых продолжается.\label{1.1.5-End}

\stepcounter{SecCounter}

\thesection.\theSecCounter. \label{1.1.6} Соблюдение профессиональных
и~технических стандартов, равно как~и~стандартов исполнения и~изложения
обеспечивается отлаженной системой регулирования и, при~необходимости,
правоприменением, а~также постепенно внедряемой \href{https://www.rics.org/eu/upholding-professional-standards/regulation/valuer-registration/}{системой зарегистрированных оценщиков RICS}~\cite{RICS:Registration}.
Всё~этов~совокупности обеспечивает оказание услуг по~оценке, соответствующих
требованиям \href{https://www.rics.org/globalassets/rics-website/media/upholding-professional-standards/sector-standards/valuation/international-valuation-standards-rics2.pdf}{МСО}~\cite{IVS-2020},
со~стороны \textsl{членов} RICS и~\href{https://www.rics.org/eu/upholding-professional-standards/regulation/}{оценочных компаний, деятельность которых регулируется RICS}~\cite{RICS:Regulation},
в~глобальном масштабе, а~также их~позиционирование в~качестве
лидирующих поставщиков оценочных услуг.\label{1.1.6-End}

\stepcounter{SecCounter}

\thesection.\theSecCounter. \label{1.1.7} Основная цель проста "---
обеспечить для~заказчиков и~пользователей результатов оказания услуг
по~\textsl{оценке} уверенность и~гарантию того, что~\textsl{оценка},
выполненная сертифицированным RICS оценщиком, в~целом будет выполнена
в~соответствии с~самыми высокими профессиональными стандартами,
в~какой~бы точке мира она~не~проводилась.\label{1.1.7-End}\label{sec:1.1_Overall_purpose-End}

\section{Область применения\label{sec:1.2_Coverage}}

\subsection{С~точки зрения лица, оказывающего услуги по~оценке\label{subsec:1.2.1_Providers_perspective}}

\stepcounter{SubSecCounter}

\thesubsection.\theSubSecCounter. \label{1.2.1.8} Для~\textsl{членов
RICS} данные Всемирные стандарты устанавливают процедурные правила
и~практические руководства, которые:
\begin{enumerate}
\item налагают на~\emph{оценщиков} и~\emph{оценочные компании}, чья~деятельность
регулируется RICS, определённые обязательства в~части компетентности,
объективности, прозрачности и~уровня исполнения;
\item создают среду, обеспечивающую унификацию и~применение передовых практик
в~сфере исполнения и~предоставления результатов услуг по~\textsl{оценке}
путём применения \href{https://www.rics.org/globalassets/rics-website/media/upholding-professional-standards/sector-standards/valuation/international-valuation-standards-rics2.}{МСО}~\cite{IVS-2020};
\item обеспечивают чёткое соблюдение \href{https://www.rics.org/ssa/upholding-professional-standards/standards-of-conduct/rules-of-conduct/}{Правил поведения RICS}~\cite{RICS:Conduct,RICS:Conduct-Firms,RICS:Conduct-Members}.\label{1.2.1.8-End}
\end{enumerate}
\stepcounter{SubSecCounter}

\thesubsection.\theSubSecCounter. \label{1.2.1.9} Данные Всемирные
стандарты \textbf{не}:
\begin{enumerate}
\item дают указаний \textsl{членам RICS} как~именно следует проводить \textsl{оценку}
в~конкретных случаях;
\item предписывают конкретные форму и~содержание \emph{отчёта об~оценке}:
\emph{отчёты} всегда должны быть адекватными и~соответствующими \emph{заданию
на~оценку} при~соблюдении обязательного соответствия требованиям
данных стандартов;
\item заменяют собой стандарты, являющиеся обязательными в~конкретных юрисдикциях,
и~не~отменяют их.\label{1.2.1.9-End}\label{subsec:1.2.1_Providers_perspective-End}
\end{enumerate}

\subsection{С~точки зрения пользователя услуг по~оценке\label{subsec:1.2.2_Users_perspective}}

\stepcounter{SubSecCounter}

\thesubsection.\theSubSecCounter. \label{1.2.2.1} Для~заказчиков
услуг по~\textsl{оценке} и~иных пользователей результатов их~оказания
данные всемирные стандарты обеспечивают оказание услуг по~\textsl{оценке},
соответствующих требованиям \href{https://www.rics.org/globalassets/rics-website/media/upholding-professional-standards/sector-standards/valuation/international-valuation-standards-rics2.}{МСО}~\cite{IVS-2020}
и, более того, обеспечивают:
\begin{enumerate}
\item последовательность действий \emph{оценщика}, способствующую пониманию
процесса \textsl{оценки}, и, следовательно, пониманию природы стоимости,
определённой в~\emph{отчёте об~оценке};
\item заслуживающие доверие и~последовательные суждения \emph{субъектов
оценочной деятельности}, имеющих соответствующую подготовку, обладающих
соответствующей квалификацией и~достаточным опытом для~выполнения
поставленной задачи, в~том числе пониманием текущей конъюнктуры рынка,
к~которому относится \emph{объект оценки};
\item независимость, объективность и~прозрачность деятельности \emph{оценщика};
\item ясность в~части условий \textsl{договора на~проведение оценки} (\emph{задания
на~оценку}), включая перечень вопросов, подлежащих рассмотрению,
и~перечень сведений, которые должны быть предоставлена;
\item ясность касательно \textsl{вида определяемой стоимости}, в~т.\,ч.~изложение
необходимых \textsl{допущений} и~соображений, которые необходимо
принимать во~внимание;
\item ясное повествование в~\emph{отчёте об~оценке}, включающее надлежащее
раскрытие информации о~том, в~каких случаях \textsl{третья сторона}
может полагаться на~результаты \textsl{оценки}.\label{1.2.2.1-End}\label{subsec:1.2.2_Users_perspective-End}
\end{enumerate}

\subsection{Внутренняя организация и~статус настоящих стандартов\label{subsec:1.2.3_Arragement_and_status}}

\stepcounter{SubSecCounter}

\thesubsection.\theSubSecCounter. \label{1.2.3.1} Основной содержательный
материал, изложенный в~данном издании, сгруппирован в~три блока:
в~Главах~\ref{chap:3_PS}. \nameref{chap:3_PS}, \ref{chap:4_VPS}.
\nameref{chap:4_VPS}, \ref{chap:5_Valuation_applications}. \nameref{chap:5_Valuation_applications}).
Детальное описание структуры материала приводится далее в~пунктах
\ref{1.2.3.1.1}--\ref{1.2.3.3.3-End} на~с.~\pageref{1.2.3.1.1}--\pageref{1.2.3.3.3-End}.
Главы \ref{chap:3_PS} и~\ref{chap:4_VPS} освещают вопросы, затрагивающие
задачи \textsl{оценки} в~целом, Глава~\ref{chap:5_Valuation_applications}
"--- касающиеся конкретных случаев \textsl{оценки}. Смысл такого
структурирования заключается в~том, чтобы яснее показать \textsl{членам
RICS}, какие положения стандартов носят обязательный, а~какие "---
рекомендательный характер, таким образом, положения, имеющие статус
обязательных, приводятся в~Главах~\ref{chap:3_PS} и~\ref{chap:4_VPS},
имеющие рекомендательный характер "--- в~Главе~\ref{chap:5_Valuation_applications}.\label{1.2.3.1-End}

\subsubsection{Стандарты профессиональной деятельности "--- обязательны к~применению\label{subsubsec:1.2.3.1_PS-mandatory}}

\stepcounter{SubSubSecCounter}

\thesubsubsection.\theSubSubSecCounter. \label{1.2.3.1.1} Стандарты
оказания профессиональных услуг и~стандарты этики, называемые в~совокупности
\textbf{\guillemotleft Стандарты профессиональной деятельности\guillemotright},
непосредственно применимы в~отношении \emph{оценщиков} и~имеют префикс
\textbf{«СПД»} перед номером (в~ориг. англ. тексте "--- \textbf{PS}).
Они~носят характер обязательных (если не~указано иное) во~всех
случаях оказания оценочных услуг, результатом которых является письменный
\emph{отчёт об~оценке}. Они~определяют критерии соответствия требованиям
RVGS/RBG в~том~числе требованиям \href{https://www.rics.org/globalassets/rics-website/media/upholding-professional-standards/sector-standards/valuation/international-valuation-standards-rics2.}{МСО}~\cite{IVS-2020},
устанавливают соответствующие нормативные требования RICS, содержат
разъяснения порядка применения \href{https://www.rics.org/ssa/upholding-professional-standards/standards-of-conduct/rules-of-conduct/}{Правил поведения RICS}~\cite{RICS:Conduct,RICS:Conduct-Firms,RICS:Conduct-Members}
в~процессе осуществления оценочной деятельности \textsl{членами RICS}.
Они состоят из:
\begin{itemize}
\item \textbf{СПД~1. Соответствие настоящим стандартам при~подготовке
письменного отчёта об~оценке} "--- описывает порядок обеспечения
соответствия стандартам RICS в~случаях проведения \textsl{оценки},
результатом которой является письменный \emph{отчёт об~оценке} (подраздел~\ref{sec:3.1_PS1_Compliance_of_written_valuation}
\vpageref{sec:3.1_PS1_Compliance_of_written_valuation}--\pageref{sec:3.1_PS1_Compliance_of_written_valuation-End});
\item \textbf{СПД~2. Этика, компетентность, объективность и~раскрытие
информации} "--- рассматривает вопросы этики, компетентности, объективности
и~раскрытия информации (подраздел~\ref{sec:3.2_PS2_Ethics_competency_objectivity}
\vpageref{sec:3.2_PS2_Ethics_competency_objectivity}--\pageref{sec:3.2_PS2_Ethics_competency_objectivity-End}).\label{1.2.3.1.1-End}\label{subsubsec:1.2.3.1_PS-mandatory-End}
\end{itemize}

\subsubsection{Стандарты проведения оценки и~подготовки отчёта об~оценке "---
обязательны к~применению\label{subsubsec:1.2.3.2_VPS-mandatory}}

\stepcounter{SubSubSecCounter}

\thesubsubsection.\theSubSubSecCounter. \label{1.2.3.2.1}\textbf{
Стандарты проведения оценки и~подготовки отчёта об~оценке} обозначаются
префиксом \textbf{«СПО»} (в~ориг. англ. тексте "--- \textbf{VPS})
перед номером (глава~\ref{chap:4_VPS} \vpageref{chap:4_VPS}--\pageref{chap:4_VPS-End}).
Они~содержат конкретные обязательные (если не~указано иное) требования
и~соответствующие указания по~проведению \textsl{оценки} в~соответствии
с~\href{https://www.rics.org/globalassets/rics-website/media/upholding-professional-standards/sector-standards/valuation/international-valuation-standards-rics2.}{МСО}~\cite{IVS-2020}.
СПО включают в~себя следующие разделы:
\begin{itemize}
\item СПО 1. Требования к~\textsl{договору на~проведение оценки} и~\emph{заданию
на~оценку} (раздел~\ref{sec:4.1_VPS1_Terms_of_engagement_Scope_of_work}
\vpageref{sec:4.1_VPS1_Terms_of_engagement_Scope_of_work}--\pageref{sec:4.1_VPS1_Terms_of_engagement_Scope_of_work-End});
\item СПО 2. Осмотры, исследования и~фиксация их~результатов (\ref{sec:4.2_VPS2_Inspections_investigations_and_records}
\vpageref{sec:4.2_VPS2_Inspections_investigations_and_records}--\pageref{sec:4.2_VPS2_Inspections_investigations_and_records-End});
\item СПО 3. Требования к~\emph{отчёту об~оценке} (\ref{sec:4.3_VPS3_Valuation_reports}
\vpageref{sec:4.3_VPS3_Valuation_reports}--\pageref{sec:4.3_VPS3_Valuation_reports-End});
\item СПО 4. \textsl{Виды стоимости}, \textsl{допущения} и~\textsl{специальные
допущения} (\ref{sec:4.4_VPS4_Bases_of_value} \vpageref{sec:4.4_VPS4_Bases_of_value}--\pageref{sec:4.4_VPS4_Bases_of_value-End});
\item СПО 5. \textsl{Подходы к~оценке} и~её~\emph{методы} (\ref{sec:4.5_VPS5_Valuation_approaches_and_methods}
\vpageref{sec:4.5_VPS5_Valuation_approaches_and_methods}--\pageref{sec:4.5_VPS5_Valuation_approaches_and_methods-End}).\label{1.2.3.2.1-End}
\end{itemize}
\stepcounter{SubSubSecCounter}

\thesubsubsection.\theSubSubSecCounter. \label{1.2.3.2.2}В~то время
как~\nameref{sec:4.1_VPS1_Terms_of_engagement_Scope_of_work}, \nameref{sec:4.4_VPS4_Bases_of_value}
и~\nameref{sec:4.5_VPS5_Valuation_approaches_and_methods} в~большей
степени посвящены вопросам самого процесса \textsl{оценки}, а~\nameref{sec:4.2_VPS2_Inspections_investigations_and_records}
и~\nameref{sec:4.3_VPS3_Valuation_reports} в~основном освещают
вопросы изложения в~\emph{отчёте об~оценке} и~его~подготовки,
дальнейшее углубление их~специфики является нецелесообразным. Организация
СПО~(VPS) повторяет организацию \href{https://www.rics.org/globalassets/rics-website/media/upholding-professional-standards/sector-standards/valuation/international-valuation-standards-rics2.}{МСО}~\cite{IVS-2020},
для~разъяснения вопросов применения которых и~были разработаны СПО~(VPS).\label{1.2.3.2.2-End}\label{subsubsec:1.2.3.2_VPS-mandatory-End}

\subsubsection{Практические руководства по~оценке приложения "--- имеют рекомендательный
характер\label{subsubsec:1.2.3.3_VPGA-advisory}}

\stepcounter{SubSubSecCounter}

\thesubsubsection.\theSubSubSecCounter. \label{1.2.3.3.1} Практические
руководства по~оценке обозначаются префиксом \textbf{«ПРО»} (в~ориг.
англ. тексте "--- \textbf{VPGA}) перед номером. Они~содержат сведения
о~практическом применении стандартов в~конкретных случаях. Таким
образом, они~охватывают вопросы \textsl{оценки} для~конкретных целей
(среди которых наиболее распространёнными являются цели \textsl{финансовой
отчётности} и~залогового кредитования), а~также аспекты \textsl{оценки}
отдельных видов активов, затрагивающие отдельные вопросы и~практические
соображения. ПРО являются воплощением передовой практики, т.\,е.~совокупности
процедур, которые, по~мнению RICS, соответствуют высоким стандартам
профессионального мастерства \emph{оценщика}.\label{1.2.3.3.1-End}

\stepcounter{SubSubSecCounter}

\thesubsubsection.\theSubSubSecCounter. \label{1.2.3.3.2} Сами по~себе
ПРО не~являются обязательными, однако они~включают в~себя ссылки
на~материал \href{https://www.rics.org/globalassets/rics-website/media/upholding-professional-standards/sector-standards/valuation/international-valuation-standards-rics2.}{МСО}~\cite{IVS-2020},
а~также на~другие части RVGS/RBG, имеющие статус обязательных. Это~сделано
для~помощи членам RICS в~выявлении материалов, относящихся к~конкретных
случаям проводимой ими~\textsl{оценки}.\label{1.2.3.3.2-End}

\stepcounter{SubSubSecCounter}

\thesubsubsection.\theSubSubSecCounter. \label{1.2.3.3.3} ПРО включают
в~себя следующие материалы:
\begin{itemize}
\item ПР 1. Оценка для~целей финансовой отчётности (\ref{sec:5.1_VPGA-1_Valuation_for_financial_statements}
\vpageref{sec:5.1_VPGA-1_Valuation_for_financial_statements}--\pageref{sec:5.1_VPGA-1_Valuation_for_financial_statements-End}).
\item ПР 2. Оценка прав в~целях залогового обеспечения (\ref{sec:5.2_VPGA-2_Valuation_for_secure_lending}
\vpageref{sec:5.2_VPGA-2_Valuation_for_secure_lending}--\pageref{sec:5.2_VPGA-2_Valuation_for_secure_lending-End}).
\item ПР 3. Оценка бизнеса и~долей в~нём (\ref{sec:5.3_VPGA-3_Valuation_of_businesses}
\vpageref{sec:5.3_VPGA-3_Valuation_of_businesses}--\pageref{sec:5.3_VPGA-3_Valuation_of_businesses-End}).
\item ПР 4. Оценка специализированной недвижимости (\ref{sec:5.4_VPGA-4_Valuation_of_trade_properties}
\vpageref{sec:5.4_VPGA-4_Valuation_of_trade_properties}--\pageref{sec:5.4_VPGA-4_Valuation_of_trade_properties-End}).
\item ПР 5. Оценка машин и~оборудования (\ref{sec:5.5_VPGA-5-Valuation_of_plant_and_equipment}
\vpageref{sec:5.5_VPGA-5-Valuation_of_plant_and_equipment}--\pageref{sec:5.5_VPGA-5-Valuation_of_plant_and_equipment-End}).
\item ПР 6. Оценка прав на~нематериальные активов (\ref{sec:5.6_VPGA-6_Valuation_of_intangible_assets}
\vpageref{sec:5.6_VPGA-6_Valuation_of_intangible_assets}--\pageref{sec:5.6_VPGA-6_Valuation_of_intangible_assets-End}).
\item ПР 7. Оценка личной собственности, в~т.\,ч.~предметов искусства
и~антиквариата (\ref{sec:5.7_VPGA-7_Valuation_of_personal_propetry}
\vpageref{sec:5.7_VPGA-7_Valuation_of_personal_propetry}--\pageref{sec:5.7_VPGA-7_Valuation_of_personal_propetry-End}).
\item ПР 8. Оценка недвижимого имущества и~прав на~него (\ref{sec:5.8_VPGA-8_Valuation_of_real_property}
\vpageref{sec:5.8_VPGA-8_Valuation_of_real_property}--\pageref{sec:5.8_VPGA-8_Valuation_of_real_property-End}).
\item ПР 9. Выявление портфелей, коллекций и~групп активов и~имущества
(\ref{sec:5.9_VPGA-9_Identification_of_portfolios} \vpageref{sec:5.9_VPGA-9_Identification_of_portfolios}--\pageref{sec:5.9_VPGA-9_Identification_of_portfolios-End}).
\item ПР 10. Причины возникновения существенной неопределённости оценки
(\ref{sec:5.10_VPGA-10_Matters_for_uncertainty} \vpageref{sec:5.10_VPGA-10_Matters_for_uncertainty}--\pageref{sec:5.10_VPGA-10_Matters_for_uncertainty-End}).\label{1.2.3.3.3-End}\label{subsubsec:1.2.3.3_VPGA-advisory-End}
\end{itemize}

\subsubsection{Национальные либо территориальные стандарты оценки\label{subsubsec:1.2.3.4_National_or_Jurisdictional}}

\stepcounter{SubSubSecCounter}

\thesubsubsection.\theSubSubSecCounter. \label{1.2.3.4.1} RICS публикует
"--- отдельно, но~в~тесной связи с~данными стандартами "--- \href{https://www.isurv.com/info/1342/rics_national_or_jurisdictional_valuation_standards}{дополнения для отдельных национальных юрисдикций}~\cite{RICS:National-Standards},
обычно называемые \guillemotleft Стандартами применения RVGS/RBG в~отдельных
юрисдикциях\guillemotright{} либо \guillemotleft Стандартами оценки
национальных ассоциаций RICS\guillemotright ,\footnote{Прим. пер.: \href{https://www.isurv.com/site/scripts/download_info.php?downloadID=2009}{соответствующий Стандарт применения RVGS/RBG в России}~\cite{RICS:Application-in-RF}
был основан на~устаревшей сейчас \href{https://www.isurv.com/downloads/download/1830/red_book_rics_valuation_-_professional_standards_global_2014_archived}{версии RVGS/RBG 2014 г.}~\cite{RICS:RVGS-2014}
и~имеет статус «утратил действие».} способствующих применению данных стандартов в~условиях работы в~конкретных
странах и~их~соотнесению с~местными особенностями. Они~созданы
для~того, чтобы, сохраняя в~целом, единство и~преемственность с~соответствующими
международными стандартами, охватывать и~соблюдать также законодательные
и~регуляторные требования, принятые в~этих странах. Данный подход
полностью соответствует добровольно применяемым \href{http://www.fao.org/3/a-i2801e.pdf}{Принципам ответственного землепользования}~\cite{FAO:Tenure}
\href{https://en.wikipedia.org/wiki/Food_and_Agriculture_Organization}{Продовольственной и сельскохозяйственной Организации Объединённых Нацистов}~\cite{FAO:Wiki},
поощряющей своих членов повышать прозрачность и~общую последовательность
при~проведении \textsl{оценки}. Вопросы соответствия требованиям
национальных стандартов рассмотрены подробнее в~разделе \ref{sec:3.1_PS1_Compliance_of_written_valuation}~\nameref{sec:3.1_PS1_Compliance_of_written_valuation}
\vpageref{sec:3.1_PS1_Compliance_of_written_valuation}--\pageref{sec:3.1_PS1_Compliance_of_written_valuation-End}.\label{1.2.3.4.1-End}

\stepcounter{SubSubSecCounter}

\thesubsubsection.\theSubSubSecCounter. \label{1.2.3.4.2} \href{https://www.isurv.com/info/1342/rics_national_or_jurisdictional_valuation_standards}{Национальные стандарты}~\cite{RICS:National-Standards}
и~сопроводительная документация к~ним~доступны в~соответствующем
\href{https://www.isurv.com/info/1342/rics_national_or_jurisdictional_valuation_standards}{разделе}~\cite{RICS:National-Standards}
на~\href{https://www.rics.org/eu/}{сайте RICS}~\cite{RICS:site}.\label{1.2.3.4.2-End}\label{subsubsec:1.2.3.4_National_or_Jurisdictional-End}\label{subsec:1.2.3_Arragement_and_status-End}

\subsection{Дата вступление в~силу настоящих стандартов, срок их~действия и~порядок
внесения поправок к~ним\label{subsec:1.2.4_Effective_date+Duration+Ammendments}}

\subsubsection{Дата вступления в~силу настоящих стандартов\label{subsubsec:1.2.4.1_Effective_date}}

\stepcounter{SubSubSecCounter}

\thesubsubsection.\theSubSubSecCounter. \label{1.2.4.1.1}Данные
стандарты вступают в~силу с~31~января~2020~г. и~должны применяться
во~всех случаях, когда \textsl{дата оценки} приходится на~эту~либо
более позднюю дату.\label{1.2.4.1.1-End}\label{subsubsec:1.2.4.1_Effective_date-End}

\subsubsection{Актуальность текста настоящих стандартов\label{subsubsec:1.2.4.2_Currency_of_the_text}}

\stepcounter{SubSubSecCounter}

\thesubsubsection.\theSubSubSecCounter. \label{1.2.4.2.21} Действующая
редакция Всемирных стандартов RICS всегда размещается в~\href{https://www.rics.org/eu/upholding-professional-standards/sector-standards/valuation/red-book/red-book-global/}{соответствующем разделе}~\cite{RVGS:current}
\href{https://www.rics.org/eu/}{сайта RICS}~\cite{RICS:site}.
Всем пользователям данного материала следует самостоятельно отслеживать
внесение последующих изменений.\label{1.2.4.2.21-End}\label{subsubsec:1.2.4.2_Currency_of_the_text-End}

\subsubsection{Внесение изменений и~размещение предварительных версий\label{subsubsec:1.2.4.3_Ammendments_and_drafts}}

\stepcounter{SubSubSecCounter}

\thesubsubsection.\theSubSubSecCounter.\label{1.2.4.3.1} Содержание
данных стандартов подлежит регулярному пересмотру "--- изменения
и~дополнения будут публиковаться по~мере необходимости. В~этом
случае \textsl{члены RICS} будут уведомлены посредством электронных
каналов связи. Подобные изменения будут незамедлительно внесены в~\href{https://www.rics.org/globalassets/rics-website/media/upholding-professional-standards/sector-standards/valuation/rics-valuation--global-standards-jan.pdf}{электронное издание Стандартов}~\cite{RICS:RVGS-2020},
однако в~печатную версию они~будут внесены только при~её~последующих
изданиях.\label{1.2.4.3.1-End}

\stepcounter{SubSubSecCounter}

\thesubsubsection.\theSubSubSecCounter. \label{1.2.4.3.2} В~тех
ситуациях, когда \uline{изменения} способны привести к~возникновению
существенных последствий, например в~случае внесения поправок в~\textbf{Стандарты
проведения оценки и~подготовки отчёта об~оценке} (СПО/VPS) либо
в~\textbf{Практические руководства по~оценке} (ПРО), \uline{они}~могут
быть опубликованы в~формате предварительных версий. Предварительные
версии будут размещаться в~\href{https://www.rics.org/eu/upholding-professional-standards/sector-standards/valuation/red-book/red-book-global/}{соответствующем разделе}~\cite{RVGS:drafts}
\href{https://www.rics.org/eu/}{официального сайта RICS}~\cite{RICS:site}
и~содержать текст, согласованный к~выпуску \href{https://www.rics.org/uk/about-rics/corporate-governance/standards-regulation-board/}{Советом RICS по всемирным стандартам оценки}~\cite{RVGS:Board}
для~их~общественного обсуждения.\label{1.2.4.3.2-End}

\stepcounter{SubSubSecCounter}

\thesubsubsection.\theSubSubSecCounter. \label{1.2.4.3.3} Целью
размещения предварительных версий является предоставление возможности
\textsl{членам RICS} и~иным заинтересованным лицам высказать своё
мнение по~предлагаемому тексту и, возможно, выявить в~нём~недостатки
до~его~официального включения в~текст RVGS/RBG. После рассмотрения
\href{https://www.rics.org/uk/about-rics/corporate-governance/standards-regulation-board/}{Советом RICS по всемирным стандартам оценки}~\cite{RVGS:Board}
всех замечаний и~итогового утверждения с~его стороны, поправки,
изложенные в~предварительной версии, получают статус вступивших в~действие
с~даты следующего обновления RVGS/RBG, выходящего после их~публикации.
Как~уже~было сказано выше, в~случае внесения изменений, \textsl{члены
RICS} будут уведомлены об~этом посредством электронных каналов связи
в~информационно-телекоммуникационной сети \guillemotleft Интернет\guillemotright{}
(далее ИТС \guillemotleft Интернет\guillemotright ).\footnote{Определение приведено согласно \href{http://docs.cntd.ru/document/901990051}{Федеральному закону «Об информации, информаци- онных технологиях и о защите информации (с изменениями на 30 декабря 2020 г.)}~\cite{FZ-149}.}\label{1.2.4.3.3-End}

\stepcounter{SubSubSecCounter}

\thesubsubsection.\theSubSubSecCounter.\label{1.2.4.3.4} \href{https://www.rics.org/uk/about-rics/corporate-governance/standards-regulation-board/}{Совет RICS по всемирным стандартам оценки}~\cite{RVGS:Board}
приветствует предложения по~включению дополнительных материалов и~будет
рад ответить на~запросы о~разъяснении положений RVGS/RBG.\label{1.2.4.3.4-End}\label{subsubsec:1.2.4.3_Ammendments_and_drafts-End}\label{subsec:1.2.4_Effective_date+Duration+Ammendments-End}

\subsection{Статус МСО в~разрезе настоящих стандартов\label{subsec:1.2.5_IVS_status}}

\subsubsection{Дата вступления в~силу, срок действия и~внесение поправок в~МСО\label{subsubsec:1.2.5.1_IVS_effective_date_and_ammendments}}

\stepcounter{SubSubSecCounter}

\thesubsubsection.\theSubSubSecCounter. \label{1.2.5.1.1} \href{https://www.rics.org/globalassets/rics-website/media/upholding-professional-standards/sector-standards/valuation/international-valuation-standards-rics2.}{Международные стандарты оценки}~\cite{IVS-2020},
полный текст которых приводится в~Части \vref{part:II_IVS}--\pageref{part:II_IVS-End}
данного материала, утверждены \href{https://www.ivsc.org/}{Советом по международным стандартам оценки}~\cite{IVSC:site}
и~вступили в~силу 31~Января~2020~г.\label{1.2.5.1.1-End}

\stepcounter{SubSubSecCounter}

\thesubsubsection.\theSubSubSecCounter. \label{1.2.5.1.2} \textsl{Членам
RICS} следует иметь ввиду, что~\href{https://www.ivsc.org/}{Совет по международным стандартам оценки}~\cite{IVSC:site}
оставляет за~собой право в~любое время вносить поправки в~\href{https://www.rics.org/globalassets/rics-website/media/upholding-professional-standards/sector-standards/valuation/international-valuation-standards-rics2.}{МСО}~\cite{IVS-2020}.
В~этом случае поправки к~данному изданию RVGS/RBG будут внесены
в~кратчайшие сроки и~опубликованы в~\href{https://www.rics.org/eu/upholding-professional-standards/sector-standards/valuation/red-book/red-book-global/}{соответствующем разделе}~\cite{RVGS:current}
\href{https://www.rics.org/eu/}{официального сайта RICS}~\cite{RICS:site}
(см.~\vpageref{subsubsec:1.2.4.2_Currency_of_the_text}--\pageref{subsubsec:1.2.4.2_Currency_of_the_text-End}
и~\vref{subsubsec:1.2.4.3_Ammendments_and_drafts}--\pageref{subsubsec:1.2.4.3_Ammendments_and_drafts-End}).\label{1.2.5.1.2-End}

\stepcounter{SubSubSecCounter}

\thesubsubsection.\theSubSubSecCounter. \label{1.2.5.1.3} \href{https://www.rics.org/globalassets/rics-website/media/upholding-professional-standards/sector-standards/valuation/international-valuation-standards-rics2.}{МСО}~\cite{IVS-2020}
принимаются и~применяются \textsl{членами RICS} через призму данных
всемирных стандартов RICS. Главы \ref{chap:3_PS}, \ref{chap:4_VPS},
\ref{chap:5_Valuation_applications} содержат для~удобства ссылки
на~соответствующие разделы \href{https://www.rics.org/globalassets/rics-website/media/upholding-professional-standards/sector-standards/valuation/international-valuation-standards-rics2.}{МСО}~\cite{IVS-2020}.\label{1.2.5.1.28-End}

\textbf{Важное замечание. }\textbf{\textsl{Члены RICS}}\textbf{ обязаны
самостоятельно отслеживать изменения в~законодательстве и~правоприменительной
практике, произошедшие после даты выхода настоящего издания Всемирных
стандартов RICS, а~также быть в~курсе изменений \href{https://www.rics.org/globalassets/rics-website/media/upholding-professional-standards/sector-standards/valuation/international-valuation-standards-rics2.}{МСО}~}\cite{IVS-2020}\textbf{
и~иных стандартов}\textbf{\textsl{ оценки}}\textbf{, имеющих отношение
к~конкретной работе по~определению стоимости. За~любыми обновлениями
материалов RICS, включая изменения, следующие из~поправок к~\href{https://www.rics.org/globalassets/rics-website/media/upholding-professional-standards/sector-standards/valuation/international-valuation-standards-rics2.}{МСО}~}\cite{IVS-2020}\textbf{,
Оценщикам следует обращаться к~\href{https://www.rics.org/eu/}{официальному сайту RICS}
в~ИТС \guillemotleft Интернет\guillemotright . В~более широком смысле,
}\textbf{\emph{оценщикам}}\textbf{ даётся напоминание об~их~ответственности
в~части соответствия \href{https://www.rics.org/ssa/upholding-professional-standards/regulation/cpd-compliance-guide/}{принципам непрерывного профессионального развития}
(далее "--- }\textbf{\emph{НПР/CPD}}\textbf{)~\cite{RICS:CPD},
соблюдение которых гарантирует соответствие широким требованиям к~знаниям,
опыту и~квалификации, ожидаемым от~}\textbf{\textsl{членов RICS}}\textbf{,
и~отражённым в~данных стандартах.}\label{subsubsec:1.2.5.1_IVS_effective_date_and_ammendments-End}\label{subsec:1.2.5_IVS_status-End}\label{sec:1.2_Coverage-End}\label{chap:1_Introduction-End}

\chapter{Глоссарий\label{chap:2_Glossary}}

\section{Глоссарий терминов RICS\label{sec:2.1_RICS_Glossary}}

Данный глоссарий\footnote{Прим. пер.: в~данном глоссарии термины приводятся в~том порядке,
в~котором они~приводятся в~оригинальном английском тексте, т.\,е.
имеет место нарушение принятого для~глоссария принципа сортировки
по~алфавиту. Это~решение обусловлено необходимостью приближения
текста к~оригинальному для~упрощения сопоставления с~ним.} содержит определения терминов, используемых в~настоящих стандартах
и~имеющих специальное либо ограниченное значение. Слова и~словосочетания,
не~встречающиеся в~Глоссарии, следует трактовать согласно их~общепринятому
лексическому смыслу. Если термин, приведённый в~данном Глоссарии,
используется дальше в~тексте стандартов, он~выделяется \textsl{наклонным
шрифтом}, который не~следует путать с~\textit{курсивом}, которым
выделяются важные понятия либо иные слова или~словосочетания, требующие
акцента внимания.

\textsl{Членам RICS} следует обратить внимание на~тот~факт, что~\href{https://www.rics.org/globalassets/rics-website/media/upholding-professional-standards/sector-standards/valuation/international-valuation-standards-rics2.}{МСО}~\cite{IVS-2020},
приведённые в~Части \ref{part:II_IVS}, включают в~себя краткий
специальный глоссарий, содержащий некоторые дополнительные определения,
специально предназначенные для~их~правильного понимания и~применения,
в~т.\,ч. содержащие сведения о~статусе того или~иного положения
\href{https://www.rics.org/globalassets/rics-website/media/upholding-professional-standards/sector-standards/valuation/international-valuation-standards-rics2.}{МСО}~\cite{IVS-2020}
в~части того, имеет~ли оно~статус обязательного, рекомендательного
и~т.\,д. Данный глоссарий не~воспроизводится здесь. Стандарты,
издаваемые \href{https://en.wikipedia.org/wiki/International_Valuation_Standards_Council}{Советом по международным стандартам оценки}~\cite{IVSC:Wiki}
отдельно, также содержат определения специфичные для~конкретных разделов
\href{https://www.rics.org/globalassets/rics-website/media/upholding-professional-standards/sector-standards/valuation/international-valuation-standards-rics2.}{МСО}~\cite{IVS-2020},
к~которым \emph{оценщикам} следует обращаться по~мере необходимости.

\href{https://www.isurv.com/info/1342/rics_national_or_jurisdictional_valuation_standards}{Стандартами применения RVGS/RBG в отдельных юрисдикциях}~\cite{RICS:National-Standards}
могут быть установлены дополнительные термины, определяемые и~утверждаемые
в~контексте конкретного национального стандарта.
\begin{description}
\item [{Допущение\label{Gloss:assumption}}] "--- предположение, принимаемое
в~качестве истины. Оно~содержит факты, условия или~обстоятельства,
затрагивающие свойства \textsl{объекта оценки} либо вопросы \emph{подходов
к~оценке}, которые, по~соглашению сторон \textsl{договора на~проведение
оценки}, не~нуждаются в~проверке со~стороны \emph{оценщика} в~рамках
осуществляемого им~процесса проведения \textsl{оценки}. Как~правило,
\textsl{допущение} вводится тогда, когда отсутствует необходимость
проведения \emph{оценщиком} специального исследования по~доказыванию
истинности того или~иного факта.\label{Glossary:assumption-End}
\item [{Вид~стоимости}] "--- утверждение об~основных \textsl{допущениях},
применяемых в~процессе измерения стоимости в~рамках проводимой \textsl{оценки}.
\item [{Затратный~подход}] "--- подход, реализующий представление о~стоимости,
основанное на~экономическом принципе, согласно которому, покупатель
не~заплатит за~актив такую цену, которая превышала~бы стоимость
приобретения актива аналогичного по~полезности путём его~покупки
либо создания.
\item [{Дата~отчёта}] "--- дата подписания \emph{оценщиком} \emph{отчёта
об~оценке}.
\item [{Дата~проведения~оценки}] "--- см.~\hyperref[Valuation_DAte]{textsl{дата оценки}}.
\item [{Отступление~от~стандартов}] "--- особые обстоятельства, при~которых
обязательное применение положений данных Всемирных стандартов может
быть неуместным либо нецелесообразным (см.~\vref{subsec:3.1.6_Departures}--\pageref{subsec:3.1.6_Departures-End}).
\item [{Остаточная~стоимость~замещения~(ОСЗ)}] "--- текущая стоимость
замещения актива его~современным аналогом, уменьшенная на~величину
его~физического износа и~всех присущих ему~устареваний и~оптимизаций.
\item [{Равновесная~стоимость~(справедливая~стоимость~в~терминологии~\href{https://www.rics.org/globalassets/rics-website/media/upholding-professional-standards/sector-standards/valuation/international-valuation-standards-rics2.}{МСО}~\cite{IVS-2020})}] "---
расчётная цена передачи актива или~обязательства между идентифицированными,
осведомлёнными и~заинтересованными в~такой передаче сторонами, отражающая
их~соответствующие интересы (см.~\vref{par:9.4.2.1.3_Equitable_Value}--\pageref{par:9.4.2.1.3_Equitable_Value-End}).
\item [{Внешний~оценщик\label{Gloss:External_valuer}}] "--- \emph{оценщик},
который, равно как~и~его~партнёры и~сотрудники, не~является аффилированным
лицом по~отношению к~заказчику либо его~представителю и~не~имеет
какого-либо интереса касательно предмета оценки.\label{External_valuer-End}
\item [{Справделивая~стоимость~в~терминологии~\href{http://docs.cntd.ru/document/420334241}{МСФО 13}~\cite{MSFO-13}~(\href{http://eifrs.ifrs.org/eifrs/bnstandards/en/IFRS13.pdf}{IFRS 13}~\cite{IFRS-13})\footnote{Прим. пер.: для~корректного доступа к~материалу \cite{IFRS-13}
по~\href{http://eifrs.ifrs.org/eifrs/bnstandards/en/IFRS13.pdf}{ссылке}
требуется \href{https://www.ifrs.org/register/}{регистрация на сайте IFRS}~\cite{IFRS:registration}.}}] "--- цена, которая была~бы получена на~\textsl{дату оценки}
при~продаже актива или~уплачена при~передаче обязательства в~рамках
сделки, заключённой согласно обычаям делового оборота, между участниками
рынка.
\item [{Финансовая~отчётность}] "--- письменные отчёты о~финансовом
положении физического или~юридического лица, а~также официальные
финансовые отчёты установленной формы и~содержания. Публикуются с~целью
предоставления сведений широкому и~неограниченному кругу \textsl{третьих
лиц}. \textsl{Финансовая отчётность} является элементом системы общественного
контроля и~ведётся в~соответствии с~правилами и~стандартами бухгалтерской
отчётности и~иными требованиями законодательства.
\item [{Оценочная~компания}] "--- хозяйственное общество либо иная организация,
с~которой \textsl{оценщик RICS} заключил трудовой договор либо через
которую он~ведёт свою деятельность.
\item [{Гудвилл}] "--- любая будущая экономическая выгода, возникающая
вследствие владения бизнесом или~долей в~нём либо вследствие использования
группы неделимых активов.
\item [{Доходный~подход}] "--- подход, реализующий представление о~стоимости
как~сумме будущих денежных потоков, приведённых к~мгновенной текущей
стоимости капитала.
\item [{Осмотр~объекта~оценки}] "--- посещение объекта и~его~физический
осмотр для~изучения и~получения соответствующих сведений с~целью
выражения \emph{суждения о~его~стоимости}. Следует отметить, что~физический
осмотр \emph{объекта оценки}, не~являющегося объектом недвижимости,
например предмета искусства или~антиквариата, не~следует трактовать
как~\textsl{осмотр объекта оценки}.
\item [{Нематериальный~актив}] --- нефинансовый актив, проявляющий свою
ценность через свои экономические свойства. Не~имеет физической формы,
но~предоставляет своему владельцу права и~(или) создаёт экономическую
выгоду.
\item [{Внутренний~оценщик\label{Gloss:Internal_valuer}}] "--- \emph{оценщик},
являющийся сотрудником организации, владеющей оцениваемыми активами,
либо компании, ответственной за~ведение финансового, бухгалтерского
и~(или) налогового учёта и~(или) подготовку финансовых, бухгалтерских
и~(или) налоговых отчётов данной организации. Как правило, внутренний
оценщик может отвечать требованиям независимости и~профессиональной
объективности, описанным в \vref{subsec:3.2.3_Independence_objectivity_confidentiality}--\pageref{subsec:3.2.3_Independence_objectivity_confidentiality-End},
но~не~всегда может отвечать дополнительным критериям независимости,
присущим отдельным видам оценочных заданий, например таким, которые
описаны в~\vref{3.2.3.4}--\pageref{3.2.3.4-End}.\label{Internal_valuer-End}
\item [{Международные~стандарты~финансовой~отчётности~(\href{http://docs.cntd.ru/document/420332842/}{МСФО}~\cite{MSFO-all}~\href{https://www.ifrs.org/issued-standards/list-of-standards/}{IFRS}~\cite{IFRS-all}))}] "---
стандарты, установленные \href{https://ru.wikipedia.org/wiki/\%D0\%A1\%D0\%BE\%D0\%B2\%D0\%B5\%D1\%82_\%D0\%BF\%D0\%BE_\%D0\%9C\%D0\%B5\%D0\%B6\%D0\%B4\%D1\%83\%D0\%BD\%D0\%B0\%D1\%80\%D0\%BE\%D0\%B4\%D0\%BD\%D1\%8B\%D0\%BC_\%D1\%81\%D1\%82\%D0\%B0\%D0\%BD\%D0\%B4\%D0\%B0\%D1\%80\%D1\%82\%D0\%B0\%D0\%BC_\%D1\%84\%D0\%B8\%D0\%BD\%D0\%B0\%D0\%BD\%D1\%81\%D0\%BE\%D0\%B2\%D0\%BE\%D0\%B9_\%D0\%BE\%D1\%82\%D1\%87\%D1\%91\%D1\%82\%D0\%BD\%D0\%BE\%D1\%81\%D1\%82\%D0\%B8}{Советом по международным стандартам финансовой отчётности}~\cite{Wiki:IASB-rus}\cite{Wiki:IASB-rus}
(\href{https://en.wikipedia.org/wiki/International_Accounting_Standards_Board}{International Accounting Standards Board — IASB}~\cite{Wiki:IASB})
с~целью достижения единообразия принципов бухгалтерского учёта. Стандарты
разрабатываются в~рамках единой концепции для~того, чтобы элементы
\textsl{финансовой отчётности} воспринимались и~обрабатывались единым
образом.
\item [{Инвестиционная~собственность}] "--- имущество, представляющее
собой земельный участок либо здание или~часть здания, либо то~и~другое
вместе взятое, содержащееся собственником в~целях получения арендной
платы либо прироста стоимости, либо в~обеих этих целях, а~не~для:\begin{itemize}
\item производства и~поставки товаров и~услуг либо административных целей;
\item продаж в~рамках обычной хозяйственной деятельности.
\end{itemize}
\item [{Инвестиционная~стоимость~или~ценность\label{Gloss:Investment_value}}] "---
стоимость актива для~владельца или~потенциального владельца, отражающая
экономический эффект его~индивидуальных инвестиционных или~операционных
целей (см.~\vref{par:9.4.2.1.4_Investment_Value}--\pageref{par:9.4.2.1.4_Investment_Value-End}).
Также может обозначаться термином «ценность».
\item [{Сравнительный~подход}] "--- подход, реализующий представление
о~стоимости на~основе сравнения оцениваемого актива с~идентичными
либо схожими активами, в~отношении которых имеются ценовая информация
и~данные.
\item [{Рыночная~арендная~плата~(РАП)}] "--- ожидаемая денежная сумма,
за~которую на~\textsl{дату оценки} недвижимое имущество может быть
сдано в~аренду в~рамках сделки между заинтересованными арендодателем
и~арендатором, заключённой после надлежащего изучения рынка на~взаимоприемлемых
условиях в~случае, когда обе~стороны являются независимыми, проявляют
должную осмотрительность и~осведомлённость, действуют разумно, в~своих
интересах и~без~принуждения (см.~\vref{par:9.4.2.1.2_Market_Rent}--\pageref{par:9.4.2.1.2_Market_Rent-End}).
\item [{Рыночная~стоимость}] "--- ожидаемая величина денежной суммы,
за~которую актив или~обязательство могут быть переданы на~\textsl{дату
оценки} в~рамках сделки, заключённой на~рыночных условиях между
заинтересованными продавцом и~покупателем после надлежащего изучения
рынка на~взаимоприемлемых условиях в~случае, когда обе стороны являются
независимыми, проявляют должную осмотрительность и~осведомлённость,
действуют разумно, в~своих интересах и~без~принуждения (см.~\vref{par:9.4.2.1.1_Market_Value}--\pageref{par:9.4.2.1.1_Market_Value-End}).
\item [{Синергетическая~стоимость}] "--- дополнительная составляющая
стоимости, создаваемая за~счёт комбинации двух или~более активов
либо имущественных прав, при~которой стоимость объединённых активов
(прав) больше суммы стоимостей каждого из~них~по~отдельности.
\item [{Член~RICS}] "--- ассоциированный (AssocRICS/TechRICS), профессиональный
(MRICS), заслуженный (FRICS) либо почётный (HonRICS) член RICS~\cite{RICS:membership}.
\item [{Личная~собственность}] "--- подразумевает активы (обязательства),
не~имеющие постоянной связи с~землёй или~зданиями.\begin{itemize}
\item \textbf{Включает}: предметы изобразительного и~декоративного искусства, антиквариат, картины, драгоценные камни и~ювелирные изделия, предметы коллекционирования, предметы интерьера и~домашнего обихода и~иные подобные предметы. Данные перечень не~является исчерпывающим.
\item \textbf{Не включает}: торговое оборудование и~инвентарь, \textit{машины и~оборудование}, права на~хозяйственные общества и~доли в~них, \textit{нематериальные активы}.
\end{itemize}
\item [{Машины~и~оборудование}] --- подразделяются на~следующие обобщённые
категории:\begin{itemize}
\item \textbf{производственный комплекс} "--- совокупность активов, включающая в~себя в~т.\,ч.~такие элементы как~объекты инфраструктуры, коммунального хозяйства, установки для~обслуживания зданий, специализированные здания и~сооружения, а~также машины и~оборудование, образующие единую производственную линию;
\item \text{машины} "--- обособленная единица, группа, парк или~массовая система настроенных машин~(технологий) (включая движимое имущество такое как~транспортные средства, подвижной железнодорожный состав, морские и~воздушные суда), используемые в~связи с~производственнами или~коммерческими процессами, в~торговле и~бизнесе непосредственно на~месте либо удалённо управляемые (машина "--- устройство, используемое для~определённого процесса);
\item \textbf{оборудование} "--- всеобъемлющий термин для~других активов таких как~прочее оборудование, инструменты, приспособления и~оснастка, мебель и~предметы интерьера, торговые принадлежности и~инвентарь, ручной инструмент и~расходные материалы, имеющие вспомогательный характер для~деятельности предприятия либо имущественного комплекса.
\end{itemize}
\item [{Недвижимое~имущество}] "--- земельные участки и~все~объекты,
являющиеся их~естественной частью (например деревья и~полезные ископаемые)
либо неразрывно связанные с~землёй объекты (например здания и~элементы
благоустройства), а~также все~капитальные сооружения (например механическая
установка или~электрическая подстанция, предназначенные для~обслуживания
здания). Следует обратить внимание на~то, что~права собственности,
распоряжения, пользования и~владения на~земельные участки и~их~улучшения
определяются как~права на~недвижимое имущество (см.~\vref{sec:10.5_IVS-400_Real_Property}--\pageref{sec:10.5_IVS-400_Real_Property-End}).
\item [{Зарегистрированный~для~регулирования~со~стороны~RICS}] "---
\begin{enumerate}
\item \textsl{оценочная компания}, зарегистрированная в~качестве регулируемой со~стороны RICS в~соответствии с~его~нормативными актами;
\item \textsl{член RICS}, зарегистрированый в~качестве оценщика согласно \href{https://www.rics.org/eu/upholding-professional-standards/regulation/valuer-registration/}{Программе регистрации оценщиков RICS}~\cite{RICS:Registration}.
\end{enumerate}
\item [{Специальное~допущение}] "--- \textsl{допущение}, основанное либо
на~фактах, отличающихся от~реально существовавших на~\textsl{дату
оценки}, либо на~фактах, которые не~были~бы приняты во~внимание
типичным участником рынка при~совершении сделки с~\emph{объектом
оценки}.
\item [{Специальный~покупатель}] "--- конкретный покупатель, для~которого
конкретный актив имеет особую ценность, вытекающую из~преимуществ,
возникающих вследствие его~права собственности, которые не~были~бы
доступны другим покупателям на~рынке.
\item [{Специальная~стоимость}] "--- значение стоимости, отражающее отдельные
свойства актива, имеющие ценность только для~\textsl{специального
покупателя}.
\item [{Специализированное~имущество}] "--- имущество, которое редко
либо никогда не~продаётся на~рынке иначе как~в~составе действующего
бизнеса либо имущественного комплекса, частью которых оно~является,
вследствие своей уникальности, обусловленной своей родовой принадлежностью,
конструкцией, конфигурацией, размером, местоположением и~(или) иными
свойствами.
\item [{Устойчивое~развитие}] "--- для~целей настоящих стандартов под~\textsl{устойчивым
развитием} понимается принятие во~внимание таких вопросов как~окружающая
среда и~изменение климата, здоровье и~благополучие человека, социальная
ответственность бизнеса (данный перечень не~является исчерпывающим),
которые могут оказать влияние в~будущем либо уже~влияют на~процедуру
\textsl{оценки} актива. В~широком смысле, это~желание осуществлять
деятельность без~истощения ресурсов и~без~оказания вредного воздействия
на~окружающую среду. \textbf{Примечание}: в~настоящее время ещё~не~существует
общепризнанного и~принятого на~всемирном уровне определения термина
\guillemotleft устойчивое развитие\guillemotright , вследствие и~по~причине
чего \textsl{члены RICS} должны проявлять осторожность в~отношении
использования этого термина без~дополнительных разъяснений.
\item [{Условия~договора~на~проведение~оценки}] "--- письменное подтверждение
условий \textsl{договора на~проведение~оценки}, предложенных \textsl{членом
RICS} либо согласованных между ним~и~заказчиком, которые должны
соблюдаться при~проведении \textsl{оценки} и~подготовке \emph{отчёта}.
В~\href{https://www.rics.org/globalassets/rics-website/media/upholding-professional-standards/sector-standards/valuation/international-valuation-standards-rics2.}{МСО}~\cite{IVS-2020}
упоминается как~\emph{задание на~оценку} (см.~\vref{sec:9.1_IVS-101_Scope_of_work}--\pageref{sec:9.1_IVS-101_Scope_of_work-End}).
\item [{Третья~сторона\label{Gloss:third_party}}] "--- любые лица, кроме
заказчика, потенциально имеющие интерес касательно \textsl{оценки}
либо её~результатов.
\item [{Специализированная~недвижимость}] "--- недвижимость, предназначенная
для~использования в~определённых предпринимательских целях, стоимость
которой отражает её~коммерческую ценность в~рамках~данного вида
деятельности.
\item [{Имущество,~предназначенное~для~реализации}] "--- имущество,
предполагаемое к~продаже в~рамках обычной хозяйственной деятельности,
например, применительно к~недвижимости, это~могут быть земельные
участки и~здания, выставленные на~продажу строительными или~девелоперскими
компаниями.
\item [{Результаты~оценки}] "--- \emph{суждение} о~значении конкретного
\textsl{вида стоимости (базы оценки)} актива или~обязательства на~определённую
дату. Если условиями \textsl{договора на~проведение оценки} не~установлено
иное, \emph{суждение} будет вынесено после \textsl{осмотра} \emph{объекта
оценки}, проведения различных исследований и~запросов информации,
которые будут уместны с~учётом характера актива (обязательства) и~целей
\textsl{оценки}.
\item [{Дата~оценки\label{Gloss:Valuation_Date}}] "--- дата, по~состоянию
на~которую \emph{оценщик} выносит своё \emph{суждение о~стоимости}.
\textsl{Дата оценки} также может содержать сведения о~конкретном
времени, если стоимость данного вида активов может существенно измениться
в~течение одних суток.
\item [{Ценность}] "--- см.\textasciitilde\hyperref[2.1.1_Investment_value]{инвестиционная стоимость}
на~с.~\pageref{Gloss:Investment_value}.\label{sec:2.1_RICS_Glossary-End}
\end{description}

\section{Пояснение наименований стандартов и~практических руководств\label{sec:2.2_Standards_naming}}

\textbf{Стандарты.}
\begin{itemize}
\item Статус: носят обязательный характер.
\item Состав: \href{https://www.rics.org/globalassets/rics-website/media/upholding-professional-standards/sector-standards/valuation/international-valuation-standards-rics2.}{Международные стандарты оценки}~\cite{IVS-2020},
издаваемые \href{https://en.wikipedia.org/wiki/International_Valuation_Standards_Council}{Советом по международным стандартам оценки}~\cite{IVSC:Wiki}
\begin{itemize}
\item \href{https://www.rics.org/globalassets/rics-website/media/upholding-professional-standards/sector-standards/valuation/international-valuation-standards-rics2.}{Международные стандарты оценки}~\cite{IVS-2020},
издаваемые \href{https://en.wikipedia.org/wiki/International_Valuation_Standards_Council}{Советом по международным стандартам оценки}~\cite{IVSC:Wiki}.
\item Стандарты проведения \textsl{оценки} и~подготовки \emph{отчёта об~оценке}
"--- обозначены префиксом \textbf{СПО}.
\item Стандарты профессиональной деятельности RICS "--- обозначены префиксом
\textbf{СПД}.
\end{itemize}
\item Комментарий: \href{https://www.rics.org/globalassets/rics-website/media/upholding-professional-standards/sector-standards/valuation/international-valuation-standards-rics2.}{МСО}~\cite{IVS-2020}
приняты и~имплементированы в~RVGS/RBG со~стороны RICS. Данное издание
содержит ссылки на~\href{https://www.rics.org/globalassets/rics-website/media/upholding-professional-standards/sector-standards/valuation/international-valuation-standards-rics2.}{МСО}~\cite{IVS-2020}
там, где~это~необходимо.
\end{itemize}
\textbf{Практические руководства.}
\begin{itemize}
\item Статус: носят рекомендательный характер.
\item Состав:
\begin{itemize}
\item Практические руководства RICS по~оценке "--- обозначены префиксом
\textbf{ПРО}.
\end{itemize}
\item Комментарий: \textbf{ПРО} имеют рекомендательный характер и~не~требуют
строгого соблюдения изложенных в~них~рекомендаций. Однако, там,
где~это~уместно, они~содержат предупреждения, о~наличии релевантных
материалов, имеющих статус обязательных, содержащихся в~RVGS/RBG,
в~т.\,ч.~соответствующих разделах \href{https://www.rics.org/globalassets/rics-website/media/upholding-professional-standards/sector-standards/valuation/international-valuation-standards-rics2.}{МСО}~\cite{IVS-2020}.
В~этих случаях всегда приводятся перекрёстные ссылки.
\end{itemize}
Помимо изложенного выше RICS также время от~времени публикует руководства
по~иным вопросам \textsl{оценки} в~форме инструкций.\label{sec:2.2_Standards_naming-End}\label{chap:2_Glossary-End}

\chapter{Стандарты профессиональной деятельности (далее --- СПД)\label{chap:3_PS}}

\section{СПД 1.~Соответствие настоящим стандартам при~подготовке письменного
отчёта об~оценке\label{sec:3.1_PS1_Compliance_of_written_valuation}}

\label{sec:3.1_PS1_Preamble}Данный обязательный стандарт:
\begin{itemize}
\item имплементирует подразделы~\ref{subsec:9.2.1_General_Principle}\nameref{subsec:9.2.1_General_Principle}\vpageref{subsec:9.2.1_General_Principle}\pageref{subsec:9.2.1_General_Principle-End},
\ref{subsec:9.2.4_Compliance_with_Other}~\nameref{subsec:9.2.4_Compliance_with_Other}
\vpageref{subsec:9.2.4_Compliance_with_Other}--\pageref{subsec:9.2.4_Compliance_with_Other-End}
\href{https://www.rics.org/globalassets/rics-website/media/upholding-professional-standards/sector-standards/valuation/international-valuation-standards-rics2.}{МСО}~\cite{IVS-2020};
\item устанавливает признание \href{https://www.rics.org/globalassets/rics-website/media/upholding-professional-standards/standards-of-conduct/international-ethics-standards-ies-rics.pdf}{Международных Стандартов Этики}~\cite{IES-2016}
и~\href{https://ipmsc.org/standards/}{Международных Сандартов Оценки Собственности}~\cite{IPMS};
\item устанавливает дополнительные обязательные требования для~\textsl{членов
RICS}.
\end{itemize}
Все~\textsl{члены RICS}, независимо от~формы осуществления оценочной
деятельности будь~то как~частная практика, так~и~работа по~трудовому
трудовому договору в~\textsl{оценочной компании}, \href{https://www.rics.org/eu/upholding-professional-standards/regulation/}{регулируемой}
со~стороны RICS~\cite{RICS:Regulation}, равно как~и~не~регулируемой
со~стороны RICS, при~составлении письменных \emph{отчётов об~оценке}
обязаны обеспечивать исполнение требований \href{https://www.rics.org/globalassets/rics-website/media/upholding-professional-standards/sector-standards/valuation/international-valuation-standards-rics2.}{МСО}~\cite{IVS-2020},
а~также данных \emph{Всемирных стандартов}, изложенных далее по~тексту.

\textsl{Члены RICS} также обязаны соблюдать условия \href{https://www.rics.org/eu/upholding-professional-standards/regulation/valuer-registration/}{Программы регистрации оценищков RICS}~\cite{RICS:Registration}.\label{sec:3.1_PS1_Preamble-End}

\subsection{Обязательное применение\label{subsec:3.1.1_Mandatory_application}}

\stepcounter{SubSecCounter}

\thesubsection.\theSubSecCounter.\label{3.1.1.1} Все~\textsl{члены
RICS} и~\href{https://www.rics.org/eu/upholding-professional-standards/regulation/}{регулируемые RICS}
\emph{оценочные компании}~\cite{RICS:Regulation} при~осуществлении
ими~\emph{оценочной деятельности} обязаны выполнять требования \hyperref[chap:3_PS]{Стандартов профессиональной деятельности}(с.~\pageref{chap:3_PS}--\pageref{chap:3_PS-End}),
а~также \hyperref[chap:4_VPS]{Стандартов проведения оценки и~подготовки отчёта об~оценке},
обозначаемых префиксами \textbf{СПД} и~\textbf{СПО} соответственно,
приведённых в~главах \ref{chap:3_PS} и~\ref{chap:4_VPS} данного
материала.\label{3.1.1.1-End}

\stepcounter{SubSecCounter}

\thesubsection.\theSubSecCounter.\label{3.1.1.2} Согласно пп.~«b»~п.
B5.2.1. «Ответственность \textsl{членов RICS}» и~п.~B5.3.1. «Ответственность
\textsl{оценочных компаний}» \href{https://www.rics.org/globalassets/rics-website/media/governance/bye-laws/}{Свода внутренних правил RICS}~\cite{RICS:Bye-Laws},
данные \emph{Всемирные стандарты} являются обязательными к~применению
со~стороны любого \textsl{члена RICS} либо \textsl{оценочной компании},
чья~деятельность \href{https://www.rics.org/eu/upholding-professional-standards/regulation/}{регулируется RICS}~\cite{RICS:Regulation},
вовлечённых в~оказание услуг по~\textsl{оценке} либо управлением
их~оказанием в~тех случаях, когда результат оказания такой услуги
выражен в~письменной форме. Совместно с~руководствами по~конкретным
случаям оценки "--- \textbf{ПР}, приведёнными в~главе~\vref{chap:5_Valuation_applications}--\pageref{chap:5_Valuation-applications-End}
данного издания, их обычно называют \textbf{«Красная книга RICS»}.\label{3.1.1.2-End}\footnote{Прим. пер.: как~уже~было сказано ранее, в~данном переводном материале
вместо обозначения \textbf{«Красная книга»/«Красная книга RICS»} используется
аббревиатура \textbf{RVGS/RBG}.}

\stepcounter{SubSecCounter}

\thesubsection.\theSubSecCounter.\label{3.1.1.3} Во~фразе \emph{«вовлечёнными
в~оказание услуг по~}\textsl{\emph{оценке}}\emph{ либо управлением
их~оказанием»} подразумеваются любые лица, ответственные по~статусу
либо добровольно принимающие на~себя ответственность в~части оценочного
анализа и~публикации \emph{суждения о~стоимости} в~письменном виде.
Сюда относятся в~т.\,ч.~лица, принимающие участие в~проведении
\textsl{оценки} и~подготовке \emph{отчёта об~оценке}, но~не~подписывающие
его, так~и~напротив "--- лица, не~готовящие \emph{отчёт об~оценке},
но~подписывающие его~в~рамках своих должностных обязанностей по~управлению
и~надзору. Для~целей настоящих стандартов, под~«письменной» формой
\emph{отчёта об~оценке} понимается передача \emph{отчёта} на~бумажном,
любом электронном или~цифровом носителе либо в~форме аудио или~видеозаписи.
Вопрос полностью устной формы изложения \emph{суждения о~стоимости}
рассмотрен в~п.~\vref{3.1.1.6}--\vref{3.1.1.6-End}. \label{3.1.1.3-End}

\stepcounter{SubSecCounter}

\thesubsection.\theSubSecCounter.\label{3.1.1.4} Для~целей настоящих
стандартов, во~избежание сомнений и~двоякого толкования, использование
и~публикация значений стоимости, возвращённых \emph{Автоматизированной
моделью оценки} (АМО) либо значений стоимости, основанных на~иных
результатах работы \emph{автоматизированных систем оценки} (см.~\vref{4.5.1}--\pageref{4.5.1-End}),
следует считать изложением \emph{письменного суждения о~стоимости}.\label{3.1.1.4-End}

\stepcounter{SubSecCounter}

\thesubsection.\theSubSecCounter.\label{3.1.1.5} Расчётное значение
стоимости замещения активов за~исключением \textsl{личного имущества},
изложенное в~рамках письменного \emph{отчёта об~оценке} либо отдельно,
определённое для~целей страхования, не~является изложением \emph{письменного
суждения о~стоимости} в~рамках оказания услуг по~\textsl{оценке},
в~том смысле, в~котором оно~описано выше в~п. \vref{3.1.1.3}--\pageref{3.1.1.3-End}.\label{3.1.1.5-End}

\stepcounter{SubSecCounter}

\thesubsection.\theSubSecCounter.\label{3.1.1.6} Во~избежание сомнений
тогда, когда "--- в~порядке исключения "--- оценочная консультация
предоставляется в~полностью устной форме, принципы, изложенные в~данном
издании, должны, тем~не~менее, соблюдаться в~максимально возможной
степени. \textsl{Членам RICS} следует помнить о~том, что~сам факт
того, что~оценочная консультация предоставляется в~устной форме,
не~означает того, что~её оказание не~налагает ответственности:
обязанности и~обязательства \emph{оценщика} всегда будут существовать
в~зависимости от~конкретных фактов и~обстоятельств случая оказания
такой консультации. В~некоторых национальных юрисдикциях предоставление
устных оценочных консультаций в~любом случае регулируется стандартами
данных юрисдикций. Кроме того, во~всех юрисдикциях \emph{оценщикам},
выступающим в~качестве судебных экспертов, следует принимать во~внимание
тот~факт, что~письменные и~устные заключения имеют одинаковое значение
и~рассматриваются по~одним и~тем~же критериям "--- см., например,
\href{https://www.rics.org/globalassets/rics-website/media/upholding-professional-standards/sector-standards/dispute-resolution/surveyors-acting-as-expert-witnesses-4th-edition-rics.pdf}{Свод правил и практических рекомендаций RICS 799 для сюрвейров, выступающих в качестве судебных экспертов, 4-е издание (2014)}~\cite{RICS:Expert-Witnesses}.\label{3.1.1.6-End}

\stepcounter{SubSecCounter}

\thesubsection.\theSubSecCounter.\label{3.1.1.7} Данные \emph{Всемирные
стандарты} написаны в~том~виде, в~котором они~применимы к~\emph{оценщику-физическому
лицу}, являющемуся \textsl{членом RICS}. Там, где~необходимо и~уместно
рассмотреть их~применение в~отношении компании, \href{https://www.rics.org/eu/upholding-professional-standards/regulation/}{зарегистрированной для регулирования её деятельности со стороны RICS}~\cite{RICS:Regulation},
они~должны толковаться соответствующим образом.\label{3.1.1.7-End}\label{subsec:3.1.1_Mandatory_application-End}

\subsection{Соблюдение стандартов оценщиками, работающими по~найму\label{subsec:3.1.2_Compliance_with_firms}}

\stepcounter{SubSecCounter}

\thesubsection.\theSubSecCounter.\label{3.1.2.1} Все~\textsl{члены
RICS} несут персональную ответственность за~соблюдение требований
данных стандартов, независимо от~того, осуществляют~ли они~частную
практику или~работают по~найму в~\textsl{оценочной компании}. В~последнем
случае, то, как~эта~ответственность реализуется на~практике в~определённой
степени зависит от~характера \textsl{оценочной компании}, в~которой
работает \textsl{член RICS}.
\begin{itemize}
\item \href{https://www.rics.org/eu/upholding-professional-standards/regulation/}{Оценочная компания, регулируемая RICS}~\cite{RICS:Regulation}:
\textsl{оценочная компания} и~все~работающие в~ней~\emph{оценщики},
являющиеся \textsl{членами RICS}, обязаны обеспечить полное соответствие
всех процессов и~оценочных процедур обязательным требованиям данных
стандартов. Данное требование также распространяется на~проведение
\textsl{оценок}, которые осуществляют работающие в~\textsl{оценочной
компании} \emph{оценщики}, не~являющиеся \textsl{членами RICS}.
\item \textbf{\textsl{Оценочная компания}}\textbf{, не~регулируемая RICS}:
в~то~время как~такие \textsl{оценочные компании} могут осуществлять
процессы по~собственным стандартам, на~которые не~распространяется
регулирование со~стороны RICS, конкретные \emph{оценщики} "--- \textsl{члены
RICS}, работающие в~них, обязаны соблюдать обязательные требования
настоящих стандартов.\label{3.1.2.1-End}
\end{itemize}
\stepcounter{SubSecCounter}

\thesubsection.\theSubSecCounter.\label{3.1.2.2} Возможны ситуации,
когда корпоративные стандарты \textsl{оценочной компании} прямо препятствуют
соблюдению того или~иного аспекта настоящих стандартов. В~таких
случаях \textsl{член RICS} вправе отступить от~требований конкретного
стандарта, при~этом он~обязан:
\begin{itemize}
\item убедиться в~том, что~отступление от~требований стандартов не~приводит
к~введению заказчика в~заблуждение либо неэтичному поведению;
\item описать в~условиях \textsl{договора на~проведение оценки} (см.~\vref{sec:4.1_VPS1_Terms_of_engagement_Scope_of_work}--\pageref{sec:4.1_VPS1_Terms_of_engagement_Scope_of_work-End})
и~тексте \emph{отчёта об~оценке} (см. \vref{sec:4.3_VPS3_Valuation_reports}--\pageref{sec:4.3_VPS3_Valuation_reports-End})
конкретные области, в~которых имел место отказ от~применения настоящих
стандартов, а~также причину этого;
\item соблюдать остальные требования данных стандартов.\label{3.1.2.2-End}
\end{itemize}
\stepcounter{SubSecCounter}

\thesubsection.\theSubSecCounter.\label{3.1.2.3} Если \textsl{член
RICS} вносит свой вклад в~\textsl{оценку}, следует также сделать
ссылку на~подраздел~\ref{subsec:3.2.2_Member qualification}~\nameref{subsec:3.2.2_Member qualification}
\vpageref{subsec:3.2.2_Member qualification}--\pageref{subsec:3.2.2_Member qualification-End}.\label{3.1.2.3-End}\label{subsec:3.1.2_Compliance_with_firms-End}

\subsection{Соответствие международным стандартам\label{subsec:3.1.3_Compliance_with_IS}}

\subsubsection{Международные стандарты оценки\label{subsubsec:3.1.3.1_Compliance_with_IVS}}

\stepcounter{SubSubSecCounter}

\thesubsubsection.\theSubSubSecCounter.\label{3.1.3.1.1} RICS признаёт
\href{https://en.wikipedia.org/wiki/International_Valuation_Standards_Council}{Международный Совет по Стандартам Оценки (IVSC/МССО)}~\cite{IVSC:Wiki}
в~качестве разработчика \href{https://www.rics.org/globalassets/rics-website/media/upholding-professional-standards/sector-standards/valuation/international-valuation-standards-rics2.}{Международных стандартов оценки}~\cite{IVS-2020},
содержащих международно признанные принципы и~термины \textsl{оценки}.
Данные \emph{Всемирные стандарты} признают и~имплементируют их, устанавливая
дополнительные конкретные требования и~практические руководства по~их~применению.
Актуальная версия \href{https://www.rics.org/globalassets/rics-website/media/upholding-professional-standards/sector-standards/valuation/international-valuation-standards-rics2.}{Международных стандартов оценки}~\cite{IVS-2020},
вступившая в~силу 31~января 2020~г, приводится в~полном объёме
в~Части~\ref{part:II_IVS} данного материала на~с.~\pageref{part:II_IVS}--\pageref{part:II_IVS-End}.\label{3.1.3.1.1-End}

\stepcounter{SubSubSecCounter}

\thesubsubsection.\theSubSubSecCounter.\label{3.1.3.1.2} В~тех~случаях,
когда при~проведении \textsl{оценки} предъявляется требование её~соответствия
положениям \href{https://www.rics.org/globalassets/rics-website/media/upholding-professional-standards/sector-standards/valuation/international-valuation-standards-rics2.}{МСО}~\cite{IVS-2020},
необходимо заверить это~в~условиях \textsl{договора на~проведение
оценки} и~в~тексте \emph{отчёта об~оценке}, примерное содержание
такого заверения приведено в~\vref{subsubsec:4.1.3.14_Confirmation_accordance_with_the_IVS}--\pageref{subsubsec:4.1.3.14_Confirmation_accordance_with_the_IVS-End}
и~\vref{subsubsec:4.3.2.11_Comfirmation_accordance_with_IVS}--\pageref{subsubsec:4.3.2.11_Comfirmation_accordance_with_IVS-End}.
В~противном случае может быть использована более общая декларация,
приведённая в~разделах~\ref{sec:4.1_VPS1_Terms_of_engagement_Scope_of_work},
\ref{sec:4.3_VPS3_Valuation_reports}, содержащая сведения о~том,
что~\textsl{оценка} будет (была) проведена в~соответствии с~\emph{«Красной
книгой RICS»} (официальное название \emph{«RICS оценка "--- Всемирные
стандарты»}).\label{3.1.3.1.2-End}\footnote{Прим. пер.: Как~уже~было сказано ранее, в~данном переводном материале
используется аббревиатура \emph{\guillemotleft RVGS/RBG\guillemotright}}

\stepcounter{SubSubSecCounter}

\thesubsubsection.\theSubSubSecCounter.\label{3.1.3.1.3} \textsl{Членам
RICS} следует помнить о~том, что~в~тех~случаях, когда они~декларируют
соответствие проводимой \textsl{оценки} требованиям \href{https://www.rics.org/globalassets/rics-website/media/upholding-professional-standards/sector-standards/valuation/international-valuation-standards-rics2.}{МСО}~\cite{IVS-2020},
подразумевается соблюдение отдельных соответствующих ситуации стандартов.
Там, где~требуется отклонение от~требований \href{https://www.rics.org/globalassets/rics-website/media/upholding-professional-standards/sector-standards/valuation/international-valuation-standards-rics2.}{МСО}~\cite{IVS-2020},
в~этой части необходимо приводить чёткое разъяснение.\label{3.1.3.1.3-End}\label{subsubsec:3.1.3.1_Compliance_with_IVS-End}

\subsubsection{Международные стандарты этики\label{subsubsec:3.1.3.2_Compliance_with_IES}}

\stepcounter{SubSubSecCounter}

\thesubsubsection.\theSubSubSecCounter.\label{3.1.3.2.1} RICS является
членом \href{https://ies-coalition.org/}{Международной Коалиции Профессиональных Организаций}~\cite{IESC-site},
созданной с~целью разработки и~внедрения первого набора всемирно
признанных стандартов этики в~сфере собственности и~связанных с~ней~профессиональных
услуг. Данные всемирные стандарты соответствуют принципам, изложенным
в~текущей версии \href{https://www.rics.org/globalassets/rics-website/media/upholding-professional-standards/standards-of-conduct/international-ethics-standards-ies-rics.pdf}{Международных стандартов этики}~\cite{IES-2016},
опубликованных \href{https://ies-coalition.org/}{Коалицией}~\cite{IESC-site},
а~также включают дополнительные более подробные требования, которые
должны соблюдаться всеми \textsl{членами RICS}.\label{3.1.3.2.1-End}\label{subsubsec:3.1.3.2_Compliance_with_IES-End}

\subsubsection{Международные стандарты оценки собственности\label{subsubsec:3.1.3.3_Compliance_with_IPMS}}

\stepcounter{SubSubSecCounter}

\thesubsubsection.\theSubSubSecCounter.\label{3.1.3.3.1} RICS также
является членом \href{https://ipmsc.org/}{Международной Коалиции Профессиональных Организаций}~\cite{IPMSC:site},
основанной для~разработки и~внедрения целостных и~прозрачных стандартов
\textsl{оценки} собственности, являющейся \textsl{недвижимостью}.
В~случае проведения \textsl{оценки} активов или~обязательств в~сфере
\textsl{недвижимости}, \textsl{члены RICS} обязаны соблюдать требования
\href{https://ipmsc.org/standards/}{Международных Стандартов Сценки Собственности}
(IPMS/МСОС)~\cite{IPMS} там, где~и~когда это~применимо. \href{https://www.rics.org/globalassets/rics-website/media/upholding-professional-standards/sector-standards/real-estate/rics-property-rement/rics-property-measurement-2nd-edition-rics.pdf}{Положение RICS об оценке стоимости недвижимого имущества}~\cite{RICS:PM}
развивает \href{https://ipmsc.org/standards/}{МСОС}~\cite{IPMS}
и~содержит детали такой \textsl{оценки}.\label{3.1.3.3.1-End}\label{subsubsec:3.1.3.3_Compliance_with_IPMS-End}\label{subsec:3.1.3_Compliance_with_IS-End}

\subsection{Соответствие национальным и~иным стандартам оценки\label{subsec:3.1.4_Compliance_with_jurisdictional_standards}}

\stepcounter{SubSecCounter}

\thesubsection.\theSubSecCounter.\label{3.1.4.1} RICS признаёт,
что~от~\textsl{члена RICS} могут потребовать подготовить \emph{отчёт
об~оценке}, соответствующим иным стандартам, отличным от~тех, что~изложены
в~RVGS/RBG. Как~правило, такое происходит по~причине наличия требований,
существующих в~конкретных национальных юрисдикциях. \textsl{Членам
RICS} следует смело соглашаться выполнять данные требования, которые,
в~частности могут касаться определения \textsl{вида стоимости}, не~указанного
в~разделе~\ref{sec:4.4_VPS4_Bases_of_value} (см.~с.~\pageref{sec:4.4_VPS4_Bases_of_value}--\pageref{sec:4.4_VPS4_Bases_of_value-End}),
при~условии, что~есть ясность, требования каких именно стандартов
следует выполнять.\label{3.1.4.1-End}

\stepcounter{SubSecCounter}

\thesubsection.\theSubSecCounter.\label{3.1.4.2} В~таких случаях
в~\textsl{договоре на~проведение оценки} и~в~тексте \emph{отчёта
об~оценке} должно приводиться заявление о~том, что~вышеуказанные
стандарты должны быть соблюдены. Если соблюдение таких стандартов
является обязательным в~данной национальной юрисдикции в~силу требований
закона, подзаконных актов либо распоряжений органов государственной
власти, это~не~исключает возможности выполнения \textsl{оценки}
также и~в~соответствии с~требованиями RVGS/RBG и, в~случае применимости,
"--- \href{https://www.rics.org/globalassets/rics-website/media/upholding-professional-standards/sector-standards/valuation/international-valuation-standards-rics2.}{МСО}~\cite{IVS-2020}.\label{3.1.4.2-End}

\stepcounter{SubSecCounter}

\thesubsection.\theSubSecCounter.\label{3.1.4.3} Для~некоторых
национальных юрисдикций RICS публикует \href{https://www.isurv.com/info/1342/rics_national_or_jurisdictional_valuation_standards}{дополнения для отдельных национальных юрисдикций}~\cite{RICS:National-Standards},
обычно называемые \emph{\guillemotleft Стандартами применения RVGS/RBG
в~отдельных юрисдикциях\guillemotright} либо \emph{\guillemotleft Стандартами
оценки национальных ассоциаций RICS\guillemotright},\footnote{Прим. пер.: соответствующий \href{https://www.isurv.com/site/scripts/download_info.php?downloadID=2009}{Стандарт применения RVGS/RBG в Ресурсной Федерации}~\cite{RICS:Application-in-RF}
был основан на~устаревшей сейчас \href{https://www.isurv.com/downloads/download/1830/red_book_rics_valuation_-_professional_standards_global_2014_archived}{версии RVGS/RBG 2014 г.}~\cite{RICS:RVGS-2014}
и~имеет статус \emph{«утратил действие»}.} способствующих применению RVGS/RBG \textsl{членами RICS} в~местном
контексте. В~соответствующих случаях данные дополнения могут выпускаться
совместно с~местными организациями \emph{оценщиков}, в~противном
случае "--- отдельно от~них, но~с~учётом их~требований к~проведению
\textsl{оценки}, в~той степени, в~которой они~не~противоречат
требованиями RICS.\label{3.1.4.3-End}

\stepcounter{SubSecCounter}

\thesubsection.\theSubSecCounter.\label{3.1.4.4} Если соблюдение
других стандартов является добровольным, т.\,е.~не~подпадает под
действие п.~\ref{3.1.4.2}, это~влечёт за~собой \textsl{отступление},
см.~подраздел~\vref{subsec:3.1.6_Departures}--\vref{subsec:3.1.6_Departures-End}.
Следует обратить внимание на~то, что~выполнение требований таких
стандартов не~может отменять соблюдение обязательных требований разделов
\ref{sec:3.1_PS1_Compliance_of_written_valuation}, \ref{sec:3.2_PS2_Ethics_competency_objectivity},
соблюдать которые \textsl{члены RICS} обязаны во~всех случаях.\label{3.1.4.4_End}

\stepcounter{SubSecCounter}

\thesubsection.\theSubSecCounter.\label{3.1.4.5} В~случае проведения
\textsl{оценки} в~отношении активов, находящихся в~двух или~более
странах или~иных территорий, имеющих разные стандарты оценки, \textsl{член
RICS} обязан согласовать с~заказчиком какие именно стандарты следует
использовать.\label{3.1.4.5-End}\label{subsec:3.1.4_Compliance_with_jurisdictional_standards-End}

\subsection{Исключения из~СПО 1--5\label{subsec:3.1.5_VPS_exceptions}}

\stepcounter{SubSecCounter}

\thesubsection.\theSubSecCounter.\label{3.1.5.1} Если \textsl{член
RICS} оказывает оценочные консультации в~письменном виде, то~все~они~должны
соответствовать хотя~бы некоторым требования RVGS/RBG "--- исключений
нет (см.~п.~\ref{3.1.1.2} на~с.~\pageref{3.1.1.2}--\pageref{3.1.1.2-End}).
Аналогичным образом, в~тех~случаях, когда оценочная консультация
даётся полностью в~устной форме, принципы, установленные RVGS/RBG,
тем~не~менее должны соблюдаться в~максимальной степени (см.~п.~\ref{3.1.1.6}
на~с.~\pageref{3.1.1.6}--\pageref{3.1.1.6-End}). Таким образом,
\hyperref[sec:3.1_PS1_Compliance_of_written_valuation]{СПД~1.}~\nameref{sec:3.1_PS1_Compliance_of_written_valuation}
и~\hyperref[sec:3.2_PS2_Ethics_competency_objectivity]{СПД~2.}~\nameref{sec:3.2_PS2_Ethics_competency_objectivity}
являются обязательными во~всех случаях (см.~п.~\vref{subsubsec:1.2.3.1_PS-mandatory}--\pageref{subsubsec:1.2.3.1_PS-mandatory-End}
и~п.~\vref{subsec:3.1.7_Regulation_monitoring}--\pageref{subsec:3.1.7_Regulation_monitoring-End}).
Иными словами, все~\textsl{члены RICS} обязаны применять их~при~осуществлении
любого вида \emph{оценочной деятельности}.\label{3.1.5.1-End}

\stepcounter{SubSecCounter}

\thesubsection.\theSubSecCounter.\label{3.1.5.2} Однако, принимая
во~внимание исключительное разнообразие деятельности \textsl{\uline{членов
RICS}}, а~также множественность юрисдикций и~национальных особенностей,
в~условиях которых \uline{они}~осуществляют оказание услуг по~\textsl{оценке}
и~стоимостному консалтингу, необходимо применять индивидуальный подход
к~конкретным задачам, при~выполнении части которых применение положений
СПО~1--5 может быть неуместным либо нецелесообразным. Несмотря на~то,
что~в~таких случаях применение настоящих стандартов не~является
обязательным, оно~всячески приветствуется, если только оно~не~противоречит
конкретным требованиям либо контексту. Данные исключения, касающиеся
СПО~1--5, будут описаны подробнее далее по~тексту. Однако, нет~возможности
описать все~предполагаемые ситуации, вследствие и~по~причине чего,
в~общем случае, требования СПО~1--5 следует считать обязательными.\label{3.1.5.2-End}

\stepcounter{SubSecCounter}

\thesubsection.\theSubSecCounter.\label{3.1.5.3} \emph{Оценщикам}
следует понимать, что~исключения, как~правило, не~являются казуистическими,
а~охватывают отдельные категории или~аспекты \emph{оценочной деятельности}
(см.~подраздел~\ref{subsec:3.1.6_Departures}~\nameref{subsec:3.1.6_Departures}
на~с.~\pageref{subsec:3.1.6_Departures}--\pageref{subsec:3.1.6_Departures-End}).
В~таких случаях \textsl{члены RICS} не~имеют права приводить в~\emph{отчёте
об~оценке} сведения о~том, что~\textsl{оценка} была проведена в~соответствии
с~\href{https://www.rics.org/globalassets/rics-website/media/upholding-professional-standards/sector-standards/valuation/international-valuation-standards-rics2.}{МСО}~\cite{IVS-2020}
(см.~главу~\ref{chap:8_IVS_Framework}~\nameref{chap:8_IVS_Framework}
на~с.~\pageref{chap:8_IVS_Framework}--\pageref{chap:8_IVS_Framework-End}).\label{3.1.5.3-End}

\stepcounter{SubSecCounter}

\thesubsection.\theSubSecCounter.\label{3.1.5.4} Далее приводится
перечень исключений, при~которых \textsl{члены RICS} могут допускать
\textsl{отступления} от~требований СПО~1--5.
\begin{itemize}
\item Предоставление агентских либо брокерских услуг при~приобретении или~продаже
актива (активов): данная деятельность регулируется \emph{\href{https://www.rics.org/globalassets/rics-website/media/upholding-professional-standards/sector-standards/real-estate/real-estate-and-agency-brokerage-3rd-edition-rics.pdf}{Практическим руководством RICS: деятельность агентства недвижимости и брокеридж, 3-е издание (2016)}}~\cite{RICS:Real&Brokerage}.
Данное исключение распространяется на~предоставление консультаций
в~ожидании, а~также в~процессе исполнения агентством поручения
о~приобретении либо продаже прав на~актив. Также оно~распространяется
на~ситуации консультирования по~вопросу того, когда стоит сделать
или~принять то~или~иное предложение по~сделке. Однако, оно~не~распространяется
на~отчёты о~покупке, включающие в~себя \textsl{оценку}.
\item Подготовка к~выступлению либо выступление в~качестве судебного эксперта
является безусловным основанием для~\textsl{отступления} от~требований.
Это~следует из~того факта, что~\textsl{член RICS}, выступающий
в~такой роли, обязан предельно точно следовать требованиям и~инструкциям,
установленным судом, арбитражным судом либо иным органом власти, для~которого
он~готовит заключение и~перед которым возможно будет выступать.
Кроме того, \textsl{член RICS} должен отвечать высоким стандартам
беспристрастности и~объективности и~соблюдать их. Полезным материалом
по~данной тематике является \href{https://www.rics.org/globalassets/rics-website/media/upholding-professional-standards/sector-standards/dispute-resolution/surveyors-acting-as-expert-witnesses-4th-edition-rics.pdf}{Свод правил и практических рекомендаций RICS для сюрвейров, выступающих в качестве судебных экспертов, 4-е издание (2014)}~\cite{RICS:Expert-Witnesses}.
\item Исполнение функций, предписанных законом, в~тех случаях, когда он~устанавливает
задачу, а~также определяет её~выполнение. В~данном случае акцент
делается на~понятии функции, т.\,е.~на~исполнении предписанной
роли или~обязанностей, связанных с~осуществлением полномочий либо
обеспечением осуществления полномочий, установленных либо признанных
законом, связанных, как~правило, с~официальным назначением лица
на~должность. Сам~факт того, что~\textsl{оценка} проводится в~соответствии
с~законом либо в~силу его~требований, не~имеет значения. Например,
проведение \textsl{оценки} для~целей включения её~в~налоговую отчётность
и~передачи в~налоговый орган, предполагающее соблюдение требований
законодательства, но~не~осуществление правоприменения, не~подпадает
под~данное исключение.
\item Проведение \textsl{оценки} исключительно для~внутренних задач заказчика
без~ответственности и~без~доведения её~результатов до~\textsl{третьих
лиц}. Данное исключение существует в~связи с~признанием того обстоятельства,
что~в~ряде случаев, \emph{оценщик}, оказывающий услуги по~\textsl{оценке}
заказчику, обращающемуся за~оценочной консультацией, "--- на~регулярной
основе в~целях переоценки портфеля активов "--- оказывает её~без~обычной
для~себя ответственности и~без~возможности использования результатов
\textsl{оценки} \textsl{третьими лицами} (например в~целях планирования
управления активами либо приобретения таковых). В~тех~случаях, когда
\textsl{члены RICS} выполняют такую работу, крайне важно, чтобы условия
\textsl{договора на~проведение оценки}, а~также сама \uline{письменная
консультация} содержали чёткие \uuline{положения}, содержащие запрет
на~раскрытие содержащихся в~ней информации и~данных любой \textsl{третьей
стороне}, использование \uline{её}~в~иных целях, а~также \uuline{касающиеся}
вопроса исключения ответственности \emph{оценщика}. Подобная консультация
часто не~предполагает отдельного денежного вознаграждения, и~данный
элемент может как~включаться, так~и~не включаться в~\textsl{условия
договора на~регулярную оценку} портфеля. Сам~факт того, что~исполнитель
\textsl{оценки} является \textsl{внутренним оценщиком}, не~означает
возможность применять данное исключение, акцент которого сосредоточен
на~сугубо внутреннем назначении \textsl{оценки}, а~не~на~способе
её~проведения и~подготовки \emph{отчёта}. Таким образом, \textsl{внешний
оценщик} может проводить \textsl{оценку} «для внутренних целей», при~этом
возрастает необходимость закрепления в~условиях \textsl{договора
на~проведение оценки} и~тексте \emph{отчёта об~оценке} абсолютно
ясных положений о~неразглашении \textsl{третьим лицам} и~исключении
ответственности \emph{оценщика}.
\item Предоставление консультаций по~вопросам \textsl{оценки} непосредственно
при~подготовке или~в~ходе переговоров либо судебных разбирательств,
в~том~числе в~случаях, когда \textsl{оценщик} выступает в~качестве
адвоката. Данное исключение относится к~оценочным консультациям о~вероятном
исходе текущих или~предстоящих переговоров или~запросов о~значениях
стоимости, на~которые сторона будет ссылаться в~процессе переговоров.
Вследствие этого:
\begin{itemize}
\item хотя неразрешённый спор может ещё~не~существовать, консультации
предоставляются непосредственно в~ходе подготовки к~переговорам
либо в~процессе их~проведения, по~итогам которых возможно согласие
и~урегулирование на~досудебной стадии либо отсутствие такого урегулирования,
вызывающее, если это~позволяет контекст, переход к~формальному процессу
урегулирования (обращение в~суд, арбитражный суд и~т.\,д.);
\item консультации в~рамках переговоров могут, а~зачастую будут, затрагивать
и~такие вопросы как~тактика и~(или) вероятный исход, и~(или) способы
урегулирования без~прибегания к~судебному разбирательству либо иным
формальным процедурам.
\end{itemize}
В~связи с~данным исключением признаётся следующее:
\item существует формальный спор, и, независимо от~того, каковы его~причины,
связанное с~ним разбирательство будет регулироваться соответствующими
законодательными актами, подзаконными актами, иными правилами и~решениями
судов как~существующими, так~и~теми, которые будут изданы, и~которые
всегда будут обладать большей силой чем~RVGS/RBG;
\item консультации, предоставляемые заказчику, могут касаться различных
вопросов, выходящих за~рамки стоимостного консультирования, например
по~вопросам тактики и~(или) вероятного исхода судебного разбирательства
и~(или) вариантам урегулирования спора и~(или) снижения судебных
расходов.\label{3.1.5.4-End}
\end{itemize}
\stepcounter{SubSecCounter}

\thesubsection.\theSubSecCounter.\label{3.1.5.5} Для~всех случаев
\emph{исключений}, кроме тех, когда вид~деятельности явно описан
в~других стандартах и~руководствах RICS, сам~факт того, что~СПО~1--5
не~носят обязательный характер, не~означает, что~их~следует просто
игнорировать "--- в~соответствии с~принципами использования передовых
практик, они~должны соблюдаться в~той~степени, в~которой это~не~исключается
конкретным требованием либо обстоятельствами.\label{3.1.5.5-End}\label{subsec:3.1.5_VPS_exceptions-End}

\subsection{Отступления от~требований настоящих стандартов\label{subsec:3.1.6_Departures}}

\stepcounter{SubSecCounter}

\thesubsection.\theSubSecCounter.\label{3.1.6.1} При~проведении
\textsl{оценки} с~предоставлением её~результатов в~письменной форме
не~допускается отступление от~требований \textbf{СПД~1}~\nameref{sec:3.1_PS1_Compliance_of_written_valuation}
и~во~всех случаях не~допускается отступление от~требований \textbf{СПД~2}~\nameref{sec:3.2_PS2_Ethics_competency_objectivity},
являющегося обязательным при~любых обстоятельствах.\label{3.1.6.1-End}

\stepcounter{SubSecCounter}

\thesubsection.\theSubSecCounter.\label{3.1.6.2} В~тех~случаях,
когда отдельно и~независимо как~от~конкретных случаев, описанных
выше в~подразделе~\ref{subsec:3.1.5_VPS_exceptions} \nameref{subsec:3.1.5_VPS_exceptions}
на~с.~\pageref{subsec:3.1.5_VPS_exceptions}--\pageref{subsec:3.1.5_VPS_exceptions-End},
так~и~от~условий, следующих из~положений подраздела~\ref{subsec:3.1.4_Compliance_with_jurisdictional_standards}~\nameref{subsec:3.1.4_Compliance_with_jurisdictional_standards},
существуют \uline{особые обстоятельства}, влекущие нецелесообразность
полного либо частичного соблюдение требований \textbf{СПО~1--5},
\uline{они}~должны быть подтверждены заказчиком и~согласованы
с~ним~в~качестве \textsl{отступлений}, а~также должно иметь место
чёткое обозначение данного обстоятельства в~условиях \textsl{договора
на~проведение оценки}, тексте \emph{отчёта об~оценке} и~любой ссылке
на~него.\label{3.1.6.2-End}

\stepcounter{SubSecCounter}

\thesubsection.\theSubSecCounter.\label{3.1.6.3} Во~избежание сомнений
и~двоякого толкования.
\begin{itemize}
\item Если \textsl{оценка} проводится в~соответствии с~предписанными законом
или~подзаконным нормативным актом процедурами либо иными официальными
требованиями, то, при~условии, что~эти~требования являются обязательными
в~конкретной ситуации либо юрисдикции, соответствие им, само по~себе,
не~является \textsl{отступлением от~стандартов} "--- тем~не~менее
в~обязательном порядке требуется прямое и~чёткое указание на~необходимость
этого.
\item В~большинстве случаев при~проведении \textsl{оценки} применимым
является один из~\textsl{видов стоимости}, предусмотренный в~подразделе~\vref{subsec:4.4.1_General_principles}--\pageref{subsec:4.4.1_General_principles-End}.
В~случае использования иного \textsl{вида стоимости}, необходимо
ясно отразить это~в~\emph{отчёте об~оценке}. В~случае, когда применение
такого \textsl{вида стоимости} является обязательным в~силу существования
определённого контекста либо требований законодательства конкретной
юрисдикции, такое применение не~считается \textsl{отступлением от~стандартов},
при~этом в~\emph{отчёте об~оценке} в~обязательном порядке приводится
описание данного требования либо контекста. RICS не~приветствует
использование \textsl{видов стоимости}, не~предусмотренных в~\vref{subsec:4.4.1_General_principles}--\pageref{subsec:4.4.1_General_principles-End}
по~инициативе \emph{оценщика} "--- такие действия всегда будут рассматриваться
как~\textsl{отступление от~стандартов}.\label{3.1.6.3-End}
\end{itemize}
\stepcounter{SubSecCounter}

\thesubsection.\theSubSecCounter.\label{3.1.6.4} \textsl{Члену RICS},
допустившему \textsl{отступление от~стандартов}, может быть предложено
обосновать причины данного действия.\label{3.1.6.4-End}\label{subsec:3.1.6_Departures-End}

\subsection{Регулирование: контроль за соблюдением настоящих стандартов\label{subsec:3.1.7_Regulation_monitoring}}

\stepcounter{SubSecCounter}

\thesubsection.\theSubSecCounter.\label{3.1.7.1} Являясь саморегулируемой
организацией, RICS несёт ответственность за~контроль \textsl{\href{https://www.rics.org/eu/upholding-professional-standards/regulation/}{соответствия своих членов и зарегистрированных для регулирования оценочных компаний}~\cite{RICS:Regulation,RICS:Registration}}
требованиям настоящих стандартов. Согласно \href{https://www.rics.org/globalassets/rics-website/media/governance/bye-laws/}{Своду Внутренних Правил RICS}~\cite{RICS:Bye-Laws},
RICS вправе запрашивать информацию у~своих \href{https://www.rics.org/eu/upholding-professional-standards/regulation/}{членов и зарегистрированных для регулирования оценочных компаний}~\textsl{\cite{RICS:Regulation,RICS:Registration}}.
Процедуры, в~рамках которых осуществляются данные полномочия, приведены
в~\href{https://www.rics.org/eu/upholding-professional-standards/regulation/}{соответствующем разделе}~\cite{RICS:Regulation}
\href{https://www.rics.org/eu/}{официального сайта RICS}~\cite{RICS:site}.\label{3.1.7.1-End}

\stepcounter{SubSecCounter}

\thesubsection.\theSubSecCounter.\label{3.1.7.2} Там, где~это~применимо,
\textsl{члены RICS} также должны отвечать требованиям, предъявляемым
к~\href{https://www.rics.org/eu/surveying-profession/career-progression/accreditations/valuer-registration-assessment/}{квалификации оценщиков}~\cite{RICS:Valuer-Assessment}.\label{3.1.7.2-End}\label{subsec:3.1.7_Regulation_monitoring-End}

\subsection{Применимость настоящих стандартов для~членов других профессиональных
сообществ оценщиков\label{subsec:3.1.8_Application_for_other}}

\stepcounter{SubSecCounter}

\thesubsection.\theSubSecCounter.\label{3.1.8.1} Данные \emph{Всемирные
Стандарты} также могут быть официально приняты другими профессиональными
сообществами \emph{оценщиков} (далее "--- \textbf{ПСО}) при~условии
предварительного согласования с~RICS и~наличии его~согласия на~это.\label{3.1.8.1-End}

\stepcounter{SubSecCounter}

\thesubsection.\theSubSecCounter.\label{3.1.8.2} За~исключением
случаев, когда RICS официально дал~своё согласие на~использование
RVGS/RBG соответствующими квалифицированными членами другого \textbf{ПСО},
ни~один \emph{оценщик}, не~являющийся \textsl{членом RICS}, не~может
утверждать, что~его~(её) \textsl{оценка} была проведена в~полном
соответствии с~требованиями данных \emph{Всемирных Стандартов}.\label{3.1.8.2-End}\label{subsec:3.1.8_Application_for_other-End}\label{sec:3.1_PS1_Compliance_of_written_valuation-End}

\newpage

\section{СПД~2.~Этика, компетентность, объективность и~раскрытие информации\label{sec:3.2_PS2_Ethics_competency_objectivity}}

Данный обязательный стандарт:
\begin{itemize}
\item имплементирует \href{https://www.rics.org/globalassets/rics-website/media/upholding-professional-standards/sector-standards/valuation/international-valuation-standards-rics2.}{МСО}~\cite{IVS-2020};
\item признаёт \href{https://www.rics.org/globalassets/rics-website/media/upholding-professional-standards/standards-of-conduct/international-ethics-standards-ies-rics.pdf}{Международные Стандарты Этики}~\cite{IES-2016}
и~\href{https://ipmsc.org/standards/}{Международные Стандарты Оценки Собственности (МСОС)}~\cite{IPMS};
\item устанавливает дополнительные обязательные требования для~\textsl{членов
RICS}
\end{itemize}
Поскольку это~имеет основополагающее значение для~чистоты процесса
\textsl{оценки}, все~\textsl{члены RICS}, практикующие в~качестве
\emph{оценщиков}, должны обладать соответствующим поставленной задаче
опытом, навыками и~рассудительностью и~обязаны~во~всех случаях
действовать профессионально и~этично, не~поддаваясь чьему-либо чрезмерному
влиянию, избегать предвзятости и~не~допускать возникновения \href{https://en.wikipedia.org/wiki/Conflict_of_interest}{конфликта интересов}~\cite{COI}.

\subsection{Профессиональные и~этические стандарты\label{subsec:3.2.1_Professional_and_ethical_standards}}

\stepcounter{SubSecCounter}

\thesubsection.\theSubSecCounter.\label{3.2.1.1} \textsl{Члены RICS}
работают в~соответствии с~самыми высокими профессиональными и~этическими
стандартами. Таким образом, критерии членства в~RICS, а~также критерии
квалификации и~практики в~качестве \emph{оценщика}, включая, там~где~это~применимо,
требования \href{https://www.rics.org/eu/upholding-professional-standards/regulation/valuer-registration/}{Программы Регистрации Оценщиков RICS}~\cite{RICS:Registration}
(см.~\hyperref[sec:3.1_PS1_Preamble]{преамбулу} раздела~\ref{sec:3.1_PS1_Compliance_of_written_valuation}~\nameref{sec:3.1_PS1_Compliance_of_written_valuation}
\vpageref{sec:3.1_PS1_Preamble}--\pageref{sec:3.1_PS1_Preamble-End}),
соответствуют либо превосходят \href{https://www.ivsc.org/standards/international-professional-standards/consultation/international-professional-standards\#tab-documents}{Cтандарты Поведения и Компетентности Профессиональных Оценщиков}~\cite{IVSC:IPS},
разрабатываемые и~внедряемые со~стороны \href{https://www.ivsc.org/}{МССО}~\cite{IVSC:Wiki,IVSC:site}.\label{3.2.1.1-End}

\stepcounter{SubSecCounter}

\thesubsection.\theSubSecCounter.\label{3.2.1.2} Данные \emph{Всемирные
стандарты} также полностью соответствуют \href{https://www.rics.org/globalassets/rics-website/media/upholding-professional-standards/standards-of-conduct/international-ethics-standards-ies-rics.pdf}{актуальной редакции Принципов Этики}~\cite{IES-2016},
опубликованных \href{https://ies-coalition.org/}{Международной Коалицией Профессиональных Организаций}~\cite{IESC-site},
участником которой является RICS.\label{3.2.1.2-End}

\stepcounter{SubSecCounter}

\thesubsection.\theSubSecCounter.\label{3.2.1.3} Помимо необходимости
соответствия этим высоких принципам и~требованиям на~всех \textsl{членов
RICS} распространяются дополнительные "--- во~многих случаях более
строгие "--- требования, изложенные ниже. Контроль соблюдения требований
осуществляется посредством \href{https://www.rics.org/eu/upholding-professional-standards/regulation/}{регуляторной функции}~\cite{RICS:Regulation}
RICS.\label{3.2.1.3-End}

\stepcounter{SubSecCounter}

\thesubsection.\theSubSecCounter.\label{3.2.1.4} Требования, устанавливаемые
настоящими стандартами, направлены непосредственно на~\textsl{членов
RICS}, осуществляющих \emph{оценочную деятельность}, т.\,е.~формирующих
\emph{суждения о~стоимости}, обладающих соответствующими техническими
навыками, опытом и~пониманием природы \emph{объекта оценки}, его~рыночного
окружения, а~также \textsl{цели оценки}.\label{3.2.1.4-End}

\stepcounter{SubSecCounter}

\thesubsection.\theSubSecCounter.\label{3.2.1.5} Во~всех случаях
\textsl{члены RICS} обязаны действовать честно и~избегать любых действий
и~ситуаций, вступающих в~противоречие их~профессиональным обязательствам.
При~работе над~конкретными оценочными заданиями, \textsl{они}~обязаны
обеспечивать необходимый уровень независимости и~объективности, применяя
профессиональный скептицизм к~\emph{информации} и~\emph{данным},\footnote{Прим. пер.: по~мнению переводчика, отличие между \emph{информацией}
и~\emph{данными} в~данном контексте заключается в~том, что~под~\emph{информацией}
понимаются:
\begin{itemize}
\item знания о~предметах, фактах, идеях и~т.\,д., которыми могут обмениваться
люди в~рамках конкретного контекста~\cite{ISO:Information_tech};
\item знания относительно фактов, событий, вещей, идей и~понятий, которые
в~определённом контексте имеют конкретный смысл~\cite{ISO:Inf_tech_voc},
\end{itemize}
таким образом, в~контексте данного материала под~\emph{информацией}
следует понимать совокупность сведений, образующих логическую схему:
теоремы, научные законы, формулы, эмпирические принципы, алгоритмы,
методы, законодательные и~подзаконные акты и~т.\,п.

\emph{Данные}~же представляют собой:
\begin{itemize}
\item формы представления информации, с~которыми имеют дело информационные
системы и~их~пользователи~\cite{ISO:Information_tech};
\item поддающееся многократной интерпретации представление информации в~формализованном
виде, пригодном для~передачи, связи или~обработки~\cite{ISO:Inf_tech_voc},
\end{itemize}
таким образом, в~контексте данного материала под~\emph{данными}
следует понимать собой совокупность результатов наблюдений о~свойствах
тех~или~иных объектов и~явлений, выраженных в~объективной форме,
предполагающей их~многократные передачу и~обработку.

Например: \emph{информацией} является знание о~том, что~для~обработки
переменных выборки аналогов, имеющих распределение отличное от~\href{http://www.machinelearning.ru/wiki/index.php?title=\%D0\%9D\%D0\%BE\%D1\%80\%D0\%BC\%D0\%B0\%D0\%BB\%D1\%8C\%D0\%BD\%D0\%BE\%D0\%B5_\%D1\%80\%D0\%B0\%D1\%81\%D0\%BF\%D1\%80\%D0\%B5\%D0\%B4\%D0\%B5\%D0\%BB\%D0\%B5\%D0\%BD\%D0\%B8\%D0\%B5}{нормального}~\cite{Distrib:Normal},
в~общем случае, некорректно использовать \href{http://www.machinelearning.ru/wiki/index.php?title=\%D0\%9A\%D0\%B0\%D1\%82\%D0\%B5\%D0\%B3\%D0\%BE\%D1\%80\%D0\%B8\%D1\%8F:\%D0\%9F\%D0\%B0\%D1\%80\%D0\%B0\%D0\%BC\%D0\%B5\%D1\%82\%D1\%80\%D0\%B8\%D1\%87\%D0\%B5\%D1\%81\%D0\%BA\%D0\%B8\%D0\%B5_\%D1\%81\%D1\%82\%D0\%B0\%D1\%82\%D0\%B8\%D1\%81\%D1\%82\%D0\%B8\%D1\%87\%D0\%B5\%D1\%81\%D0\%BA\%D0\%B8\%D0\%B5_\%D1\%82\%D0\%B5\%D1\%81\%D1\%82\%D1\%8B}{параметрические методы}~\cite{Stat.test:par}
статистического анализа; \emph{данные} в~этом случае "--- это~непосредственно
сама выборка.

Иными словами, оперируя терминологией \href{https://ru.wikipedia.org/wiki/\%D0\%90\%D1\%80\%D1\%85\%D0\%B8\%D1\%82\%D0\%B5\%D0\%BA\%D1\%82\%D1\%83\%D1\%80\%D0\%B0_\%D0\%BA\%D0\%BE\%D0\%BC\%D0\%BF\%D1\%8C\%D1\%8E\%D1\%82\%D0\%B5\%D1\%80\%D0\%B0}{архитектуры ЭВМ}~\cite{Wiki:Computer_architecture_rus},
\emph{данные} "--- набор значений переменных, \emph{информация} "---
набор инструкций.

Во~избежании двусмысленности в~тексте данного переводного материала
эти~термины приводятся именно в~тех~смыслах, которые описаны выше.
В~случае необходимости также используется более общий термин \guillemotleft сведения\guillemotright ,
обобщающий оба~вышеуказанных понятия. В~ряде случае, термины используются
в~соответствии с~принятым значением в~контексте устоявшихся словосочетаний.
Например, термин \guillemotleft раскрытие информации\guillemotright{}
в б\'{о}льшей степени относится к~данным, однако, в~силу устоявшегося
значения, в~данном материале он~приводится в~своей общепринятой
форме.} используемым в~качестве доказательств. Профессиональный скептицизм
"--- это~подход, включающий профессиональное любопытство, критическое
восприятие \emph{информации} и~данных, используемых в~ходе проведения
\textsl{оценки}, и~бдительность в~отношении условий, способных повлечь
за~собой введение в~заблуждение. \textsl{Члены RICS} обязаны не~допускать
возникновение ситуаций, при~которых \emph{\href{https://en.wikipedia.org/wiki/Conflict_of_interest}{конфликт интересов}~\cite{COI}}
превалирует над~их~профессиональными и~деловыми суждениями и~обязательствами,
а~также не~разглашать конфиденциальные сведения. На~всех \textsl{членов
RICS} распространяется действие \href{https://www.rics.org/ssa/upholding-professional-standards/standards-of-conduct/rules-of-conduct/}{Правил Поведения RICS}~\cite{RICS:Conduct,RICS:Conduct-Firms,RICS:Conduct-Members},
вследствие чего они~обязаны соблюдать положения Свода Профессиональных
Правил RICS (далее "--- \textbf{СПП}) \href{https://www.rics.org/globalassets/rics-website/media/upholding-professional-standards/standards-of-conduct/conflicts-of-interest/conflicts_of_interest_global_1st-edition_dec_2017_revisions_pgguidance_2017_rw.pdf}{«Конфликт интересов»}~\cite{RICS:Rule:Conflicts-of-interest}
Дополнительная информация доступна в~\href{https://www.rics.org/eu/upholding-professional-standards/standards-of-conduct/ethics/}{соответствующем разделе}~\cite{RICS:ethics-page}
\href{https://www.rics.org/eu/}{сайта RICS}~\cite{RICS:site}.\label{3.2.1.5-End}\label{subsec:3.2.1_Professional_and_ethical_standards-End}

\subsection{Квалификация членов RICS\label{subsec:3.2.2_Member qualification}}

\stepcounter{SubSecCounter}

\thesubsection.\theSubSecCounter.\label{3.2.2.1}

Определение того, является~ли лицо достаточно квалифицированным для~того,
чтобы нести ответственность за~проверку входных \emph{данных} для~\textsl{оценки}
либо осуществлять контроль за~этим процессом, основывается на~проверке
соответствия следующим критериям:
\begin{itemize}
\item наличие соответствующей образовательной (профессиональной) подготовки,
обеспечивающей достаточную компетентность в~технических вопросах;
\item наличие членства в~профессиональном сообществе, обеспечивающее приверженность
нормам этики;
\item достаточное понимание природы оцениваемого актива и~его~рыночного
окружения на~локальном, национальном либо международном уровне (в~зависимости
от~задачи), а~также наличие навыков и~профессионального кругозора,
необходимых для~качественного проведения \textsl{оценки};
\item соответствие требованиям национального либо территориального законодательства,
регулирующего вопросы права на~проведение \textsl{оценки};
\item там, где~это~применимо, соответствие требованиям \href{https://www.rics.org/eu/upholding-professional-standards/regulation/valuer-registration/}{Программы Регистрации Оценщиков RICS}~\cite{RICS:Registration}.\label{3.2.2.1-End}
\end{itemize}
\stepcounter{SubSecCounter}

\thesubsection.\theSubSecCounter.\label{3.2.2.2} Поскольку \textsl{члены
RICS} осуществляют деятельность по~широкому спектру специальностей
на~многих рынках, членство в~RICS (включая наличие квалификации)
или~\href{https://www.rics.org/eu/upholding-professional-standards/regulation/valuer-registration/}{регистрация в качестве оценщика}~\cite{RICS:Registration}
сами по~себе не~означают, что~лицо обязательно обладает практическим
опытом проведения \emph{оценки} в~определённой отрасли или~на~определённом
рынке "--- такие вопросы в~обязательном порядке требуют соответствующего
подтверждения.\label{3.2.2.2-End}

\stepcounter{SubSecCounter}

\thesubsection.\theSubSecCounter.\label{3.2.2.3} В~ряде юрисдикций
\emph{оценщики} обязаны проходить процедуру сертификации или~лицензирования
для~проведения определённых видов \textsl{оценки}. В~таких случаях
действуют правила, установленные подразделом \ref{subsec:3.1.4_Compliance_with_jurisdictional_standards}~\nameref{subsec:3.1.4_Compliance_with_jurisdictional_standards}~\vpageref{subsec:3.1.4_Compliance_with_jurisdictional_standards}--\pageref{subsec:3.1.4_Compliance_with_jurisdictional_standards-End}.
Кроме того, заказчиком либо \href{https://www.isurv.com/info/1342/rics_national_or_jurisdictional_valuation_standards}{Стандартами Национальных Ассоциаций}~\cite{RICS:National-Standards}
могут быть предусмотрены более жёсткие требования. В~таких случаях
в~условиях \textsl{договора на~проведение оценки} и~тексте \emph{отчёта
об~оценке} должны присутствовать положения о~соблюдении названных
стандартов, см.~п.~\pageref{3.1.4.2-End}--\vref{3.1.4.2}.\label{3.2.2.3-End}

\stepcounter{SubSecCounter}

\thesubsection.\theSubSecCounter.\label{3.2.2.4} Если \textsl{член
RICS} не~обладает уровнем знаний и~опытом, необходимыми для~надлежащего
решения того или~иного аспекта оценочного задания, то~ему~(ей)
следует определиться с~тем, какая помощь необходима. В~случае необходимости,
на~основании прямого соглашения с~заказчиком, \textsl{члену RICS}
следует заказать, собрать и~интерпретировать \emph{сведения}, полученные
от~других специалистов, таких как, например, \emph{оценщики}, экологи,
бухгалтеры или~юристы.\label{3.2.2.4-End}\footnote{Прим. пер.: из~ориг. англ. текста не~следует является~ли данный
перечень закрытым либо нет. По~мнению переводчика, данный перечень
не~является закрытым.}

\stepcounter{SubSecCounter}

\thesubsection.\theSubSecCounter.\label{3.2.2.5} Требования, предъявляемые
к~знаниям и~навыкам специалиста, могут быть выполнены путём совместной
работы нескольких \textsl{членов RICS}, являющихся сотрудниками \textsl{оценочной
компании} и~отвечающих в~совокупности таковым, при~условии, что~каждый
из~них~соответствует остальным требованиям \textbf{СПД~2}.\label{3.2.2.5-End}

\stepcounter{SubSecCounter}

\thesubsection.\theSubSecCounter.\label{3.2.2.6} В~случае, если
\textsl{член RICS} планирует привлечь другую \textsl{оценочную компанию}
для~выполнения всех либо некоторых \textsl{оценок}, проведение которых
установлено \textsl{договором на~проведение оценки}, он~должен получить
на~это~согласие заказчика (см.~секцию~\ref{subsubsec:4.3.2.1_Identification_and_status_of_the_valuer}~\nameref{subsubsec:4.3.2.1_Identification_and_status_of_the_valuer}~\vpageref{subsubsec:4.3.2.1_Identification_and_status_of_the_valuer}--\pageref{subsubsec:4.3.2.1_Identification_and_status_of_the_valuer-End}).\label{3.2.2.6-End}

\stepcounter{SubSecCounter}

\thesubsection.\theSubSecCounter.\label{3.2.2.7} В~случае проведения
\textsl{оценки} либо внесения вклада в~её~проведение более чем~одним
\emph{оценщиком}, перечень этих \emph{оценщиков} должен храниться
среди рабочих материалов вместе с~подтверждением того, что~все~они~соответствовали
требованиям, изложенным в~разделе~\ref{sec:3.1_PS1_Compliance_of_written_valuation}~\nameref{sec:3.1_PS1_Compliance_of_written_valuation}~\vpageref{sec:3.1_PS1_Compliance_of_written_valuation}--\pageref{sec:3.1_PS1_Compliance_of_written_valuation-End}.\label{3.2.2.7-End}

\stepcounter{SubSecCounter}

\thesubsection.\theSubSecCounter.\label{3.2.2.8} \textsl{Член RICS},
ответственный за~управление (см.~п.~\pageref{3.1.1.1-End}--\vref{3.1.1.1}),
обязан обеспечить:
\begin{itemize}
\item надлежащий уровень контроля на~всех стадиях руководства \textsl{оценкой},
обеспечивающий её~надлежащую доказательную силу и~способность пройти
проверку и~устоять в~случае её~опротестования в~будущем, особенно
в~тех~случаях, когда выполнение оценочного задания связано с~удалёнными
местами и~(или) несколькими юрисдикциями;
\item принятие ответственности и~подотчётности за~\emph{отчёт об~оценке}
и~его~содержание, способность обосновывать выводы, содержащиеся
в~нём, а~также защищать его~в~случае оспаривания "--- крайне
важно, чтобы процесс руководства \textsl{оценкой} не~воспринимался
как~её~простое формальное автоматическое утверждение и~заверение
подписью без~надлежащего изучения её~содержания.\label{3.2.2.8-End}\label{subsec:3.2.2_Member qualification-End}
\end{itemize}

\subsection{Независимость, объективность, конфиденциальность, выявление и~урегулирование
конфликтов интересов\label{subsec:3.2.3_Independence_objectivity_confidentiality}}

\stepcounter{SubSecCounter}

\thesubsection.\theSubSecCounter.\label{3.2.3.1} Вопросы независимости
и~объективности неразрывно связаны с~надлежащим обеспечением конфиденциальности
сведений, ставших известными в~связи с~осуществлением \emph{оценочной
деятельности}, а~также с~более широкой проблемой выявления и~урегулирования
\href{https://en.wikipedia.org/wiki/Conflict_of_interest}{конфликтов интересов}~\cite{COI,RICS:Rule:Conflicts-of-interest}.
\textsl{Члены RICS} обязаны соблюдать требования, предусмотренные
\textbf{СПП} \textit{\href{https://www.rics.org/globalassets/rics-website/media/upholding-professional-standards/standards-of-conduct/conflicts-of-interest/conflicts_of_interest_global_1st-edition_dec_2017_revisions_pgguidance_2017_rw.pdf}{«Конфликт интересов»}~\cite{RICS:Rule:Conflicts-of-interest}},
и~внимательно относиться к~сопровождающим его~указаниям. Текст
в~нижеследующей части данного подраздела специально предназначен
для~решения вопросов, возникающих в~связи с~осуществлением \emph{оценочной
деятельности}, и~является дополнительным по~отношению к~\textbf{\href{https://www.rics.org/globalassets/rics-website/media/upholding-professional-standards/standards-of-conduct/conflicts-of-interest/conflicts_of_interest_global_1st-edition_dec_2017_revisions_pgguidance_2017_rw.pdf}{вышеуказанному СПП}~\cite{RICS:Rule:Conflicts-of-interest}}.\label{3.2.3.1-End}

\stepcounter{SubSecCounter}

\thesubsection.\theSubSecCounter.\label{3.2.3.2} \emph{Оценщикам}
следует помнить о~двух основополагающих требованиях, содержащихся
в~\textbf{СПП} \textit{\href{https://www.rics.org/globalassets/rics-website/media/upholding-professional-standards/standards-of-conduct/conflicts-of-interest/conflicts_of_interest_global_1st-edition_dec_2017_revisions_pgguidance_2017_rw.pdf}{«Конфликт интересов»}~\cite{RICS:Rule:Conflicts-of-interest}}.
\begin{enumerate}
\item Ни~один \textsl{член RICS} не~должен консультировать или~представлять
заказчика в~тех~случаях, когда осуществление этого вызывает \emph{\href{https://en.wikipedia.org/wiki/Conflict_of_interest}{конфликт интересов}~\cite{COI,RICS:Rule:Conflicts-of-interest}}
либо существенный риск его~возникновения, за~исключением тех~случаев,
когда все~лица, чьи~интересы будут либо могут быть затронуты, предварительно
дали своё информированное согласие. Сторона, чьи~интересы затронуты
либо могут быть затронуты, может дать информированное согласие только
в~том случае, если лицо, разъясняющее положение дел, действует полностью
прозрачно, а~также, если данное лицо уверено в~том, что~такая сторона
осознаёт значение и~последствия своих действий, включая связанные
с~этим риски, а~также возможные альтернативные варианты, и~даёт
такое согласие добровольно. Информированное согласие может быть запрошено
только в~том~случае, если \textsl{член RICS} убеждён в~том, что~продолжение
процесса, несмотря на~наличие \textit{\href{https://en.wikipedia.org/wiki/Conflict_of_interest}{конфликта интересов}}~\cite{COI,RICS:Rule:Conflicts-of-interest},
отвечает интересам всех, кого это~затрагивает либо может затронуть.
\item \textsl{Члены RICS} должны хранить записи о~решениях, принятых относительно
того соглашаться~ли на~выполнение (продолжение выполнения) конкретного
профессионального задания, о~получении информированных согласий,
а~также о~мерах, предпринятых во~избежание \textit{\href{https://en.wikipedia.org/wiki/Conflict_of_interest}{конфликта интересов}}~\cite{COI,RICS:Rule:Conflicts-of-interest}.\label{3.2.3.2-End}
\end{enumerate}
\stepcounter{SubSecCounter}

\thesubsection.\theSubSecCounter.\label{3.2.3.3} Придание необходимого
уровня независимости и~объективности при~выполнении конкретных заданий,
уважение и~обеспечение требований конфиденциальности, а~также выявление
и~урегулирование потенциальных либо фактически существующих \emph{\href{https://en.wikipedia.org/wiki/Conflict_of_interest}{конфликтов интересов}~\cite{COI,RICS:Rule:Conflicts-of-interest}}
имеют решающее значение. Работа \emph{оценщика} зачастую сопряжена
с~особенной сложностью и~чувствительностью в~таких вопросах, вследствие
и~по~причине чего от~\textsl{членов RICS }требуются действия, строго
соответствующие нижеописанным как~общим принципам, так~и~критериям,
являющимся специфичными для~\textsl{оценки}.\label{3.2.3.3-End}

\stepcounter{SubSecCounter}

\thesubsection.\theSubSecCounter.\label{3.2.3.4} В~случае некоторых
\emph{целей оценки}, требований закона, нормативных актов, распоряжений
органов власти или~при~наличии специальных требований заказчика
(например при~проведении \textsl{оценки} для~целей залогового обеспечения
обязательства "--- см.~раздел~\ref{sec:5.2_VPGA-2_Valuation_for_secure_lending}~\nameref{sec:5.2_VPGA-2_Valuation_for_secure_lending}\vpageref{sec:5.2_VPGA-2_Valuation_for_secure_lending}--\pageref{sec:5.2_VPGA-2_Valuation_for_secure_lending-End})
для~достижения определённого уровня независимости могут быть установлены
специфические критерии, которым должен отвечать \textsl{член RICS},
являющиеся дополнительными по~отношению к~остальным, изложенным
далее по~тексту. Зачастую такие дополнительные критерии содержат
определение \emph{приемлемого уровня независимости} и~в~них~могут
быть использованы такие термины как~\textit{«независимый эксперт»},
\textit{«эксперт-оценщик»}, \textit{«независимый оценщик»}, \textsl{«постоянный
независимый оценщик»} или~\textit{«приемлемый оценщик»}. Важно, чтобы
\textsl{член RICS} подтвердил своё соответствие этим критериям как~при~принятии
оценочного задания, так~и~в~тексте \emph{отчёта об~оценке}, таким
образом, чтобы заказчик либо любая \textsl{третья сторона}, полагающиеся
на~\emph{отчёт}, могли~бы быть уверены в~том, что~дополнительные
критерии были соблюдены.\label{3.2.3.4-End}

\stepcounter{SubSecCounter}

\thesubsection.\theSubSecCounter.\label{3.2.3.5} Согласно определению,
данному в~\textbf{СПП} \href{https://www.rics.org/globalassets/rics-website/media/upholding-professional-standards/standards-of-conduct/conflicts-of-interest/conflicts_of_interest_global_1st-edition_dec_2017_revisions_pgguidance_2017_rw.pdf}{«Конфликт интересов»}~\cite{RICS:Rule:Conflicts-of-interest},
под~\emph{конфиденциальной информацией} понимается \textsl{«конфиденциальная
информация, хранящаяся либо распространяемая в~электронном или~устном
виде либо на~материальном носителе»}. Существует общая обязанность
относиться ко~всей информации, касающейся заказчика, как~к~\emph{конфиденциальной},
если эта~информация стала известна в~результате профессиональных
отношений и~не~является общедоступной. Информация и~данные, собранные
в~ходе проведения работы по~\textsl{оценке}, могут быть чувствительными
для~рынка, вследствие чего вышеуказанная обязанность имеет особое
значение.\label{3.2.3.5-End}

\stepcounter{SubSecCounter}

\thesubsection.\theSubSecCounter.\label{3.2.3.6} В~частности, в~обязательном
порядке необходимо проявлять особую осторожность в~части сохранения
конфиденциальности сведений при~обращении к~заказчику согласно п.~\vref{4.3.2.16.1}--\pageref{4.3.2.16.1-End}
в~части, касающейся «ключевых исходных данных». Согласно \textbf{СПП}
\href{https://www.rics.org/globalassets/rics-website/media/upholding-professional-standards/standards-of-conduct/conflicts-of-interest/conflicts_of_interest_global_1st-edition_dec_2017_revisions_pgguidance_2017_rw.pdf}{«Конфликт интересов»}~\cite{RICS:Rule:Conflicts-of-interest},
обязанность по~соблюдению конфиденциальности всегда превалирует над~обязанностью
по~\emph{раскрытию информации} при~условии соблюдения требований
закона.\label{3.2.3.6-End}

\stepcounter{SubSecCounter}

\thesubsection.\theSubSecCounter.\label{3.2.3.7} Риск раскрытия
\emph{конфиденциальной информации} также является существенным фактором,
который \emph{оценщик} обязан принимать во~внимание при~принятии
решения относительно того, имеет~ли место потенциальный \emph{конфликт
интересов} или, в~терминах, установленных пп.~«c» п.~4.2 \textbf{СПП~}\href{https://www.rics.org/globalassets/rics-website/media/upholding-professional-standards/standards-of-conduct/conflicts-of-interest/conflicts_of_interest_global_1st-edition_dec_2017_revisions_pgguidance_2017_rw.pdf}{«Конфликт интересов»}~\cite{RICS:Rule:Conflicts-of-interest},
«конфликт конфиденциальной информации». Иногда бывает необходимо раскрыть
некоторые детали участия \emph{оценщика} в~отношении \emph{объекта
оценки}. Если адекватное \emph{раскрытие информации} не~может быть
сделано без~нарушения обязательств о~сохранении конфиденциальности,
следует отказаться от~выполнения такой работы.\label{3.2.3.7-End}

\stepcounter{SubSecCounter}

\thesubsection.\theSubSecCounter.\label{3.2.3.8} Обязательство по~сохранению
конфиденциальности действует бессрочно и~непрерывно и~распространяет
своё действие на~нынешних, бывших и~даже потенциальных заказчиков.\label{3.2.3.8-End}

\stepcounter{SubSecCounter}

\thesubsection.\theSubSecCounter.\label{3.2.3.9} В~то~время как~в~контексте
\textsl{оценки} невозможно составить исчерпывающий список ситуаций,
способных создать угрозу для~независимости или~объективности \textsl{члена
RICS}, следующий перечень содержит условия, которые всегда следует
рассматривать в~качестве представляющих реальную либо потенциальную
угрозу, и, следовательно, требующих принятия необходимых мер, установленных
соответствующим \href{https://www.rics.org/globalassets/rics-website/media/upholding-professional-standards/standards-of-conduct/conflicts-of-interest/conflicts_of_interest_global_1st-edition_dec_2017_revisions_pgguidance_2017_rw.pdf}{Сводом правил}~\cite{RICS:Rule:Conflicts-of-interest}:
\begin{itemize}
\item действия в~интересах покупателя и~продавца объекта недвижимости
либо иного актива в~рамках одной сделки;
\item действия в~интересах двух или~более сторон, конкурирующих в~рамках
конкретной ситуации;\footnote{Прим. пер.: в~данном случае подразумевается не~конкуренция между
рыночными агентами как~таковая, а~конкуренция за~конкретную возможность,
т.\,е.~обычная работа \emph{оценщика} в~интересах, например, \label{https://www.bspb.ru/en/about/}~\cite{PJSC_BSPB:official_site}
и~\href{https://www.vikingbank.ru/Viking_new2.nsf/main/en}{АО «КАБ ,,Викинг''»}~\cite{JSC_Bank_Viking:official_site}
по~\textsl{оценке} разных залогов в~рамках обычного сотрудничества
с~ними не~является потенциальной угрозой в~контексте данного подраздела,
тогда как~проведение \textsl{оценки} для~каждого из~банков одного
и~того~же земельного участка, в~покупке которого они~заинтересованы,
и~будут либо потенциально могут стать конкурентами на~торгах по~его~купле-продаже,
является таковой.}
\item проведение \textsl{оценки} в~интересах кредитора при~одновременном
оказании консультаций заёмщику или~брокеру;\footnote{Прим. пер.: здесь также имеется в~виду одновременная работа на~две
стороны одной сделки, а~не~сотрудничество с~ними вообще.}
\item проведение \textsl{оценки} объекта недвижимости либо иного актива,
\textsl{оценка} которого ранее выполнялась для~другого заказчика
теми~же \emph{оценщиком} либо \textsl{оценочной компанией};
\item проведение \textsl{оценки}, результаты которой будет либо может использовать
\textsl{третья сторона}, в~то~время как~\textsl{оценочная компания}
имеет иные коммерческие отношения с~заказчиком;
\item оценка в~интересах обеих сторон сделки по~аренде.
\end{itemize}
\textsl{Членам RICS} следует помнить о~том, что~интересы любой \textsl{третьей
стороны} в~проводимой \textsl{оценке}, а~также возможность того,
что~она~может полагаться на~результаты \textsl{оценки}, следует
принимать во~внимание в~качестве фактора, требующего учёта в~контексте
данного подраздела.\label{3.2.3.9-End}

\stepcounter{SubSecCounter}

\thesubsection.\theSubSecCounter.\label{3.2.3.10} Угроза для~объективности
\textsl{члена RICS} может возникнуть в~том~случае, если результат
\textsl{оценки} до~её~завершения обсуждается с~заказчиком либо
иной стороной, имеющей интерес касательно предмета \textsl{оценки}.
Хотя такие обсуждения не~являются неприемлемыми и~могут быть полезны
как~для~\emph{оценщика} так~и~для~заказчика, \textsl{член RICS}
обязан внимательно относиться к~потенциальному влиянию, которое такие
обсуждения могут оказать на~его~основную обязанность "--- вынесение
объективного \emph{суждения о~стоимости}. В~случае, если такие обсуждения
имеют место, \textsl{член RICS} обязан вести письменную отчётность
о~подобных встречах и~беседах, а, в~случае если \textsl{член RICS}
примет решение изменить предварительные результаты \textsl{оценки}
под~их~влиянием, основания для~этого также должны быть тщательно
зафиксированы.\label{3.2.3.10-End}

\stepcounter{SubSecCounter}

\thesubsection.\theSubSecCounter.\label{3.2.3.11} У~\textsl{члена
RICS} может возникнуть потребность в~обсуждении различных вопросов
в~т.\,ч. таких как проверка фактов и~иных относящихся к~\textsl{оценке}
информации и~данных (например подтверждение уровня текущих арендных
платежей или~уточнение границ объекта) прежде, чем~он~сформирует
предварительное мнение о~стоимости. На~любой стадии процесса \textsl{оценки}
такие обсуждения дают заказчику возможность понять точку зрения \textsl{члена
RICS} и~ход его~рассуждений. Предполагается, что~заказчик будет
раскрывать факты и~информацию, в~т.\,ч.~касающиеся сделок с~имуществом,
активами или~обязательствами, имеющие отношение к~задаче по~\textsl{оценке}.\label{3.2.3.11-End}

\stepcounter{SubSecCounter}

\thesubsection.\theSubSecCounter.\label{3.2.3.12} Предоставляя заказчику
предварительную консультацию, проект \emph{отчёта} либо \textsl{оценку}
до~её~завершения, \textsl{член RICS} обязан сообщить следующее:
\begin{itemize}
\item предоставленное мнение носит предварительный характер и~подлежит
доработке при~подготовке итогового \emph{отчёта};
\item данная консультация предоставляется исключительно для~внутренних
целей заказчика;
\item любая предварительная версия ни~в~коем случае не~подлежит публикации
либо иному обнародованию.
\end{itemize}
Если какие-либо вопросы, имеющие фундаментальную важность, не~были
отражены, необходимо отразить наличие данного упущения.\label{3.2.3.12-End}

\stepcounter{SubSecCounter}

\thesubsection.\theSubSecCounter.\label{3.2.3.13} В~тех~случаях,
когда обсуждения с~заказчиком происходят после предоставления предварительных
суждений либо материалов \textsl{оценки} важно, чтобы такие обсуждения
не~приводили и~не~могли~бы привести к~возникновению предположения
о~том, что~они~повлияли на~мнение \textsl{члена RICS}, за~исключением
вопросов исправления неточностей и~внесения дополнительно предоставленных
сведений.\label{3.2.3.13-End}

\stepcounter{SubSecCounter}

\thesubsection.\theSubSecCounter.\label{3.2.3.14} Для~демонстрации
того, что~такие обсуждения не~поставили под~угрозу независимость
\textsl{члена RICS}, рабочие записи, содержащие сведения о~предоставлении
предварительных версий \emph{отчёта об~оценке} либо \emph{суждения
о~стоимости}, должны включать в~себя:
\begin{itemize}
\item перечень предоставленных информации и~данных и~(или) высказанных
предложений со~стороны заказчика;
\item описание того, как~эти~информация и~данные были использованы при~рассмотрении
вопросов о~внесении существенных изменений в~мнение \emph{оценщика};
\item причины, по~которым результаты \textsl{оценки} были либо не~были
изменены.\label{3.2.3.14-End}
\end{itemize}
\stepcounter{SubSecCounter}

\thesubsection.\theSubSecCounter.\label{3.2.3.15} При~необходимости,
данные записи следует предоставить аудиторам либо любой другой стороне,
имеющей законный и~существенный интерес в~проведённой \textsl{оценке}.\label{3.2.3.15-End}\label{subsec:3.2.3_Independence_objectivity_confidentiality-End}

\subsection{Обеспечение строгого разделения между консультантами\label{subsec:3.2.4_Advisers_strict_separation}}

\stepcounter{SubSecCounter}

\thesubsection.\theSubSecCounter.\label{3.2.4.1} RICS имеет строгие
своды правил в~отношении минимальных стандартов, которые должны быть
приняты организациями после получения \emph{информированного согласия}
в~соответствии с~\textbf{СПП}~\href{https://www.rics.org/globalassets/rics-website/media/upholding-professional-standards/standards-of-conduct/conflicts-of-interest/conflicts_of_interest_global_1st-edition_dec_2017_revisions_pgguidance_2017_rw.pdf}{«Конфликт интересов»}~\cite{RICS:Rule:Conflicts-of-interest}
при~разделении консультантов, действующих в~интересах сторон, имеющих
противоположные интересы. Механизм, называемый в~ряде юрисдикций
«китайской стеной», созданный для~такого разделения, должен быть
достаточно устойчивым для~того, чтобы исключить возможность передачи
информации или~данных от~одной такой группы консультантов к~другой.
Это~строгий критерий: простого принятия «разумных мер» для~обеспечения
разделения недостаточно.\label{3.2.4.1-End}

\stepcounter{SubSecCounter}

\thesubsection.\theSubSecCounter.\label{3.2.4.2} Соответственно
любая договорённость, созданная и~согласованная с~заказчиками, чьи~интересы
затрагиваются, подлежит обязательному надзору со~стороны «специалиста
по~внутреннему контролю» так, как~описано ниже, и~должна соответствовать
всем нижеследующим критериям:
\begin{enumerate}
\item лица, действующие в~интересах клиентов, имеющих противоположные интересы,
должны быть разными, причём данное требование распространяется также
и~на~секретарский и~иной вспомогательный персонал;
\item такие специалисты либо их~группы должны быть физически разделены
по~крайней мере на~уровне отдельных помещений, если нет~возможности
обеспечить их~нахождение в~разных зданиях;
\item любая информация либо данные, в~какой~бы форме они~не~хранились,
должны быть недоступны «другой стороне» в~любой момент времени, и,
если они~хранятся в~письменной форме на~материальном носителе,
их~следует безопасно размещать в~закрытом отдельном помещении в~соответствии
с~требованиями специалиста по~внутреннем контролю или~иного руководящего
лица организации, действующего независимо;
\item специалист по~внутреннему контролю или~иное руководящее лицо организации:
\begin{enumerate}
\item осуществляет контроль за~созданием и~работой механизма разделения,
принимая требуемые меры и~проводя проверки его~эффективности;
\item не~должен иметь отношения к~выполнению самих заданий;
\item обладает достаточно высоким должностным статусом в~организации, позволяющим
осуществлять свою деятельность беспрепятственно;
\end{enumerate}
\item в~\textsl{оценочной компании} должна существовать соответствующая
образовательная программа подготовки по~принципам и~практике управления
в~сфере \emph{конфликтов интересов}.\label{3.2.4.2-End}
\end{enumerate}
\stepcounter{SubSecCounter}

\thesubsection.\theSubSecCounter.\label{3.2.4.3} Эффективные механизмы
управления \emph{конфликтами интересов} не~могут работать без~существенного
планирования, поэтому они~должны становиться неотъемлемой частью
корпоративной культуры. Вследствие этого небольшим \textsl{оценочным
компаниям} и~частнопрактикующим \emph{оценщиками} сложнее, а~порой
и~невозможно, обеспечить эффективную работу таких механизмов.\label{3.2.4.3-End}\label{subsec:3.2.4_Advisers_strict_separation-End}

\subsection{Раскрытие информации при~общественном интересе либо для~информирования
третьей стороны\label{subsec:3.2.5_Disclosures_for_public_interest}}

\subsubsection{Требования к~раскрытию информации\label{subsubsec:3.2.5.1_Disclosure_requirements}}

\stepcounter{SubSubSecCounter}

\thesubsection.\theSubSubSecCounter.\label{3.2.5.1.1} Некоторые
виды \textsl{оценки} могут использоваться сторонами, не~являющимися
заказчиком \emph{отчёта} либо лицом, для~которого этот \emph{отчёт}
предназначен. Примерами таких \textsl{оценок} являются \textsl{оценки},
выполняемые для:
\begin{itemize}
\item целей публикуемой \textsl{финансовой отчётности};
\item фондовой биржи либо аналогичных организаций;
\item публикаций, коммерческих информационных или рекламных целей;
\item разработки схем финансирования (называемых в~Западном полушарии \emph{«инвестиционными
программами»}), которые могут принимать различные формы в~зависимости
от~юрисдикции;
\item сделок по~\href{https://ru.wikipedia.org/wiki/\%D0\%A1\%D0\%BB\%D0\%B8\%D1\%8F\%D0\%BD\%D0\%B8\%D1\%8F_\%D0\%B8_\%D0\%BF\%D0\%BE\%D0\%B3\%D0\%BB\%D0\%BE\%D1\%89\%D0\%B5\%D0\%BD\%D0\%B8\%D1\%8F}{слиянию и поглощению}~\cite{Wiki:M&A-rus}.
\end{itemize}
В~тех случаях, когда и~если \textsl{оценка} проводится в~отношении
актива, который ранее уже~был оценён этими~же \emph{оценщиком} либо
\textsl{оценочной компанией} для~любых целей, то~в~\textsl{договоре
на~проведение оценки}, тексте \emph{отчёта об~оценке}, а~также
любой опубликованной ссылке на~\textsl{оценку}, в~зависимости от~обстоятельств,
должна быть раскрыта следующая информация:
\begin{itemize}
\item характер отношений с~заказчиком и~прежнее сотрудничество с~ним;
\item политика ротации;
\item продолжительность сотрудничества;
\item доля выручки, приходящаяся на~одного заказчика.\label{3.2.5.1.1-End}
\end{itemize}
\stepcounter{SubSubSecCounter}

\thesubsection.\theSubSubSecCounter.\label{3.2.5.1.2}

Требования к~\emph{раскрытию информации}, установленные настоящими
стандартами, могут быть изменены либо расширены требованиями, применяемыми
в~конкретных странах или~территориях, либо могут быть включены в~соответствующие
\href{https://www.isurv.com/info/1342/rics_national_or_jurisdictional_valuation_standards}{национальные стандарты}~\cite{RICS:National-Standards},
в~случае применения положений подраздела~\vref{subsec:3.1.4_Compliance_with_jurisdictional_standards}--\pageref{subsec:3.1.4_Compliance_with_jurisdictional_standards-End}.\label{3.2.5.1.2-End}

\stepcounter{SubSubSecCounter}

\thesubsection.\theSubSubSecCounter.\label{3.2.5.1.3} В~части,
касающейся проведения \textsl{оценки} для~целей залогового кредитования,
изменённые либо расширенные требования к~\textsl{оценке} в~части
\emph{раскрытия информации} приведены в~разделе~\ref{sec:5.2_VPGA-2_Valuation_for_secure_lending}~\nameref{sec:5.2_VPGA-2_Valuation_for_secure_lending}
\vpageref{sec:5.2_VPGA-2_Valuation_for_secure_lending}--\pageref{sec:5.2_VPGA-2_Valuation_for_secure_lending-End}.\label{3.2.5.1.3_End}\label{subsubsec:3.2.5.1_Disclosure_requirements-End}

\subsubsection{Использование результатов оценки третьими сторонами\label{subsubsec:3.2.5.2_Reliance_by_third_parties}}

\stepcounter{SubSubSecCounter}

\thesubsection.\theSubSubSecCounter.\label{3.2.5.2.1} В~случаях,
когда результаты \textsl{оценки} могут использоваться \textsl{третьей
стороной} (в~\hyperref[def:third_party]{определении},
данном в~\hyperref[chap:2_Glossary]{Глоссарии} на~с.~\pageref{Gloss:third_party}),
которая изначально может быть идентифицирована, информация, подлежащая
раскрытию согласно положениям данного подраздела, должна быть предоставлена
такой стороне до~проведения \textsl{оценки}. В~дополнение к~обычному
\emph{раскрытию информации}, также должна быть раскрыта информация
о~любых обстоятельствах, при~которых \emph{оценщик} либо \textsl{оценочная
компания} получают дополнительную материальную выгоду от~выполнения
такой работы сверх обычного вознаграждения либо размера комиссионных.
Это~даст \textsl{третьей стороне} возможность возразить против привлечения
к~выполнению этого задания данного \emph{оценщика} либо \textsl{оценочной
компании}, если она~считает, что~независимость и~объективность
\textsl{члена RICS} могут быть поставлены под~угрозу.\label{3.2.5.2.1-End}

\stepcounter{SubSubSecCounter}

\thesubsection.\theSubSubSecCounter.\label{3.2.5.2.2} Однако, во~многих
случаях, \textsl{третья сторона} представляет собой группу физических
лиц, например акционеров компании, \emph{раскрытие информации} для~которых
на~начальном этапе явно было~бы нецелесообразным. В~таких случаях,
самая ранняя возможность \emph{раскрытия информации} возникает при~выпуске
\emph{отчёта} либо публикации ссылки на~него. Таким образом, на~\textsl{члене
RICS} лежит большая ответственность: прежде чем~принять задание,
необходимо понять: смогут~ли \textsl{третьи лица}, использующие результаты
\textsl{оценки}, согласиться с~тем, что~любое участие данного \textsl{члена
RICS}, требующее \emph{раскрытия информации}, не~нанесёт неоправданного
ущерба его~объективности и~независимости. Далее в~подразделе~\ref{subsec:3.2.8_Responsibility_for_the_valuation}~\nameref{subsec:3.2.8_Responsibility_for_the_valuation}
\vpageref{subsec:3.2.8_Responsibility_for_the_valuation}--\pageref{subsec:3.2.8_Responsibility_for_the_valuation-End}
приводятся подробности применительно к~отдельным категориям \textsl{оценок}.\label{3.2.5.2.2-End}

\stepcounter{SubSubSecCounter}

\thesubsection.\theSubSubSecCounter.\label{3.2.5.2.3} \textsl{Оценки},
затрагивающие общественные интересы либо используемые \textsl{третьими
лицами}, часто являются объектом законодательного либо иного нормативного
регулирования. Зачастую имеются конкретные положения, которым должен
соответствовать \textsl{член RICS}, для~того, чтобы его~считали
подходящим для~того, чтобы высказать действительно независимое и~объективное
\emph{суждение о~стоимости}. В~противном случае, ответственность
за~обеспечение информирования о~потенциальных конфликтах и~угрозах
независимости и~объективности лежит на~самом \textsl{члене RICS}.\label{3.2.5.2.3-End}\label{subsubsec:3.2.5.2_Reliance_by_third_parties-End}

\subsubsection{Отношения с~клиентом и~прежнее сотрудничество\label{subsubsec:3.2.5.3_Relations_with_client_and_past}}

\stepcounter{SubSubSecCounter}

\thesubsection.\theSubSubSecCounter.\label{3.2.5.3.1} Несмотря на~однозначность
требования к~\textsl{членам RICS} действовать независимо, честно
и~объективно, как~это~описано выше, нет~необходимости раскрывать
все~детали отношений между \textsl{членом RICS} и~заказчиком. \textsl{Член
RICS} должен принимать во~внимание и~соблюдать положения \textbf{СПП}
~\href{https://www.rics.org/globalassets/rics-website/media/upholding-professional-standards/standards-of-conduct/conflicts-of-interest/conflicts_of_interest_global_1st-edition_dec_2017_revisions_pgguidance_2017_rw.pdf}{«Конфликт интересов»}~\cite{RICS:Rule:Conflicts-of-interest}.
В~случае наличия неустранимых сомнений рекомендуется \emph{раскрывать
информацию}.\label{3.2.5.3.1-End}

\stepcounter{SubSubSecCounter}

\thesubsection.\theSubSubSecCounter.\label{3.2.5.3.2} С~целью выявления
какого-либо потенциального \href{https://en.wikipedia.org/wiki/Conflict_of_interest}{конфликта интересов}~\cite{COI,RICS:Rule:Conflicts-of-interest},
в~условиях, когда \textsl{член RICS} либо \textsl{оценочная компания}
были вовлечены в~сделку по~приобретению одного или~нескольких активов
для~данного заказчика в~течение 12~(двенадцати) месяцев либо иного
более длительного срока, установленного в~конкретной юрисдикции,
предшествующего дате задания либо согласования условий \textsl{договора
на~проведение оценки~}(в~зависимости от~того, что~произойдёт
ранее), \textsl{член RICS} в~отношении этих активов обязан раскрыть
следующую информацию:
\begin{itemize}
\item получение предоплаты;
\item участие в~переговорах по~сделке на~стороне заказчика.\label{3.2.5.3.2-End}
\end{itemize}
\stepcounter{SubSubSecCounter}

\thesubsection.\theSubSubSecCounter.\label{3.2.5.3.3} Для~целей
настоящего стандарта \emph{раскрытия информации}, необходимо дать
определения терминам \emph{«заказчик»} и~\emph{«оценочная компания»}.\label{3.2.5.3.3-End}

\stepcounter{SubSubSecCounter}

\thesubsection.\theSubSubSecCounter.\label{3.2.5.3.4} Существует
множество различных отношений, которые могут быть квалифицированы
как~отношения \emph{заказчика} и~\textsl{оценочной компании}. В~соответствии
с~\hyperref[sec:4.1_VPS1_Terms_of_engagement_Scope_of_work]{минимальными требованиями к~содержанию \textsl{договора на~проведение оценки}},
установленными \hyperref[sec:4.1_VPS1_Terms_of_engagement_Scope_of_work]{СПО~1}на~\vpageref{sec:4.1_VPS1_Terms_of_engagement_Scope_of_work}--\pageref{sec:4.1_VPS1_Terms_of_engagement_Scope_of_work-End},
и~\hyperref[sec:4.3_VPS3_Valuation_reports]{требованиями к~подготовке \emph{отчёта об~оценке}},
установленными \hyperref[sec:4.3_VPS3_Valuation_reports]{СПО~3}
на~с.~\vpageref{sec:4.3_VPS3_Valuation_reports}--\pageref{sec:4.3_VPS3_Valuation_reports-End},
под~\emph{\guillemotleft заказчиком\guillemotright} понимается лицо,
согласовывающее условия \textsl{договора на~проведение оценки}, и~которому
адресован \emph{отчёт об~оценке}. \textsl{\guillemotleft Оценочной
компанией\guillemotright} является лицо, указанное в~\textsl{договоре
на~проведение оценки} и~\emph{отчёте об~оценке}.\label{3.2.5.3.4-End}

\stepcounter{SubSubSecCounter}

\thesubsection.\theSubSubSecCounter.\label{3.2.5.3.5} Тесно связанные
хозяйственные общества, входящие в~\emph{\href{https://de.wikipedia.org/wiki/Unternehmensverbindung}{группу компаний}~\cite{Wiki:holding}},
должны рассматриваться как~один заказчик или~\textsl{оценочная компания}.
Однако, в~связи с~зачастую сложным характером современного бизнеса,
нередки случаи, когда другие организации имеют лишь отдалённые юридические
либо коммерческие связи с~заказчиком, для~которого \textsl{оценочная
компания} \textsl{члена RICS} также оказывает услуги. На~практике
могут возникнуть сложности при~выявлении таких отношений, например
в~случае их~наличия между партнёрами \textsl{оценочной компании}
\textsl{члена RICS} в~других странах либо территориях и~заказчиком.
Иногда именно коммерческие отношения \textsl{члена RICS} со~стороной,
не~являющейся заказчиком, могут создавать существенную угрозу независимости
первого.\label{3.2.5.3.5-End}

\stepcounter{SubSubSecCounter}

\thesubsection.\theSubSubSecCounter.\label{3.2.5.3.6} Ожидается,
что~\textsl{член RICS} запросит сведение, уместные в~конкретных
обстоятельствах: нет~необходимости пытаться установить наличие всех
возможных отношений, которые могут иметь место, при~условии, что~\textsl{член
RICS} придерживается принципов, установленных данным стандартом.\label{3.2.5.3.6-End}

\stepcounter{SubSubSecCounter}

\thesubsection.\theSubSubSecCounter.\label{3.2.5.3.7} Ниже приводятся
примеры, когда требования по~\emph{раскрытию информации}, относятся
к~сторонам, отличным от~лица, от~которого исходит оценочное задание:
\begin{itemize}
\item дочерние структуры \href{https://de.wikipedia.org/wiki/Unternehmensverbindung}{холдинговой компании}~\emph{\cite{Wiki:holding}
}"--- заказчика \textsl{оценки};
\item дочерняя структура \href{https://de.wikipedia.org/wiki/Unternehmensverbindung}{холдинга}~\emph{\cite{Wiki:holding}},
формирующая оценочное задание, при~условии, что~существуют отношения
между \textsl{оценочной компанией} \textsl{члена RICS} и~другими
компаниями, связанными с~\href{https://de.wikipedia.org/wiki/Mutterunternehmen}{материнской компанией}~\cite{Wiki:parent_company}
данного \href{https://de.wikipedia.org/wiki/Unternehmensverbindung}{холдинга}~\emph{\cite{Wiki:holding}};
\item \textsl{третья сторона}, дающая оценочные задания в~качестве агента,
действующего от~имени и~в~интересах различных компаний, например,
менеджмент фонда, управляющего активами.\label{3.2.5.3.7-End}
\end{itemize}
\stepcounter{SubSubSecCounter}

\thesubsection.\theSubSubSecCounter.\label{3.2.5.3.8} Аналогичные
соображения применимы при~определении степени \emph{раскрытия информации}
в~отношении \textsl{оценочной компании} \textsl{члена RICS} в~случаях,
когда речь идёт об~отдельных юридических лицах, действующих на~различных
территориях и~(или) выполняющих разные виды работ. Может быть нецелесообразно
включать в~периметр \emph{раскрытия информации} все~организации,
связанные с~\textsl{оценочной компанией}, проводящей \textsl{оценку},
в~случаях, когда их~деятельность не~имеет существенного значения
для~рассматриваемой \textsl{оценки} либо далека от~неё, например,
вообще не~связана с~\textsl{оценкой} либо оказанием сопутствующих
консультаций. Однако, в~случае существования \href{https://de.wikipedia.org/wiki/Unternehmensverbindung}{группы тесно связанных юридических лиц}~\cite{Wiki:holding},
осуществляющих деятельность с~использованием общих \href{https://en.wikipedia.org/wiki/Trademark}{товарных знаков (знаков обслуживания)}~\cite{Wiki:trademark},
необходимо раскрыть характер и~степень связей заказчика со~всеми
такими лицами "--- например, в~случае, если \textsl{оценочная компания}
осуществляет свою деятельность таким образом, при~котором одно её~подразделение\footnote{Прим. пер.: под~\emph{\guillemotleft подразделением\guillemotright}
понимается как~структурные подразделения одного хозяйственного общества,
так~и~иное "--- формально независимое, но~\href{https://ru.wikipedia.org/wiki/\%D0\%90\%D1\%84\%D1\%84\%D0\%B8\%D0\%BB\%D0\%B8\%D1\%80\%D0\%BE\%D0\%B2\%D0\%B0\%D0\%BD\%D0\%BD\%D0\%BE\%D0\%B5_\%D0\%BB\%D0\%B8\%D1\%86\%D0\%BE}{аффилированное}~\cite{Wiki:affiliate}
общество.} проводит \textsl{оценку}, а~другое~(другие) оказывает~(оказываю)
все~прочие услуги по~управлению имуществом и~консультациями в~этой
области.\label{3.2.5.3.8-End}

\stepcounter{SubSubSecCounter}

\thesubsection.\theSubSubSecCounter.\label{3.2.5.3.9} \href{https://www.isurv.com/info/1342/rics_national_or_jurisdictional_valuation_standards}{Стандартами оценки национальных ассоциаций RICS}~\cite{RICS:National-Standards}
равно как~и~стандартами \textsl{оценки} и~иными нормативными актами,
принятыми в~конкретной национальной юрисдикции могут, устанавливаться
дополнительные требования к~\emph{раскрытию информации}, расширяющие
требования данного стандарта.\label{3.2.5.3.9-End}\label{subsubsec:3.2.5.3_Relations_with_client_and_past-End}

\subsubsection{Политика ротации\label{subsubsec:3.2.5.4_Rotation_policy}}

\stepcounter{SubSubSecCounter}

\thesubsection.\theSubSubSecCounter.\label{3.2.5.4.1} Обязанность
\emph{раскрывать информацию} в~отношении политики ротации \textsl{оценочной
компании} возникает у~неё~только в~том~случае, если \textsl{член
RICS} выполнил ряд~\textsl{оценок} в~течение периода времени. В~случае
первого либо разового поручения отсутствует необходимость \emph{раскрывать
информацию} о~политике ротации, принятой в~\textsl{оценочной компании}.\label{3.2.5.4.1-End}

\stepcounter{SubSubSecCounter}

\thesubsection.\theSubSubSecCounter.\label{3.2.5.4.2} В~случаях,
когда \textsl{член RICS}, отвечающий за~\textsl{оценку}, проводимую
в~соответствии с~настоящими стандартами, осуществляет её~в~течение
многих лет, знакомство с~заказчиком либо знания об~активе могут
создать впечатление, что~его~независимость и~объективность могли
быть скомпрометированы. Данная проблема может быть решена путём организации
системы ротации \textsl{членов RICS}, отвечающих за~\textsl{оценку}
для~конкретных клиентов либо конкретных активов.\label{3.2.5.4.2-End}

\stepcounter{SubSubSecCounter}

\thesubsection.\theSubSubSecCounter.\label{3.2.5.4.3} Метод организации
ротации лиц, ответственных за~проведение \textsl{оценки} в~\textsl{оценочной
компании}, устанавливается ей~самостоятельно после обсуждения с~заказчиком,
если таковое необходимо. Вместе с~тем, рекомендация RICS заключается
в~том, что~лицо "--- \textsl{член RICS}, ответственное за~подписание
\emph{отчёта об~оценке}, независимо от~своего положения в~\textsl{оценочной
компании}, несло такую ответственность в~течение лишь ограниченного
времени. Конкретный лимит времени зависит от:
\begin{itemize}
\item регулярности проведения \textsl{оценки};
\item наличия процедур контроля и~надзора, например работы «комиссии по~\textsl{оценке}»,
способствующих точности и~объективности процесса \textsl{оценки};
\item сложившихся \href{https://neg.by/novosti/otkrytj/obychaj-delovogo-oborota}{обычаев делового оборота}~\cite{Obychay_del}.
\end{itemize}
RICS полагает, что~хорошей, но~необязательной бизнес-практикой является
ротация \emph{оценщиков} не~реже одного раза в~7~(семь) лет.\label{3.2.5.4.3-End}

\stepcounter{SubSubSecCounter}

\thesubsection.\theSubSubSecCounter.\label{3.2.5.4.4} Если размер
\textsl{оценочной компании} слишком мал для~того, чтобы проводить
ротацию подписантов \emph{отчётов} или~создать «комиссию по~\textsl{оценке}»,
возможно принятие иных мер, способствующих соблюдению требований данного
стандарта. Например, когда одно и~то~же~оценочное задание выполняется
на~регулярной основе, порядок проведения \textsl{оценки} должен перепроверяться
другим \textsl{членом RICS} с~периодичностью не~реже 1~(одного)
раза в~7~(семь) лет, что~продемонстрирует, что~\textsl{член RICS},
проводящий \textsl{оценку}, принимает меры для~сохранения объективности,
что, в~свою очередь, позволит сохранить доверие к~нему со~стороны
тех, кто~использует результаты \textsl{оценки}.\label{3.2.5.4.4-End}\label{subsubsec:3.2.5.4_Rotation_policy-End}

\subsubsection{Период пребывания в~статусе подписанта отчётов об~оценке\label{subsubsec:3.2.5.5_Time_as_signatory}}

\stepcounter{SubSubSecCounter}

\thesubsection.\theSubSubSecCounter.\label{3.2.5.5.1} Целью требований,
изложенных в~данной секции, является информирование любой \textsl{третьей
стороны} о~продолжительности периода, в~течение которого \textsl{член
RICS} непрерывно выступал в~роли подписанта \emph{отчётов об~оценке}
для~одной цели. Также \emph{раскрытию} подлежит информация, касающаяся
продолжительности периода выполнения \textsl{оценок} \textsl{оценочной
компанией}, в~которой работает \textsl{член RICS}, одного и~того~же
актива для~данного заказчика, а~также о~характере, степени и~продолжительности
их~сотрудничества.\label{3.2.5.5.1-End}

\stepcounter{SubSubSecCounter}

\thesubsection.\theSubSubSecCounter.\label{3.2.5.5.2} В~отношении
\textsl{члена RICS} \emph{раскрытие информации} осуществляется относительно
непрерывного периода ответственности за~проведение \textsl{оценки},
предшествующего дате подписания \emph{отчёта об~оценке}. Возможна
ситуация, при~которой \textsl{член RICS} ранее подписывал \emph{отчёты
об~оценке} для~той~же цели, но~в~связи с~\hyperref[subsubsec:3.2.5.4_Rotation_policy]{политикой ротации},
принятой в~\textsl{оценочной компании}, имел место период, в~течение
которого он~был освобождён от~данной обязанности. Нет~необходимости
раскрывать информацию о~таких предшествующих периодах.\label{3.2.5.5.2-End}

\stepcounter{SubSubSecCounter}

\thesubsection.\theSubSubSecCounter.\label{3.2.5.5.3} \textsl{Член
RICS} не~имеет обязанности предоставлять исчерпывающий отчёт обо~всех
работах, когда-либо выполненных его~\textsl{оценочной компанией}
для~данного заказчика. Простое лаконичное изложение, раскрывающее
характер ранее проделанной работы, а~также продолжительность взаимоотношений,
является достаточным.\label{3.2.5.5.3-End}

\stepcounter{SubSubSecCounter}

\thesubsection.\theSubSubSecCounter.\label{3.2.5.5.4} В~случае
отсутствия каких-либо связей между \textsl{членом RICS} и~заказчиком
вне~рамок оценочного задания, об~этом также следует сделать заявление.\label{3.2.5.5.4-End}\label{subsubsec:3.2.5.5_Time_as_signatory-End}

\subsubsection{Предшествующее сотрудничество\label{subsubsec:3.2.5.6_Previous_involvement}}

\stepcounter{SubSubSecCounter}

\thesubsection.\theSubSubSecCounter.\label{3.2.5.6.1} Целью требований,
изложенных в~данной секции, является выявление любого потенциального
\href{https://en.wikipedia.org/wiki/Conflict_of_interest}{конфликта интересов}~\cite{COI,RICS:Rule:Conflicts-of-interest}
в~тех~случаях, когда \textsl{член RICS} либо \textsl{оценочная компания},
в~которой он~работает, ранее проводили \textsl{оценку} актива для~тех~же
целей либо принимали участие в~его~приобретении на~стороне заказчика
в~течение 12~(двенадцати) месяцев, предшествующих \textsl{дате оценки},
либо иного срока и~условий, предписанных либо принятых в~национальной
либо территориальной юрисдикции.\label{3.2.5.6.1-End}

\stepcounter{SubSubSecCounter}

\thesubsection.\theSubSubSecCounter.\label{3.2.5.6.2} В~случаях,
когда \textsl{оценка} проводится с~целью её~включения в~публикуемый
документ, затрагивающий общественные интересы либо предназначенный
для~его~использования \textsl{третьей стороной}, \textsl{член RICS}
обязан раскрывать следующую информацию.
\begin{enumerate}
\item В случае проведения оценки актива, который ранее уже~оценивался \textsl{членом
RICS} либо его~\textsl{оценочной компанией} для~тех~же целей:
\begin{enumerate}
\item в~условиях \textsl{договора на~проведение оценки} приводятся сведения
о~\hyperref[subsubsec:3.2.5.4_Rotation_policy]{политике ротации}в~отношении
\emph{оценщиков}, ответственных за~проведение \textsl{оценки}, принятой
в~\textsl{оценочной компании};
\item в~\emph{отчёте об~оценке} и~публикуемых ссылках на~него приводятся
сведения о~продолжительности периода, в~течение которого \emph{оценщик}
непрерывно нёс ответственность за~подписание \emph{отчётов об~оценке},
выпускаемых для~того~же заказчика и~в~тех~же целях, что~и~текущий
\emph{отчёт}, а~также о~продолжительности периода, в течение которого
\textsl{оценочная компания}, в~которой работает \emph{оценщик}, непрерывно
принимала оценочные задания от~этого заказчика.
\end{enumerate}
\item Сведения о~характере, степени и~продолжительности любых отношений,
имевших место между \textsl{оценочной компанией} и~заказчиком независимо
от~их~целей.
\item В случае, если \emph{отчёт об~оценке} либо какая-либо опубликованная
ссылка на~него затрагивают актив~(активы), приобретённые заказчиком
в~течение периода, установленного согласно п.~\vref{3.2.5.6.1}--\pageref{3.2.5.6.1-End},
при~условии, что~\textsl{член RICS} либо его~\textsl{оценочная
компания} имеют отношение к~данному активу~(активам) путём:
\begin{enumerate}
\item получения предоплаты;
\item совершения действий от~имени заказчика при~совершении сделки с~ним~(ними),
\end{enumerate}
следует сделать соответствующее заявление, включая, когда это~уместно,
утверждение \emph{отчёта об~оценке}, осуществляемое в~порядке, установленном
в~секции~\vref{subsec:3.2.5.7_Proportion_of_fees}--\pageref{subsec:3.2.5.7_Proportion_of_fees-End}.\label{3.2.5.6.2-End}
\end{enumerate}
\stepcounter{SubSubSecCounter}

\thesubsection.\theSubSubSecCounter.\label{3.2.5.6.3} \href{https://www.isurv.com/info/1342/rics_national_or_jurisdictional_valuation_standards}{Стандартами оценки национальных ассоциаций RICS}~\cite{RICS:National-Standards}
равно как~и~стандартами оценки и~иными нормативными актами, принятыми
в~конкретной национальной юрисдикции, могут устанавливаться дополнительные
критерии, расширяющие требования данного стандарта.\label{3.2.5.6.3-End}

\stepcounter{SubSubSecCounter}

\thesubsection.\theSubSubSecCounter.\label{3.2.5.6.4} Раздел~\ref{sec:5.2_VPGA-2_Valuation_for_secure_lending}~\nameref{sec:5.2_VPGA-2_Valuation_for_secure_lending}
\vpageref{sec:5.2_VPGA-2_Valuation_for_secure_lending}--\pageref{sec:5.2_VPGA-2_Valuation_for_secure_lending-End}
содержит дополнительные либо модифицированные требования, применимые
при~проведении \textsl{оценки} для~целей залогового обеспечения.\label{3.2.5.6.4-End}\label{subsubsec:3.2.5.6_Previous_involvement-End}

\subsubsection{Доля выручки\label{subsec:3.2.5.7_Proportion_of_fees}}

\stepcounter{SubSubSecCounter}

\thesubsection.\theSubSubSecCounter.\label{3.2.5.7.1} \emph{Раскрытию}
подлежит \emph{информация} о~том, является~ли доля общей суммы платежей
заказчика за~предыдущий год относительно всей выручки \textsl{оценочной
компании} \textsl{члена RICS} за~тот~же период \emph{минимальной},
\emph{существенной} либо \emph{значительной}.\label{3.2.5.7.1-End}

\stepcounter{SubSubSecCounter}

\thesubsection.\theSubSubSecCounter.\label{3.2.5.7.2} Доля платежей
со~стороны одного заказчика в~размере менее 5\,\% считается \emph{минимальной},
5\,\%--25\,\% "--- \emph{существенной}, свыше 25\,\% "--- \emph{значительной}.\label{3.2.5.7.2-End}

\stepcounter{SubSubSecCounter}

\thesubsection.\theSubSubSecCounter.\label{3.2.5.7.3} \href{https://www.isurv.com/info/1342/rics_national_or_jurisdictional_valuation_standards}{Стандартами оценки национальных ассоциаций RICS}~\cite{RICS:National-Standards}
равно как~и~стандартами оценки и~иными нормативными актами, принятыми
в~конкретной национальной юрисдикции, могут устанавливаться дополнительные
критерии, расширяющие требования данного стандарта.\label{3.2.5.7.3-End}\label{subsec:3.2.5.7_Proportion_of_fees-End}

\subsubsection{Иные случаи раскрытия информации\label{subsubsec:3.2.5.8_Other_disclosures}}

\stepcounter{SubSubSecCounter}

\thesubsection.\theSubSubSecCounter.\label{3.2.5.8.1} Необходимо
убедиться в~том, что~помимо обязательного \emph{раскрытия} различной
\emph{информации}, установленного требованиями разделов~\ref{sec:4.1_VPS1_Terms_of_engagement_Scope_of_work}--\ref{sec:4.3_VPS3_Valuation_reports}
на~с.~\pageref{sec:4.1_VPS1_Terms_of_engagement_Scope_of_work}--\pageref{sec:4.3_VPS3_Valuation_reports-End},
также были сделаны и~другие \emph{раскрытия информации}, требуемые
для~конкретной \textsl{оценки} либо для~определённой цели. Требования
к~\emph{раскрытию информации}, которые могут потребовать предоставление
более конкретных сведений, касающихся цели \textsl{оценки}, затрагивающих
следующие аспекты:
\begin{itemize}
\item существенность участия;
\item статус \textsl{члена RICS};
\item специальные требования к~независимости;
\item знания и навыки \textsl{члена RICS};
\item объём исследований;
\item управление \href{https://en.wikipedia.org/wiki/Conflict_of_interest}{конфликтами интересов}~\cite{COI};
\item подходы к~\textsl{оценке};
\item \emph{раскрытие информации} по~требованию регулирующих органов, регламентирующих
цель оценки.\label{3.2.5.8.1-End}\label{3.2.5.8-End}\label{subsec:3.2.5_Disclosures_for_public_interest-End}
\end{itemize}

\subsection{Рецензирование отчётов об~оценке, выполненных другими оценщиками\label{subsec:3.2.6_Reviewing_another_valuation}}

\stepcounter{SubSecCounter}

\thesubsection.\theSubSecCounter.\label{3.2.6.1} Довольно распространена
практика, когда к~\emph{оценщику} обращаются для~того, чтобы он~провёл
проверку другой \textsl{оценки} либо её~части, выполненной иным \emph{оценщиком}
в~следующих ситуациях, перечень которых не~является исчерпывающим:
\begin{itemize}
\item содействие при рассмотрении \href{https://en.wikipedia.org/wiki/Risk_assessment}{оценки рисков}~\cite{Wiki:Risk_ass};
\item предоставление комментариев к~опубликованной \textsl{оценке}, например
в~ситуациях со~\href{https://www.investopedia.com/terms/t/takeover.asp}{сделками поглощения}~\cite{Takeover};
\item рецензирование \textsl{оценок}, подготовленных для~целей использования
в~судопроизводстве;
\item содействие при~проведении аудита.\label{3.2.6.1-End}
\end{itemize}
\stepcounter{SubSecCounter}

\thesubsection.\theSubSecCounter.\label{3.2.6.2} Важно проводить
чёткое разграничение между критическим рецензированием \textsl{оценки}
и~её~аудитом "--- с~одной стороны "--- и~отдельной новой независимой
\textsl{оценкой} имущества, актива либо обязательства, оформленной
в~\emph{отчёте} иного \emph{оценщика} "--- с~другой.\label{3.2.6.2-End}

\stepcounter{SubSecCounter}

\thesubsection.\theSubSecCounter.\label{3.2.6.3} Ожидается, что~\textsl{член
RICS} при~проведении рецензирования, ссылаясь на~\textsl{дату оценки},
а~также существовавшие в~этот момент факты и~обстоятельства, относящиеся
к~\emph{объекту оценки}, выполняет следующее:
\begin{itemize}
\item формирует своё мнение касательно того, является~ли надлежащим анализ,
проведённый в~исследуемой работе;
\item рассматривает вопрос об~убедительности суждений и~выводов, приведённых
в~ней;
\item рассматривает вопрос о~том, является~ли \emph{отчёт об~оценке}
в~целом приемлемым и~не~вводит~ли он~в~заблуждение.\label{3.2.6.3-End}
\end{itemize}
\stepcounter{SubSecCounter}

\thesubsection.\theSubSecCounter.\label{3.2.6.4} Рецензирование
должно быть выполнено с~учётом контекста требований, применявшихся
при~подготовке рассматриваемого \emph{отчёта об~оценке}, а~\textsl{член
RICS} обязан вырабатывать и~излагать мнения и~выводы, приводя при~этом
причины своего несогласия с~\emph{отчётом}, если таковое имеет место.\label{3.2.6.4-End}

\stepcounter{SubSecCounter}

\thesubsection.\theSubSecCounter.\label{3.2.6.5} \textsl{Член RICS}
может проводить критическое рецензирование \textsl{оценки}, выполненной
другим \emph{оценщиком}, предназначенное для~публикации или~иной
формы \emph{раскрытия информации}, только в~том~случае, если он~(\textsl{член
RICS}) располагает всеми данными и~информацией, на~которые опирался
\emph{оценщик}, выполнивший \emph{отчёт}.\label{3.2.6.5-End}\label{subsec:3.2.6_Reviewing_another_valuation-End}

\subsection{Условия договора на~проведение оценки (задания на~оценку\label{subsec:3.2.7_Terms_of_engagement_Scope_of_work})}

\stepcounter{SubSecCounter}

\thesubsection.\theSubSecCounter.\label{3.2.7.1} Согласно различным
требованиям, изложенным выше, для~обеспечения того, чтобы все~соответствующие
вопросы были освещены, крайне важно, чтобы к~моменту завершения работы
по~письменной \textsl{оценке}, но~до~выпуска \emph{отчёта}, все~детали,
имеющие для~заказчика существенное значение, были в~полном объёме
доведены до~его~сведения и~надлежащим образом задокументированы.
Это~необходимо для~того, чтобы текст \emph{отчёта об~оценке} не~содержал
таких отклонений от~начальных условий \textsl{договора на~проведение
оценки}, о~которых заказчик не~был осведомлён.\label{3.2.7.1-End}

\stepcounter{SubSecCounter}

\thesubsection.\theSubSecCounter.\label{3.2.7.2} \textsl{Члены RICS}
обязаны заботиться о~том, чтобы в~полной мере понимать требования
и~нужды своих заказчиков, а~также осознавать, что~иногда от~них~самих
потребуется направлять своих заказчиков в~вопросах выбора наиболее
подходящей консультации в~конкретных обстоятельствах.\label{3.2.7.2-End}

\stepcounter{SubSecCounter}

\thesubsection.\theSubSecCounter.\label{3.2.7.3} \hyperref[sec:4.1_VPS1_Terms_of_engagement_Scope_of_work]{Минимальные требования к~содержанию \textsl{договора на~проведение оценки}}
приводятся в~разделе~\ref{sec:4.1_VPS1_Terms_of_engagement_Scope_of_work}~\nameref{sec:4.1_VPS1_Terms_of_engagement_Scope_of_work}
\vpageref{sec:4.1_VPS1_Terms_of_engagement_Scope_of_work}--\pageref{sec:4.1_VPS1_Terms_of_engagement_Scope_of_work-End}.
В~случае, когда его~требования не~являются обязательными (например
в~случаях, перечисленных в~подразделе~\ref{subsec:3.1.5_VPS_exceptions}~\nameref{subsec:3.1.5_VPS_exceptions}
\vpageref{subsec:3.1.5_VPS_exceptions}--\pageref{subsec:3.1.5_VPS_exceptions-End},
соответствующие условия \textsl{договора на~проведение оценки}, тем~не~менее,
должны быть подготовлены в~соответствии со~спецификой конкретного
случая. RICS признаёт, что, с~учётом существенного разнообразия \emph{оценочной
деятельности} и~широты географии работы \textsl{членов RICS} в~области
\textsl{оценки} и~стоимостного консалтинга, условия \textsl{договора}
должны быть соотносимыми с~потребностями заказчика, однако во~всех
случаях \textsl{члены RICS} обязаны обеспечить доведение до~его~сведения
всех вопросов, имеющих существенное значение для~подготовки \emph{отчёта
об~оценке}.\label{3.2.7.3-End}

\stepcounter{SubSecCounter}

\thesubsection.\theSubSecCounter.\label{3.2.7.4} Поскольку спорные
ситуации могут возникнуть по~прошествии многих лет после завершения
проведения \textsl{оценки}, крайне важно, чтобы соглашение об~условиях
\textsl{договора на~проведение оценки} содержалось в~исчерпывающей
документации, хранимой в~приемлемой \href{https://neg.by/novosti/otkrytj/obychaj-delovogo-oborota}{обычаями делового оборота}~\cite{Obychay_del}
и~читаемой форме, либо прямо следовало из~его~содержания.\label{3.2.7.4-End}\label{subsec:3.2.7_Terms_of_engagement_Scope_of_work-End}

\subsection{Ответственность за~оценку\label{subsec:3.2.8_Responsibility_for_the_valuation}}

\stepcounter{SubSecCounter}

\thesubsection.\theSubSecCounter.\label{3.2.8.1} Во~избежание сомнений,
после должного рассмотрения вышеупомянутых вопросов, каждое оценочное
задание, к~которому применимы данные стандарты, должно быть выполнено
непосредственно конкретным персонифицированным \emph{оценщиком}, имеющим
должную квалификацию, либо под~его~надзором, при~этом в~обоих
случаях он~обязан принимать на~себя всю~полноту ответственности
за~выполненную \textsl{оценку}.\label{3.2.8.1-End}

\stepcounter{SubSecCounter}

\thesubsection.\theSubSecCounter.\label{3.2.8.2} В~тех случаях,
когда \textsl{оценка} была выполнена при~участии других \textsl{членов
RICS} или~\emph{оценщиков}, либо в~итоговый \emph{отчёт об~оценке}
включался отдельный \emph{отчёт}, посвящённый какому-либо её~аспекту,
ответственность за~итоговую \textsl{оценку} всё~равно несёт конкретный
\emph{оценщик}, указанный в~п.~\ref{3.2.8.1} выше, однако указание
на~других соисполнителей также необходимо, поскольку оно~обеспечивает
гарантию выполнения требований секции~\vref{subsubsec:4.3.2.1_Identification_and_status_of_the_valuer}~\pageref{subsubsec:4.3.2.8_Nature_and_source_of_the_information-End}.\label{3.2.8.2-End}

\stepcounter{SubSecCounter}

\thesubsection.\theSubSecCounter.\label{3.2.8.3} RICS не~допускает
проведение \emph{оценки} \textsl{оценочной компанией} как~таковой,
а~не~конкретным \emph{оценщиком} (даже, если это~допускается \href{https://www.rics.org/globalassets/rics-website/media/upholding-professional-standards/sector-standards/valuation/international-valuation-standards-rics2.pdf}{МСО}~\cite{IVS-2020}).
Однако использование формулировки \emph{«от~имени и~по~поручению»}
ниже подписи ответственного \emph{оценщика} является приемлемой заменой.\label{3.2.8.3-End}

\stepcounter{SubSecCounter}

\thesubsection.\theSubSecCounter.\label{3.2.8.4} \textsl{Членам
RICS} не~рекомендуется называть какую-либо \textsl{оценку} «официальной»
либо «неофициальной», поскольку использование таких терминов может
привести к~недопониманию, особенно в~отношении объёма проведённых
либо не~проведённых исследований и~(или) сделанных либо не~сделанных
\textsl{допущений} с~его~стороны.\label{3.2.8.4-End}

\stepcounter{SubSecCounter}

\thesubsection.\theSubSecCounter.\label{3.2.8.5} \textsl{Членам
RICS} в~обязательном порядке следует проявлять большую осторожность
прежде чем~разрешать использование результатов \textsl{оценки} для~целей
отличных от~изначально согласованных. Существует вероятность, что~пользователь
или~читатель \emph{отчёта об~оценке} не~в~полной мере воспримут
пределы применимости \textsl{оценки} и~изложенные в~\emph{отчёте}~ограничительные
условия, а~также то, что~его~содержание может быть искажено вне~изначального
контекста. Кроме того, в~таких случаях, может возникнуть \href{https://en.wikipedia.org/wiki/Conflict_of_interest}{конфликт интересов}~\cite{COI,RICS:Rule:Conflicts-of-interest},
который не~мог~бы иметь места в~случае использования \emph{отчёта
об~оценке} в~соответствии с~его~изначальным назначением. Вследствие
этого крайне важно, чтобы условия \textsl{договора на~проведение
оценки} и~текст \emph{отчёта об~оценке} учитывали данный риск надлежащим
образом. В~этой связи также следует учитывать положения подраздела~\ref{subsec:3.2.4_Advisers_strict_separation}~\nameref{subsec:3.2.4_Advisers_strict_separation}
\vpageref{subsec:3.2.4_Advisers_strict_separation}--\pageref{subsec:3.2.4_Advisers_strict_separation-End}.\label{subsec:3.2.8_Responsibility_for_the_valuation-End}\label{sec:3.2_PS2_Ethics_competency_objectivity-End}\label{chap:3_PS-End}

\newpage

\chapter{Стандарты проведения оценки и~подготовки отчёта об~оценке (СПО)\label{chap:4_VPS}}

Как~было сказано ранее в~п.~\hyperref[1.3.13]{13} и~\hyperref[1.3.14]{14}
Главы~\ref{chap:1_Introduction}~\nameref{chap:1_Introduction}~\vpageref{1.2.3.2.1}--\pageref{1.2.3.2.2-End},
всемирные стандарты проведения \textsl{оценки} и~подготовки \emph{отчёта
об~оценке}, которым обязаны следовать \textsl{члены RICS}, изложены
в~СПО 1--5, приведённых ниже. В~то~время как~СПО 1,~4~и~5
в~большей степени сосредоточены на~вопросах проведения \textsl{оценки},
а~СПО~2~и~3 преимущественно затрагивают вопросы изложения и~подготовки
\emph{отчётов об~оценке}, дальнейшая их~спецификация является нецелесообразной.
Их~текущий структура повторяет структуру \href{https://www.rics.org/globalassets/rics-website/media/upholding-professional-standards/sector-standards/valuation/international-valuation-standards-rics2.pdf}{Международных стандартов оценки}~\cite{IVS-2020},
принимаемых и~внедряемых посредством СПО. Следует обращать внимание
на~предисловие к~каждому СПО.

\newpage

\section{СПО 1. Условия договора на~проведение оценки (Задания на\_оценку)\label{sec:4.1_VPS1_Terms_of_engagement_Scope_of_work}}

Данный обязательный стандарт:
\begin{itemize}
\item имплементирует \hyperref[sec:9.1_IVS-101_Scope_of_work]{Международный стандарт оценки 101 \textbf{Задание на~оценку}};
\item устанавливает дополнительные обязательные для~\textsl{членов RICS}
требования, разработанные с~целью:
\begin{itemize}
\item улучшения понимания заказчиком сути оказываемой ему~услуги, в~т.\,ч.~\emph{вида
стоимости}, определяемой в~\emph{отчёте об~оценке};
\item обеспечения уверенности в~том, что~работа, выполняемая \textsl{членами
RICS}, соответствует высоким профессиональным стандартам, подкреплённым
эффективным регулированием;
\item рассмотрения отдельных аспектов внедрения \href{https://www.rics.org/globalassets/rics-website/media/upholding-professional-standards/sector-standards/valuation/international-valuation-standards-rics2.pdf}{МСО}~\cite{IVS-2020},
которые могут возникнуть в~конкретных случаях.
\end{itemize}
\end{itemize}

\subsection{Основные принципы\label{subsec:4.1.1_General_principles}}

\stepcounter{SubSecCounter}

\thesubsection.\theSubSecCounter.\label{4.1.1.1} Как~правило, условия
\textsl{договора на~проведение оценки} устанавливаются заказчиком
и~\emph{оценщиком} при~первичном направлении и~принятии оценочного
задания соответственно. Однако, нельзя не~признать, что~оценочное
задание может включать в~себя \textsl{оценку} как~отдельного актива,
так~и~крупного портфеля активов, поэтому степень того, насколько
все~основные условия \textsl{договора на~проведение оценки} могут
быть согласованы на~начальном этапе, может варьироваться.\label{4.1.1.1-End}

\stepcounter{SubSecCounter}

\thesubsection.\theSubSecCounter.\label{4.1.1.2}\emph{ Оценщикам}
следует обеспечить полное понимание со~~своей стороны потребностей
и~требований заказчиков, а~также осознавать, что~в~ряде случаев
от~них~может потребоваться направлять заказчиков в~вопросе выбора
наиболее подходящей в~конкретных обстоятельствах оценочной консультации.\label{4.1.1.2-End}

\stepcounter{SubSecCounter}

\thesubsection.\theSubSecCounter.\label{4.1.1.3} В~целом, условия
\textsl{договора на~проведение оценки} должны обеспечивать чёткое
понимание требований к~результатам \textsl{оценки} и~процессу~её
проведения, а~сами они~должны быть сформулированы таким образом
и~с~использование таких терминов, при~которых они~могут быть прочитаны
и~поняты лицом, не~обладающим предварительными знаниями ни~об~оцениваемом
активе, ни~о~процессе \textsl{оценки}.\label{4.1.1.3-End}

\stepcounter{SubSecCounter}

\thesubsection.\theSubSecCounter.\label{4.1.1.4} Формат и~детали
предполагаемого \emph{отчёта об~оценке} являются предметом соглашения
между \emph{оценщиком} и~заказчиком, фиксируемым в~письменной форме
в~условиях \textsl{договора на~проведение оценки}. Они~во~всех
случаях должны быть соразмерны задаче и, как~и~сама \textsl{оценка}
в~целом, соответствовать её~целям с~профессиональной точки зрения.
Для~большей ясности, стандарты, которые должны соблюдаться при~составлении
\emph{отчёта об~оценке}, изложены в~Разделе~\ref{sec:4.3_VPS3_Valuation_reports}~\nameref{sec:4.3_VPS3_Valuation_reports}
\vpageref{sec:4.3_VPS3_Valuation_reports}--\pageref{sec:4.3_VPS3_Valuation_reports-End}.
В~целом они~отражают требования, изложенные в~данном разделе, однако
имеют дополнительные детали.\label{4.1.1.4-End}

\stepcounter{SubSecCounter}

\thesubsection.\theSubSecCounter.\label{4.1.1.5} Всякий раз, когда
\emph{оценщик} либо заказчик определяют, что~\textsl{оценка}, возможно,
должна отражать существующие либо предполагаемые рыночные ограничения,
детали таких ограничений подлежат обязательному согласованию и~включению
в~условия \textsl{договора на~проведение оценки}. Запрещается использовать
термин «стоимость при~вынужденной продаже» (см.~подраздел~\ref{subsec:4.4.9_Reflection_of_market_constraint}~\nameref{subsec:4.4.9_Reflection_of_market_constraint}
\vpageref{subsec:4.4.9_Reflection_of_market_constraint}--\pageref{subsec:4.4.9_Reflection_of_market_constraint-End}).\label{4.1.1.5-End}

\stepcounter{SubSecCounter}

\thesubsection.\theSubSecCounter.\label{4.1.1.6} К~моменту завершения
проведения \textsl{оценки}, но~до~выпуска \emph{отчёта об~оценке}
все~соответствующие вопросы должны быть доведены до~сведения заказчика
и~надлежащим образом задокументированы. Это~необходимо для~того,
чтобы \emph{отчёт об~оценке} не~содержал никаких отступлений от~первоначальных
условий \textsl{договора на~проведение оценки}, о~которых не~было~бы
известно заказчику.\label{4.1.1.6-End}\label{subsec:4.1.1_General_principles-End}

\subsection{Формат договора на~проведение оценки\label{subsec:4.1.2_Terms_of_engagement_format}}

\stepcounter{SubSecCounter}

\thesubsection.\theSubSecCounter.\label{4.1.2.1}\textsl{ Оценочные
компании} могут иметь типовую форму \textsl{договора на~проведения
оценки} либо его постоянные условия, которые могут включать часть
минимальных требований, предусмотренных данными всемирными стандартами.
Возможно, \emph{оценщику} потребуется внести изменения в~данную форму
для~того, чтобы учесть вопросы, которые будут прояснены позднее.\label{4.1.2.1-End}

\stepcounter{SubSecCounter}

\thesubsection.\theSubSecCounter.\label{4.1.2.2} Хотя точная форма
\textsl{договора на проведение оценки} может варьироваться "--- например
некоторые «внутренние \textsl{оценки}» могут проводится на~основании
существующих в~организации инструкций либо иных внутренних политик
и~процедур "--- \emph{оценщики} обязаны иметь заготовленные письменные
шаблоны \textsl{договоров на~проведение оценки} для~всех видов работ
по~\textsl{оценке}. Нельзя переоценить потенциальные риски, которые
могут возникнуть в~случае возникновения в~дальнейшем вопросов к~\textsl{оценке}
при~условии, что~параметры оценочного задания были недостаточно
задокументированы.\label{4.1.2.2-End}\label{subsec:4.1.2_Terms_of_engagement_format-End}

\subsection{Условия договора на~проведения оценки (Задания на~оценку)\label{subsec:4.1.3_Terms_of_engagement_scope_of_work}}

\stepcounter{SubSecCounter}

\thesubsection.\theSubSecCounter.\label{4.1.3.1} Условия \textsl{договора
на~проведение оценки} должны охватывать следующие вопросы:
\begin{enumerate}
\item сведения об~\emph{оценщике} и~его~статусе;
\item сведения о~заказчике \textsl{оценки};
\item сведения о~других предполагаемых пользователях \emph{отчёта об~оценке};
\item сведения об~оцениваемом активе либо обязательстве;
\item валюта \textsl{оценки};
\item цель \textsl{оценки};
\item \textsl{вид(ы) определяемой стоимости};
\item \textsl{дата оценки};
\item характер и~объём работы \emph{оценщика}, в~т.\,ч. исследований,
а~также любые их~ограничения;
\item характер и~источники(и) информации, использованной \emph{оценщиком};
\item принимаемые \textsl{допущения}, в~т.\,ч. специальные;
\item формат \emph{отчёта об~оценке};
\item ограничения на~использование, распространение и~публикацию \emph{отчёта
об~оценке};
\item подтверждение того, что~\textsl{оценка} будет проводиться в~соответствии
с~требованиями \href{https://www.rics.org/globalassets/rics-website/media/upholding-professional-standards/sector-standards/valuation/international-valuation-standards-rics2.pdf}{МСО}~\cite{IVS-2020};
\item обоснование размера вознаграждения за~проведение \textsl{оценки};
\item в~случае, если \emph{оценочная компания} \href{https://www.rics.org/eu/upholding-professional-standards/regulation/}{зарегистрирована для регулирования со стороны RICS}~\cite{RICS:Regulation},
ссылка на~процедуру рассмотрения жалоб, предполагающую получение
копии по~запросу;
\item заявление о~том, что~соблюдение данных всемирных стандартов, может
быть объектом контроля в~соответствии с~\href{https://www.rics.org/ssa/upholding-professional-standards/standards-of-conduct/rules-of-conduct/}{Правилами поведения и дисциплинарными нормами RICS}~\cite{RICS:Conduct},
разработанными как~для \href{https://www.rics.org/globalassets/rics-website/media/upholding-professional-standards/regulation/rules-of-conduct-for-members_2020.pdf}{субъектов оценочной деятельности "--- оценщиков физических лиц}~\cite{RICS:Conduct-Members},
так~и~для~\href{https://www.rics.org/globalassets/rics-website/media/upholding-professional-standards/regulation/rules-of-conduct-for-firms_2020.pdf}{оценочных компаний}~\cite{RICS:Conduct-Firms};
\item положение, описывающее все~согласованные ограничения ответственности.\label{4.1.3.1-End}
\end{enumerate}
\stepcounter{SubSecCounter}

\thesubsection.\theSubSecCounter.\label{4.1.3.2} Каждый пункт далее
рассмотрен более подробно. Ключевые принципы выделены \textbf{полужирным
начертанием}. Сопутствующий текст разъясняет как~именно эти~принципы
должны толковаться и~реализовываться в~отдельных случаях.\label{4.1.3.2-End}

\subsubsection{Сведения об~оценщике и~его~статусе\label{subsubsec:4.1.3.1_Identification_and_status_of_the_valuer}}

\textbf{Включает положения, подтверждающие что:}
\begin{enumerate}
\item \textbf{ответственность за~}\textbf{\textsl{оценку}}\textbf{ несёт
конкретный именованный }\textbf{\emph{оценщик}}\textbf{: RICS не~допускает
проведение }\textbf{\textsl{оценки}}\textbf{ }\textbf{\textsl{оценочной
компанией}}\textbf{ как~таковой;}
\item \textbf{\emph{оценщик}}\textbf{ в~состоянии обеспечить проведение
объективной и~беспристрастной }\textbf{\textsl{оценки}}\textbf{;}
\item \textbf{\emph{оценщик}}\textbf{ имеет либо не~имеет существенной
заинтересованности касательно }\textbf{\emph{объекта оценки}}\textbf{
либо иных аспектов оценочного задания, в случае наличия каких-либо
иных факторов, способных ограничить возможность }\textbf{\emph{оценщика}}\textbf{
проводить беспристрастную и~независимую }\textbf{\textsl{оценку}}\textbf{,
такие факторы подлежат обязательному раскрытию;}
\item \textbf{\emph{оценщик}}\textbf{ достаточно компетентен для~того,
чтобы выполнить оценочное задание, в~случае, если }\textbf{\emph{оценщику}}\textbf{
в~связи с~отдельными аспектами оценочного задания необходимо обратиться
за~существенной помощью к~другим специалистам, характер такого содействия,
а~также степень доверия к~его~результатам должны быть ясно описаны,
согласованы и~зафиксированы.}
\end{enumerate}
\textbf{Реализация.}

\stepcounter{SubSubSecCounter}

\thesubsubsection.\theSubSubSecCounter.\label{4.1.3.1.1} Использование
формулировки \emph{«от~имени и~по~поручению }\textsl{\emph{оценочной
компании}}\emph{»} является приемлемой заменой для~лица, ответственного
за~подписание \emph{отчёта об~оценке}. В~том~случае, если \textsl{оценка}
проводилась \textsl{членом RICS} под~надзором достаточно квалифицированного
\emph{оценщика}, при~этом, последний обязан обеспечить и~проверить
выполнение \textsl{оценки} на~соблюдение минимальных стандартов,
так, как~если~бы он~единолично выполнял эту~\textsl{оценку} и~был
ответственен за~неё.\label{4.1.3.1.1-End}

\stepcounter{SubSubSecCounter}

\thesubsubsection.\theSubSubSecCounter.\label{4.1.3.1.2} Для~некоторых
целей от~\emph{оценщика} может потребоваться указание того, действует~ли
он~(она) в~качестве внутреннего или~внешнего \emph{оценщика}. В~том~случае,
если \emph{оценщик} обязан соблюдать дополнительные требования в~отношении
своей независимости, следует применять положения секции~\ref{subsec:3.2.3_Independence_objectivity_confidentiality}
\nameref{subsec:3.2.3_Independence_objectivity_confidentiality} \vpageref{subsec:3.2.3_Independence_objectivity_confidentiality}--\pageref{subsec:3.2.3_Independence_objectivity_confidentiality-End}.\label{4.1.3.1.2-End}

\stepcounter{SubSubSecCounter}

\thesubsubsection.\theSubSubSecCounter.\label{4.1.3.1.3} При~рассмотрении
степени любой возможной существенной заинтересованности, существовавшей
в~прошлом, существующей в~ настоящем либо потенциально возможной
в~будущем, \emph{оценщик} обязан указать на~наличие такой заинтересованности
в~\emph{договоре на~проведение оценки}. В~тех случаях, когда в~прошлом
не~было какого-либо существенного участия, заявление об~этом должно
быть сделано в~условиях \emph{договора на~проведение оценки}, а~также
в~\emph{отчёте об~оценке} (см.~п.~\vref{4.3.2.1.4}--\pageref{4.3.2.1.4-End}).
Расширенные указания по~обеспечению независимости и~объективности
приведены в~подразделе~\ref{sec:3.2_PS2_Ethics_competency_objectivity}~\nameref{sec:3.2_PS2_Ethics_competency_objectivity}
\vpageref{sec:3.2_PS2_Ethics_competency_objectivity}--\pageref{sec:3.2_PS2_Ethics_competency_objectivity-End}.\label{4.1.3.1.3-End}

\stepcounter{SubSubSecCounter}

\thesubsubsection.\theSubSubSecCounter.\label{4.1.3.1.4} В части,
касающейся вопросов компетенции \emph{оценщика}, заявление может быть
ограничено подтверждением того, что~\emph{оценщик} обладает достаточными
знаниями местного, национального и~международного (если это~необходимо)
рынка, а~также пониманием того, как~компетентно выполнить оценку
и~достаточно развитыми навыками для~этого. Нет~необходимости приводить
какие-либо детали в~этой части. Там, где~применяются положения секции~\ref{subsec:3.2.3_Independence_objectivity_confidentiality},
следует делать соответствующее \emph{раскрытие информации}.\label{4.1.3.1.4-End}\label{subsubsec:4.1.3.1_Identification_and_status_of_the_valuer-End}

\subsubsection{Сведения о~Заказчике\label{subsubsec:4.1.3.2_Identification_of_the_client}}

\textbf{Установление тех, для~кого подготавливается }\textbf{\emph{задание
на~оценку}}\textbf{, является важным с~точки зрения определения
формы и~содержания }\textbf{\emph{отчёта об~оценке}}\textbf{, который
должен удовлетворять их~потребностям. Любые ограничения круга пользователей
}\textbf{\emph{отчёта}}\textbf{, которые могут полагаться на~него,
должны быть согласованы с~заказчиком и~зафиксированы в~письменной
форме.}

\textbf{Реализация.}

\stepcounter{SubSubSecCounter}

\thesubsubsection.\theSubSubSecCounter.\label{4.1.3.2.1}Запросы
на~проведение \textsl{оценки} часто поступают от~представителей
заказчика, в~таких случаях \emph{оценщик} должен убедиться в~том,
что~сам заказчик правильно идентифицирован. Особенно актуально это~в~следующих
ситуациях:
\begin{enumerate}
\item запрос на~проведение \textsl{оценки} сделан исполнительным органом
юридического лица, но~заказчиком является оно~само, и~при~этом
её~исполнительный орган имеет отдельный юридический статус;
\item проведение \textsl{оценки} необходимо для~целей кредитования, и,
хотя она~проводится по~поручению заёмщика или~организации, действующей
от~имени кредитора (например управляющей компании), \emph{отчёт}
может быть составлен, например, для~кредитора, его~дочерних структур,
участников консолидированных групп, таким образом, крайне важно определить
истинного заказчика;
\item \textsl{оценка} требуется для~целей управления имуществом либо для~отчётности
о~доходах, связанных с~ним, и, несмотря на~то, что~она~может
быть заказана финансовым консультантом или~доверенным лицом, \emph{отчёт}
предназначен для~собственника этого имущества, являющегося истинным
заказчиком.\label{4.1.3.2.1-End}\label{subsubsec:4.1.3.2_Identification_of_the_client-End}
\end{enumerate}

\subsubsection{Сведения об~иных предполагаемых пользователях результатов оценки\label{subsubsec:4.1.3.3_Identification_of_other_intended_users}}

\textbf{Важно понимать, существуют~ли иные пользователи }\textbf{\emph{отчёта
об~оценке}}\textbf{, установить их~данные и~потребности, что~позволит
убедиться в~том, что~содержание и~формат }\textbf{\emph{отчёта}}\textbf{
удовлетворяют потребностям таких пользователей.}

\textbf{Реализация.}

\stepcounter{SubSubSecCounter}

\thesubsubsection.\theSubSubSecCounter.\label{4.1.3.3.1}\emph{ Оценщик}
должен указать, могут~ли другие стороны, помимо заказчика, полагаться
на~результаты~\textsl{оценки}.\label{4.1.3.3.1-End}

\stepcounter{SubSubSecCounter}

\thesubsubsection.\theSubSubSecCounter.\label{4.1.3.3.2} Во~многих
случаях единственным лицом, полагающимся на~\textsl{оценку}, будет
заказчик. Согласие на~использование \textsl{оценки} \textsl{третьими
лицами} существенно повышает риски для~\emph{оценщика}.\label{4.1.3.3.2-End}

\stepcounter{SubSubSecCounter}

\thesubsubsection.\theSubSubSecCounter.\label{4.1.3.3.3} По~умолчанию
\emph{оценщику} следует указывать в~\textsl{договоре на~проведение
оценки}, что~\textsl{он~}не~разрешает использовать \emph{отчёт
об~оценке} какой-либо \textsl{третьей стороне}. Любое разрешение
на~использование \textsl{оценки} \textsl{третьей стороной} должно
быть тщательно обдумано, а~условия разрешения такого использования
"--- зафиксированы в~письменной форме. Особое внимание следует уделить
тому, чтобы \emph{оценщик} не~подвергался случайному риску со~стороны
\textsl{третьих лиц}, для~чего следует указывать на~необходимость
проявления должной осмотрительности с~их~стороны при~использовании
\textsl{оценки}, а~также применять к~ним соответствующие условия
использования \emph{отчёта об~оценке}, например такие как~ограничение
ответственности \emph{оценщика}. В~таких ситуациях \emph{оценщикам}
следует рассмотреть вопрос получения консультации юриста.\label{4.1.3.3.3-End}

\stepcounter{SubSubSecCounter}

\thesubsubsection.\theSubSubSecCounter.\label{4.1.3.3.4}\emph{ Оценщики}
должны проявлять осторожность при~рассмотрении вопроса о~разрешении
уступки права по~\textsl{договору на~проведение оценки} (замены
стороны договора) (не~путать с~правом использования \textsl{оценки}
\textsl{третьей стороной}), поскольку это~может подвергнуть \emph{оценщиков
}дополнительным рискам. \emph{Оценщикам} следует убедиться в~том,
что~лимит страхового покрытия их~профессиональной ответственности
предусматривает необходимое покрытие в~случаях, когда такая замена
стороны договора была разрешена.\label{4.1.3.3.4-End}\label{subsubsec:4.1.3.3_Identification_of_other_intended_users-End}

\subsubsection{Сведения об~оцениваемом активе (активах) или~обязательстве (обязательствах)\label{subsubsec:4.1.3.4_Identification_of_the asset_or_liability}}

\textbf{\emph{Объект оценки}}\textbf{ "--- актив или~обязательство
"--- должен быть чётко идентифицирован в~}\textbf{\textsl{договоре
на~проведение оценки}}\textbf{, при~этом следует проводить разграничение
между активом (обязательством), конкретным правом на~них и~правом
пользования ими.}

\textbf{В~случае проведения }\textbf{\textsl{оценки}}\textbf{ в~отношении
актива или~обязательства, используемых в~сочетании с~другими активами
или~обязательствами, необходимо установить в~их~отношении принадлежность
к~одному из~следующих статусов:}
\begin{itemize}
\item \textbf{включены в~периметр }\textbf{\textsl{оценки}}\textbf{ в~рамках
}\textbf{\textsl{договора на~проведение оценки}}\textbf{;}
\item \textbf{исключены из~периметра }\textbf{\textsl{оценки}}\textbf{,
но~предполагается их~доступность;}
\item \textbf{исключены из~периметра }\textbf{\textsl{оценки}}\textbf{,
и~предполагается их~недоступность.}
\end{itemize}
\textbf{В~случае проведения }\textbf{\textsl{оценки}}\textbf{ в~отношении
доли в~активе или~обязательстве, необходимо уточнить взаимосвязь
оцениваемой доли со~всеми другими долями в~активе~(обязательстве),
а~также взаимные права и~обязанности, существующие между всеми владельцами
долей в~них.}

\textbf{Необходимо уделять отдельное внимание вопросам выявления портфелей,
коллекций и~иных групп имущества. Важно рассмотреть вопросы классификации
и~группировки, идентифицировать различные категории имущества или~активов,
разработать систему }\textbf{\textsl{допущений}}\textbf{ и~}\textbf{\textsl{специальных
допущений}}\textbf{, относящихся к~обстоятельствам, при~которых
имущество, активы, обязательства или~их~группы могут быть выставлены
на~рынке.}

\textbf{Реализация.}

\stepcounter{SubSubSecCounter}

\thesubsubsection.\theSubSubSecCounter.\label{4.1.3.4.1} В~отношении
каждого актива или~обязательства необходимо привести описание существующих
на~них~прав. Необходимо проводить разграничение и~ понимать различие
между \emph{активом как~таковым} и~оцениваемым \emph{правом (долей
в~праве) на~этот актив}.\label{4.1.3.4.1-End}

\stepcounter{SubSubSecCounter}

\thesubsubsection.\theSubSubSecCounter.\label{4.1.3.4.2} При \textsl{оценке}
прав на~недвижимое имущество, сдаваемое в~аренду, может возникнуть
потребность в~выявлении улучшений, созданных арендатором, а~также
в~установлении того, должны~ли эти~улучшения приниматься во~внимание
при~продлении или~пересмотре условий договора аренды, а~также в~установлении
возможности требования компенсации со~стороны арендатора при~освобождении
им~данного объекта.\label{4.1.3.4.2-End}

\stepcounter{SubSubSecCounter}

\thesubsubsection.\theSubSubSecCounter.\label{4.1.3.4.3} При~\textsl{оценке}
доли в~праве на~недвижимое имущество, \emph{оценщику} также следует
установить степень контроля, которую предоставляет оцениваемый размер
доли, и~прав\'{а}, которыми обладают другие владельцы долей, обременяющие
оборотоспособность оцениваемой доли (в~качестве примера можно привести
право преимущественной покупки, возникающее при~продаже такого актива).\label{4.1.3.4.3-End}

\stepcounter{SubSubSecCounter}

\thesubsubsection.\theSubSubSecCounter.\label{4.1.3.4.4} При~наличии
сомнений относительно того, что~представляет собой отдельный актив
или~имущество, \emph{оценщику} при~проведении \textsl{оценки} необходимо
сформировать «лот на~продажу» или~провести группировку имущества
таким образом, чтобы он~максимально близко соответствовал ситуации
реальной продажи доли. Однако, \emph{оценщику} следует всегда обсуждать
подобные варианты с~заказчиком, в~обязательном порядке фиксировать
их~в~\textsl{договоре на~проведение оценки} и~лишь затем применять
в~\emph{отчёте об~оценке}.\label{4.1.3.4.4-End}

\stepcounter{SubSubSecCounter}

\thesubsubsection.\theSubSubSecCounter.\label{4.1.3.4.5} Дополнительные
руководства по~\textsl{оценке} портфелей, коллекций и~групп активов,
включая рекомендации по~содержанию соответствующих \emph{отчётов
об~оценке}, приводятся в~разделе~\ref{sec:5.9_VPGA-9_Identification_of_portfolios}~\nameref{sec:5.9_VPGA-9_Identification_of_portfolios}
\vpageref{sec:5.9_VPGA-9_Identification_of_portfolios}--\pageref{sec:5.9_VPGA-9_Identification_of_portfolios-End}.\label{4.1.3.4.5-End}

\stepcounter{SubSubSecCounter}

\thesubsubsection.\theSubSubSecCounter.\label{4.1.3.4.6} В~части,
касающейся \textsl{оценки} нефинансовых обязательств, см.~раздел~\ref{sec:10.3_IVS-220_Non-Financial_Liabilities}~\nameref{sec:10.3_IVS-220_Non-Financial_Liabilities}
\vpageref{sec:10.3_IVS-220_Non-Financial_Liabilities}--\pageref{sec:10.3_IVS-220_Non-Financial_Liabilities-End}.\label{4.1.3.4.6-End}

\subsubsection{Валюта оценки\label{subsubsec:4.1.3.5_Valuation_currency}}

\textbf{Необходимо указывать валюту, в~которой должна быть номинирована
стоимость актива или~обязательства.}

\textbf{Данное требование является особенно важным в~части оценочных
заданий, касающихся активов~(обязательств), существующих более чем~в~одной
юрисдикции и~(или) денежных потоков, номинированных в~различных
валютах.}

\textbf{Реализация.}

\stepcounter{SubSubSecCounter}

\thesubsubsection.\theSubSubSecCounter.\label{4.1.3.5.1} Если результаты
\textsl{оценки} необходимо привести в~валюте отличной от~валюты
страны, в~которой расположен оцениваемый актив, порядок определения
обменного курса подлежит обязательному согласованию.\label{4.1.3.5.1-End}\label{subsubsec:4.1.3.5_Valuation_currency-End}

\subsubsection{Цель оценки\label{subsubsec:4.1.3.6_Purpose_of_the_valuation}}

\textbf{Цель проведения }\textbf{\textsl{оценки}}\textbf{ должна быть
чётко определена и~указана, поскольку важно, чтобы результаты }\textbf{\textsl{оценки}}\textbf{
не~использовались вне контекста или~для~целей, для~которых они~не~предназначены.}

\textbf{Цель оценки также, как правило, оказывает решающее влияние
на~выбор вида~(видов) определяемой стоимости.}

\textbf{Реализация.}

\stepcounter{SubSubSecCounter}

\thesubsubsection.\theSubSubSecCounter.\label{4.1.3.6.1}\emph{ Оценщикам}
следует понимать, что~в~тех~случаях, когда заказчик отказывается
раскрывать цели \textsl{оценки}, могут возникнуть затруднения с~выполнением
всех требований настоящих всемирных стандартов. Если \emph{оценщик}
готов выполнить такую \textsl{оценку}, ему~необходимо в~письменной
форме уведомить заказчика о~том, что~данный пробел будет упомянут
в~\emph{отчёте об~оценке}. В~этом случае \emph{отчёт об~оценке}
не~может быть публично обнародован либо предоставлен какой-либо \textit{третьей
стороне}.\label{4.1.3.6.1-End}

\stepcounter{SubSubSecCounter}

\thesubsubsection.\theSubSubSecCounter.\label{4.1.3.6.2} В~случае
проведения \textsl{оценки} для~нестандартных целей, условия \textsl{договора
на~проведение оценки} должны содержать требование о~том, что~результаты
такой \textsl{оценки} не~могут использоваться для~целей отличных
от~изначально согласованных с~заказчиком.\label{4.1.3.6.2-End}\label{subsubsec:4.1.3.6_Purpose_of_the_valuation-End}

\subsubsection{Вид~(виды) определяемой стоимости\label{subsec:4.1.3.7_Basis_of_value}}

\textsl{Вид определяемой стоимости} должен соответствовать цели \textsl{оценки}.
Необходимо приводить ссылку на~источник определения понятия используемого
\textsl{вида стоимости} либо приводить его~описание в~тексте \emph{отчёта
об~оценке}. Данное требование не~применяется к~рецензированию \textsl{оценки},
по~результатам которого не~требуется формирование мнения о~стоимости,
и~перед рецензентом не~стоит задача выносить суждение относительно
корректности выбранного \textsl{вида стоимости}.

\textbf{Реализация.}

\stepcounter{SubSubSecCounter}

\thesubsubsection.\theSubSubSecCounter.\label{4.1.3.7.1} В~тех~случаях,
когда \textsl{вид определяемой стоимости} прямо определён в~настоящих
всемирных стандартах (включая \textsl{виды стоимости}, установленные
\href{https://www.rics.org/globalassets/rics-website/media/upholding-professional-standards/sector-standards/valuation/international-valuation-standards-rics2.pdf}{Международными стандартами оценки}~\cite{IVS-2020}),
текст определения должен быть полностью приведён в~\emph{отчёте об~оценке}.
Если определение дополняется подробной концептуальной основой или~другим
дополнительным материалом, нет~необходимости в~их~воспроизведении
в~тексте \emph{отчёта об~оценке}. Однако, если \emph{оценщик} сочтёт,
что~это~может помочь заказчику более полно понять принципы, лежащие
в~основе выбора соответствующего \textsl{вида стоимости}, он~может
привести такие дополнения по~своему усмотрению.\label{4.1.3.7.1-End}

\stepcounter{SubSubSecCounter}

\thesubsubsection.\theSubSubSecCounter.\label{4.1.3.7.2} В~ряде
случаев проведения \textsl{оценки}, например для~целей \textsl{финансовой
отчётности}, составляемой согласно \href{http://docs.cntd.ru/document/420332842/}{МСФО}~\cite{MSFO-all}~(\href{https://www.ifrs.org/issued-standards/list-of-standards/}{IFRS}~\cite{IFRS-all})
либо в~условиях проведения \textsl{оценки} в~отдельных юрисдикциях,
предъявляющих специфические требования, может быть оговорено использование
конкретного \textsl{вида стоимости}. Во~всех остальных случаях вопрос
выбора подходящего \textsl{вида стоимости} является вопросом профессионального
суждения \emph{оценщика}.\label{4.1.3.7.2-End}

\stepcounter{SubSubSecCounter}

\thesubsubsection.\theSubSubSecCounter.\label{4.1.3.7.3} RICS признаёт,
что~для~некоторых целей помимо определения текущей стоимости также
может потребоваться определение прогнозной стоимости. Такие прогнозные
значения должны соответствовать применимым требованиям страновых стандартов,
а~также \href{https://www.isurv.com/info/1342/rics_national_or_jurisdictional_valuation_standards}{стандартам национальных ассоциаций RICS}~\cite{RICS:National-Standards}
(см.~раздел~\ref{sec:4.4_VPS4_Bases_of_value}~\nameref{sec:4.4_VPS4_Bases_of_value}
\vpageref{sec:4.4_VPS4_Bases_of_value}--\pageref{sec:4.4_VPS4_Bases_of_value-End}).\label{4.1.3.7.3-End}\label{subsec:4.1.3.7_Basis_of_value-End}

\subsubsection{Дата оценки\label{subsubsec:4.1.3.8_Valuation_date}}

\textbf{\textsl{Дата оценки~(дата проведения оценки)}}\textbf{ может
отличаться от~даты выпуска }\textbf{\emph{отчёта об~оценке}}\textbf{
или~даты, когда проводились либо были завершены исследования, выполненные
в~рамках проводимой }\textbf{\textsl{оценки}}\textbf{. В~таких случаях
следует проводить чёткое разграничение между ними.}

\textbf{Реализация.}

\stepcounter{SubSubSecCounter}

\thesubsubsection.\theSubSubSecCounter.\label{4.1.3.8.1} Конкретная
\textsl{дата оценки} должна быть согласована с~заказчиком "--- \textsl{допущение}
о~том, что~\textsl{дата оценки} соответствует дате составления \emph{отчёта},
является неприемлемым.\label{4.1.3.8.1-End}

\stepcounter{SubSubSecCounter}

\thesubsubsection.\theSubSubSecCounter.\label{4.1.3.8.2} В~тех
исключительных случаях, когда оценочная консультация касается даты,
имеющей место в~будущем, \emph{оценщику} следует руководствоваться
положениями п.~\ref{subsubsec:4.3.2.6_Valuation_date} \vpageref{subsubsec:4.3.2.6_Valuation_date}--\pageref{subsubsec:4.3.2.6_Valuation_date-End},
а~также подраздела~\ref{subsec:4.4.10_Assumptions_for_projected_values}~\nameref{subsec:4.4.10_Assumptions_for_projected_values}
\vpageref{subsec:4.4.10_Assumptions_for_projected_values}--\pageref{subsec:4.4.10_Assumptions_for_projected_values-End}.\label{4.1.3.8.2-End}\label{subsubsec:4.1.3.8_Valuation_date-End}

\subsubsection{Характер и~объём работы оценщика, в~т.\,ч.~объём исследований
и~любые ограничения в~этой области\label{subsubsec:4.1.3.9_Nature_and_extent_of_the_valuer=002019s_work}}

\textbf{Любые ограничения или~запреты в~части }\textbf{\textsl{осмотра}}\textbf{,
запросов и~(или) проводимого }\textbf{\emph{оценщиком}}\textbf{ анализа,
выполняемых в~целях выполнения }\textbf{\textsl{оценки}}\textbf{,
должны быть оговорены и~зафиксированы в~условиях договора на~проведение
оценки.}

\textbf{В~случае, если соответствующая информация недоступна вследствие
и~по~причине того что~условия задания ограничивают возможный объём
исследований, то, в~случае принятия такого задания, данные~ограничения,
а~также любые необходимые }\textbf{\textsl{допущения}}\textbf{ и~}\textbf{\textsl{специальные
допущения}}\textbf{, следующие из~данного ограничения, должны быть
идентифицированы и~зафиксированы в~условиях договора на~проведение
оценки.}

\textbf{Реализация.}

\stepcounter{SubSubSecCounter}

\thesubsubsection.\theSubSubSecCounter.\label{4.1.3.9.1} Заказчик
может запросить оказание услуги в~ограниченном объёме: например,
короткий срок, предоставленный на~выполнение \emph{отчёта об~оценке},
может сделать невозможным установление фактов, которые в~обычной
ситуации устанавливаются осмотром или~соответствующим запросом; либо
может иметь место запрос на~предоставление результатов \textsl{оценки},
основанной на~применении результатов \emph{автоматизированной модели
оценки} \emph{(АМО)}. Следует обратить внимание на~то, что~в~целях
настоящих всемирных стандартов, предоставление результатов, полученных
\emph{автоматизированной моделью оценки}, рассматривается как предоставление
письменной оценки (см.~п.~\hyperref[3.1.1.4]{4} подраздела~\ref{subsec:3.1.1_Mandatory_application}
\vpageref{3.1.1.4}--\vpageref{3.1.1.4-End}). Соответственно, \emph{оценщикам}
следует осознавать последствия применения результатов \emph{АМО} равно
как~и~их~ручной корректировки и~относиться к~этому внимательно.
Услуга также будет считаться оказываемой в~ограниченном объёме в~случае
ограничений на~применение \textsl{допущений}, сделанных в~соответствии
с~разделом~\ref{sec:4.2_VPS2_Inspections_investigations_and_records}~\nameref{sec:4.2_VPS2_Inspections_investigations_and_records}
\vpageref{sec:4.2_VPS2_Inspections_investigations_and_records}--\pageref{sec:4.2_VPS2_Inspections_investigations_and_records-End}.\label{4.1.3.9.1-End}

\stepcounter{SubSubSecCounter}

\thesubsubsection.\theSubSubSecCounter.\label{4.1.3.9.2} RICS допускает,
что~иногда у~заказчика может возникнуть потребность в~услугах такого
характера, однако \emph{оценщик} обязан обсудить потребности и~требования
заказчика до~представления \emph{отчёта}. Подобные оценочные задания,
применительно к~недвижимости часто называют «экспресс-оценкой», «камеральной
оценкой», «оценкой на~ходу».\label{4.1.3.9.2-End}

\stepcounter{SubSubSecCounter}

\thesubsubsection.\theSubSubSecCounter.\label{4.1.3.9.3}\emph{ Оценщику}
следует рассмотреть вопрос обоснованности ограничения с~точки зрения
цели, для~которой требуется проведение \textsl{оценки}. \emph{Оценщик}
может рассмотреть возможность принятия оценочного задания при~соблюдении
определённых условий, например, условия, согласно которому, \emph{отчёт
об~оценке} не~может быть публично обнародован либо раскрыт \textsl{третьим
лицам}.\label{4.1.3.9.3-End}

\stepcounter{SubSubSecCounter}

\thesubsubsection.\theSubSubSecCounter.\label{4.1.3.9.4} Если \emph{оценщик}
считает, что~проведение \textsl{оценки} невозможно даже на~ограниченной
основе, следует отказаться от~принятия оценочного задания.\label{4.1.3.9.4-End}

\stepcounter{SubSubSecCounter}

\thesubsubsection.\theSubSubSecCounter.\label{4.1.3.9.5} В~случае
подтверждения принятия такого оценочного задания \emph{оценщик} обязан
чётко описать в~\emph{отчёте об~оценке} характер этих ограничений,
вытекающие из~них \textsl{допущения}, а~также степень их~влияния
на~точность оценки~(cм.~раздел~\ref{sec:4.3_VPS3_Valuation_reports}~\nameref{sec:4.3_VPS3_Valuation_reports}
\vpageref{sec:4.3_VPS3_Valuation_reports}--\pageref{sec:4.3_VPS3_Valuation_reports-End}).\label{4.1.3.9.5-End}

\stepcounter{SubSubSecCounter}

\thesubsubsection.\theSubSubSecCounter.\label{4.1.3.9.6} Раздел~\ref{sec:4.2_VPS2_Inspections_investigations_and_records}~\nameref{sec:4.2_VPS2_Inspections_investigations_and_records}
\vpageref{sec:4.2_VPS2_Inspections_investigations_and_records}--\pageref{sec:4.2_VPS2_Inspections_investigations_and_records-End}
содержит основные требования в~части вопросов проведения \emph{осмотра}
\emph{объекта оценки}.\label{4.1.3.9.6-End}\label{subsubsec:4.1.3.9_Nature_and_extent_of_the_valuer=002019s_work-End}

\subsubsection{Характер и~источники информации, используемой оценщиком\label{subsubsec:4.1.3.10_Nature_and_sources_of_information}}

\textbf{Характер и~источники соответствующих информации и~данных,
а~также степень их~верификации, осуществляемой при~проведении }\textbf{\textsl{оценки}}\textbf{,
необходимо определить и~описать в~}\textbf{\textsl{договоре на~проведение
оценки}}\textbf{. В~целях данного раздела под~}\textbf{\emph{информацией}}\textbf{
следует понимать }\textbf{\emph{исходные данные}}\textbf{ и~иные
подобные вводные.}\footnote{Прим.~пер.: таким образом, в~контексте данного подраздела происходит
смешение понятий \emph{\guillemotleft информация\guillemotright}
и~\emph{\guillemotleft данные\guillemotright ,} различие между которыми
было описано в~п.~\hyperref[3.2.1.5]{5}~подраздела~\ref{subsec:3.2.1_Professional_and_ethical_standards}
\vpageref{3.2.1.5}--\pageref{3.2.1.5-End}. Вследствие этого в~данном
переводном материале данные понятия будут применять в~том смысле,
в~котором они~были описаны в~вышеуказанном пункте.}

\textbf{Реализация.}

\stepcounter{SubSubSecCounter}

\thesubsubsection.\theSubSubSecCounter.\label{4.1.3.10.1} В~том~случае,
когда заказчик предоставляет \emph{данные}, на~которые следует полагаться
\emph{оценщику}, последний обязан чётко указать эти~\emph{данные
}в~условиях \textsl{договора на~проведение оценки} и, в~соответствующих
случаях, их~источник. В~каждом конкретном случае \emph{оценщик}
должен выносить суждение о~том, насколько вероятно, что~предоставляемые
\emph{данные} являются достоверными, не~забывая при~этом о~пределах
своих квалификации и~опыта и~не~выходя за~их~рамки.\label{4.1.3.10.1-End}

\stepcounter{SubSubSecCounter}

\thesubsubsection.\theSubSubSecCounter.\label{4.1.3.10.2} Заказчик
может ожидать, что~\emph{оценщик} выразит своё мнение по~социальным,
экологическим и~правовым вопросам, оказывающим влияние на~\textsl{оценку},
а~\emph{оценщик}, в~свою очередь, может иметь такое желание. Вследствие
этого \emph{оценщику} необходимо обозначить в~\emph{отчёте} те~\emph{информацию}
и~\emph{данные}, которые должны быть проверены юристами заказчика
либо иного пользователя \emph{отчёта об~оценке} прежде, чем~он~может
быть использован по~назначению либо публично обнародован.\label{4.1.3.10.2-End}
\label{subsubsec:4.1.3.10_Nature_and_sources_of_information-End}

\subsubsection{Необходимые допущения и~специальные допущения\label{subsubsec:4.1.3.11_Assumptions_and_special_assumptions}}

\textbf{Все~необходимые для~проведения }\textbf{\textsl{оценки}}\textbf{
и~составления }\textbf{\emph{отчёта}}\textbf{ }\textbf{\textsl{допущения}}\textbf{
и~}\textbf{\textsl{специальные допущения}}\textbf{ должны быть определены
и~зафиксированы в~письменном виде.}
\begin{itemize}
\item \textbf{\textsl{Допущения}}\textbf{ "--- обстоятельства, которые
следует принимать в~качестве установленных фактов в~контексте оценочного
задания без~проведения соответствующего исследования либо проверки.
Будучи однажды установленными, они~должны приниматься для~понимания
}\textbf{\textsl{оценки}}\textbf{ либо иной консультации.}
\item \textbf{\textsl{Специальные допущения}}\textbf{ "--- }\textbf{\textsl{допущения}}\textbf{
которые либо опираются на~обстоятельства заведомо отличные от~реально
существующих на дату оценки, либо не~соответствующие поведению типичного
рыночного агента в~рассматриваемой гипотетической сделке в~период,
соотносимый с~}\textbf{\textsl{датой оценки}}\textbf{.}
\end{itemize}
\textbf{Следует вводить только те~}\textbf{\textsl{допущения}}\textbf{
и~}\textbf{\textsl{специальные допущения}}\textbf{, которые являются
разумными и~уместными применительно к~цели и~назначению }\textbf{\textsl{оценки}}\textbf{.}

\textbf{Реализация.}

\stepcounter{SubSubSecCounter}

\thesubsubsection.\theSubSubSecCounter.\label{4.1.3.11.1}\textsl{
Специальные допущения} зачастую используются для~демонстрации влияния
изменившихся обстоятельств на~стоимость. Некоторые примеры подобных
\textsl{специальных допущений}:
\begin{itemize}
\item предполагаемое строительство было завершено на~\textsl{дату оценки};
\item по~состоянию на~\textsl{дату оценки} имели место некие договорные
обязательства, не~существовавшие в~действительности;
\item финансовый инструмент оценивается с~использованием кривой доходности,
отличной от~той, которая~бы использовалась участником рынка.\label{4.1.3.11.1-End}
\end{itemize}
\stepcounter{SubSubSecCounter}

\thesubsubsection.\theSubSubSecCounter.\label{4.1.3.11.2} Дальнейшие
указания применительно к~\textsl{допущениям} и~\textsl{специальным
допущениям}, включая случай с~прогнозной стоимостью (т.\,е.~будущим
состоянием актива или~любых факторов, относящихся к~его~\textsl{оценке})
приводятся в~разделе~\ref{sec:4.4_VPS4_Bases_of_value}~\nameref{sec:4.4_VPS4_Bases_of_value}
\vref{sec:4.4_VPS4_Bases_of_value}--\pageref{sec:4.4_VPS4_Bases_of_value-End}.\label{4.3.1.11.2-End}
\label{subsubsec:4.1.3.11_Assumptions_and_special_assumptions-End}

\subsubsection{Содержание отчёта об~оценке\label{subsubsec:4.1.3.12_Format_of_the_report}}

\textbf{\emph{Оценщик}}\textbf{ должен установить формат }\textbf{\emph{отчёта
об~оценке}}\textbf{, а~также способ передачи }\textbf{\emph{данных}}\textbf{
в~рамках проведения }\textbf{\textsl{оценки}}\textbf{.}

\textbf{Реализация.}

\stepcounter{SubSubSecCounter}

\thesubsubsection.\theSubSubSecCounter.\label{4.1.3.12.1} Раздел~\ref{sec:4.3_VPS3_Valuation_reports}~\nameref{sec:4.3_VPS3_Valuation_reports}
\vref{sec:4.3_VPS3_Valuation_reports}--\pageref{sec:4.3_VPS3_Valuation_reports-End}
устанавливает, как~следует из~его~названия, обязательные требования
к~\emph{отчёту об~оценке}. В~тех случаях, когда в~порядке исключения
согласовывается неприменение какого-либо из~минимальных требований
к~\emph{отчёту об~оценке}, принимаемое в~качестве \textsl{отступления},
при~условии согласования этого в~\textsl{договоре на~проведение
оценки} и~описания данного факта в~\emph{отчёте об~оценке}, оно~не~должно
приводить к~введению в~заблуждение его~читателя, а~также быть
явно неадекватным относительно цели проведения \textsl{оценки}.\label{4.1.3.12.1-End}

\stepcounter{SubSubSecCounter}

\thesubsubsection.\theSubSubSecCounter.\label{4.1.3.12.2}\emph{
Отчёт об~оценке}, подготовленный в~соответствии с~данным стандартом,
а~также разделом~\ref{sec:4.3_VPS3_Valuation_reports}~\nameref{sec:4.3_VPS3_Valuation_reports}
\vref{sec:4.3_VPS3_Valuation_reports}--\pageref{sec:4.3_VPS3_Valuation_reports-End},
не~может быть и~обозначен и~называться как~\guillemotleft сертификат
стоимости\guillemotright{} либо \guillemotleft декларация стоимости\guillemotright ,
поскольку использование таких формулировок подразумевает гарантию
либо такой уровень уверенности, которые часто являются неприемлемыми.
Однако \emph{оценщик} может использовать термин «заверенный» или~аналогичные
слова в~тексте \emph{отчёта}, если известно, что~\textsl{оценка}
выполнена для~целей, требующих официального заверения \emph{суждения
о~стоимости}.\label{4.1.3.12.2-End}

\stepcounter{SubSubSecCounter}

\thesubsubsection.\theSubSubSecCounter.\label{4.1.3.12.3}\emph{
Оценщикам} следует знать о~том, что~термины «сертификат стоимости»,
«сертификат оценки» и~«заключение о~стоимости» имеют специфическое
значение в~ряде национальных юрисдикций и~употребляются в~ряде
законодательно установленных документов. Одним из~общих свойств таких
документов является то, что~они~предполагают только простое подтверждение
самой стоимости без~каких-либо требований к~пониманию контекста,
фундаментальных допущений или~аналитических процессов, стоящих за~представленными
цифрами. \emph{Оценщик}, ранее выполнявший \textsl{оценку} либо консультировавший
по~сделке, связанной с~активом, может подготовить такой документ,
если заказчик обязан предоставить его~в~силу закона.\label{4.1.3.12.3-End}\label{subsubsec:4.1.3.12_Format_of_the_report-End}

\subsubsection{Ограничения на~использование, распространение и~публикацию отчёта
об~оценке\label{subsubsec:4.1.3.13_Restrictions_on_use_of_the_report}}

\textbf{В~тех~случаях, когда необходимо либо желательно ограничить
использование результатов }\textbf{\textsl{оценки}}\textbf{ либо круг
её~пользователей, такие ограничения должны быть зафиксированы в~}\textbf{\textsl{договоре
на~проведение оценки}}\textbf{ в~явном виде.}

\textbf{Реализация.}

\stepcounter{SubSubSecCounter}

\thesubsubsection.\theSubSubSecCounter.\label{4.1.3.13.1}\emph{
Оценщик} обязан привести сведения о~разрешённом использовании и~возможности
обнародования и~распространения \emph{отчёта об~оценке}.\label{4.1.3.13.1-End}

\stepcounter{SubSubSecCounter}

\thesubsubsection.\theSubSubSecCounter.\label{4.1.3.13.2} Ограничения
имеют силу только в~том случае, если заказчик был~заранее уведомлен
о~них.\label{4.1.3.13.2-End}

\stepcounter{SubSubSecCounter}

\thesubsubsection.\theSubSubSecCounter.\label{4.1.3.13.3}\emph{
Оценщику} следует иметь ввиду, что~условия страхования его\emph{~}профессиональной
ответственности, защищающие его~от~последствий собственных неосторожных
либо небрежных действий, могут предусматривать требования в~части
его~квалификации, а~также включения соответствующих ограничительных
оговорок в~каждый \emph{отчёт об~оценке}. В~этом случае соответствующие
формулировки должны повторяться при~составлении каждого \emph{отчёта
об~оценке}, если только страховщик не~даст своё согласие на~их~изменение
либо полный отказ от~них. В~случае возникновения сомнений в~этой
части \emph{оценщику} следует изучить правила страхования прежде чем~согласиться
на~выполнение оценочного задания.\label{4.1.3.13.3-End}

\stepcounter{SubSubSecCounter}

\thesubsubsection.\theSubSubSecCounter.\label{4.1.3.13.4} В~ряде
случаев, \textsl{оценки} проводятся для~целей, при~которых исключение
ответственности перед третьими лицами запрещено законом или~регулятором.
В~остальных случаях вопрос ответственности \emph{оценщика} рассматривается
и~согласовывается с~заказчиком с~учётом мнения \emph{оценщика}.\label{4.1.3.13.4-End}

\stepcounter{SubSubSecCounter}

\thesubsubsection.\theSubSubSecCounter.\label{4.1.3.13.5} Особое
внимание вопросам ответственности перед \textsl{третьими лицами} следует
уделять в~случаях проведения \textsl{оценки} для~целей залогового
кредитования.\label{4.1.3.13.5-End}\label{subsubsec:4.1.3.13_Restrictions_on_use_of_the_report-End}

\subsubsection{Подтверждение того, что~оценка будет проводиться в~соответствии
с~требованиями МСО\label{subsubsec:4.1.3.14_Confirmation_accordance_with_the_IVS}}

\textbf{Оценщик обязан привести:}
\begin{itemize}
\item \textbf{подтверждение того, что~}\textbf{\textsl{оценка}}\textbf{
будет выполнена в~соответствии с~требованиями \href{https://www.rics.org/globalassets/rics-website/media/upholding-professional-standards/sector-standards/valuation/international-valuation-standards-rics2.pdf}{МСО}~\cite{IVS-2020},
а~}\textbf{\emph{оценщик}}\textbf{ проведёт проверку применимости
всех существенных }\textbf{\emph{исходных данных}}
\end{itemize}
\textbf{либо, в~зависимости от~конкретных требований заказчика:}
\begin{itemize}
\item \textbf{подтверждение того, что~}\textbf{\textsl{оценка}}\textbf{
будет выполнена в~соответствии с~требованиями }\textbf{\emph{Всемирных
стандартов оценки RICS}}\textbf{, включающих в~себя \href{https://www.rics.org/globalassets/rics-website/media/upholding-professional-standards/sector-standards/valuation/international-valuation-standards-rics2.pdf}{МСО}~\cite{IVS-2020}
и, там~где~это~уместно, \href{https://www.isurv.com/info/1342/rics_national_or_jurisdictional_valuation_standards}{Стандартов национальных ассоциаций RICS}~\cite{RICS:National-Standards},
а~также дополнений для~от- дельных юрисдикций. Там~где~это~уместно,
данное утверждение может быть сокращено до~одной ссылки на~использование
}\textbf{\emph{Всемирных стандартов RICS}}\textbf{.}
\end{itemize}
\textbf{В~обоих случаях необходимо привести сопроводительные сведения
и~пояснения в~части любых }\textbf{\textsl{отступлений}}\textbf{
от~требований \href{https://www.rics.org/globalassets/rics-website/media/upholding-professional-standards/sector-standards/valuation/international-valuation-standards-rics2.pdf}{МСО}~\cite{IVS-2020}
либо RVGS/RBG. Любое такое }\textbf{\textsl{отступление}}\textbf{
должно быть идентифицировано и~обосновано. }\textbf{\textsl{Отступление}}\textbf{
не~может считаться оправданным, если оно~приводит к~введению в~заблуждение
пользователей результатов }\textbf{\textsl{оценки}}\textbf{.}

\textbf{Реализация.}

\stepcounter{SubSubSecCounter}

\thesubsubsection.\theSubSubSecCounter.\label{4.1.3.14.1} Существенная
разница между двумя вышеуказанными способами подтверждения соответствия
отсутствует, они~могут использоваться в~зависимости от~конкретных
требований оценочного задания. Некоторые заказчики хотели~бы иметь
подтверждение того, что~\textsl{оценка} была проведена в~соответствии
с~\href{https://www.rics.org/globalassets/rics-website/media/upholding-professional-standards/sector-standards/valuation/international-valuation-standards-rics2.pdf}{Международными стандартами оценки}~(\href{https://www.rics.org/globalassets/rics-website/media/upholding-professional-standards/sector-standards/valuation/international-valuation-standards-rics2.pdf}{МСО})~\cite{IVS-2020},
которое конечно~же должно быть предоставлено. Во~всех остальных
случаях подтверждение того, что~\textsl{оценка} была выполнена в~соответствии
с~RVGS/RBG несёт в~себе двойную гарантию соответствия в~целом как~\href{https://www.rics.org/globalassets/rics-website/media/upholding-professional-standards/sector-standards/valuation/international-valuation-standards-rics2.pdf}{МСО}~\cite{IVS-2020}
так~и~данным профессиональным стандартам RICS.\label{4.1.3.14.1-End}

\stepcounter{SubSubSecCounter}

\thesubsubsection.\theSubSubSecCounter.\label{4.1.3.14.2} Под~ссылкой
на~RVGS/RBG без~ссылки на~год выпуска понимается версия стандартов
RICS, действующая на~\textsl{дату оценки}, при~условии, что~\textsl{она}
соответствует \textsl{дате отчёта} либо предшествует~ей. В~случае
определения «прогнозной стоимости», т.\,е.~относящейся к~будущему
относительно \textsl{даты отчёта}, последняя является определяющей
в~части определения применимой версии RVGS/RBG.\label{4.1.3.14.2-End}

\stepcounter{SubSubSecCounter}

\thesubsubsection.\theSubSubSecCounter.\label{4.1.3.14.3} Заявление
о~соответствии должно акцентировать внимание на~\textsl{отступлениях}
(см.~подраздел~\ref{subsec:3.1.6_Departures}~\nameref{subsec:3.1.6_Departures}
\vref{subsec:3.1.6_Departures}--\pageref{subsec:3.1.6_Departures-End}).
В~случае наличия \textsl{отступлений}, не~являющихся реально необходимыми,
невозможно обеспечить соответствие \emph{отчёта} требованиям \href{https://www.rics.org/globalassets/rics-website/media/upholding-professional-standards/sector-standards/valuation/international-valuation-standards-rics2.pdf}{Международных стандартов оценки}~(\href{https://www.rics.org/globalassets/rics-website/media/upholding-professional-standards/sector-standards/valuation/international-valuation-standards-rics2.pdf}{МСО})~\cite{IVS-2020}.\label{4.1.3.14.3-End}

\stepcounter{SubSubSecCounter}

\thesubsubsection.\theSubSubSecCounter.\label{4.1.3.14.4} В~случаях,
когда предполагается соблюдение других стандартов оценки, специфических
для~конкретной национальной юрисдикции, это~должно быть согласовано
в~процессе заключения \textsl{договора на~проведение оценки}.\label{4.1.3.14.4-End}\label{subsubsec:4.1.3.14_Confirmation_accordance_with_the_IVS-End}

\subsubsection{Определение размера вознаграждения за~проведение оценки\label{subsubsec:4.1.3.15_The_basis_of_the_fee}}

\textbf{Реализация.}

\stepcounter{SubSubSecCounter}

\thesubsubsection.\theSubSubSecCounter.\label{4.1.3.15.1} \uline{Размер
вознаграждения} за~проведение \textsl{оценки} определяется на~основе
соглашения с~заказчиком за~исключением случаев, когда \uline{он}~предписан
внешними сторонами, ограничивающими обеим сторонам свободу договора
в~данной части. RICS не~публикует какие-либо рекомендованные значения
размера вознаграждения.\label{4.1.3.15.1-End}\label{subsubsec:4.1.3.15_The_basis_of_the_fee-End}

\subsubsection{Ссылка на~процедуру рассмотрения жалоб фирмой, зарегистрированной
для~регулирования со~стороны RICS, позволяющую получить копию результатов
её~рассмотрения\label{subsubsec:4.1.3.16_Firm=002019s_complaints_handling_procedure}}

\textbf{Реализация.}

\stepcounter{SubSubSecCounter}

\thesubsubsection.\theSubSubSecCounter.\label{4.1.3.16.1} Включение
данного пункта призвано акцентировать внимание на~обязанность \textsl{оценочных
компаний, чья~деятельности регулируется RICS}, соблюдать \textsl{\href{https://www.rics.org/globalassets/rics-website/media/upholding-professional-standards/regulation/rules-of-conduct-for-firms_2020.pdf}{Правила поведения RICS для оценочных компаний}~\cite{RICS:Conduct-Firms}}.\label{4.1.3.16.1-End}\label{subsubsec:4.1.3.16_Firm=002019s_complaints_handling_procedure-End}

\subsubsection{Заявление о~том, что~соблюдение данных стандартов может быть предметом
контроля в~соответствии с~правилами поведения и~дисциплинарными
нормами RICS\label{subsubsec:4.1.3.17_Compliance_with_RICS_conduct_regulations}}

\textbf{Реализация.}

\stepcounter{SubSubSecCounter}

\thesubsubsection.\theSubSubSecCounter.\label{4.1.3.17.1} Заявление
о~том, что~соблюдение данных стандартов может быть предметом контроля
в~соответствии с~\href{http://.rics.org/ssa/upholding-professional-standards/standards-of-conduct/rules-of-conduct/}{Правилами поведения и дисциплинарными нормами RICS}~\cite{RICS:Conduct,RICS:Conduct-Firms,RICS:Conduct-Members}.\label{4.1.3.17.1-End}

\stepcounter{SubSubSecCounter}

\thesubsubsection.\theSubSubSecCounter.\label{4.1.3.17.2} Руководство
по~осуществлению режима контроля в~т.\,ч.~в~части вопросов конфиденциальности
доступно в~\href{https://www.rics.org/eu/upholding-professional-standards/regulation/}{соответствующем разделе}
официального \href{https://www.rics.org/eu/}{сайта RICS} в~ИТС
«Интернет»~\cite{RICS:Regulation}.\label{4.1.3.17.2-End}

\stepcounter{SubSubSecCounter}

\thesubsubsection.\theSubSubSecCounter.\label{4.1.3.17.3} Заказчикам
следует иметь в~виду, что~данное заявление не~может быть сделано
\emph{оценщиком}, не~являющимся \textsl{членом RICS} и~не~работающим
в~компании, чья~деятельность \href{https://www.rics.org/eu/upholding-professional-standards/regulation/}{регулируется со стороны RICS}~\cite{RICS:Regulation},
либо осуществляющим деятельность в~порядке, предусмотренном подразделом~\ref{subsec:3.1.8_Application_for_other}~\nameref{subsec:3.1.8_Application_for_other}
\vpageref{subsec:3.1.8_Application_for_other}--\pageref{subsec:3.1.8_Application_for_other-End}.\label{4.1.3.17.3-End}

\subsubsection{Заявление, содержащее согласованные условия ограничения ответственности\label{subsubsec:4.1.3.18_Statement_setting_limitations_on_liability}}

\textbf{Реализация.}

\stepcounter{SubSubSecCounter}

\thesubsubsection.\theSubSubSecCounter.\label{4.1.3.18.1}

Вопросы риска, ответственности и~страхования профессиональной ответственности
тесно взаимосвязаны. В~ожидании выхода глобального руководства RICS
по~данным вопросам членам RICS следует сверяться с~последними руководствами
RICS в~данной части, применяемыми в~их~юрисдикции, размещёнными
в~\href{https://www.rics.org/uk/upholding-professional-standards/regulation/regulatory-support/professional-indemnity/pii-and-valuation-guidance/}{соответствующем разделе}
официального сайта RICS в~ИТС «Интернет»~\cite{RICS:prof_indemnity}.\label{4.1.3.18.1-End}\label{subsubsec:4.1.3.18_Statement_setting_limitations_on_liability-End}

\label{subsec:4.1.3_Terms_of_engagement_scope_of_work-End}

\label{sec:4.1_VPS1_Terms_of_engagement_Scope_of_work-End}

\newpage

\section{СПО 2. Осмотры, исследования и~фиксация их~результатов\label{sec:4.2_VPS2_Inspections_investigations_and_records}}

Данный обязательный стандарт:
\begin{itemize}
\item имплементирует \hyperref[sec:9.2_IVS-102_Investigations]{МСО 102 Исследования и соответствие};
\item устанавливает дополнительные обязательные требования для~\textsl{членов
RICS}, разработанные с~целью улучшения понимания заказчиками процесса
\textsl{оценки} и~\emph{отчёта об~оценке};
\item содержит рассмотрение аспектов имплементации, возникающих в~конкретных
случаях.
\end{itemize}
\textbackslash newpage

\subsection{Осмотры и~исследования\label{subsec:4.2.1_Inspections_and_investigations}}

\textbf{\textsl{Осмотры}}\textbf{ и~исследования во~всех случаях
необходимо осуществлять в~той степени, в~которой это~необходимо
для~обеспечения соответствия }\textbf{\textsl{оценки}}\textbf{ поставленной
цели с~точки зрения профессиональных требований. }\textbf{\emph{Оценщик}}\textbf{
обязан принять разумные меры по~проверке информации и~данных, на~которые
он~полагается при~проведении }\textbf{\textsl{оценки}}\textbf{,
и, если это~ещё~не~было сделано, уточнить и~согласовать с~заказчиком
все~необходимые }\textbf{\textsl{допущения}}\textbf{.}

\textbf{Данные общие принципы дополняются дополнительными требованиями,
изложенными в~\nameref{sec:4.1_VPS1_Terms_of_engagement_Scope_of_work}
и~\nameref{sec:4.3_VPS3_Valuation_reports}:}
\begin{itemize}
\item \textbf{любые ограничения и~запреты на~осмотр, исследования и~анализ
в~рамках работ по~}\textbf{\textsl{оценке}}\textbf{ должны быть
определены и~зафиксированы в~}\textbf{\textsl{договоре на~проведение
оценки}}\textbf{ (см.~п.~\ref{subsubsec:4.1.3.9_Nature_and_extent_of_the_valuer=002019s_work}~\nameref{subsubsec:4.1.3.9_Nature_and_extent_of_the_valuer=002019s_work}
\vpageref{subsubsec:4.1.3.9_Nature_and_extent_of_the_valuer=002019s_work}--\pageref{subsubsec:4.1.3.9_Nature_and_extent_of_the_valuer=002019s_work-End}),
а~также в~тексте самого }\textbf{\emph{отчёта об~оценке}}\textbf{
(см.~п.~\ref{subsubsec:4.3.2.7_Extent_of_investigation} \vpageref{subsubsec:4.3.2.7_Extent_of_investigation}--\pageref{subsubsec:4.3.2.7_Extent_of_investigation-End});}
\item \textbf{если соответствующие информация либо данные недоступны вследствие
и~по~причине запрета на~исследования, установленного условиями
}\textbf{\textsl{договора на~проведение оценки}}\textbf{, то, в~случае
принятия такой работы, данный запрет и~любые связанные с~ним }\textbf{\textsl{допущения}}\textbf{
или~}\textbf{\textsl{специальные допущения}}\textbf{ необходимо идентифицировать
и~зафиксировать в~}\textbf{\textsl{договоре на~проведение оценки}}\textbf{
(см.~п.~\ref{subsubsec:4.1.3.9_Nature_and_extent_of_the_valuer=002019s_work}~\nameref{subsubsec:4.1.3.9_Nature_and_extent_of_the_valuer=002019s_work}
\vpageref{subsubsec:4.1.3.9_Nature_and_extent_of_the_valuer=002019s_work}--\pageref{subsubsec:4.1.3.9_Nature_and_extent_of_the_valuer=002019s_work-End}),
а также в тексте самого отчёта об оценке (см.~п.~\ref{subsubsec:4.3.2.7_Extent_of_investigation}
\vpageref{subsubsec:4.3.2.7_Extent_of_investigation}--\pageref{subsubsec:4.3.2.7_Extent_of_investigation-End}).}
\end{itemize}
\textbf{Реализация.}

\stepcounter{SubSecCounter}

\thesubsection.\theSubSecCounter.\label{4.2.1.1} В~процессе утверждения
условий \textsl{договора на~проведение оценки} \emph{оценщику} необходимо
согласовать в~т.\,ч.~объём исследований, касающихся \emph{объекта
оценки}, а~также вопросы его~\textsl{осмотра} "--- см.~раздел~\ref{sec:4.1_VPS1_Terms_of_engagement_Scope_of_work}~\nameref{sec:4.1_VPS1_Terms_of_engagement_Scope_of_work}
\vpageref{sec:4.1_VPS1_Terms_of_engagement_Scope_of_work}--\pageref{sec:4.1_VPS1_Terms_of_engagement_Scope_of_work-End}.\label{4.2.1.1-End}

\stepcounter{SubSecCounter}

\thesubsection.\theSubSecCounter.\label{4.2.1.2} При определении
объёма необходимых сведений, имеющих доказательное значения для~целей
проведения \textsl{оценки}, необходимо профессиональное суждение \emph{оценщика}
на~предмет того, насколько собранные информация и~данные соответствует
её целям, а~также \textsl{виду определяемой стоимости}. В~каждом
конкретном случае \emph{оценщику} необходимо выносить суждение по~вопросу
степени надёжности предоставленных информации и~данных, учитывая
при~этом пределы своих квалификации и~опыта и~не~выходя за~их~рамки.\label{4.2.1.2-End}

\stepcounter{SubSecCounter}

\thesubsection.\theSubSecCounter.\label{4.2.1.3} При~\textsl{осмотре}
и~иных исследованиях объектов недвижимого имущества либо иных физических
существующих активов степень таких проверок будет варьироваться в~зависимости
от~характера актива и~цели \textsl{оценки}. За~исключением случаев,
описанных ниже в~подразделе~\ref{subsec:4.2.2_Revaluation_without_re-inspection}~\nameref{subsec:4.2.2_Revaluation_without_re-inspection}
\vpageref{subsec:4.2.2_Revaluation_without_re-inspection}--\pageref{subsec:4.2.2_Revaluation_without_re-inspection-End},
\emph{оценщикам} следует помнить о~том, что~добровольный отказ от~изучения
и~\textsl{осмотра} материальных активов может привнести неприемлемый
уровень риска в~оказываемую оценочную консультацию, вследствие чего
данный риск необходимо детально изучить до~начала работы "--- см.~параграф~\ref{subsubsec:4.1.3.9_Nature_and_extent_of_the_valuer=002019s_work}
\vpageref{subsubsec:4.1.3.9_Nature_and_extent_of_the_valuer=002019s_work}--\pageref{subsubsec:4.1.3.9_Nature_and_extent_of_the_valuer=002019s_work-End},
затрагивающий вопросы оказания услуг в~ограниченном объёме, включая
использование \emph{автоматизированных моделей оценки}.\label{4.2.1.3-End}

\stepcounter{SubSecCounter}

\thesubsection.\theSubSecCounter.\label{4.2.1.4} В~тех~случаях,
когда существует потребность провести либо проверить измерения, \textsl{членам
RICS} необходимо, там~где~это~уместно, придерживаться положений
\href{https://ipmsc.org/standards/}{Международных стандартов оценки собственности}~\cite{IPMS}.
\href{https://www.rics.org/globalassets/rics-website/media/upholding-professional-standards/sector-standards/real-estate/rics-property-rement/rics-property-measurement-2nd-edition-rics.pdf}{Положение RICS об оценке стоимости недвижимого имущества}~\cite{RICS:PM}
содержит больше деталей.\label{4.2.1.4-End}

\stepcounter{SubSecCounter}

\thesubsection.\theSubSecCounter.\label{4.2.1.5} Раздел~\ref{sec:5.8_VPGA-8_Valuation_of_real_property}~\nameref{sec:5.8_VPGA-8_Valuation_of_real_property}
\vpageref{sec:5.8_VPGA-8_Valuation_of_real_property}--\pageref{sec:5.8_VPGA-8_Valuation_of_real_property-End}
содержит подробные комментарии по~вопросам как~очевидным так~и~подлежащим
рассмотрению в~процессе \textsl{осмотра} и~изучения объектов недвижимого
имущества, включая вопросы, относящиеся к~общей теме защиты окружающей
среды и~обеспечения \textsl{устойчивого развития}. Данные факторы
имеют всё~большее значение с~точки зрения их~восприятия рыночными
агентами и~влияния на~сам~рынок, вследствие и~по~причине чего
важно, чтобы \emph{оценщики} учитывали их~в~процессе работы над~конкретными
оценочными заданиями.\label{4.2.1.5-End}

\stepcounter{SubSecCounter}

\thesubsection.\theSubSecCounter.\label{4.2.1.6} \emph{Оценщику}
необходимо принимать разумные меры для~проверки информации~и~данных,
на~которые он~опирался в~процессе проведения \textsl{оценки}, учитывая
при~этом положения подраздела~\ref{subsec:3.2.2_Member qualification}
\vpageref{subsec:3.2.2_Member qualification}--\pageref{subsec:3.2.2_Member qualification-End},
а~также параграфа\vref{subsubsec:4.1.3.10_Nature_and_sources_of_information}--\pageref{subsubsec:4.1.3.10_Nature_and_sources_of_information-End},
и, если это~ещё~не~было сделано, уточнить совместно с~заказчиком
любые необходимые \textsl{допущения}. В~то~время как~заказчик может
дать согласие на~использование допущения либо сам~попросить ввести
его, тем~не~менее, если после \textsl{осмотра} либо иного исследования
\emph{оценщик} приходит к~выводу о~том, что~оно~противоречит имеющимся
фактам, его~использование в~качестве \textsl{специального допущения}
возможно при~условии его~реалистичности, действительности и~актуальности
в~контексте обстоятельств конкретной \textsl{оценки} (см.~подраздел~\ref{subsec:4.4.8_Special_assumptions}~\nameref{subsec:4.4.8_Special_assumptions}
\vpageref{subsec:4.4.8_Special_assumptions}--\pageref{subsec:4.4.8_Special_assumptions-End}).\label{4.2.1.6-End}

\stepcounter{SubSecCounter}

\thesubsection.\theSubSecCounter.\label{4.2.1.7} В~случае согласия
на~работу, когда соответствующие данные недоступны вследствие наличия
ограничений в~условиях \textsl{договора на~проведение оценки} либо
в~случае наличия договорённости, ограничивающей \textsl{осмотры}
и~исследования, \textsl{оценка} проводится на~основе таких ограниченных
данных с~учётом требований параграфа~\ref{subsubsec:4.1.3.10_Nature_and_sources_of_information}~\nameref{subsubsec:4.1.3.10_Nature_and_sources_of_information}
\vpageref{subsubsec:4.1.3.10_Nature_and_sources_of_information}--\pageref{subsubsec:4.1.3.10_Nature_and_sources_of_information-End}.
Любые ограничения на~\textsl{осмотр} или~проверку данных равно как~и~их~недостаток
либо отсутствие должны быть отражены в~условиях \textsl{договора
на~проведение оценки} и~тексте \emph{отчёта об~оценке}. Если \emph{оценщик}
считает, что~проведение \textsl{оценки} невозможно даже на~ограниченной
основе, следует отказаться от~принятия такой работы.\label{4.2.1.7-End}

\stepcounter{SubSecCounter}

\thesubsection.\theSubSecCounter.\label{4.2.1.8} В~случае, если
условия \textsl{договора на~проведение оценки} предполагают использование
информации или~данных, предоставленных внешним по~отношению к~\emph{оценщику}
источником, ему~следует рассмотреть вопрос достоверности такой информации~(данных),
а~ также возможности их~использования без~ущерба достоверности
суждения \emph{оценщика}. В~подобных случаях возможно продолжение
работы. Существенные исходные данные, предоставленные \emph{оценщику}
(например менеджментом либо собственниками), значительно влияющие
на~результат \textsl{оценки}, в~отношении которых, по~мнению \emph{оценщика},
существуют сомнения, требуют проверки, исследования и~(или) подтверждения
в~зависимости от~обстоятельств. В~ситуациях, когда надёжность или~достоверность
предоставленной информации~(данных) не~может быть подтверждена,
от~её~использования следует отказаться.\label{4.2.1.8-End}

\stepcounter{SubSecCounter}

\thesubsection.\theSubSecCounter.\label{4.2.1.9} Поскольку \emph{оценщику}
следует проявлять должную осмотрительность при~проверке любой предоставленной
или~собранной информации (данных), любые ограничения данной обязанности
должны быть чётко указаны в~\textsl{договоре на~проведение оценки}
(см.~раздел~\ref{sec:4.1_VPS1_Terms_of_engagement_Scope_of_work}~\nameref{sec:4.1_VPS1_Terms_of_engagement_Scope_of_work}\vpageref{sec:4.1_VPS1_Terms_of_engagement_Scope_of_work}\pageref{sec:4.1_VPS1_Terms_of_engagement_Scope_of_work-End}
). В~случае подготовки \textsl{оценки} для~целей финансовой отчётности
\emph{оценщик} должен быть готов обсуждать уместность любых допущений
с~аудитором заказчика, другим профессиональным консультантом либо
представителями регулирующих органов.\label{4.2.1.9-End}

\stepcounter{SubSecCounter}

\thesubsection.\theSubSecCounter.\label{4.2.1.10}\emph{ Оценщик},
соответствующий требованиям, установленным подразделом~\ref{subsec:3.2.2_Member qualification}~\nameref{subsec:3.2.2_Member qualification}
\vpageref{subsec:3.2.2_Member qualification}--\pageref{subsec:3.2.2_Member qualification-End},
обязан обладать, если и~не~глубокими экспертными, то, как~минимум,
достаточно полными знаниями аспектов, влияющих на~данный вид актива,
включая, там~где~это~уместно, вопросы его~местоположения. В~случаях,
когда некие фактические либо потенциальные обстоятельства, способные
повлиять на~стоимость, становятся очевидны для~\emph{оценщика} в~силу
его~профессиональных знаний либо вследствие \textsl{осмотра} \emph{объекта
оценки} или~изучения сведений о~нём, включая, как~натурные обследования,
так~и~обычные запросы, \emph{оценщику} следует обратить внимание
заказчика на~эти~обстоятельства не~позднее чем~в момент передачи
\emph{отчёта}, а~в~идеале, до~момента выпуска \emph{отчёта} в~случаях,
когда такое влияние является существенным.\label{4.2.1.10-End}\label{subsec:4.2.1_Inspections_and_investigations-End}

\subsection{Переоценка ранее оцениваемого недвижимого имущества без~осмотра\label{subsec:4.2.2_Revaluation_without_re-inspection}}

\textbf{Реализация.}

\stepcounter{SubSecCounter}

\thesubsection.\theSubSecCounter.\label{4.2.2.1} Переоценка недвижимого
имущества и~прав на~него, ранее уже~оцениваемых \emph{оценщиком}
либо \textsl{оценочной компанией}, без~повторного \textsl{осмотра}
не~должна проводиться до~тех~пор, пока \emph{оценщик} не~убедится
в~том, что~с~момента предыдущей \textsl{оценки} не~произошло существенных
изменений физических свойств данного имущества либо характеристик
его~местоположения.\label{4.2.2.1-End}

\stepcounter{SubSecCounter}

\thesubsection.\theSubSecCounter.\label{4.2.2.2} RICS признаёт,
что~у заказчика может существовать потребность в~регулярной актуализации
\textsl{оценки} его~недвижимости, а~также, что~повторный \textsl{осмотр}
в~каждом случае может быть излишним. Переоценка без~повторного \textsl{осмотра}
может быть проведена в~том~случае, если \emph{оценщик} ранее осматривал
данный объект недвижимости, а~заказчик предоставил подтверждение
того, что~с~этого момента не~произошло существенных изменений физических
свойств объекта, а~также территории, где~он~расположен. Условия
\textsl{договора на~проведение оценки} должны содержать соответствующее
\textsl{допущение}.\label{4.2.2.2-End}

\stepcounter{SubSecCounter}

\thesubsection.\theSubSecCounter.\label{4.2.2.3}\emph{ Оценщик}
обязан получить от~заказчика сведения о~текущих и~предполагаемых
изменениях арендного дохода от~инвестиционной недвижимости, а~также
о~любых существенных изменениях нефизических характеристик объекта
недвижимости таких как: изменение условий аренды, согласование проектировочных
решений, уведомления органов власти и~т.\,п. \emph{Оценщик} также
обязан рассмотреть вопрос возможности изменения факторов, обеспечивающих
\emph{устойчивое развитие}, оказывающих влияние на~\textsl{оценку}.\label{4.2.2.3-End}

\stepcounter{SubSecCounter}

\thesubsection.\theSubSecCounter.\label{4.2.2.4} Если заказчик сообщает
о~существенных изменениях, либо если \emph{оценщик} узнаёт о~них~или~имеет
достаточные основания полагать, что~такие изменения произошли, \emph{оценщик}
обязать провести \textsl{осмотр} объекта недвижимости. Во~всех остальных
ситуациях, определение интервала между \textsl{осмотрами} относится
к~области \emph{профессионального суждения} \emph{оценщика}, который
среди прочих соображений обязан учитывать тип объекта, а~также его~местоположение.\label{4.2.2.4-End}

\stepcounter{SubSecCounter}

\thesubsection.\theSubSecCounter.\label{4.2.2.5} В~случае, если
\emph{оценщик} считает, что~проведение переоценки без~повторного
\textsl{осмотра} неприемлемо вследствие и~по~причине произошедших
существенных изменений, длительности периода, прошедшего с~момента
прежней \textsl{оценки} либо иных обстоятельств, он, тем~не~менее,
может взяться за~выполнение такой работы без~проведения \textsl{осмотра}
при~условии, что~заказчик в~письменной форме и~до~момента передачи
ему~\emph{отчёта} подтвердит, что~переоценка нужна ему~исключительно
для~внутренних целей корпоративного управления, а~также, что~\emph{отчёт}
не~будет опубликован или~передан \textsl{третьим лицам}, а~сам~заказчик
принимает на~себя ответственность за~сопутствующий риск. Данное
заявление, а~также запрет на~публикацию \emph{отчёта} должны быть
недвусмысленно указаны в~его~тексте.\label{4.2.2.5-End}\label{subsec:4.2.2_Revaluation_without_re-inspection-End}

\subsection{Протоколирование исследований оценщика\label{subsec:4.2.3_Valuation_records}}

\textbf{Записи, отражающие результаты }\textbf{\textsl{осмотров}}\textbf{,
исследований и~сбора ключевой исходной информации и~данных, должны
вестись в~надлежащем деловом формате.}

\textbf{Реализация.}

\stepcounter{SubSecCounter}

\thesubsection.\theSubSecCounter.\label{4.2.3.1} Подробности \textsl{осмотра,}
равно как и~других исследований, подлежат обязательной чёткой и~аккуратной
фиксации, осуществляемой таким образом, чтобы они~не~были ни~двусмысленными,
ни~вводящими в~заблуждение, а~также не~создавали ложного впечатления.\label{4.2.3.1-End}

\stepcounter{SubSecCounter}

\thesubsection.\theSubSecCounter.\label{4.2.3.2} Для~обеспечения
должного уровня протоколирования и~способности эффективно реагировать
на~будущие запросы по~результатам \textsl{осмотра} необходимо делать
удобочитаемые записи (которые могут включать в~себя фотографии и~иные
графические материалы), включающие также описание имевших место ограничений
\textsl{осмотра} и~обстоятельств, при~которых он~проводился. Данные
записи должны также включать в~себя основные исходные данные, все~расчёты,
исследования и~аналитические выводы, учитываемые в~дальнейшем в~процессе
\textsl{оценки}.\label{4.2.3.2-End}

\stepcounter{SubSecCounter}

\thesubsection.\theSubSecCounter.\label{4.2.3.3} Хотя это~не~является
обязательным, \emph{оценщикам} всё~же настоятельно рекомендуется
собирать и~фиксировать сведения, затрагивающие вопросы \emph{устойчивого
развития}, по~мере их~поступления, в~надлежащем качестве и~в~достаточном
объёме. Даже, если эти~сведения не~оказывают влияния на~стоимость
в~настоящее время, они~потребуются для~обеспечения сопоставимости
в~будущем. Это~может особенно пригодиться в~случаях, когда \emph{оценщик}
на~постоянной основе привлекается к~выполнению работ для~заказчика.\label{4.2.3.3-End}

\stepcounter{SubSecCounter}

\thesubsection.\theSubSecCounter.\label{4.2.3.4} Все~примечания
и~записи следует поддерживать в~установленном деловом формате. Надлежащий
срок их~хранения зависит от~\emph{цели проведения оценки} и~обстоятельств
оказания услуги, но~всегда должен соотноситься с~соответствующими
требованиями закона, подзаконных актов, а~также органов власти. \label{4.2.3.4-End}\label{subsec:4.2.3_Valuation_records-End}\label{sec:4.2_VPS2_Inspections_investigations_and_records-End}

\newpage

\section{СПО 3. Требования к~содержанию отчёта об~оценке\label{sec:4.3_VPS3_Valuation_reports}}

\textbf{Данный обязательный стандарт:}
\begin{itemize}
\item \textbf{имплементирует \hyperref[sec:9.3_IVS-103_Reporting]{МСО~103. Составление отчёта об~оценке};}
\item \textbf{устанавливает для~}\textbf{\textsl{членов RICS}}\textbf{
дополнительные обязательные требования, направленные на~улучшение
понимания заказчиками }\textbf{\emph{отчётов об~оценке}}\textbf{
и~увеличение их~полезности;}
\item \textbf{содержит конкретные детали имплементации, возникающие в~отдельных
случаях.}
\end{itemize}

\subsection{Основные принципы\label{subsec:4.3.1_General_principles}}

\textbf{В~}\textbf{\emph{отчёте об~оценке}}\textbf{:}
\begin{itemize}
\item \textbf{в~чёткой и~точной форме должны быть изложены выводы по~результатам
проведения }\textbf{\textsl{оценки}}\textbf{, при~этом повествование
не~должно содержать каких-либо двусмысленных либо вводящих в~заблуждение
выводов, а~также формировать ложное восприятия суждений }\textbf{\emph{оценщика}}\textbf{.
При~необходимости }\textbf{\emph{оценщик}}\textbf{ обязан обратить
внимание на~любые вопросы, оказывающие влияние на~степень определённости~(неопределённости)
}\textbf{\textsl{оценки}}\textbf{, и~дать соответствующие комментарии
согласно параграфу~\ref{subsubsec:4.3.2.15_Commentary_ on_any_material_uncertainty}
\vpageref{subsubsec:4.3.2.15_Commentary_ on_any_material_uncertainty}--\pageref{subsubsec:4.3.2.15_Commentary_ on_any_material_uncertainty-End});}
\item \textbf{должны быть отражены все~вопросы, установленные соглашением
с~заказчиком в~условиях }\textbf{\textsl{договора на~проведение
оценки}}\textbf{ }\textbf{\textsl{(заданием на оценку)}}\textbf{ (см.~раздел~\ref{sec:4.1_VPS1_Terms_of_engagement_Scope_of_work}~\nameref{sec:4.1_VPS1_Terms_of_engagement_Scope_of_work}
\vpageref{sec:4.1_VPS1_Terms_of_engagement_Scope_of_work}--\pageref{sec:4.1_VPS1_Terms_of_engagement_Scope_of_work-End}).}
\end{itemize}
\stepcounter{SubSecCounter}

\thesubsection.\theSubSecCounter.\label{4.3.1.1} Во~вводной части
\emph{отчёт об~оценке} должен обеспечивать чёткое понимание изложенного
в~нём~мнения \emph{оценщика} и~должен быть изложен в~таких терминах
и~формулировках, которые могут быть прочитаны и~поняты читателем,
не~имеющим предварительных знаний о~рассматриваемом активе или~обязательстве.\label{4.3.1.1-End}

\stepcounter{SubSecCounter}

\thesubsection.\theSubSecCounter.\label{4.3.1.2} Формат \emph{отчёта
об~оценке} и~его~детали являются предметом соглашения между \emph{оценщиком}
и~заказчиком, устанавливаемого условиями \textsl{договора на~проведение
оценки}. Во~всех случаях эти~аспекты должны быть адекватны задаче,
и, как~и~сама \textsl{оценка} "--- адекватны \emph{цели проведения
оценки} с~профессиональной точки зрения. В~тех случаях, когда \emph{отчёт}
должен быть составлен по~шаблону или~формату, предписанному заказчиком,
в~которых отсутствуют один или~более разделов, установленных требованиями
подраздела~\ref{subsec:4.3.2_Report_content}~\nameref{subsec:4.3.2_Report_content}
\vpageref{subsec:4.3.2_Report_content}--\pageref{subsec:4.3.2_Report_content-End},
данный вопрос в~обязательном порядке подлежит чёткому урегулированию
либо в~условиях первоначального договора комплексного оказания услуг,
либо в~\textsl{договоре на~проведение оценки} конкретного объекта,
либо в~соответствующей комбинации условий их~обоих. Невыполнение
данного условия приведёт к~тому, что~выполненную таким образом \textsl{оценку}
нельзя будет признать соответствующей данным всемирным стандартам.
В~этой части следует руководствоваться положениями параграфа~\ref{subsubsec:4.3.2.12_Valuation_approach_and_reasoning}~\nameref{subsubsec:4.3.2.12_Valuation_approach_and_reasoning}
\vpageref{subsubsec:4.3.2.12_Valuation_approach_and_reasoning}--\pageref{subsubsec:4.3.2.12_Valuation_approach_and_reasoning-End}.\label{4.3.1.2-End}

\stepcounter{SubSecCounter}

\thesubsection.\theSubSecCounter.\label{4.3.1.3} В~тех случаях,
когда в~течение определённого периода времени одному и~тому~же
заказчику должно быть предоставлено несколько \emph{отчётов об~оценке}
на~основании \textsl{договоров на~проведение оценки}, содержащих
идентичные условия, данное обстоятельство должно быть ясно доведено
до~сведения как~самого заказчика, так~и~иных пользователей, равно
как~и~тот~факт, что~формат \emph{отчёта об~оценке} не~может
рассматриваться в~отрыве от~условий этих \textsl{договоров}.\label{4.3.1.3-End}

\stepcounter{SubSecCounter}

\thesubsection.\theSubSecCounter.\label{4.3.1.4}\emph{ Оценщик}
вправе предоставить заказчику предварительную консультацию о~стоимости
либо проект \emph{отчёта}, либо предварительные результаты \emph{оценки}
до~момента составления итогового \emph{отчёта об~оценке} (см.~п.~\vref{3.2.3.12}--\pageref{3.2.3.12-End}).
Однако, крайне важно, чтобы в~период ожидания итоговой версии такой
предварительный статус консультации был~бы чётко определён и~обозначен.\label{4.3.1.4-End}

\stepcounter{SubSecCounter}

\thesubsection.\theSubSecCounter.\label{4.3.1.5}\textsl{ Членам
RICS} следует помнить о~том, что~любая консультация по~\textsl{оценке},
предоставленная в~любой форме, несёт в~себе угрозу потенциальной
ответственности перед заказчиком, а, при~определённых обстоятельствах,
также и~перед \textsl{третьими сторонами}. Вследствие этого следует
проявлять большую осмотрительность при~рассмотрении вопроса о~том,
когда и~каким именно образом возникают либо могут возникнуть такие
риски, а~также каков их~потенциальный масштаб (см.~параграф~\ref{subsubsec:4.3.2.16_Limitations_on_liability}
\vpageref{subsubsec:4.3.2.16_Limitations_on_liability}--\pageref{subsubsec:4.3.2.16_Limitations_on_liability-End}).\label{4.3.1.5-End}

\stepcounter{SubSecCounter}

\thesubsection.\theSubSecCounter.\label{4.3.1.6} При~оказании консультаций
в~сфере стоимостного консультирования следует избегать использования
терминов «сертификат стоимости», «сертификат оценки» и~«декларация
стоимости». Тем~не~менее \emph{оценщик} может использовать термин
«заверенная стоимость» или~схожие по~смыслу термины в~тексте \emph{отчёта
об~оценке}, если ему~известно, что~\textsl{оценка} выполняется
для~целей, требующих официального заверения \emph{суждения о~стоимости}
(см.~параграф~\ref{subsubsec:4.1.3.12_Format_of_the_report}~ \vpageref{subsubsec:4.1.3.12_Format_of_the_report}--\pageref{subsubsec:4.1.3.12_Format_of_the_report-End}).\label{4.3.1.6-End}\label{subsec:4.3.1_General_principles-End}

\subsection{Содержание отчёта об~оценке\label{subsec:4.3.2_Report_content}}
\begin{itemize}
\item \label{4.3.2.0.1} В~\emph{отчёте об~оценке} должны содержаться
следующие сведения, являющиеся обязательными к~включению в~\textsl{договор
на~проведение оценки} \emph{(задание на оценку)} согласно требованиям
раздела~\ref{subsec:4.1.1_General_principles}~\nameref{subsec:4.1.1_General_principles}
\vpageref{subsec:4.1.1_General_principles}--\pageref{sec:4.1_VPS1_Terms_of_engagement_Scope_of_work-End}.
Хотя \emph{отчёты об~оценке} часто начинаются с~описания оцениваемого
актива~(обязательства) и~подтверждения \emph{цели проведения оценки},
\emph{оценщикам} настоятельно рекомендуется придерживаться следующего
перечня разделов, обеспечивающего отражение всех существенных вопросов,
подлежащих рассмотрению в~\emph{отчёте об~оценке}.
\begin{enumerate}
\item сведения об~\emph{оценщике} и~его~статусе;
\item сведения о~заказчике и~иных предполагаемых пользователях \emph{отчёта};
\item цель проведения \textsl{оценки};
\item описание оцениваемого актива~(активов) либо обязательства~(обязательств);
\item \textsl{вид~(ы) определяемой стоимости};
\item \textsl{дата оценки};
\item объём исследований;
\item характер и~источники использованных информации и~данных;
\item \textsl{допущения} и специальные допущения;
\item ограничения на~использование, обнародование и~распространение \emph{отчёта};
\item подтверждение факта проведения \textsl{оценки} в~соответствии с~\href{https://www.rics.org/globalassets/rics-website/media/upholding-professional-standards/sector-standards/valuation/international-valuation-standards-rics2.pdf}{МСО$\sim$(IVS)})~\cite{IVS-2020};
\item обоснование выбранного подхода к~\textsl{оценке} и~описание расчётов,
проведённых в~его~рамках;
\item результат~(результаты) \textsl{оценки};
\item дата составления \emph{отчёта};
\item комментарии касательно любой существенной неопределённости в~\textsl{оценке}
в~тех случаях, когда они~необходимы для~обеспечения её~ясного
понимания со~стороны пользователя \emph{отчёта};
\item заявление, содержащее все~ранее согласованные ограничения ответственности.\label{4.3.2.0.1-End}
\end{enumerate}
\item \label{4.3.2.0.2}Каждый из~вышеприведённых разделов далее будет
рассмотрен подробнее. Текст, выделенный \textbf{полужирным} начертанием,
содержит ключевые принципы. В~нижеследующем за~ним~тексте приводятся
уточнения, как~именно следует понимать эти~принципы, и~как~их~следует
реализовывать.\label{4.3.2.0.2-End}
\end{itemize}

\subsubsection{Сведения об~оценщике\label{subsubsec:4.3.2.1_Identification_and_status_of_the_valuer}}

\textbf{\emph{Оценщик}}\textbf{ может осуществлять свою деятельность
как~самостоятельно, так~и~посредством работы в~}\textbf{\textsl{оценочной
компании}}\textbf{. В~обоих случаях }\textbf{\emph{отчёт}}\textbf{
должен содержать:}
\begin{itemize}
\item \textbf{подпись физического лица, ответственного за~проведение }\textbf{\textsl{оценки}}\textbf{;}
\item \textbf{заявление о~том, что~}\textbf{\emph{оценщик}}\textbf{ достаточно
компетентен для~проведения }\textbf{\textsl{оценки}}\textbf{, а~также
в~состоянии провести её~объективно и~беспристрастно.}
\end{itemize}
\textbf{В~случае, если при~проведении }\textbf{\textsl{оценки}}\textbf{
}\textbf{\emph{оценщик}}\textbf{ прибегал к~какому-либо существенному
содействию со~стороны иных лиц, в~}\textbf{\emph{отчёте}}\textbf{
необходимо указать характер такого содействия, а~также степень доверия
к~его~результатам.}

\textbf{Реализация.}

\stepcounter{SubSubSecCounter}

\thesubsubsection.\theSubSubSecCounter.\label{4.3.2.1.1} Ответственность
за~\textsl{оценку} несёт \textsl{член RICS}, являющийся физическим
лицом. RICS не~допускает проведение \textsl{оценки} \textsl{оценочной
компанией} как~таковой, при этом допускается использование формулировки
«\textsl{оценка} выполнена от~имени и~по~поручению \textsl{оценочной
компании}» в~сочетании с~подписью \emph{оценщика} "--- физического
лица.\label{4.3.2.1.1-End}

\stepcounter{SubSubSecCounter}

\thesubsubsection.\theSubSubSecCounter.\label{4.3.2.1.2} Во~всех
случаях необходимо ясно указывать профессиональный статус (например
MRICS) либо иное обозначение соответствующей профессиональной квалификации.\label{4.3.2.1.2-End}

\stepcounter{SubSubSecCounter}

\thesubsubsection.\theSubSubSecCounter.\label{4.3.2.1.3} В~случае
наличия соответствующих особых требований, \emph{оценщик} обязан указывать
действует~ли он~в~качестве \hyperref[Internal_valuer]{внутреннего}
либо \hyperref[External_valuer]{внешнего} \emph{оценщика}
в~определениях, данных в~\hyperref[chap:2_Glossary]{Глоссарии}
на~с.~\pageref{Gloss:Internal_valuer}, \pageref{Gloss:External_valuer}
соответственно. Однако в~определённых случаях в~ряде юрисдикций
могут использоваться другие определения данных терминов, которые в~таких
случаях необходимо приводить в~\textsl{договоре на~проведение оценки}
(при~условии, что~\emph{оценщик} соответствует критериям, приведённым
в~\hyperref[chap:2_Glossary]{Глоссарии}), а~также в~тексте
\emph{отчёта}. В~случае использования других критериев, касающихся
статусов \emph{оценщика}, они~подлежат обязательному включению в~\emph{отчёт}
совместно с~заявлением о~соответствии им~\emph{оценщика}.\label{4.3.2.1.3-End}

\stepcounter{SubSubSecCounter}

\thesubsubsection.\theSubSubSecCounter.\label{4.3.2.1.4} При~рассмотрении
вопроса существенности сотрудничества по~другим договорам между заказчиком
и~\emph{оценщиком} в~прошлом, настоящем или~будущем необходимо
применять положения параграфа~\ref{subsubsec:3.2.5.6_Previous_involvement}~\nameref{subsubsec:3.2.5.6_Previous_involvement}
\vpageref{subsubsec:3.2.5.6_Previous_involvement}--\pageref{subsubsec:3.2.5.6_Previous_involvement-End}.\footnote{Прим.~пер.: в~ориг. англ. тексте вместо ссылки на~параграф \ref{subsubsec:3.2.5.6_Previous_involvement}~\nameref{subsubsec:3.2.5.6_Previous_involvement}
приводится ссылка на~подраздел~\ref{subsec:3.2.8_Responsibility_for_the_valuation}~\nameref{subsec:3.2.8_Responsibility_for_the_valuation}.
По~мнению переводчика, это~является опечаткой.} Любые \textsl{раскрытия информации} и~заявления, имеющиеся в~условиях
\textsl{договора на~проведение оценки} и~составленные в~соответствии
с~п.~\vref{4.1.3.1.3}--\pageref{4.1.3.1.3-End}, должны также
быть воспроизведены в~тексте \emph{отчёта об~оценке}. В~случае,
если между заказчиком \textsl{оценки} и~\emph{оценщиком} отсутствует
какая-либо существенная связь вне~рамок рассматриваемого \textsl{договора
на~проведение оценки}, заявление об~этом также подлежит обязательному
включению в~\emph{отчёт}. В~части, касающейся вопросов разрешения
\textsl{конфликтов интересов}, следует применять положения раздела~\ref{sec:3.2_PS2_Ethics_competency_objectivity}~\nameref{sec:3.2_PS2_Ethics_competency_objectivity}
\vpageref{sec:3.2_PS2_Ethics_competency_objectivity}--\pageref{sec:3.2_PS2_Ethics_competency_objectivity-End}.\label{4.3.2.1.4-End}

\stepcounter{SubSubSecCounter}

\thesubsubsection.\theSubSubSecCounter.\label{4.3.2.1.5} Необходимо
сделать заявление о~том, что~\emph{оценщик} обладает достаточным
пониманием рынка на~региональном, национальном и~международном уровне
(в~зависимости от~того, что~применимо), а~также навыками и~знаниями,
необходимыми для~выполнения \textsl{оценки} на~должном профессиональном
уровне. В~случае, если в~проведении \textsl{оценки} участвовало
несколько \emph{оценщиков}, являющихся сотрудниками \textsl{оценочной
компании}, в~тексте \emph{отчёта} необходимо привести подтверждение
соблюдения требований, установленных п.~\vref{3.2.2.7}--\pageref{3.2.2.7-End},
при~этом отсутствует необходимость сообщать какие-либо подробности
в~данной части.\label{4.3.2.1.5-End}

\stepcounter{SubSubSecCounter}

\thesubsubsection.\theSubSubSecCounter.\label{4.3.2.1.6} 

В~случае, когда \emph{оценщик} включает в~\emph{отчёт} \textsl{оценку},
выполненную другим \emph{оценщиком} либо \textsl{оценочной компанией},
как~на~условиях субподряда, так~и~в~случае привлечения в~качестве
внешнего эксперта по~одному или~нескольким аспектам, следует применять
положения п.~\vref{4.3.2.10.4}--\pageref{4.3.2.10.4-End}.\label{4.3.2.1.6-End}

\stepcounter{SubSubSecCounter}

\thesubsubsection.\theSubSubSecCounter.\label{4.3.2.1.7} В~некоторых
странах и~территориях соответствующие стандарты оценки, специфические
для~данных юрисдикций, могут устанавливать дополнительные требования
в~части \emph{раскрытия информации} об~\emph{оценщике} и~его~статусе.\label{4.3.2.1.7-End}\label{subsubsec:4.3.2.1_Identification_and_status_of_the_valuer-End}

\subsubsection{Сведения о~заказчике и~иных предполагаемых пользователях результатов
проведения оценки\label{subsubsec:4.3.2.2_Identification_of_other_intended_users}}

\textbf{Необходимо привести сведения о~лице, являющимся заказчиком
}\textbf{\textsl{договора на~проведение оценки}}\textbf{, а~также
о~любых иных лицах, являющихся потенциальными пользователями результатов
проведения }\textbf{\textsl{оценки}}\textbf{. См.~также параграф~\ref{subsubsec:4.3.2.10_Restrictions_on_use_of_the_report}~\nameref{subsubsec:4.3.2.10_Restrictions_on_use_of_the_report}
\vpageref{subsubsec:4.3.2.10_Restrictions_on_use_of_the_report}--\pageref{subsubsec:4.3.2.10_Restrictions_on_use_of_the_report-End}.}

\textbf{Реализация.}

\stepcounter{SubSubSecCounter}

\thesubsubsection.\theSubSubSecCounter.\label{4.3.2.2.1}\emph{ Отчёт}
должен быть адресован заказчику \textsl{оценки} либо его~представителям.
В~случае, если в~\textsl{договоре на~проведение оценки} в~качестве
заказчика указана сторона, отличная от~той, которой адресован \emph{отчёт
об~оценке}, сведения о~конечном пользователе подлежат обязательному
раскрытию. Также следует привести сведения обо~всех известных \emph{оценщику}
предполагаемых пользователях результатов \textsl{оценки}.\label{4.3.2.2.1-End}

\stepcounter{SubSubSecCounter}

\thesubsubsection.\theSubSubSecCounter.\label{4.3.2.2.2} В~ряде
случаев \emph{оценщики} не~могут отказаться от~ответственности перед
\textsl{третьими лицами} (см.~подраздел~\ref{subsec:3.2.5_Disclosures_for_public_interest}~\nameref{subsec:3.2.5_Disclosures_for_public_interest}
\vpageref{subsec:3.2.5_Disclosures_for_public_interest}--\pageref{subsec:3.2.5_Disclosures_for_public_interest-End}).
Следует приводить сведения обо~всех ограничениях раскрытия результатов
\textsl{оценки}, основанной на~сведениях либо указаниях, имеющих
ограниченный доступ (см.~параграф~\ref{subsubsec:4.1.3.10_Nature_and_sources_of_information}~\nameref{subsubsec:4.1.3.10_Nature_and_sources_of_information}
\vpageref{subsubsec:4.1.3.10_Nature_and_sources_of_information}--\pageref{subsubsec:4.1.3.10_Nature_and_sources_of_information-End}).\label{4.3.2.2.2-End}\label{subsubsec:4.3.2.2_Identification_of_other_intended_users-End}

\subsubsection{Цель проведения оценки\label{subsubsec:4.3.2.3_Purpose_ of_the_valuation}}

\textbf{Цель проведения }\textbf{\textsl{оценки}}\textbf{ необходимо
указывать в~точном соответствии с~}\textbf{\textsl{договором на~её~проведение}}\textbf{.}

\textbf{Реализация.}

\stepcounter{SubSubSecCounter}

\thesubsubsection.\theSubSubSecCounter.\label{4.3.2.3.1}\emph{ Отчёт
об~оценке} должен не~допускать его~неоднозначного толкования. В~случае,
если \emph{цель проведения оценки} не~разглашается заказчиком, \emph{оценщику}
следует получить разъяснение касательно причин этого. \emph{Отчёт
об~оценке} должен содержать соответствующее заявление, проясняющие
такие обстоятельства.\label{4.3.2.3.1-End}\label{subsubsec:4.3.2.3_Purpose_ of_the_valuation-End}

\subsubsection{Сведения об~оцениваемом активе~(активах) либо обязательстве~(обязательствах)\label{subsubsec:4.3.2.4_Identification_of_the_asset_or_liability}}

\textbf{Оцениваемый актив либо обязательство должны быть чётко идентифицированы.
Может потребоваться разъяснение, разграничивающее такие понятия как~\guillemotleft актив\guillemotright ,
\guillemotleft имущественные права на~актив\guillemotright{} и~\guillemotleft право
использования актива\guillemotright .}

\textbf{В~случае проведения }\textbf{\textsl{оценки}}\textbf{ актива,
используемого совместно с~другими активами, в~их~отношении следует
провести уточнение касательно того, что~они:}
\begin{itemize}
\item \textbf{включены в~периметр }\textbf{\textsl{оценки}}\textbf{ согласно
условиям }\textbf{\textsl{договора на~проведение оценки}}\textbf{;}
\item \textbf{исключены из~периметра }\textbf{\textsl{оценки}}\textbf{,
но~предполагается их~доступность;}
\item \textbf{исключены из~периметра оценки, и~предполагается их~недоступность.}
\end{itemize}
\textbf{В~случае проведения }\textbf{\textsl{оценки}}\textbf{ в~отношении
доли в~активе или~обязательстве, необходимо описать взаимосвязь
оцениваемой доли со~всеми другими долями в~праве на~актив~(обязательство),
а~также права и~обязанности, существующие между всеми владельцами
долей на~них.}

\textbf{Необходимо уделять особое внимание описанию портфелей, коллекций
и~групп активов. Крайне важно рассмотреть вопросы, касающиеся группировки
активов, определения различных категорий имущества либо активов, а~также
любые }\textbf{\textsl{допущения}}\textbf{ либо }\textbf{\textsl{особые
допущения}}\textbf{, касающиеся обстоятельств, при~которых имущество,
активы, обязательства либо их~группы могут быть выведены на~открытый
рынок.}

\textbf{Реализация.}

\stepcounter{SubSubSecCounter}

\thesubsubsection.\theSubSubSecCounter.\label{4.3.2.4.1} В~отношении
каждого актива либо обязательства необходимо указать существующие
права на~него. Крайне важно привести разъяснение, проводящее разграничение
между активом как~таковым и~конкретным подлежащем \textsl{оценке}
правом на~него. В~случае, если актив представляет собой недвижимое
имущество, необходимо указать степень его~арендной загрузки, а~также
возможность его~освобождения от~арендаторов (в~случае наличия такой
необходимости).\label{4.3.2.4.1-End}

\stepcounter{SubSubSecCounter}

\thesubsubsection.\theSubSubSecCounter.\label{4.3.2.4.2} Если активы
расположены более чем~в~одной стране или~территории, необходимо
привести список активов по~каждой из~них, также, как~правило, осуществляется
группировка активов, исходя из~их~расположения по~странам. При
этом в~отношении каждого актива указываются существующие на~него
права.\label{4.3.2.4.2-End}

\stepcounter{SubSubSecCounter}

\thesubsubsection.\theSubSubSecCounter.\label{4.3.2.4.3} В~случае,
когда условиями \textsl{договора на~проведение оценки} предусмотрен
раздельный учёт активов, основанный на~критериях их~назначения,
типа и~вида, их~описание в~\emph{отчёте об~оценке} должно быть
структурировано соответствующим образом.\label{4.3.2.4.3-End}

\stepcounter{SubSubSecCounter}

\thesubsubsection.\theSubSubSecCounter.\label{4.3.2.4.4} В~случае
возникновения сомнений относительно того, куда следует относить тот~или~иной
актив либо имущество, \emph{оценщику}, как~правило, следует группировать
их~таким образом, который в~наибольшей степени соответствует той~группировке,
которая~бы имела место при~продаже этого имущества~(передаче прав
на~него). Однако \emph{оценщику} следует обсудить это~с~заказчиком
и~в~обязательном порядке получить от~него одобрение на~применение
способа группировки, описанного в~\textsl{договоре на~проведение
оценки} и~\emph{отчёте об~оценке}. Дополнительные рекомендации по~\textsl{оценке}
портфелей, коллекций и~групп активов содержатся в~разделе~\ref{sec:5.9_VPGA-9_Identification_of_portfolios}~\nameref{sec:5.9_VPGA-9_Identification_of_portfolios}
\vpageref{sec:5.9_VPGA-9_Identification_of_portfolios}--\pageref{sec:5.9_VPGA-9_Identification_of_portfolios-End}.\label{4.3.2.4.4-End}\label{subsubsec:4.3.2.4_Identification_of_the_asset_or_liability-End}

\subsubsection{Вид(ы) определяемой стоимости\label{subsubsec:4.3.2.5_Basis_of_value}}

\textbf{\textsl{Вид стоимости}}\textbf{ должен соответствовать }\textbf{\emph{цели
проведения оценки}}\textbf{. Необходимо привести ссылку на~источник
определения используемого }\textbf{\textsl{вида стоимости}}\textbf{
либо разъяснить его~суть в~тексте самого }\textbf{\emph{отчёта об~оценке}}\textbf{. }

\textbf{Данное требование не~применяется в~случае оказания услуги
по~рецензированию другой }\textbf{\textsl{оценки}}\textbf{ в~случаях,
когда не~требуется вынесение нового }\textbf{\emph{суждения о~стоимости}}\textbf{
либо формирования мнения относительно правильности выбора использованного
}\textbf{\textsl{вида стоимости}}\textbf{.}

\textbf{Реализация.}

\stepcounter{SubSubSecCounter}

\thesubsubsection.\theSubSubSecCounter.\label{4.3.2.5.1} В~\emph{отчёте
об~оценке} необходимо приводить полные данные о~\textsl{виде определяемой
в~нём~стоимости}, а~также его~определение (без~описания всей
его~концепции и~приведения иных материалов, разъясняющих его~суть).\label{4.3.2.5.1-End}

\stepcounter{SubSubSecCounter}

\thesubsubsection.\theSubSubSecCounter.\label{4.3.2.5.2} Если условиями
\textsl{договора на~проведение оценки} не~предусмотрено иное, от~\emph{оценщика}
не~требуется проведение \textsl{оценки} стоимости отличной от~\textsl{рыночной}.
Однако в~тех~случаях, когда \textsl{вид~определяемой стоимости}
не~основан на~количественных данных открытого рынка, а~методика
\textsl{оценки} существенно отличается от~методики определения \textsl{рыночной
стоимости}, может быть уместно приведение разъяснений в~этой части.
Это~требуется для~того, чтобы обеспечить понимание читателем \emph{отчёта
об~оценке} того факта, что, хотя результаты \textsl{оценки} и~применимы
для~заявленных в~нём~целей, они~могут не~соотноситься с~той~стоимостью,
которая была~бы получена в~случае продажи оцениваемого актива либо
обязательства на~открытом рынке.\label{4.3.2.5.2-End}

\stepcounter{SubSubSecCounter}

\thesubsubsection.\theSubSubSecCounter.\label{4.3.2.5.3} В~исключительных
случаях, когда \textsl{оценка} проводится на~будущую дату, данное
обстоятельство должно быть чётко отражено в~тексте \emph{отчёта}
(см.~параграф~\ref{subsubsec:4.3.2.6_Valuation_date} \vpageref{subsubsec:4.3.2.6_Valuation_date}--\pageref{subsubsec:4.3.2.6_Valuation_date-End},
а~также п\@.~\vref{4.4.1.5}--\vref{4.4.1.5-End}). Отражение
данного обстоятельства следует приводить вместе с~подтверждением
того, что~такая \textsl{оценка} приемлема с~точки зрения требований
национальных стандартов оценки, а~также \href{https://www.isurv.com/info/1342/rics_national_or_jurisdictional_valuation_standards}{Стандартов национальных ассоциаций}~\cite{RICS:National-Standards}.
Прогнозная стоимость может иметь различные формы и, как правило, не~образует
самостоятельный \textsl{вид стоимости}. Поскольку прогнозная стоимость
в~значительной степени основывается на~\textsl{специальных допущениях},
которые могут как~реализоваться в~реальности, так~и~нет, она~имеет
другой характер нежели стоимость, определённая на~текущую либо прошлую
дату, вследствие чего нельзя говорить о~её~равнозначности относительно
них. В~частности такую стоимость никогда нельзя называть \textsl{рыночной
стоимостью} без~соответствующих комментариев.\label{4.3.2.5.3-End}\label{subsubsec:4.3.2.5_Basis_of_value-End}

\subsubsection{Дата оценки\label{subsubsec:4.3.2.6_Valuation_date}}

\textbf{\textsl{Дата оценки}}\textbf{ может отличаться от~даты выпуска
}\textbf{\emph{отчёта}}\textbf{, а~также от~даты проведения либо
завершения исследований. Необходимо проводить чёткое разграничение
между этими датами там, где~это~требуется.}

\textbf{Требование данного параграфа не~распространяются на~услуги
по~рецензированию }\textbf{\emph{отчётов об~оценке}}\textbf{ в~случаях,
когда от~рецензента не~требуется формирование мнения относительно
корректности использованной }\textbf{\textsl{даты оценки}}\textbf{.}

\textbf{Реализация.}

\stepcounter{SubSubSecCounter}

\thesubsubsection.\theSubSubSecCounter.\label{4.3.2.6.1} В~\emph{отчёте
об~оценке} необходимо указывать \textsl{дату оценки} (см.~параграф~\ref{subsubsec:4.1.3.8_Valuation_date}
\vpageref{subsubsec:4.1.3.8_Valuation_date}--\pageref{subsubsec:4.1.3.8_Valuation_date-End}).\label{4.3.2.6.1-End}

\stepcounter{SubSubSecCounter}

\thesubsubsection.\theSubSubSecCounter.\label{4.3.2.6.2} В~случае,
если в~период, прошедший между \textsl{датой оценки} (если она~предшествует
дате составления \emph{отчёта}) и~датой составления \emph{отчёта},
произошло существенное изменение рыночных условий или~свойств имущества,
актива или~портфеля, \emph{оценщик} должен выразить своё отношение
к~данному обстоятельству. В~ряде случаев также бывает уместно обратить
внимание заказчика на~тот~факт, что~стоимость имеет свойство изменяться
с~течением времени, вследствие и~по~причине чего \textsl{оценка},
выполненная на~конкретную дату, может быть недействительной по~состоянию
на~более раннюю либо более позднюю дату.\label{4.3.2.6.2-End}

\stepcounter{SubSubSecCounter}

\thesubsubsection.\theSubSubSecCounter.\label{4.3.2.6.3} В~случае
предоставления результатов \textsl{оценки} на~будущую дату, следует
проявить дополнительную осмотрительность в~вопросах информирования
заказчика о~том, что~фактическая стоимость (любого вида) может и,
скорее всего, будет отличаться от~них в~случае изменения рыночного
окружения либо свойств \emph{объекта оценки} относительно тех, которые
были указаны в~\emph{отчёте} в~качестве \textsl{специальных допущений}.
См.~также параграф~\ref{subsubsec:4.3.2.5_Basis_of_value} \vpageref{subsubsec:4.3.2.5_Basis_of_value}--\pageref{subsubsec:4.3.2.5_Basis_of_value-End}.\label{4.3.2.6.3-End}\label{subsubsec:4.3.2.6_Valuation_date-End}

\subsubsection{Объём исследований\label{subsubsec:4.3.2.7_Extent_of_investigation}}

\textbf{Сведения об~объёме проведённых исследований, включая ограничения
на~их~проведение, установленные условиями }\textbf{\textsl{договора
на~проведение оценки}}\textbf{ (}\textbf{\emph{заданием на~оценку}}\textbf{)
подлежат обязательному раскрытию в~тексте }\textbf{\emph{отчёта}}\textbf{.}

\textbf{Реализация.}

\stepcounter{SubSubSecCounter}

\thesubsubsection.\theSubSubSecCounter.\label{4.3.2.7.1} В случаях,
когда \emph{объектом оценки} является недвижимое имущество~(права
на~него), в~\emph{отчёте об~оценке} необходимо привести сведения
о~дате и~обстоятельствах проведения \textsl{осмотра}, включая данные
об~отсутствии доступа к~какой-либо части исследуемого объекта (в~случае
наличия таких фактов). (См.~также раздел~\ref{sec:4.2_VPS2_Inspections_investigations_and_records}~\nameref{sec:4.2_VPS2_Inspections_investigations_and_records}).
В~отношении \textsl{личного имущества}, имеющего материальную форму,
следует применять аналогичные принципы с~учётом его~класса и~особенностей.\label{4.3.2.7.1-End}

\stepcounter{SubSubSecCounter}

\thesubsubsection.\theSubSubSecCounter.\label{4.3.2.7.2} В~случаях,
когда отсутствовала возможность проведения \textsl{осмотра} либо иного
обследования \emph{объекта оценки} надлежащим образом, \emph{оценщик}
обязан привести в~\emph{отчёте} указание на~данное обстоятельство
(см.~п.~\ref{4.2.1.2}, \ref{4.2.1.7} \vpageref{4.2.1.2}--\pageref{4.2.1.2-End},
\pageref{4.2.1.7}--\pageref{4.2.1.7-End} соответственно).\label{4.3.2.7.2-End}

\stepcounter{SubSubSecCounter}

\thesubsubsection.\theSubSubSecCounter.\label{4.3.2.7.3} В~случае
проведения переоценки в~тексте \emph{отчёта} следует приводить ссылку
на~соглашение с~заказчиком, устанавливающее необходимость и~частоту
проведения повторного \textsl{осмотра} (см.~раздел~\ref{sec:4.2_VPS2_Inspections_investigations_and_records}~\nameref{sec:4.2_VPS2_Inspections_investigations_and_records}
\vpageref{sec:4.2_VPS2_Inspections_investigations_and_records}--\pageref{sec:4.2_VPS2_Inspections_investigations_and_records-End}).\label{4.3.2.7.3-End}

\stepcounter{SubSubSecCounter}

\thesubsubsection.\theSubSubSecCounter.\label{4.3.2.7.4} При~проведении
\textsl{оценки} значительного числа объектов допускается обобщённое
описание аспектов \textsl{осмотра} при~условии, что~оно~не~приведёт
к~введению в~заблуждение пользователя \emph{отчёта}.\label{4.3.2.7.4-End}

\stepcounter{SubSubSecCounter}

\thesubsubsection.\theSubSubSecCounter.\label{4.3.2.7.5} В~случаях,
когда \emph{объект оценки} не~является недвижимым имуществом либо
\textsl{личной собственностью}, имеющей материальную форму, следует
проявлять особую внимательность при~описании возможных способов его~обследования.\label{4.3.2.7.5-End}

\stepcounter{SubSubSecCounter}

\thesubsubsection.\theSubSubSecCounter.\label{4.3.2.7.6} При~проведении
\textsl{оценки} на~основе ограниченных сведений либо при~проведении
переоценки без~\textsl{осмотра}, в~\emph{отчёте об~оценке} должно
приводиться исчерпывающее описание этих обстоятельств (см.~также
параграф~\ref{subsubsec:4.1.3.9_Nature_and_extent_of_the_valuer=002019s_work}
\vpageref{subsubsec:4.1.3.9_Nature_and_extent_of_the_valuer=002019s_work}--\pageref{subsubsec:4.1.3.9_Nature_and_extent_of_the_valuer=002019s_work-End}).\label{4.3.2.7.6-End}\label{subsubsec:4.3.2.7_Extent_of_investigation-End}

\subsubsection{Используемые оценщиком информация и~данные и~их~источники\label{subsubsec:4.3.2.8_Nature_and_source_of_the_information}}

\textbf{Необходимо приводить сведения о~характере и~источниках всех
сведений, использованных при~проведении }\textbf{\emph{оценки}}\textbf{,
а~также степени глубины их~проверки. Также приводятся сведения о~факте
использования сведений, предоставленных заказчиком либо иной стороной,
без~их~проверки со~стороны }\textbf{\emph{оценщика}}\textbf{ с~указанием
ссылки на~предоставленные материалы. В~контексте данного параграфа
термин «информация» следует понимать как~набор исходных данных (но~не~инструкций).}\footnote{Прим.~пер: вопрос отличия терминов \guillemotleft информация\guillemotright{}
и~\guillemotleft данные\guillemotright{} был рассмотрен в~п.~\ref{3.2.1.5}\vpageref{3.2.1.5}\pageref{3.2.1.5-End}.
С~учётом данного контекста здесь, ранее и~далее термин \foreignlanguage{english}{\guillemotleft information\guillemotright}
чаще всего переводится как~\guillemotleft сведения\guillemotright{}
либо \guillemotleft данные\guillemotright .}

\textbf{Реализация.}

\stepcounter{SubSubSecCounter}

\thesubsubsection.\theSubSubSecCounter.\label{4.3.2.8.1} Если какие-либо
\uline{данные}, используемые при~проведении \textsl{оценки}, были
предоставлены заказчиком, \emph{оценщик} обязан указать на~то~обстоятельство,
что~это~следует из~условий \textsl{договора на~проведение оценки}
(см.~раздел~\ref{sec:4.1_VPS1_Terms_of_engagement_Scope_of_work}~\nameref{sec:4.1_VPS1_Terms_of_engagement_Scope_of_work}
\vpageref{sec:4.1_VPS1_Terms_of_engagement_Scope_of_work}--\pageref{sec:4.1_VPS1_Terms_of_engagement_Scope_of_work-End}),
а~также привести ссылку на~\uline{их}~источник. В~любом случае
\emph{оценщик} обязан иметь собственное мнение относительно достоверности
таких данных, а~также вопроса необходимости принятия разумных мер
по~их~проверке.\label{4.3.2.8.1-End}

\stepcounter{SubSubSecCounter}

\thesubsubsection.\theSubSubSecCounter.\label{4.3.2.8.2} В~случае,
если \textsl{оценка} проводилась без~тех~данных, которые обычно
являются доступными (либо к~которым обычно может быть получен доступ),
данное обстоятельство необходимо отражать в~тексте \emph{отчёта}.
Также в~тексте \emph{отчёта} необходимо отражать потребность в~проверке
каких-либо использованных в~нём данных либо \textsl{допущений} (при~наличии
возможности таких проверок), а~также факт непредставления какой-либо
существенной их части.\label{4.3.2.8.2.-End}

\stepcounter{SubSubSecCounter}

\thesubsubsection.\theSubSubSecCounter.\label{4.3.2.8.3} В~случаях,
когда подобные данные либо \textsl{допущение} оказывают существенное
влияние на~стоимость \emph{объекта оценки}, текст \emph{отчёта об~оценке}
должен содержать положение о~том, что~результаты \textsl{оценки}
не~могут считаться заслуживающими доверия без~проведения соответствующей
проверки (см.~параграф~\ref{subsubsec:4.1.3.10_Nature_and_sources_of_information}
\vpageref{subsubsec:4.1.3.10_Nature_and_sources_of_information}--\pageref{subsubsec:4.1.3.10_Nature_and_sources_of_information-End}).
В~случае проведения переоценки в~\emph{отчёт об~оценке} следует
включать справку об~отсутствии существенных изменений свойств \emph{объекта
оценки}, составленную заказчиком, либо \textsl{допущение} о~том,
что~подобные изменения не~имели места.\label{4.3.2.8.3-End}

\stepcounter{SubSubSecCounter}

\thesubsubsection.\theSubSubSecCounter.\label{4.3.2.8.4} Заказчик
может ожидать от~\emph{оценщика} выражения последним~своей позиции
по~юридическим вопросам, оказывающим влияние на~результаты \textsl{оценки},
а~\emph{оценщик}, в~свою очередь, может иметь такое желание. В~этом
случае, \emph{оценщик} обязан указать перечень сведений, требующих
обязательной проверки со~стороны юристов заказчика либо иной заинтересованной
стороны, прежде чем~результаты \textsl{оценки} можно будет считать
достаточно надёжными (например для~их~публичного обнародования).\label{4.3.2.8.4-End}

\stepcounter{SubSubSecCounter}

\thesubsubsection.\theSubSubSecCounter.\label{4.3.2.8.5} Любые изначально
доступные \emph{оценщику} либо собранные им~в~процессе проведения
\textsl{оценки} сведения, способные, по~его~мнению, существенным
образом улучшить восприятие \emph{отчёта об~оценке} и~повысить его~пользу
для~заказчика в~контексте цели проведения \textsl{оценки}, подлежат
включению в~\emph{отчёт об~оценке}.\label{4.3.2.8.5-End}\label{subsubsec:4.3.2.8_Nature_and_source_of_the_information-End}

\subsubsection{Допущения и~специальные допущения\label{subsubsec:4.3.2.9_Assumptions_and_special_assumptions}}

\textbf{Все~}\textbf{\textsl{допущения}}\textbf{ и~}\textbf{\textsl{специальные
допущения}}\textbf{ в~обязательном порядке приводятся в~}\textbf{\emph{отчёте
об~оценке}}\textbf{.}

\textbf{Реализация.}

\stepcounter{SubSubSecCounter}

\thesubsubsection.\theSubSubSecCounter.\label{4.3.2.9.1} Все~\textsl{допущения}
и~\textsl{специальные допущения} в~обязательном порядке приводятся
в~тексте \emph{отчёта} в~полном объёме и~со~всеми необходимыми
оговорками, а~также с~указанием на~то~обстоятельство, что~они~были
согласованы с~заказчиком. В~разделе \emph{отчёта об~оценке}, содержащем
его~выводы, равно как~и~в~предоставляемом отдельно от~него сопроводительном
письме (если предоставление такового имеет место) должны быть приведены
все~использованные \textsl{специальные допущения}. В~случае наличия
различий в~применяемых \textsl{допущениях} в~зависимости от~страны
или~территории, данное обстоятельство отражается в~\emph{отчёте}.\label{4.3.2.9.1-End}\label{subsubsec:4.3.2.9_Assumptions_and_special_assumptions-End}

\subsubsection{Ограничения на~использование, распространение и~публикацию отчёта
об~оценке\label{subsubsec:4.3.2.10_Restrictions_on_use_of_the_report}}

\textbf{В~случае наличия необходимости ограничить варианты применения
результатов }\textbf{\textsl{оценки}}\textbf{ либо круг её~пользователей,
данное обстоятельство должно быть изложено в~тексте }\textbf{\emph{отчёта}}\textbf{.}

\textbf{Реализация.}

\stepcounter{SubSubSecCounter}

\thesubsubsection.\theSubSubSecCounter.\label{4.3.2.10.1}\emph{
Оценщик} в~обязательном устанавливает и~указывает разрешённые варианты
использования, распространения и~публикации \emph{отчёта об~оценке}.
\label{4.3.2.10.1-End}

\stepcounter{SubSubSecCounter}

\thesubsubsection.\theSubSubSecCounter.\label{4.3.2.10.2} Если цели
\emph{отчёта об~оценке} предполагают публикацию ссылки на~него,
\emph{оценщик} обязан подготовить сопроводительный текст для~такой
ссылки, который в~т.\,ч.~может быть подготовлен в~форме отдельного
документа, прилагаемого к~\emph{отчёту}.\label{4.3.2.10.2-End}

\stepcounter{SubSubSecCounter}

\thesubsubsection.\theSubSubSecCounter.\label{4.3.2.10.3}\emph{
Отчёт} может быть опубликован полностью (например в~качестве приложения
к~годовой бухгалтерской~(налоговой) отчётности, но, как~правило,
приводится только ссылка на~него. В~этом случае важно, чтобы \emph{оценщик}
принимал непосредственное участие в~составлении сопроводительного
текста к~публикации в~целях обеспечения точности содержания ссылок
на~\emph{отчёт} и~устранения риска введения в~заблуждение читателя.
Это~становится особенно важным тогда, когда на~\emph{оценщика} возлагают
ответственность за~публикации, касающиеся выполненного им~\emph{отчёта,}
либо какую-либо их~часть.\label{4.3.2.10.3-End}

\stepcounter{SubSubSecCounter}

\thesubsubsection.\theSubSubSecCounter.\label{4.3.2.10.4} Если полный
текст \emph{отчёта} не~предназначен для~публичного обнародования,
проект сопроводительного текста должен быть подготовлен как~отдельный
документ и~предоставлен заказчику вместе с~\emph{отчётом}. Содержание
такого документа может являться предметом регулирования со~стороны
государственных органов, но~во~всех случаях он~должен содержать
следующие минимальные сведения:
\begin{itemize}
\item инициалы \emph{оценщика} и~сведения о~его~квалификации либо аналогичная
сведения относительно \textsl{оценочной компании};
\item данные о~том, является~ли \emph{оценщик} \hyperref[Internal_valuer]{внутренним}
либо \hyperref[External_valuer]{внешним}, а~также, в~случае
необходимости, сведения о~соответствии критериям, специфичным для
каждого варианта;
\item \textsl{дата оценки} и~\textsl{вид определённой в~отчёте стоимости}
совместно со~\textsl{специальными допущениями};
\item сведения о~том, в~какой степени стоимость была определена на~основе
данных, непосредственно наблюдаемых на~открытом рынке, а~в~какой
"--- с~помощью иных методов;
\item подтверждение соответствия \textsl{оценки} требованиям настоящих стандартов
либо причину(ы) и~степень отступления от~них;
\item сведения о~частях \emph{отчёта}, подготовленных другим \emph{оценщиком}
либо иным специалистом.\label{4.3.2.10.4-End}
\end{itemize}
\stepcounter{SubSubSecCounter}

\thesubsubsection.\theSubSubSecCounter.\label{4.3.2.10.5} В~случае
проведения \textsl{оценки}, затрагивающей общественные интересы, либо
предполагающей возможность её~использования лицами "--- отличными
от~заказчика, для~которого предназначен \emph{отчёт}, \emph{оценщик}
обязан делать дополнительное \emph{раскрытие информации} в~тексте
\emph{отчёта} и~любых опубликованных ссылках на~него. Данное требование
следует из~подраздела~\ref{subsec:3.2.5_Disclosures_for_public_interest}~\nameref{subsec:3.2.5_Disclosures_for_public_interest}
\vpageref{subsec:3.2.5_Disclosures_for_public_interest}--\pageref{subsec:3.2.5_Disclosures_for_public_interest-End}.\label{4.3.2.10.5-End}

\stepcounter{SubSubSecCounter}

\thesubsubsection.\theSubSubSecCounter.\label{4.3.2.10.6} Под~«публикацией
\emph{отчёта}» не~следует понимать предоставление его~самого либо
результатов определения стоимости ипотечному заёмщику либо его~кредитору.\label{4.3.2.10.6-End}

\stepcounter{SubSubSecCounter}

\thesubsubsection.\theSubSubSecCounter.\label{4.3.2.10.7}\emph{
Оценщик} обязан проверить точность любого другого подлежащего публикации
материала, относящегося к~\textsl{оценке} либо к~оцениваемому имуществу.\label{4.3.2.10.7-End}

\stepcounter{SubSubSecCounter}

\thesubsubsection.\theSubSubSecCounter.\label{4.3.2.10.8}\emph{
Оценщику} также рекомендуется полностью прочесть документ, в~составе
которого публикуется \emph{отчёт} либо ссылка на~него. Это~необходимо
для~того, чтобы убедиться в~отсутствии искажения мнений по~тем~вопросам,
о~которых \emph{оценщик} имеет представление.\label{4.3.2.10.8-End}

\stepcounter{SubSubSecCounter}

\thesubsubsection.\theSubSubSecCounter.\label{4.3.2.10.9}\emph{
}\emph{\uline{Оценщик}} должен настоять на~получении итоговой
версии документа либо ссылки на~него до~момента их~публикации и~приложить
их~к~своему письму, содержащему \uline{его}~согласие на~публикацию.
Следует пресекать любое давление со~стороны \textsl{третьих лиц}
равно как~и~предложения о~делегировании права подписи такого согласия.\label{4.3.2.10.9-End}

\stepcounter{SubSubSecCounter}

\thesubsubsection.\theSubSubSecCounter.\label{4.3.2.10.10} Допускается
исключение из~текста \emph{отчёта}, публикуемого в~полном виде,
сведений, имеющих статус конфиденциальных, в~силу наличия соответствующих
требований законодательства отдельных стран и~территорий.\label{4.3.2.10.10-End}

\stepcounter{SubSubSecCounter}

\thesubsubsection.\theSubSubSecCounter.\label{4.3.2.10.11} Выраженное
мнение о~стоимости в~случае его~включения в~опубликованный документ,
может оказать некоторое влияние на~предмет спора или~переговоров
либо на~аспекты определённых взаимоотношений между собственником
актива и~\textsl{третьими сторонами} (например мнение о~величине
арендной платы либо стоимости касательно недвижимости, в~отношении
которой предстоит пересмотр величины арендной платы). Также \emph{отчёт}
может содержать обычно не~разглашаемые сведения о~деятельности компании.
Такие сведения являются коммерческой тайной, и~заказчик, на~основании
консультаций с~аудитором и~регуляторами, в~обязательном порядке
принимает решение об~их~включении либо невключении в~публикацию.\label{4.3.2.10.11-End}

\stepcounter{SubSubSecCounter}

\thesubsubsection.\theSubSubSecCounter.\label{4.3.2.10.12} В~сопроводительном
тексте к~публикации~(ссылке) \emph{оценщик} обязан привести данные
о~том, какие именно сведения были исключены, а~также указать на~то~обстоятельство,
что~данные действия были выполнены по~указанию заказчика и~с~одобрения
регулятора и~(или) аудитора. В~случае отсутствия подобного разъяснения
\emph{оценщик} может невольно оказаться в~ситуации, когда в~его~адрес
будет высказана необоснованная критика.\label{4.3.2.10.12-End}

\stepcounter{SubSubSecCounter}

\thesubsubsection.\theSubSubSecCounter.\label{4.3.2.10.13} В~тех
случаях, когда не~публикуется полный текст \emph{отчёта}, сопроводительный
текст должен содержать сведения обо~всех \textsl{специальных допущениях},
а~также дополнительно проведённых \textsl{оценках}. Аналогичным образом
следует приводить достаточные сведения обо~всех допущенных \textsl{отступлениях}.\label{4.3.2.10.13-End}

\stepcounter{SubSubSecCounter}

\thesubsubsection.\theSubSubSecCounter.\label{4.3.2.10.14} Во~всех
случаях определение степени достаточности сопроводительных сведений
является прерогативой \emph{оценщика} и~относится к~сфере его~ответственности.
Такие сведения не~могут считаться достаточными, если в~них~не~раскрыты
вопросы, имеющие принципиальное значение для~понимания \textsl{вида
определённой стоимости}, а~также результата \textsl{оценки}, либо,
если они~не~содержат данных, достаточных для~того, чтобы исключить
риск введения в~заблуждение читателя.\label{4.3.2.10.14-End}

\stepcounter{SubSubSecCounter}

\thesubsubsection.\theSubSubSecCounter.\label{4.3.2.10.15} Предполагается,
что, как~правило, \emph{оценщик} не~даёт согласия на~публикацию
прогнозной стоимости. В~ тех~исключительных случаях, когда согласие
на~это~всё~же~даётся, ему следует проявить большую осмотрительность
и~бдительность в~вопросах точного воспроизведения всех связанных
с~ней оговорок и~сделать соответствующее заявление об~отказе от~ответственности.\label{4.3.2.10.15-End}\label{subsubsec:4.3.2.10_Restrictions_on_use_of_the_report-End}

\subsubsection{Подтверждение факта проведения оценки в~соответствии с~МСО\label{subsubsec:4.3.2.11_Comfirmation_accordance_with_IVS}}

\emph{Оценщик} приводит заявление о~том, что:
\begin{itemize}
\item \textbf{\textsl{оценка}}\textbf{ была проведена в~соответствии \href{https://www.rics.org/globalassets/rics-website/media/upholding-professional-standards/sector-standards/valuation/international-valuation-standards-rics2.pdf}{с Международными стандартами оценки (МСО 2020)}~\cite{IVS-2020},
а~все~существенные для~её~проведения исходные данные были проанализированы
}\textbf{\emph{оценщиком}}\textbf{ и~признаны подходящими для~проведения
}\textbf{\textsl{оценки}}
\end{itemize}
\textbf{либо (в~зависимости от~конкретных требований заказчика)}
\begin{itemize}
\item \textbf{подтверждение того, что~}\textbf{\textsl{оценка}}\textbf{
была выполнена в~соответствии с~требованиями }\textbf{\emph{Всемирных
стандартов оценки RICS}}\textbf{, включающих в~себя \href{https://www.rics.org/globalassets/rics-website/media/upholding-professional-standards/sector-standards/valuation/international-valuation-standards-rics2.pdf}{МСО}~\cite{IVS-2020}
и, там~где~это~уместно, \href{https://www.isurv.com/info/1342/rics_national_or_jurisdictional_valuation_standards}{стандартов национальных ассоциаций RICS}~\cite{RICS:National-Standards},
а~также дополнений для~отдельных юрисдикций. Там~где~это~уместно,
данное утверждение может быть сокращено до~одной ссылки на~использование
}\textbf{\emph{Всемирных стандартов RICS}}.
\end{itemize}
\textbf{В~обоих случаях необходимо включить сопроводительный текст,
а~также обоснование любых }\textbf{\textsl{отступлений}}\textbf{
от~требований \href{https://www.rics.org/globalassets/rics-website/media/upholding-professional-standards/sector-standards/valuation/international-valuation-standards-rics2.pdf}{МСО}~\cite{IVS-2020}
и~настоящих }\textbf{\emph{Всемирных стандартов}}\textbf{. }\textbf{\textsl{Отступление}}\textbf{
не~может считаться оправданным, если оно~приводит к~тому, что~}\textbf{\textsl{оценка}}\textbf{
начинает вводить в~заблуждение.}

\textbf{Реализация.}

\stepcounter{SubSubSecCounter}

\thesubsubsection.\theSubSubSecCounter.\label{4.3.2.11.1} Между
двумя вышеприведёнными формами подтверждения не~существует какой-либо
значимой разницы "--- выбор между ними должен основываться на~требованиях
конкретного \emph{задания на~оценку}. Со~стороны ряда заказчиков
может поступить прямое требование предоставить им~подтверждение того,
что~\textsl{оценка} была проведена в~соответствии с~\href{https://www.rics.org/globalassets/rics-website/media/upholding-professional-standards/sector-standards/valuation/international-valuation-standards-rics2.pdf}{МСО}~\cite{IVS-2020},
которое, естественно, должно быть предоставлено. Во~всех остальных
случаях приведение сведений о~том, что~\textsl{оценка} была проведена
в~соответствии с~требованиями настоящих \emph{Всемирных стандартов
RICS} подразумевает её~соответствие как~\href{https://www.rics.org/globalassets/rics-website/media/upholding-professional-standards/sector-standards/valuation/international-valuation-standards-rics2.pdf}{МСО}~\cite{IVS-2020},
так~и~профессиональным стандартам RICS.\label{4.3.2.11.1-End}

\stepcounter{SubSubSecCounter}

\thesubsubsection.\theSubSubSecCounter.\label{4.3.2.11.2} Ссылку
на~стандарты оценки RICS без~указания года их~издания следует трактовать
как~ссылку на~версию стандартов, действующих на~\textsl{дату оценки}
при~условии, что~она~соответствует дате выпуска \emph{отчёта} либо
предшествует ей.\label{4.3.2.11.2-End}

\stepcounter{SubSubSecCounter}

\thesubsubsection.\theSubSubSecCounter.\label{4.3.2.11.3} В~тексте
заявления о~соответствии следует акцентировать внимание на~любых
\textsl{отступлениях} (см.~подраздел~\ref{subsec:3.1.6_Departures}~\nameref{subsec:3.1.6_Departures}
\vpageref{subsec:3.1.6_Departures}--\pageref{subsec:3.1.6_Departures-End}).
В~случае, если допущенные \textsl{отступления} не~были необходимы,
нельзя говорить о~соответствии \textsl{оценки} требованиям \href{https://www.rics.org/globalassets/rics-website/media/upholding-professional-standards/sector-standards/valuation/international-valuation-standards-rics2.pdf}{МСО}~\cite{IVS-2020}.\label{4.3.2.11.3-End}

\stepcounter{SubSubSecCounter}

\thesubsubsection.\theSubSubSecCounter.\label{4.3.2.11.4} В~случаях,
когда применялись стандарты, специфичные для~конкретной юрисдикции,
возможно включить официальное заявление о~соответствии требованиям
таких стандартов.\label{4.3.5.11.4-End}

\stepcounter{SubSubSecCounter}

\thesubsubsection.\theSubSubSecCounter.\label{4.3.2.11.5} В~случае,
если \emph{оценщик} включает в~свой \emph{отчёт} \textsl{оценку},
выполненную другим \emph{оценщиком} либо \textsl{оценочной компанией}
"--- как~в~роли субподрядчика, так~и~в~качестве стороннего эксперта,
"--- необходимо подтверждение того, что~такая \textsl{оценка} была
выполнена в~соответствии с~данными стандартами либо иными, применимыми
в~конкретных обстоятельствах.\label{4.3.2.11.5-End}

\stepcounter{SubSubSecCounter}

\thesubsubsection.\theSubSubSecCounter.\label{4.3.2.11.6} От~\emph{оценщика}
могут потребовать включить в\emph{~отчёт} \textsl{оценку}, предоставленную
ему~непосредственно заказчиком. В~таких случаях \emph{оценщику}
необходимо убедиться в~том, что~\textsl{она}~соответствует настоящим
стандартам.\label{4.3.2.11.6-End}\label{subsubsec:4.3.2.11_Comfirmation_accordance_with_IVS-End}

\subsubsection{Подходы к~оценке и~аргументация\label{subsubsec:4.3.2.12_Valuation_approach_and_reasoning}}

\textbf{Для~обеспечения понимания результатов }\textbf{\textsl{оценки}}\textbf{
в~контексте }\textbf{\emph{отчёта}}\textbf{ в~его~тексте необходимо
привести сведения о~применённом }\textbf{\emph{подходе~(подходах)}}\textbf{,
использованном }\textbf{\emph{методе~(методах)}}\textbf{, ключевых
исходных информации и~данных, а~также основных причинах тех~или~иных
выводов.}

\textbf{Если результатом оказания услуг является не~}\textbf{\textsl{оценка}}\textbf{,
а~рецензирование другой }\textbf{\textsl{оценки}}\textbf{, заключение
должно содержать выводы рецензента относительно рассматриваемой работы,
в~том числе необходимые обоснования и~доказательства.}

\textbf{Данные требования не~распространяются на~случаи, когда условиями
}\textbf{\textsl{договора на~проведение оценки}}\textbf{ (}\textbf{\emph{заданием
на~оценку}}\textbf{) установлено, что~}\textbf{\emph{отчёт}}\textbf{
предоставляется без~обоснований и~приведения иной подтверждающей
информации и~данных.}

\textbf{Реализация.}

\stepcounter{SubSubSecCounter}

\thesubsubsection.\theSubSubSecCounter.\label{4.3.2.12.1} В~тех~случаях,
когда для~\textsl{оценки} различных активов требуется использование
различных \emph{подходов к~оценке} и~\textsl{допущений}, важно идентифицировать
такие активы и~привести в~тексте \emph{отчёта} отдельные описания
для~каждого из~них.\label{4.3.2.12.1-End}

\stepcounter{SubSubSecCounter}

\thesubsubsection.\theSubSubSecCounter.\label{4.3.2.12.2} Различия
между понятиями \emph{подход к~оценке} и~\emph{метод оценки} описаны
в~разделе~\ref{sec:4.5_VPS5_Valuation_approaches_and_methods}~\nameref{sec:4.5_VPS5_Valuation_approaches_and_methods}
\vpageref{sec:4.5_VPS5_Valuation_approaches_and_methods}--\pageref{sec:4.5_VPS5_Valuation_approaches_and_methods-End}.
Степень детализации их~описания должна быть соразмерна задачам конкретной
\textsl{оценки} и, в~первую очередь, отвечать цели обеспечения её~понимания
со~стороны заказчика \textsl{оценки} и~иных предполагаемых пользователей
\emph{отчёта}. В~соответствующих случаях описания обоснований сделанных
выводов и~их~причин должны включать в~себя объяснение отклонений
от~общепринятых профессиональных практик, допущенных \emph{оценщиком}
при~проведении \textsl{оценки}.\label{4.3.2.12.2-End}

\stepcounter{SubSubSecCounter}

\thesubsubsection.\theSubSubSecCounter.\label{4.3.2.12.3} В~случаях
проведения \textsl{оценки} активов или~обязательств, относящихся
к~недвижимому имуществу, следует принимать во~внимание положения
п.~\vref{4.2.1.5}--\pageref{4.2.1.5-End}, а~также, там где~это~уместно,
тот~факт, что~вопросы \emph{устойчивого развития} и~защиты окружающей
среды являются в~наше время настолько актуальными и~значимыми, что~должны
являться неотъемлемой частью любого \emph{подхода к~оценке} и~аргументации,
подтверждающей её~итоговые результаты.\label{4.3.2.12.3-End}\label{subsubsec:4.3.2.12_Valuation_approach_and_reasoning-End}

\subsubsection{Величина определённой оценщиком стоимости~(стоимостей)\label{subsec:4.3.2.13_Amount_of_the_valuation}}

\textbf{Итоговое значение стоимости должно быть номинировано в~применимой
валюте.}

\textbf{Данное требование не~распространяется на~случаи проведения
рецензирования другой }\textbf{\textsl{оценки}}\textbf{, когда от~выполняющего
работу }\textbf{\emph{оценщика}}\textbf{ не~требуется формирование
собственного }\textbf{\emph{суждения о~стоимости}}\textbf{.}

\textbf{Реализация.}

\stepcounter{SubSubSecCounter}

\thesubsubsection.\theSubSubSecCounter.\label{4.3.2.13.1} Результаты
\textsl{оценки} следует указывать как~цифрами так~и~прописью в~основном
тексте \emph{отчёта}.\label{4.3.2.13.1-End}

\stepcounter{SubSubSecCounter}

\thesubsubsection.\theSubSubSecCounter.\label{4.3.2.13.2} В~случае
проведения \textsl{оценки} активов, относящихся к~различным классам,
исходя из~их назначения, либо расположенных в~разных географических
локациях, вопрос её~проведения отдельно по~каждому активу либо иным
образом следует рассматривать исходя из~целей проведения \textsl{оценки},
обстоятельств и~предпочтений заказчика. В~случае, если группа оцениваемых
активов включает в~себя единицы, в~отношении которых существуют
различные права, либо сроки этих прав (в~случае права аренды), стоимость
активов, относящихся к~одной группе может быть выделена в~отдельный
подпункт в~дополнение к~общей стоимости всех активов.\label{4.3.2.13.2-End}

\stepcounter{SubSubSecCounter}

\thesubsubsection.\theSubSubSecCounter.\label{4.3.2.13.3} Заказчики,
как~правило, требуют выражения стоимости \uline{активов~(обязательств)}
в~валюте той~страны, в~которой \uline{они}~расположены. В~терминах
\textsl{финансовой отчётности} это~звучит как~\guillemotleft валюта
баланса\guillemotright . Независимо от~местонахождения заказчика,
\textsl{оценку} необходимо проводить в~валюте страны, в~которой
находится актив или~обязательство.\label{4.3.2.13.3-End}

\stepcounter{SubSubSecCounter}

\thesubsubsection.\theSubSubSecCounter.\label{4.3.2.13.4} В~случае,
если заказчик требует перевода результатов \textsl{оценки} в~другую
валюту (например в~валюту отчётности), обменный курс принимается
как~курс на~момент закрытия валютных торгов на~\textsl{дату оценки}
(т.\,н.~спотовый курс), если не~согласовано иное.\label{4.3.2.13.4-End}

\stepcounter{SubSubSecCounter}

\thesubsubsection.\theSubSubSecCounter.\label{4.3.2.13.5} Если \emph{заданием
на~оценку} установлена необходимость выражения результатов \textsl{оценки}
более чем~в~одной валюте (например при~\textsl{оценке} группы активов,
расположенных в~разных странах), в~тексте \emph{отчёта} должны быть
указаны все~использованные валюты, а~само значение стоимости, номинированное
в~каждой из~этих валют, указано цифрами и~прописью. Кроме того,
использованный в~\emph{отчёте} обменный курс должен быть принят на~\textsl{дату
оценки} и~также указан в~его~тексте.\label{4.3.2.13.5-End}

\stepcounter{SubSubSecCounter}

\thesubsubsection.\theSubSubSecCounter.\label{4.3.2.13.6} Если описание
отдельных активов и~результаты их~\textsl{оценки} приводятся в~приложениях
к~\emph{отчёту} в~табличной форме, краткие данные об~их~стоимости
также должны приводиться в~основном тексте \emph{отчёта}.\label{4.3.2.13.6-End}

\stepcounter{SubSubSecCounter}

\thesubsubsection.\theSubSubSecCounter.\label{4.3.2.13.7} В~тех~случаях,
когда в~период, прошедший между \textsl{датой оценки}, предшествующей
дате составления \emph{отчёта}, и~датой составления \emph{отчёта},
произошли существенные изменения рыночных условий либо свойств \emph{объекта
оценки} \emph{(объектов оценки)}, \emph{оценщик} обязан обратить внимание
на~данное обстоятельство в~тексте \emph{отчёта}. В~соответствующих
случаях \emph{оценщик} также может обратить заказчика на~тот~факт,
что~стоимость имеет свойство меняться с~течением времени, вследствие
и~по~причине чего \textsl{оценка}, выполненная на~определённую
дату, может быть недействительной на~более раннюю или~позднюю дату.\label{4.3.2.13.7-End}

\stepcounter{SubSubSecCounter}

\thesubsubsection.\theSubSubSecCounter.\label{4.3.2.13.8} Возможна
ситуация, при~которой возникает \guillemotleft отрицательная стоимость\guillemotright ,
представляющая собой обязательство. В~этом случае её~всегда необходимо
отражать отдельно. Сальдировние подобных отрицательных стоимостей
с~положительными стоимостями других объектов не~допускается.\label{4.3.2.13.8-End}\label{subsec:4.3.2.13_Amount_of_the_valuation-End}

\subsubsection{Дата отчёта об~оценке\label{subsec:4.3.2.14_Date_of_report}}

\textbf{Необходимо указать дату выпуска }\textbf{\emph{отчёта об~оценке}}\textbf{.
Она~может отличаться от~}\textbf{\textsl{даты оценки}}\textbf{ (см.~пар.~\ref{subsubsec:4.3.2.6_Valuation_date}
\vpageref{subsubsec:4.3.2.6_Valuation_date}--\pageref{subsubsec:4.3.2.6_Valuation_date-End}).}\label{subsec:4.3.2.14_Date_of_report-End}

\subsubsection{Комментарии, затрагивающие вопросы существенной неопределённости,
необходимые для~обеспечения ясности для~пользователей результатов
оценки\label{subsubsec:4.3.2.15_Commentary_ on_any_material_uncertainty}}

\textbf{Реализация.}

\stepcounter{SubSubSecCounter}

\thesubsubsection.\theSubSubSecCounter.\label{4.3.2.15.1} Данное
требование является обязательным только тогда, когда неопределённость
является существенной. В~целях настоящих стандартов, критерием признания
неопределённости существенной для~\textsl{оценки}, является такая
её~степень, при~которой значения используемых переменных выходят
за~границы диапазона, ожидаемого и~принимаемого в~повседневной
практике бизнеса.\label{4.3.2.15.1-End}

\stepcounter{SubSubSecCounter}

\thesubsubsection.\theSubSubSecCounter.\label{4.3.2.15.2} Любая
\textsl{оценка} является выражением профессионального \emph{\uline{суждения}}\emph{
о~мгновенном значении стоимости} конкретного \textsl{вида}, \uline{основанного}
в~т.\,ч.~на~\textsl{допущениях} и~(или) \textsl{специальных допущениях},
также в~обязательном порядке включаемых в~текст \emph{отчёта} "---
\textsl{оценка} не~является отражением фактической стоимости и~не~являет
собой факт существования именно такого значения стоимости. Как~и~в~случае
с~любым мнением~(суждением), степень субъективности неизбежно будет
разной в~каждом конкретном случае. Также разной будет и~степень
\guillemotleft определённости\guillemotright{} "--- например вероятности
того, что~\emph{суждение оценщика} о~\textsl{рыночной стоимости}
будет точно совпадать с~фактической ценой продажи на~\textsl{дату
оценки}, даже если все~обстоятельства, предусмотренные определением
\textsl{рыночной стоимости} и~допущениями, будут идентичны обстоятельствам
фактической продажи. Также большая часть \textsl{оценок} одного и~того~же
актива или~обязательства на~одну и~ту~же дату будут отличаться
в~некоторой степени в~зависимости от~\emph{суждений} конкретных
оценщиков. Данный принцип хорошо признаётся судами во~многих юрисдикциях.\label{4.3.2.15.2-End}

\stepcounter{SubSubSecCounter}

\thesubsubsection.\theSubSubSecCounter.\label{4.3.2.15.3} В~целях
обеспечение понимания \emph{отчётов} и~формирования доверия пользователей
к~\textsl{оценкам} \emph{оценщику} следует поддерживать достаточный
уровень ясности и~прозрачности, отсюда вытекает общее требование
параграфа~\ref{subsubsec:4.3.2.12_Valuation_approach_and_reasoning}
о~том, что~\emph{отчёт} должен содержать описание использованного
\emph{подхода}~\emph{(подходов)}, основные использованные исходные
данные, а~также описание основных причин сделанных выводов, обеспечивая
тем~самым понимание результатов \textsl{оценки} читателем в~контексте
процесса её~проведения. Вопрос детализации необходимых пояснений
и~степень глубины описания доказательств, \textsl{подходов к~оценке}
и~конкретного рыночного контекста относится к~сфере суждения \emph{оценщика}
в~каждом конкретном случае.\label{4.3.2.15.3-End}

\stepcounter{SubSubSecCounter}

\thesubsubsection.\theSubSubSecCounter.\label{4.3.2.15.4} Как~правило,
нет~необходимости приводить какие-либо объяснения либо разъяснения
сверх тех, которые следуют из~общих требований, упомянутых выше в~п.~\vref{4.3.2.15.3}--\pageref{4.3.2.15.3-End}.
Однако, в~ряде случаев неопределённость результатов \textsl{оценки}
может быть б\'{о}льшей относительно обычной для~данного вида \emph{объектов
оценки}. Если эта~неопределённость является существенной, то во~избежание
введения в~заблуждение пользователя \emph{отчёта} необходимо добавить
дополнительные соответствующие комментарии, а~также отразить в~нём
сам~факт наличия такой высокой степени неопределённости. \emph{Оценщикам}
не~следует рассматривать наличие подобного заявления, выражающего
меньшую чем~обычно уверенность в~\textsl{оценке}, как~признание
слабости, поскольку оно~не~отражает их~квалификацию или~качество
суждений, а~в~значительной мере относится к~области вопросов \emph{раскрытия
информации}. Напротив, в~том~случае, если отсутствие акцента на~наличие
существенной неопределённости создаст у~заказчика впечатление о~том,
что~\emph{суждению оценщика} следует придавать больший вес, чем~это~стоит
делать на~самом деле, \emph{отчёт} будет вводить в~заблуждение.\label{4.3.2.15.4-End}

\stepcounter{SubSubSecCounter}

\thesubsubsection.\theSubSubSecCounter.\label{4.3.2.15.5} Дальнейшие
разъяснения, касающиеся существенной неопределённости результатов
\textsl{оценки}, содержатся в~разделе~\ref{sec:5.10_VPGA-10_Matters_for_uncertainty}~\nameref{sec:5.10_VPGA-10_Matters_for_uncertainty}
\vpageref{sec:5.10_VPGA-10_Matters_for_uncertainty}--\pageref{sec:5.10_VPGA-10_Matters_for_uncertainty-End}.\label{4.3.2.15.5-End}\label{subsubsec:4.3.2.15_Commentary_ on_any_material_uncertainty-End}

\subsubsection{Положение о~согласованных ограничениях ответственности\label{subsubsec:4.3.2.16_Limitations_on_liability}}

\textbf{Реализация.}

\stepcounter{SubSubSecCounter}

\thesubsubsection.\theSubSubSecCounter.\label{4.3.2.16.1} Вопросы
риска, ответственности и~страхования тесно связаны между собой. В~ожидании
выхода соответствующего всемирного руководства, \textsl{членам RICS}
следует ознакомиться с~\href{https://www.rics.org/uk/upholding-professional-standards/regulation/regulatory-support/professional-indemnity/pii-and-valuation-guidance/}{последней версией соответствующего руководства RICS},
применимого для~их~юрисдикции~\cite{RICS:prof_indemnity}.\label{4.3.2.16.1-End}\label{subsubsec:4.3.2.16_Limitations_on_liability-End}\label{subsec:4.3.2_Report_content-End}\label{sec:4.3_VPS3_Valuation_reports-End}

\newpage

\section{СПО 4. Виды стоимости, допущения и~специальные допущения\label{sec:4.4_VPS4_Bases_of_value}}

\textbf{Данный обязательный стандарт:}
\begin{itemize}
\item \textbf{имплементирует \hyperref[sec:9.4_IVS-104_Bases_of_Value]{МСО 104. Виды стоимости};}
\item \textbf{устанавливает дополнительные обязательные требования для~}\textbf{\textsl{членов
RICS}}\textbf{;}
\item \textbf{рассматривает конкретные аспекты имплементации, возникающие
в~конкретных случаях.}
\end{itemize}
\textbf{\emph{Оценщик}}\textbf{ обязан убедиться в~том, что~выбранный
им~}\textbf{\textsl{вид определяемой стоимости}}\textbf{ соответствует
}\textbf{\emph{цели оценки}}\textbf{.}

\textbf{В~случае применения одного из~}\textbf{\textsl{видов стоимости}}\textbf{,
предусмотренного настоящими }\textbf{\emph{Всемирными стандартами}}\textbf{
(включая }\textbf{\textsl{виды стоимости}}\textbf{, установленные
\href{https://www.rics.org/globalassets/rics-website/media/upholding-professional-standards/sector-standards/valuation/international-valuation-standards-rics2.pdf}{МСО}~\cite{IVS-2020}),
его~следует осуществлять в~порядке, предусмотренном соответствующим
определением и~руководствами, включая принятие любых уместных }\textbf{\textsl{допущений}}\textbf{
и~(или) }\textbf{\textsl{специальных допущений}}\textbf{.}

\textbf{В~случае применения }\textbf{\textsl{вида стоимости}}\textbf{,
не~предусмотренного настоящими }\textbf{\emph{Всемирными стандартами}}\textbf{
(включая }\textbf{\textsl{виды стоимости}}\textbf{, установленные
\href{https://www.rics.org/globalassets/rics-website/media/upholding-professional-standards/sector-standards/valuation/international-valuation-standards-rics2.pdf}{МСО}~\cite{IVS-2020}),
}\textbf{\textsl{он}}\textbf{~должен быть чётко обозначен и~описан
в~}\textbf{\emph{отчёте}}\textbf{, в~котором также должен быть отражён
факт того, что~применение подобного }\textbf{\textsl{вида стоимости}}\textbf{
является }\textbf{\textsl{отступлением}}\textbf{ в~случае, если оно~является
добровольным, а~не~проистекает из~обязанности.}

\textbf{В~случае наличия }\textbf{\textsl{отступлений}}\textbf{,
не~следующих из~обязанности, нельзя говорить о~соответствии }\textbf{\emph{отчёта}}\textbf{
требованиям \href{https://www.rics.org/globalassets/rics-website/media/upholding-professional-standards/sector-standards/valuation/international-valuation-standards-rics2.pdf}{МСО}~\cite{IVS-2020}.}

\subsection{Основные принципы\label{subsec:4.4.1_General_principles}}

\stepcounter{SubSubSecCounter}

\thesubsubsection.\theSubSubSecCounter.\label{4.4.1.1} Вид стоимости
"--- совокупность фундаментальных \textsl{допущений}, используемых
при~проведении \textsl{оценки}.\label{4.4.1.1-End}

\stepcounter{SubSubSecCounter}

\thesubsubsection.\theSubSubSecCounter.\label{4.4.1.2} Следующие
виды стоимости установлены \href{https://www.rics.org/globalassets/rics-website/media/upholding-professional-standards/sector-standards/valuation/international-valuation-standards-rics2.pdf}{Международными Стандартами Оценки}
(см.~\ref{subsubsec:9.4.2.1_IVS-Defined_Bases}~\nameref{subsubsec:9.4.2.1_IVS-Defined_Bases}
\vpageref{subsubsec:9.4.2.1_IVS-Defined_Bases}--\pageref{subsubsec:9.4.2.1_IVS-Defined_Bases-End})
и~б\'{о}льшая часть из~них~имеет широкое применение, хотя они~и~могут
быть не~приняты на~некоторых рынках:
\begin{itemize}
\item \textsl{рыночная стоимость} (см.~подраздел~\ref{subsec:4.4.3_Market-value}~\nameref{subsec:4.4.3_Market-value}
\vpageref{subsec:4.4.3_Market-value}--\pageref{subsec:4.4.3_Market-value-End});
\item \textsl{рыночная арендная плата} (см.~подраздел~\ref{subsec:4.4.4_Market-rent}~\nameref{subsec:4.4.4_Market-rent}
\vpageref{subsec:4.4.4_Market-rent}--\pageref{subsec:4.4.4_Market-rent-End});
\item \textsl{инвестиционная стоимость~(ценность)} (см.~подраздел~\ref{subsec:4.4.5_Investment-value}~\nameref{subsec:4.4.5_Investment-value}
\vpageref{subsec:4.4.5_Investment-value}--\pageref{subsec:4.4.5_Investment-value-End});
\item \textsl{равновесная стоимость} (ранее определялась \href{https://www.rics.org/globalassets/rics-website/media/upholding-professional-standards/sector-standards/valuation/international-valuation-standards-rics2.pdf}{МСО}~\cite{IVS-2020}
как~\emph{справедливая стоимость});
\item \emph{синергетическая стоимость};
\item \emph{ликвидационная стоимость}.
\end{itemize}
Особое внимание следует уделять правильной интерпретации заказчиком
понятия \emph{синергетической стоимости} в~тех~случаях, когда она~используется.\label{4.4.1.2-End}

\stepcounter{SubSubSecCounter}

\thesubsubsection.\theSubSubSecCounter.\label{4.4.1.3} Кроме того,
для~целей \textsl{финансовой отчётности} широко используется \emph{справедливая
стоимость} (установленная \href{http://docs.cntd.ru/document/420334241}{МСФО 13}~(\href{http://eifrs.ifrs.org/eifrs/bnstandards/en/IFRS13.pdf}{IFRS 13})~\cite{MSFO-13},
\cite{IFRS-13}), признаваемая многими участниками рынка (в~т.\,ч.~RICS),
хотя и~она~также не~универсальна "--- см.~подраздел~\ref{subsec:4.4.6_Fair_value}~\nameref{subsec:4.4.6_Fair_value}
\vpageref{subsec:4.4.6_Fair_value}--\pageref{subsec:4.4.6_Fair_value-End}.\label{4.4.1.3-End}

\stepcounter{SubSubSecCounter}

\thesubsubsection.\theSubSubSecCounter.\label{4.4.1.4} В~ряде случаев
проведения \textsl{оценки}, в~особенности касающихся некоторых юрисдикций,
в~которых могут существовать правовые нормы, устанавливающие иные
обязательные требования, возможно применение иных \textsl{видов стоимости}:
обязательных либо допустимых (\textsl{членам RICS} следует обратить
внимание, что~в~МСО~104 приводятся некоторые примеры таких \textsl{видов
стоимости} "--- см.~\ref{subsec:9.4.2.2_Other_Bases}~\nameref{subsec:9.4.2.2_Other_Bases}
\vpageref{subsec:9.4.2.2_Other_Bases}--\pageref{subsec:9.4.2.2_Other_Bases-End}).
В~таких случаях, \emph{оценщику} необходимо дать чёткое определение
применяемого \textsl{вида стоимости}, а~в~случае его~применения,
не~следующего из~обязательных требований, привести в~тексте \emph{отчёта}
обоснование неприменимости \textsl{видов стоимости}, предусмотренных
данными \emph{Всемирными стандартами} (включая \href{https://www.isurv.com/info/1342/rics_national_or_jurisdictional_valuation_standards}{дополнения для отдельных национальных юрисдикций}~\cite{RICS:National-Standards})
"--- см.~подраздел~\ref{subsec:3.1.4_Compliance_with_jurisdictional_standards}~\nameref{subsec:3.1.4_Compliance_with_jurisdictional_standards}
\vpageref{subsec:3.1.4_Compliance_with_jurisdictional_standards}--\pageref{subsec:3.1.4_Compliance_with_jurisdictional_standards-End}.\label{4.4.1.4-End}

\stepcounter{SubSubSecCounter}

\thesubsubsection.\theSubSubSecCounter.\label{4.4.1.5}

Поскольку рынки продолжают развиваться и~совершенствоваться, а~потребности
заказчиков становятся всё~более изощрёнными, перед \emph{оценщиками}
всё~чаще встаёт вопрос необходимости удовлетворения потребностей
в~консультациях, несущих в~себе некоторые элементы прогнозирования
или~предсказания. Необходимо тщательно следить за~тем, чтобы подобные
консультации не~были неверно поняты либо искажены. Также с~целью
недопущения подрыва основ \textsl{вида стоимости} необходимо проводить
детальный анализ чувствительности модели определения стоимости.\label{4.4.1.5-End}

\stepcounter{SubSubSecCounter}

\thesubsubsection.\theSubSubSecCounter.\label{4.4.1.6}\emph{ Оценщикам}
следует иметь ввиду, что~применение не~общепризнанных либо специально
разработанных под~конкретную работу \textsl{видов стоимости} без~веских
на~то~причин может повлечь за~собой нарушение требования о~том,
что~\emph{отчёт об~оценке} не~должен допускать неоднозначного толкования
либо вводить в~заблуждение (см.~подраздел~\ref{subsec:4.3.1_General_principles}~\nameref{subsec:4.3.1_General_principles}
\vpageref{subsec:4.3.1_General_principles}--\pageref{subsec:4.3.1_General_principles-End}).\label{4.4.1.6-End}

\stepcounter{SubSubSecCounter}

\thesubsubsection.\theSubSubSecCounter.\label{4.4.1.7} Следует обратить
внимание на~тот~факт, что~МСО 104 (раздел~\ref{sec:9.4_IVS-104_Bases_of_Value}~\nameref{sec:9.4_IVS-104_Bases_of_Value}
\vpageref{sec:9.4_IVS-104_Bases_of_Value}--\pageref{sec:9.4_IVS-104_Bases_of_Value-End})
содержит материал о~\emph{\guillemotleft предпосылках стоимости\guillemotright},
который не~воспроизводится в~настоящих \emph{Всемирных стандартах}.\label{4.4.1.7-End}\label{subsec:4.4.1_General_principles-End}

\subsection{Виды стоимости\label{subsec:4.4.2_Bases_of_value}}

\stepcounter{SubSubSecCounter}

\thesubsubsection.\theSubSubSecCounter.\label{4.4.2.1}\emph{ Оценщик}
несёт ответственность за~то, чтобы использованный им~\textsl{вид
стоимости} соответствовал \emph{цели оценки} и~был уместен в~конкретных
обстоятельствах "--- данная ответственность является вопросом соответствия
любым обязательным требованиям, например, установленным законом. Важно,
чтобы используемый \textsl{\uline{вид стоимости}} изначально был
обсуждён и~согласован с~заказчиком во~всех случаях, когда \uline{его}~выбор
не~является очевидным.\label{4.4.2.1-End}

\stepcounter{SubSubSecCounter}

\thesubsubsection.\theSubSubSecCounter.\label{4.4.2.2} Следует отметить,
что~\textsl{виды стоимости} не~являются взаимоисключающими. Например,
\textsl{ценность} объекта недвижимости либо актива для~конкретной
стороны или~\textsl{равновесная стоимость} этого объекта либо актива
в~обмене между двумя конкретными сторонами может совпадать с~их~\textsl{рыночной
стоимостью} даже если используются разные критерии определения стоимости.\label{4.4.2.2-End}

\stepcounter{SubSubSecCounter}

\thesubsubsection.\theSubSubSecCounter.\label{4.4.2.3} Поскольку
результатом определения \textsl{вида стоимости} отличного от~\textsl{рыночной},
может быть такое её~значение, которое может и~не~быть получено
при~фактической сделке на~общем либо ином рынке, \emph{оценщику}
необходимо чётко обозначить \textsl{допущения} и~\textsl{специальные
допущения}, используемые вместо и~(или) в~дополнение к~тем, которые
были~бы уместны при~определении \textsl{рыночной стоимости}. Примеры
подобных типовых \textsl{допущений} и~\textsl{специальных допущений}
рассмотрены ниже в~подразделах~\ref{subsec:4.4.7_Assumptions} \vpageref{subsec:4.4.7_Assumptions}--\pageref{subsec:4.4.7_Assumptions-End},
\ref{subsec:4.4.8_Special_assumptions} \vpageref{subsec:4.4.8_Special_assumptions}--\pageref{subsec:4.4.8_Special_assumptions-End}
соответственно.\label{4.4.2.3-End}

\stepcounter{SubSubSecCounter}

\thesubsubsection.\theSubSubSecCounter.\label{4.4.2.4} Во~всех
случаях \emph{оценщику} необходимо убедиться в~том, что~\textsl{вид
стоимости} описан либо ясно обозначен как~в~условиях \textsl{договора
на~проведение оценки} (\emph{задании на~оценку}), так~и~в~тексте
\emph{отчёта об~оценке}.\label{4.4.2.4-End}

\stepcounter{SubSubSecCounter}

\thesubsubsection.\theSubSubSecCounter.\label{4.4.2.5}\emph{ Оценщик}
может получить законное указание провести \textsl{оценку} на~основе
иных критериев, и~тогда альтернативные \textsl{виды стоимости} могут
являться уместными. В~таких случаях дефиниция применяемого \textsl{вида
стоимости} должна быть приведена в~полном виде и~сопровождаться
необходимыми пояснениями. В~тех~случаях, когда применяемый \textsl{вид
стоимости} существенным образом отличается от~\textsl{рыночной стоимости}
рекомендуется привести краткий комментарий, описывающий такие отличия.\label{4.4.2.5-End}\label{subsec:4.4.2_Bases_of_value-End}

\subsection{Рыночная стоимость\label{subsec:4.4.3_Market-value}}

\textbf{Определение рыночной стоимости содержится в~IVS~104~(см.~п.~\vref{9.4.2.1.1.1}--\pageref{9.4.2.1.1.1-End}).}
\begin{description}
\item [{Рыночная~стоимость}] "--- расчётная величина денежной суммы,
в~обмен на~которую актив или~обязательство могут быть переданы
в~рамках сделки, заключённой на~\textsl{дату оценки} на~рыночных
условиях между заинтересованными продавцом и~покупателем после надлежащего
изучения рынка на~взаимоприемлемых условиях в~случае, когда обе
стороны являются независимыми, проявляют должную осмотрительность
и~осведомлённость, действуют разумно, в~своих интересах и~без~принуждения.
\end{description}
\stepcounter{SubSubSecCounter}

\thesubsubsection.\theSubSubSecCounter.\label{4.4.3.1}\textsl{~
Рыночная стоимость} представляет собой \textsl{вид стоимости}, признанный
на~международном уровне и~имеющий давно устоявшееся определение.
Она~отражает обмен между сторонами, не~связанными между собой и~свободно
действующими на~рынке, и~представляет такое значение цены, которое
имело~бы место в~гипотетической сделке купли-продажи либо аналогичной
по~смыслу юридически значимой сделке, заключённой на~\textsl{дату
оценки}, и~учитывает при~этом влияние всех факторов, которые принимали~бы
во~внимание при~формировании своих предложений участники рынка в~целом,
основываясь при~этом на~варианте \emph{наилучшего и~наиболее эффективного
использования} актива. \emph{Наилучшее и~наиболее эффективное использование
актива} представляет собой такой вариант его~использования, при~котором
функция его~полезности достигает максимального значения, и~который
является физически возможным, юридически допустимым и~финансово осуществимым
"--- данная предпосылка стоимости подробно рассмотрена в~секции~\ref{subsubsec:9.4.3.1_Highest_and_Best_use}
\vpageref{subsubsec:9.4.3.1_Highest_and_Best_use}--\pageref{subsubsec:9.4.3.1_Highest_and_Best_use-End}.\label{4.4.3.1-End}

\stepcounter{SubSubSecCounter}

\thesubsubsection.\theSubSubSecCounter.\label{4.4.3.2} Данный \textsl{вид~стоимости}
позволяет исключить любые искажения стоимости, характерные для~\textsl{специальной
стоимости} (стоимости, отражающей особые характеристики актива, имеющие
ценность только для~\textsl{определённого покупателя}) либо \textsl{синергетической
стоимости}. Она~представляет собой значение цены, являющееся наиболее
вероятной в~условиях широкого набора обстоятельств. При~определении
рыночной \textsl{арендной платы} (см.~подраздел~\ref{subsec:4.4.4_Market-rent}
\vpageref{subsec:4.4.4_Market-rent}--\pageref{subsec:4.4.4_Market-rent-End})
аналогичные критерии применяются для~определении величины периодических
платежей, а~не~стоимости самого актива.\label{4.4.3.2-End}

\stepcounter{SubSubSecCounter}

\thesubsubsection.\theSubSubSecCounter.\label{4.4.3.3} Величина
определённой \textsl{рыночной стоимости} должна отражать состояние
рынка и~обстоятельства, существовавшие на~\textsl{дату оценки}.
Полное описание концептуальной основы \textsl{рыночной стоимости}
приводится в~п.~\ref{9.4.2.1.1.2} \vpageref{9.4.2.1.1.2}--\pageref{9.4.2.1.1.2-End}.\label{4.4.3.3-End}

\stepcounter{SubSubSecCounter}

\thesubsubsection.\theSubSubSecCounter.\label{4.4.3.4} Несмотря
на~отказ от~учёта факторов характерных для~\textsl{специальной
стоимости}, в~тех~случаях, когда цены спроса на~открытом рынке
учитывают ожидания изменений условий использования актива в~будущем,
влияние таких ожиданий отражается и~на~\textsl{рыночной стоимости}.
Ниже приводятся примеры случаев, когда~ожидание возникновения или~получения
дополнительной стоимости в~будущем может оказать влияние на~текущую
\textsl{рыночную стоимость}:
\begin{itemize}
\item перспектива реализации девелоперского проекта в~условиях, когда на~текущий
момент отсутствует соответствующее разрешение;
\item перспектива возникновения в~будущем \emph{синергетической стоимости}
вследствие объединения с~другим имуществом или~активом либо другой
долей в~оцениваемом имуществе (активе).\label{4.4.3.4-End}
\end{itemize}
\stepcounter{SubSubSecCounter}

\thesubsubsection.\theSubSubSecCounter.\label{4.4.3.5} Следует разделять
эффект возникновения дополнительной стоимости вследствие введения
\textsl{допущений} и~(или) \textsl{специальных допущений} и~её~возникновение,
проистекающее вследствие наличия \textsl{специального покупателя}.\label{4.4.3.5-End}

\stepcounter{SubSubSecCounter}

\thesubsubsection.\theSubSubSecCounter.\label{4.4.3.6} В~ряде юрисдикций
практикуется использование такого вида стоимости как~\guillemotleft стоимость
при~наилучшем и~наиболее эффективном использовании\guillemotright ,
установленного законодательно либо следующего из~сложившейся практики
в~отдельных странах либо на~отдельных территориях.\label{4.4.3.6-End}\label{subsec:4.4.3_Market-value-End}

\subsection{Рыночная арендная плата\label{subsec:4.4.4_Market-rent}}

\textbf{Определение рыночной арендной платы содержится в~IVS~104~(см.~п.~\vref{9.4.2.1.2.1}--\pageref{9.4.2.1.2.1-End}).}
\begin{description}
\item [{Рыночная~арендная~плата}] "--- расчётная величина денежной суммы,
за~которую права на~недвижимое имущество могут быть переданы в~рамках
сделки по~его аренде, заключённой на~\textsl{дату оценки} на~рыночных
условиях между заинтересованными арендодателем и~арендатором после
надлежащего изучения рынка на~взаимоприемлемых условиях в~случае,
когда обе стороны являются независимыми, проявляют должную осмотрительность
и~осведомлённость, действуют разумно, в~своих интересах и~без~принуждения.
\end{description}
\stepcounter{SubSubSecCounter}

\thesubsubsection.\theSubSubSecCounter.\label{4.4.4.1} Величина
\textsl{рыночной арендной платы} будет значительно отличаться в зависимости
от условий предполагаемого договора аренды. Принимаемые условия аренды
как~правило отражают сложившуюся на~\textsl{дату оценки} практику
на~рынке, на~котором находится объект недвижимости, хотя для~определённых
целей может возникнуть потребность предусмотреть необычные условия.
На~величину \textsl{рыночной арендной платы} оказывают влияние в~т.\,ч.~такие
аспекты как~срок аренды, частота пересмотра величины платы, обязанности
сторон по~обслуживанию и~оплате операционных расходов. В~ряде юрисдикций
возможность согласования некоторых условий договора аренды может быть
ограничена законодательством, которое также может прямо предписывать
некоторые из~них. При~необходимости это~следует принимать во~внимание.\label{4.4.4.1-End}

\stepcounter{SubSubSecCounter}

\thesubsubsection.\theSubSubSecCounter.\label{4.4.4.2} Под~\textsl{рыночной
арендной платой}, как~правило, понимается сумма, за~которую свободная
недвижимость может быть сдана в~аренду, или~за~которую уже~арендованная
недвижимость может быть повторно сдана в~аренду по~истечении срока
действующего договора аренды. \textsl{Рыночная арендная плата} не~является
подходящей основой для~определения величины арендной платы, устанавливаемой
в~рамках её~периодического пересмотра согласно условиям действующего
договора аренды. В~таких случаях следует опираться на~условия договора,
а~также установленные в~нём допущения и~определения.\label{4.4.4.2-End}

\stepcounter{SubSubSecCounter}

\thesubsubsection.\theSubSubSecCounter.\label{4.4.4.3} Таким образом,
\emph{оценщикам} необходимо позаботиться о~том, чтобы в~\emph{отчёте}
были приведены предполагаемые условия аренды, с~учётом которых проводится
определение величины \textsl{рыночной арендной платы}. В~тех~случаях,
когда сложившаяся практика заключения договоров аренды предполагает
наличие выплаты либо скидки одной стороны другой стороне в~качестве
стимулирующей меры для~побуждения к~их~заключению, оказывающих
влияние на~размер согласованной величины арендной платы, определение
величины рыночной арендной платы должно осуществляться с~учётом такой
практики. \emph{Оценщик} обязан описать характер такой стимулирующей
меры наряду с~предполагаемыми условиями договора аренды.\label{4.4.4.3-End}\label{subsec:4.4.4_Market-rent-End}

\subsection{Инвестиционная стоимость\label{subsec:4.4.5_Investment-value}}

\textbf{Определение рыночной арендной платы содержится в~IVS~104~(см.~п.~\vref{9.4.2.1.4.1}--\pageref{9.4.2.1.4.1-End}).}
\begin{description}
\item [{Инвестиционная~стоимость}] "--- стоимость актива для~конкретного
владельца либо потенциального владельца с~учётом его~конкретных
инвестиционных либо операционных целей.
\end{description}
\stepcounter{SubSubSecCounter}

\thesubsubsection.\theSubSubSecCounter.\label{4.4.5.1}

Как~следует из~определения, и~в~отличие от~\textsl{рыночной стоимости},
данный \textsl{вид стоимости} не~основан на~предпосылке о~гипотетической
сделке, а~является стоимостной мерой выгод от~владения для~нынешнего
или~потенциального владельца, допуская при~этом, что~они~могут
отличаться от~выгод типичного участника рынка. Данный вид стоимости
часто используется для~оценки эффективности владения активом в~соответствии
с~индивидуальными инвестиционными критериями его~владельца.\label{4.4.5.1-End}\label{subsec:4.4.5_Investment-value-End}

\subsection{Справедливая стоимость\label{subsec:4.4.6_Fair_value}}

Определение справедливой стоимости установлено \href{https://www.ifrs.org/}{Советом по Международным стандартам финансовой отчётности}~\cite{Wiki:IASB,Wiki:IASB-rus}
и~содержится в~\href{http://docs.cntd.ru/document/420334241}{Международном стандарте финансовой отчётности 13}~\cite{MSFO-13}
(\href{http://eifrs.ifrs.org/eifrs/bnstandards/en/IFRS13.pdf}{IFRS 13})~\cite{IFRS-13}.
\begin{description}
\item [{Справедливая~стоимость}] "--- цена, которая была~бы получена
при~продаже актива либо~уплачена при~передаче обязательства в~ходе
упорядоченной сделки между участниками рынка на~\textsl{дату оценки}.
\end{description}
\stepcounter{SubSubSecCounter}

\thesubsubsection.\theSubSubSecCounter.\label{4.4.6.1} \href{http://docs.cntd.ru/document/420334241}{МСФО 13}~\cite{MSFO-13}
(\href{http://eifrs.ifrs.org/eifrs/bnstandards/en/IFRS13.pdf}{IFRS 13})~\cite{IFRS-13}
содержит руководство в~т.\,ч.~в~части подходов по~определению
\textsl{справедливой стоимости}.\label{4.4.6.1-End}

\stepcounter{SubSubSecCounter}

\thesubsubsection.\theSubSubSecCounter.\label{4.4.6.2} Целью \textsl{оценки}
\textsl{справедливой стоимости} является расчётное определение цены,
при~которой упорядоченная сделка по~продаже актива или~передаче
обязательства состоялась~бы между рыночными агентами на~\textsl{дату
оценки} при~текущих рыночных условиях. Таким образом, данный подход
иногда описывается как~\guillemotleft оценка по~рынку\guillemotright .
В~действительности отсылки, содержащиеся в~\href{http://docs.cntd.ru/document/420334241}{МСФО 13}~\cite{MSFO-13}
к~\guillemotleft рыночным агентам\guillemotright{} и~\guillemotleft продаже\guillemotright{}
позволяют сделать вывод о~том, что~для~большинства практических
целей концепция \textsl{справедливой стоимости} соответствует концепции
\textsl{рыночной стоимости}, вследствие и~по~причине чего результатом
их~определения будет одно и~то~же значение стоимости.\label{4.4.6.2-End}

\stepcounter{SubSubSecCounter}

\thesubsubsection.\theSubSubSecCounter.\label{4.4.6.3} При~проведении
\textsl{оценки} \textsl{справедливой стоимости} от~\emph{оценщика}
требуется определить всё~нижеперечисленное:
\begin{itemize}
\item конкретный актив или~обязательство, являющийся~(являющееся) \emph{объектом
оценки}, а~также соответствующую расчётную единицу;
\item для~нефинансовых активов "--- соответствующие \emph{предпосылки
оценки} в~соответствии с~их~наилучшим и~наиболее эффективным использованием;
\item основной либо наиболее выгодный рынок для~актива или~обязательства;
\item применимый(е) с~учётом наличия исходных данных метод(ы) \textsl{оценки},
на~основе которого~(которых) можно разработать вводные, реализованные
в~форме \textsl{допущений}, которые рыночные агенты использовали~бы
при~определении цены актива или~обязательства, а~также степень
упорядочивания \textsl{справедливой стоимости}, в~соответствии с~которой
проводится классификация исходных данных.\label{4.4.6.3-End}
\end{itemize}
\stepcounter{SubSubSecCounter}

\thesubsubsection.\theSubSubSecCounter.\label{4.4.6.4} \emph{Оценщикам},
проводящим \textsl{оценку} для~целей включения в~\textsl{финансовую
отчётность}, следует ознакомиться с~требованиями \hyperref[sec:5.1_VPGA-1_Valuation_for_financial_statements]{ПР~1}~см.~раздел~\ref{sec:5.1_VPGA-1_Valuation_for_financial_statements}~\nameref{sec:5.1_VPGA-1_Valuation_for_financial_statements}\vpageref{sec:5.1_VPGA-1_Valuation_for_financial_statements}\pageref{sec:5.1_VPGA-1_Valuation_for_financial_statements-End}.\label{4.4.6.4-End}\label{subsec:4.4.6_Fair_value-End}

\subsection{Допущения\label{subsec:4.4.7_Assumptions}}

\textbf{\textsl{Допущения}}\textbf{ вводятся в~тех~случаях, когда
}\textbf{\emph{оценщику}}\textbf{ представляется разумным принять
что-либо в~качестве истины без~необходимости проведения соответствующего
исследования либо проверки.}

\textbf{Любое подобное }\textbf{\textsl{допущение}}\textbf{ должно
быть обоснованным и~уместным с~точки зрения цели, для~которой требуется
}\textbf{\textsl{оценка}}\textbf{.}

\stepcounter{SubSubSecCounter}

\thesubsubsection.\theSubSubSecCounter.\label{4.4.7.1} \hyperref[chap:2_Glossary]{Глоссарий} 
RICS содержит следующее полное \hyperref[Glossary:asumption]{определение}.
\begin{description}
\item [{Допущение}] "--- предположение, принимаемое в~качестве истины.
Оно~содержит факты, условия или~обстоятельства, затрагивающие свойства
\emph{объекта оценки} либо вопросы подходов к~\textsl{оценке}, которые,
по~соглашению сторон \textsl{договора на~проведение оценки}, не~нуждаются
в~проверке со~стороны \emph{оценщика} в~рамках осуществляемого
им~процесса проведения \textsl{оценки}. Как~правило, \textsl{допущение}
вводится тогда, когда отсутствует необходимость проведения \emph{оценщиком}
специального исследования по~доказыванию истинности того или~иного
факта.\label{4.4.7.1-End}
\end{description}
\stepcounter{SubSubSecCounter}

\thesubsubsection.\theSubSubSecCounter.\label{4.4.7.2} \textsl{Вид~определяемой
стоимости} практически во~всех случаях необходимо соотносить с~соответствующими
\textsl{допущениями} (либо \textsl{специальными допущениями} (см.~подраздел~\ref{subsec:4.4.8_Special_assumptions}\nameref{subsec:4.4.8_Special_assumptions}\vpageref{subsec:4.4.8_Special_assumptions}\pageref{subsec:4.4.8_Special_assumptions-End}),
описывающими статус либо состояние имущества либо актива на~\textsl{дату
оценки}.\label{4.4.7.2-End}

\stepcounter{SubSubSecCounter}

\thesubsubsection.\theSubSubSecCounter.\label{4.4.7.3} Допущение
часто бывает связано с~ограничением объёма исследований или~запросов
которые могут быть выполнены \emph{оценщиком} "--- см.~\ref{sec:4.2_VPS2_Inspections_investigations_and_records}~\nameref{sec:4.2_VPS2_Inspections_investigations_and_records}
\vpageref{sec:4.2_VPS2_Inspections_investigations_and_records}--\pageref{sec:4.2_VPS2_Inspections_investigations_and_records-End}.
Вследствие и~по~причине этого все~\textsl{допущения}, которые вероятно
могут быть включены в~\emph{отчёт}, должны быть согласованы с~заказчиком
и~содержаться в~\textsl{договоре на~проведение оценки}. В~тех~случаях,
когда отсутствует возможность включить \textsl{допущения} в~условия
\textsl{договора~на~проведение оценки}, они~должны быть согласованы
с~заказчиком в~письменном виде до~составления \emph{отчёта об~оценке}.\label{4.4.7.3-End}

\stepcounter{SubSubSecCounter}

\thesubsubsection.\theSubSubSecCounter.\label{4.4.7.4} В~тех~случаях,
когда по~результатам \textsl{осмотра} либо исследования \emph{оценщик}
предполагает, что~ранее согласованное с~\uline{заказчиком} \textsl{допущение}
скорее всего окажется неуместным либо его~следует перевести в~разряд
\textsl{специальных допущений}, подобные пересмотры подлежат обязательному
обсуждению с~\uline{последним} до~момента завершения работы и~передачи
\emph{отчёта}.\label{4.4.7.4-End}

\stepcounter{SubSubSecCounter}

\thesubsubsection.\theSubSubSecCounter.\label{4.4.7.5} Вопросы практического
применения \textsl{допущений} в~случае \textsl{оценки} недвижимого
имущества и~прав на~него рассмотрены в~разделе~\ref{sec:5.8_VPGA-8_Valuation_of_real_property}~\nameref{sec:5.8_VPGA-8_Valuation_of_real_property}
\vpageref{sec:5.8_VPGA-8_Valuation_of_real_property}--\pageref{sec:5.8_VPGA-8_Valuation_of_real_property-End}.\label{subsec:4.4.7_Assumptions-End}

\subsection{Специальные допущения\label{subsec:4.4.8_Special_assumptions}}

\textbf{\textsl{Специальное допущение}}\textbf{ вводится }\textbf{\emph{оценщиком}}\textbf{
в~том~случае, когда }\textbf{\textsl{допущение}}\textbf{~предполагает
либо факты, отличающиеся от~существующих на~}\textbf{\textsl{дату
оценки}}\textbf{, либо совершение действий, которые не~были~бы сделаны
типичным участником рынка при~совершении сделки на~}\textbf{\textsl{дату
оценки}}\textbf{.}

\textbf{В~случаях, когда для~предоставления заказчику }\textbf{\textsl{оценки}}\textbf{
в~соответствии с~его~потребностями требуется введение }\textbf{\textsl{специальных
допущений}}\textbf{, они~должны быть в~явном виде согласованы с~ним
и~подтверждены в~письменной форме до~момента выпуска }\textbf{\emph{отчёта}}\textbf{.}

\textbf{\textsl{Специальные допущения}}\textbf{ могут вводиться только
в~том случае, когда их~можно обоснованно считать реалистичными,
уместными и~действительными при~конкретных обстоятельствах }\textbf{\textsl{оценки}}\textbf{.}

\textbf{Реализация.}

\stepcounter{SubSubSecCounter}

\thesubsubsection.\theSubSubSecCounter.\label{4.4.8.1} \emph{Оценщик}
может включить в~\emph{отчёт} собственную оценку вероятности выполнения
\textsl{специального допущения} либо соответствующий комментарий в~этой
части. Например, \textsl{специальное допущение} о~том, что~имеется
разрешение на~застройку земельного участка, может отражать влияние
на~стоимость любых связанных с~этим потенциальных условий такого
разрешения.\label{4.4.8.1-End}

\stepcounter{SubSubSecCounter}

\thesubsubsection.\theSubSubSecCounter.\label{4.4.8.2} Типичным
\textsl{специальным допущением} является предположение о~том, что~свойства
либо состояние имущества или~актива были изменены определённым образом,
например \textsl{\guillemotleft рыночная стоимость} при~\textsl{специальном
допущении} о~том, что~работы были завершены\guillemotright . Иными
словами \textsl{специальное допущение} предполагает факты "--- отличные
от~реально существовавших на~\textsl{дату оценки}.\label{4.4.8.2-End}

\stepcounter{SubSubSecCounter}

\thesubsubsection.\theSubSubSecCounter.\label{4.4.8.3} В~случае,
когда заказчик запрашивает проведение \textsl{оценки}, основанной
на~\textsl{специальном допущении}, которое \emph{оценщик} считает
нереалистичным, ему~следует отказаться от~такого задания.\label{4.4.8.3-End}

\stepcounter{SubSubSecCounter}

\thesubsubsection.\theSubSubSecCounter.\label{4.4.8.4} Ниже приводится
перечень обстоятельств, при~которых введение \textsl{особых допущений}
может являться целесообразным:
\begin{itemize}
\item получение предложения о~покупке от~\textsl{специального покупателя}
либо обоснованное ожидание получения такого предложения;
\item ситуация, при~которой оцениваемые права не~могут быть переданы в~рамках
сделки, совершаемой на~открытом и~свободном рынке;
\item наличие изменений физических свойств имущества либо актива, произошедших
в~прошлом в~условиях, когда \emph{оценщику} необходимо исходить
из~того, что~такие изменения не~происходили;
\item предстоящее изменение физических свойств имущества, например, строительство
нового здания либо реконструкция или~снос существующего;
\item ожидаемые изменения характера использования недвижимого имущества
либо осуществляемой в~нём деятельности;
\item совершение действий в~части, касающейся перепланировок и~улучшений
недвижимого имущества, осуществляемых в~соответствии с~условиями
договора аренды;
\item воздействие на~имущество экологических факторов, включая природные
(например, наводнение), антропогенные (например, загрязнение) либо
факторов, связанных с~проблемами текущей эксплуатации (например,
недобросовестный пользователь).\label{4.4.8.4-End}
\end{itemize}
\stepcounter{SubSubSecCounter}

\thesubsubsection.\theSubSubSecCounter.\label{4.4.8.5} Ниже приводятся
примеры \textsl{специальных допущений} в~отношении недвижимого имущества:
\begin{itemize}
\item имеется либо будет получено согласование на~реализацию девелоперского
проекта (включая изменение вида разрешённого использования) в~отношении
объекта недвижимости;
\item строительство здания либо иной вариант девелопмента были завершены
в~соответствии с~определённым планом и~проектом;
\item объект недвижимости претерпел определённые изменения, например был~выполнен
демонтаж технологического оборудования;
\item объект недвижимости свободен, хотя в~реальности на~\textsl{дату
оценки} он~занят и~используется;
\item объект недвижимости сдаётся в~аренду на~определённых условиях, при~том,
что~в~действительности он~пустует;
\item обмен происходит между сторонами, одна или~несколько из~которых~имеют
\uline{особые интересы}, в~результате сочетания \uline{которых}
возникает дополнительная синергетическая стоимость.\label{4.4.8.5-End}
\end{itemize}
\stepcounter{SubSubSecCounter}

\thesubsubsection.\theSubSubSecCounter.\label{4.4.8.6} В~случаях,
когда имело место повреждение недвижимого имущества, могут быть приняты
следующие \textsl{специальные допущения}:
\begin{itemize}
\item рассмотрение объекта недвижимости как~восстановленного (с~учётом
вопросов требования страхового возмещения);
\item оценка объекта как~расчищенного и~свободного земельного участка
с~учётом наличия разрешения на~застройку в~соответствии с~текущим
видом разрешённого использования;
\item реконструкция либо перестройка для~другого вида использования с~учётом
вероятности перспективы получения необходимых разрешений и~согласований.\label{4.4.8.6-End}
\end{itemize}
\stepcounter{SubSubSecCounter}

\thesubsubsection.\theSubSubSecCounter. \label{4.4.8.7} Использование
подобных \textsl{специальных допущений} может определять возможность
применения \textsl{рыночной стоимости}. Зачастую они~бывают особенно
полезны в~тех~случаях, когда заказчик является кредитором, и~\textsl{специальные
допущения} используются для~демонстрации потенциального влияния изменившихся
обстоятельств на~стоимость недвижимого имущества, предоставляемого
в~качестве залога.\label{4.4.8.7-End}

\stepcounter{SubSubSecCounter}

\thesubsubsection.\theSubSubSecCounter.\label{4.4.8.8} При~проведении
\textsl{оценки} для~целей \textsl{финансовой отчётности} используемый
при~этом \textsl{вид стоимости}, как~правило, исключает возможность
существования дополнительной стоимости, возникающей вследствие применения
\textsl{специальных допущений}. Однако, если, в~порядке исключения,
\textsl{специальное допущение} всё~же было принято, указание на~это~обстоятельство
должно содержаться в~тексте отчёта и~любой ссылке на~него. См.~параграфы~\ref{subsubsec:4.3.2.9_Assumptions_and_special_assumptions},
\ref{subsubsec:4.3.2.12_Valuation_approach_and_reasoning} \vpageref{subsubsec:4.3.2.9_Assumptions_and_special_assumptions}--\pageref{subsubsec:4.3.2.9_Assumptions_and_special_assumptions-End},
\pageref{subsubsec:4.3.2.12_Valuation_approach_and_reasoning}--\pageref{subsubsec:4.3.2.12_Valuation_approach_and_reasoning-End}
соответственно.\label{4.4.8.8-End}\label{subsec:4.4.8_Special_assumptions-End}

\subsection{Оценки, отражающие существующие или~предполагаемые ограничения оборотоспособности
объекта оценки либо вынужденность продажи\label{subsec:4.4.9_Reflection_of_market_constraint}}

\textbf{В~тех~случаях, когда }\textbf{\emph{оценщиком}}\textbf{
либо заказчиком установлено, что~}\textbf{\textsl{оценка}}\textbf{
может потребовать отражения фактических либо предполагаемых ограничений
оборотоспособности }\textbf{\emph{объекта оценки}}\textbf{, детали
этих ограничений должны быть согласованы и~изложены в~условиях }\textbf{\textsl{договора
на~проведение оценки}}\textbf{.}

\textbf{Реализация.}

\stepcounter{SubSubSecCounter}

\thesubsubsection.\theSubSubSecCounter.\label{4.4.9.1} \emph{Оценщику}
может быть поручено провести \textsl{оценку}, отражающую фактическое
или~предполагаемое ограничение оборотоспособности \emph{объекта оценки},
которое может быть реализовано в~одной из~многих различных форм.\label{4.4.9.1-End}

\stepcounter{SubSubSecCounter}

\thesubsubsection.\theSubSubSecCounter.\label{4.4.9.2} Если имущество
либо актив не~могут быть свободно и~полноценно представлены на~рынке,
это, скорее всего, негативно скажется на~их~цене. Прежде чем~принять
в~работу задание, включающее в~себя консультацию о~вероятном влиянии
ограничений, \emph{оценщик} должен выяснить, следуют~ли они~из неотъемлемых
характеристик оцениваемого актива или~права либо из~обстоятельств,
относящихся к~заказчику, либо из~некоторой комбинации обеих причин.\label{4.4.9.2-End}

\stepcounter{SubSubSecCounter}

\thesubsubsection.\theSubSubSecCounter.\label{4.4.9.3} Как~правило,
существует возможность определить влияние на~стоимость неотъемлемого
ограничения, существующего на~\textsl{дату оценки}. Следует отражать
наличие такого ограничения в~условиях \textsl{договора на~проведение
оценки} таким образом, чтобы его~существование следовало в~явной
форме. Для~демонстрации влияния ограничения на~стоимость целесообразным
может быть проведение альтернативной \textsl{оценки}, основанной на~\textsl{специальном
допущении} о~том, что~оно~отсутствовало на~\textsl{дату оценки}.\label{4.4.9.3-End}

\stepcounter{SubSubSecCounter}

\thesubsubsection.\theSubSubSecCounter.\label{4.4.9.4} Необходимо
проявлять осторожность в~случаях, когда неотъемлемое ограничение
не~существует на~\textsl{дату оценки}, но~является предсказуемым
следствием определённого события или~последовательности событий.
С~другой стороны, заказчик может запросить проведение \textsl{оценки},
основанной на~определённых ограничениях оборотоспособности. В~любом
случае \textsl{оценка} будет проводиться на~основе \textsl{специального
допущения} о~том, что~ограничение возникло в~\textsl{дату оценки}.
Точное описание ограничений должно быть включено в~условия \textsl{договора
на~проведение оценки}. Для~демонстрации потенциального влияния ограничения
может быть целесообразно провести \textsl{оценку} без~\textsl{специального
допущения} о~его~существовании.\label{4.4.9.4-End}

\stepcounter{SubSubSecCounter}

\thesubsubsection.\theSubSubSecCounter.\label{4.4.9.5} \textsl{Специальное
допущение}, связанное с~ограниченным сроком экспозиции, не~содержащее
объяснения причин такого срока, не~является разумным \textsl{допущением}.
Без~чёткого понимания причин ограничения \emph{оценщик} не~может
определить влияние, которое оно~оказывает на~ликвидность, ход~переговоров
о~продаже и~достижимую цену, и~не~сможет предоставить содержательную
консультацию.\label{4.4.9.5-End}

\stepcounter{SubSubSecCounter}

\thesubsubsection.\theSubSubSecCounter.\label{4.4.9.6} Не~следует
смешивать понятие ограниченной оборотоспособности с~ситуацией вынужденной
продажи. Наличие ограничения может как~повлечь за~собой вынужденную
продажу, так~и~существовать, не~принуждая собственника к~таковой.\label{4.4.9.6-End}

\stepcounter{SubSubSecCounter}

\thesubsubsection.\theSubSubSecCounter.\label{4.4.9.7} Необходимо
отказаться от~использования термина \guillemotleft стоимость при~вынужденной
продаже\guillemotright . Понятие \guillemotleft вынужденная продажа\guillemotright{}
описывает ситуацию, в~условиях которой происходит сделка, не~образуя
при~этом самостоятельный \textsl{вид стоимости}. Вынужденная продажа
возникает тогда, когда продавец находится под~давлением, вынуждающим
его~совершить продажу к~определённому моменту времени, например
вследствие наличия необходимости привлечь средства или~погасить обязательства
к~определённой дате. Факт вынужденности продажи означает, что~продавец
находится под~воздействием юридических, личных либо коммерческих
обстоятельств, вследствие и~по~причине чего ограничение по~дате
продажи не~является простым предпочтением продавца. Также при~определении
стоимости в~данных обстоятельствах следует принимать во~внимание
характер таких внешних факторов и~последствия для~собственника в~случае
неудачи с~продажей в~установленный срок.\label{4.4.9.7-End}

\stepcounter{SubSubSecCounter}

\thesubsubsection.\theSubSubSecCounter.\label{4.4.9.8} Хотя \emph{оценщик}
может оказать продавцу содействие в~определении цены, которую следует
установить в~условиях вынужденной продажи, всё~же данный вопрос
относится к~сфере суждения предпринимателя. Любое отношение цены
вынужденной продажи объекта к~его~\textsl{рыночной стоимости} носит
случайный характер "--- оно~не~является расчётной величиной, значение
которой возможно заранее определить, представляя собой значение, отражающее
ценность объекта для~конкретного продавца в~конкретный момент времени
в~конкретном контексте. Как~было подчёркнуто в~п.~\ref{4.4.9.7}
\vpageref{4.4.9.7}--\pageref{4.4.9.7-End}, хотя предоставление
консультации о~вероятных параметрах сделки вынужденной продажи является
возможным, использование такой формулировки необходимо рассматривать
исключительно в~качестве описания обстоятельств сделки, но~не~для~описания
либо формирования \textsl{вида стоимости}.\label{4.4.9.8-End}

\stepcounter{SubSubSecCounter}

\thesubsubsection.\theSubSubSecCounter.\label{4.4.9.9} Существует
распространённое заблуждение о~том, что~на~узком или~падающем
рынке будет непременно мало \guillemotleft заинтересованных продавцов\guillemotright ,
и, как~следствие, большинство сделок на~рынке являются результатом
\guillemotleft вынужденных продаж\guillemotright . Соответственно,
\emph{оценщику} может быть предложено предоставить консультацию по~вынужденной
продаже на~такой основе. Данный аргумент имеет мало оснований, поскольку
предполагает, что~\emph{оценщик} должен игнорировать свидетельства
того, что~происходит на~рынке. Подраздел~\ref{subsec:4.4.3_Market-value}~\nameref{subsec:4.4.3_Market-value-End}
содержит комментарии, сопровождающие определение \textsl{рыночной
стоимости}, ясно указывающие на~то, что~заинтересованный продавец
намерен продать объект на~наилучших для~себя условиях, доступных
на~рынке после его надлежащего изучения, какой~бы ни~была цена
такой продажи. \emph{Оценщику} следует проявлять осторожность и~не~принимать
в~работу поручения, основанные на~заблуждениях, разъясняя заказчикам,
что, в~отсутствие определённых ограничений, оказывающих влияние на~актив
либо продавца, подходящим \textsl{видом стоимости} является \textsl{рыночная
стоимость}. На~депрессивном рынке значительная часть продаж может
осуществляться продавцами, имеющими обязанность совершать их, например
конкурсными управляющими, ликвидаторами и~правопреемниками. Однако
такие продавцы обычно обязаны получить наилучшую цену, являющуюся
возможной в~сложившихся обстоятельствах, и~не~могут по~своей воле
устанавливать необоснованные рыночные условия или~ограничения. Вследствие
этого, как правило, такие продажи всё~же соответствуют критериям
\textsl{рыночной стоимости}.\label{4.4.9.9-End}\label{subsec:4.4.9_Reflection_of_market_constraint-End}

\subsection{Допущения и~специальные допущения, связанные с~прогнозной стоимостью\label{subsec:4.4.10_Assumptions_for_projected_values}}

\textbf{Любые }\textbf{\textsl{допущения}}\textbf{, }\textbf{\textsl{специальные}}\textbf{
либо иные, относящиеся к~прогнозируемой стоимости, должны быть согласованы
с~заказчиком до~вынесения }\textbf{\emph{суждения о~стоимости}}\textbf{.}

\textbf{В~}\textbf{\emph{отчёте об~оценке}}\textbf{ должно быть
сделано упоминание о~том, что~определение прогнозной стоимости связано
с~относительно высокой степенью неопределённости, вызванной принципиальной
недоступность сопоставимых данных.}

\textbf{Реализация.}

\stepcounter{SubSubSecCounter}

\thesubsubsection.\theSubSubSecCounter.\label{4.4.10.1} По~своей
природе прогнозная стоимость полностью основывается на~\textsl{допущениях},
которые могут включать в~т.\,ч.~некоторые существенные \textsl{специальные
допущения}. Например, \emph{оценщик} может ввести ряд \textsl{допущений},
описывающих параметры рынка в~будущем такие~как~рыночная доходность,
рост арендной платы, процентные ставки и~т.\,д., которые должны
быть подкреплены достоверными исследованиями или~прогнозами, основанными
на перспективах экономики.\label{4.4.10-1-End}

\stepcounter{SubSubSecCounter}

\thesubsubsection.\theSubSubSecCounter.\label{4.4.10.2} Необходимо
удостовериться в~том, что~принятые \textsl{допущения} отвечают следующим
критериям:
\begin{itemize}
\item соответствие применимым стандартам, действующим в~соответствующей
стране или~юрисдикции;
\item реалистичность и~правдоподобность;
\item ясное и~всестороннее изложение в~\emph{отчёте}.\label{4.4.10.2-End}
\end{itemize}
\stepcounter{SubSubSecCounter}

\thesubsubsection.\theSubSubSecCounter.\label{4.4.10.3} При~принятии
\textsl{специальных допущений} также необходимо проявлять большую
осторожность в~части надёжности и~точности любых методов, инструментов
или~данных, используемых для~прогнозирования или~экстраполяции.\label{4.4.10.3-End}\label{subsec:4.4.10_Assumptions_for_projected_values-End}\label{sec:4.4_VPS4_Bases_of_value-End}

\newpage

\section{Подходы к~оценке и~её методы\label{sec:4.5_VPS5_Valuation_approaches_and_methods}}

\textbf{Данный обязательный стандарт:}
\begin{itemize}
\item \textbf{имплементирует \hyperref[sec:9.5_IVS-105_Valuation_Approaches]{МСО~105 Подходы и~методы оценки}
, см.~раздел~\ref{sec:9.5_IVS-105_Valuation_Approaches}~\nameref{sec:9.5_IVS-105_Valuation_Approaches}
\vpageref{sec:9.5_IVS-105_Valuation_Approaches}--\pageref{sec:9.5_IVS-105_Valuation_Approaches-End};}
\item \textbf{рассматривает отдельные аспекты имплементации, возникающие
в~конкретных случаях.}
\end{itemize}
\textbf{\emph{Оценщик}}\textbf{ отвечает за~выбор }\textbf{\emph{подходов}}\textbf{
и~}\textbf{\emph{методов}}\textbf{ }\textbf{\textsl{оценки}}\textbf{,
применяемых при~выполнении~конкретных заданий и, при~необходимости,
приводит обоснование такого выбора. Во~всех случаях необходимо принимать
во~внимание:}
\begin{itemize}
\item \textbf{природу актива~(обязательства);}
\item \textbf{цель, предполагаемое использование и~контекст конкретного
задания;}
\item \textbf{любые законодательные и~(или) иные обязательные требования,
действующие в~соответствующей юрисдикции.}
\end{itemize}
\textbf{Также }\textbf{\emph{\uline{оценщикам}}}\textbf{ следует
учитывать признанную передовую практику оценочной деятельности и~(или)~той~области
специализации, к~которой }\textbf{\uline{они}}\textbf{~обращаются
при~осуществлении своей практики, однако это~не~должно ограничивать
}\textbf{\uline{их}}\textbf{~в~применении }\textbf{\uuline{собственных
суждений}}\textbf{ в~рамках отдельных заданий по~}\textbf{\textsl{оценке}}\textbf{,
}\textbf{\uuline{обеспечивающих}}\textbf{ выполнение работы в~соответствии
с~требованиями профессии и~поставленной цели проведения }\textbf{\textsl{оценки}}\textbf{.}

\textbf{Никакой }\textbf{\emph{подход к~оценке}}\textbf{ или~какой-либо
конкретный }\textbf{\emph{метод оценки}}\textbf{ не~имеют априорной
преимущественной силы перед другими, если это~прямо не~предусмотрено
законом либо иными обязательными требованиями. При~проведении }\textbf{\textsl{оценки}}\textbf{
в~некоторых юрисдикциях и~(или) для~определённых задач в~целях
формирования сбалансированного }\textbf{\emph{суждения о~стоимости}}\textbf{
может ожидаться либо требоваться применение более чем~одного подхода.
В~связи с~этим }\textbf{\emph{оценщик}}\textbf{ всегда должен быть
готов привести обоснования выбора применённых им~}\textbf{\emph{подхода~(подходов)}}\textbf{
и~}\textbf{\emph{метода~(методов)}}\textbf{ }\textbf{\textsl{оценки}}\textbf{.}

\textbf{Реализация.}

\stepcounter{SecCounter}

\thesection.\theSecCounter.\label{4.5.1} Хотя не~существует формального
общепризнанного определения понятия \emph{подход к~оценке}, как~правило
под~этим термином понимается обобщённое описание способа выполнения
задачи по~\textsl{оценке} конкретного актива либо обязательства.
Термин \emph{метод оценки} как~правило используется для~обозначения
конкретной процедуры или~техники, используемой при~проведении \textsl{оценки}
и~расчёте её~результатов.\label{4.5.1-End}

\stepcounter{SecCounter}

\thesection.\theSecCounter.\label{4.5.2} Проведение \textsl{оценки}
осуществляется в~отношении широкого спектра прав и~активов и~для~множества
целей. Принимая во~внимание такое разнообразие, следует признать,
что~применение одного \emph{подхода к~оценке} может быть уместным
в~одних и~неуместным в~других случаях, что~в~ещё~большей степени
справедливо для~\emph{методов оценки}. Используя рабочее определение,
приведённое выше в~п.~\ref{4.5.1}, в~целом можно говорить о~том,
что~любой \emph{подход к~оценке} может быть отнесён к~одной из~трёх
категорий.
\begin{itemize}
\item \textsl{Рыночный подход} основан на~сравнении \emph{объекта оценки}
с~идентичными либо аналогичными активами (или~обязательствами),
в~отношении которых имеется ценовая информация, например, сравнении
с~фактическими сделками по~такому~же или~близкому типу актива
(или~обязательства) в~течение сопоставимого периода времени.
\item \textsl{Доходный подход} основан на~различных вариантах капитализации
или~приведения текущих и~прогнозируемых доходов (денежных потоков)
к~мгновенной текущей стоимости. В~зависимости от~типа актива и~того,
что~является распространённой практикой рыночных агентов, можно считать
уместными такие формы реализации как~капитализация типичного на~рынке
дохода или~дисконтирование конкретного прогнозного дохода.
\item \textsl{Затратный подход} основан на~экономическом принципе, согласно
которому, покупатель заплатит за~актив сумму не~большую, чем~затраты
на~получение актива равной полезности путём его~покупки либо создания.\label{4.5.2-End}
\end{itemize}
\stepcounter{SecCounter}

\thesection.\theSecCounter.\label{4.5.3} В~основе любого \emph{подхода
к~оценке} и~любого её~\emph{метода} лежит необходимость проведения
практически осуществимого сравнительного анализа, поскольку это~является
основным компонентом, необходимым для~получения \textsl{оценки},
обоснованной данными, наблюдаемыми на~рынке. Вполне вероятно, что~можно
придти к~единому \emph{суждению о~стоимости} на~основе применения
более чем~одного \emph{подхода}, а~также \emph{метода} или~техники,
если только закон или~какой-либо регулирующий орган не~устанавливают
конкретное требование в~этой части. Необходимо проявлять большую
осторожность при~использовании \textsl{затратного подхода} в~качестве
основного либо единственного, поскольку связь между затратами и~стоимостью
редко бывает прямой.\label{4.5.3-End}

\stepcounter{SecCounter}

\thesection.\theSecCounter.\label{4.5.4} \emph{Методы оценки} могут
включать ряд~аналитических инструментов или~методик, а~также различные
формы моделирования, многие из~которых предполагают использование
передовых численных и~статистических методов. В~целом можно говорить
о~том~что, чем~совершеннее применяемый \emph{метод}, тем~б\'{о}льшую
бдительность следует проявлять в~вопросах обеспечения отсутствия
внутренней несогласованности, например в~отношении принятых \textsl{допущений}.\label{4.5.4-End}

\stepcounter{SecCounter}

\thesection.\theSecCounter.\label{4.5.5} Дальнейшие детали применения
\emph{подходов} и~\emph{методов} можно найти в~\hyperref[sec:9.5_IVS-105_Valuation_Approaches]{МСО~105}~см.~раздел~\ref{sec:9.5_IVS-105_Valuation_Approaches}~\nameref{sec:9.5_IVS-105_Valuation_Approaches}
\vpageref{sec:9.5_IVS-105_Valuation_Approaches}--\pageref{sec:9.5_IVS-105_Valuation_Approaches-End},
включая обязанности \emph{оценщиков} в~части, касающейся моделирования
в~\textsl{оценке}. Необходимо подчеркнуть, что, в~конечном итоге,
\emph{оценщик} самостоятельно несёт ответственность за~выбор \emph{подхода}
\emph{(подходов)} и~\emph{метода} \emph{(методов)}, используемых
при~выполнении конкретных работ, если только закон либо регулятор
не~устанавливают конкретные требования в~этой части.\label{4.5.5-End}\label{sec:4.5_VPS5_Valuation_approaches_and_methods-End}\label{chap:4_VPS-End}

\chapter{Практические руководства по~оценке\label{chap:5_Valuation_applications}}

\subsection{Введение\label{5.0.1}}

Данная часть \emph{Всемирных стандартов RICS} посвящена их~имплементации
и~применению в~конкретных случаях как~для~определённых целей,
так~и~в~отношении \textsl{оценки} определённых видов активов.

Нижеследующие руководства RICS по~оценочной практике "--- \textbf{Практические
руководства по~проведению оценки}~\textbf{(ПР)} предназначены для~изложения
ключевых вопросов, которые необходимо принимать во~внимание, а~также
углублённого рассмотрения практического применения стандартов в~конкретных
условиях. Хотя сами по~себе \textbf{ПР} не~носят обязательный характер,
они~содержат ссылки на~важные материалы \href{https://www.rics.org/globalassets/rics-website/media/upholding-professional-standards/sector-standards/valuation/international-valuation-standards-rics2.pdf}{МСО}~\cite{IVS-2020},
а~также на~другие разделы настоящих \emph{Всемирных стандартов},
являющиеся обязательными. Данные ссылки предназначены для~того, чтобы
помочь \textsl{членам RICS} найти материал, относящийся к~конкретной
выполняемой ими~работе по~\textsl{оценке}.

Следует напомнить, что~\href{https://www.rics.org/globalassets/rics-website/media/upholding-professional-standards/sector-standards/valuation/international-valuation-standards-rics2.pdf}{МСО}~\cite{IVS-2020}
содержат следующие стандарты оценки активов, тексты которых полностью
воспроизведены в~главе~\ref{chap:10_Asset_Standards}~\nameref{chap:10_Asset_Standards}
\vpageref{chap:10_Asset_Standards}--\pageref{chap:10_Asset_Standards-End}
настоящего материала.
\begin{itemize}
\item \hyperref[sec:10.1_IVS-200_Businesses]{МСО~200. Оценка бизнеса и~долей в~нём}~(раздел~\ref{sec:10.1_IVS-200_Businesses}~\nameref{sec:10.1_IVS-200_Businesses}
\vpageref{sec:10.1_IVS-200_Businesses}--\pageref{sec:10.1_IVS-200_Businesses-End});
\item \hyperref[sec:10.2_IVS-210_Intangible_Assets]{МСО~210. Оценка прав на~нематериальные активы}~(раздел~\ref{sec:10.2_IVS-210_Intangible_Assets}~\nameref{sec:10.2_IVS-210_Intangible_Assets}
\vpageref{sec:10.2_IVS-210_Intangible_Assets}--\pageref{sec:10.2_IVS-210_Intangible_Assets-End});
\item \hyperref[sec:10.3_IVS-220_Non-Financial_Liabilities]{МСО~220. Оценка нефинансовых обязательств}~(раздел~\ref{sec:10.3_IVS-220_Non-Financial_Liabilities}\nameref{sec:10.3_IVS-220_Non-Financial_Liabilities}\vpageref{sec:10.3_IVS-220_Non-Financial_Liabilities}\pageref{sec:10.3_IVS-220_Non-Financial_Liabilities-End});
\item \hyperref[sec:10.4_IVS-300_Plant_and_Equipment]{МСО~300. Оценка машин и~оборудования}~(раздел~\ref{sec:10.4_IVS-300_Plant_and_Equipment}~\nameref{sec:10.4_IVS-300_Plant_and_Equipment}
\vpageref{sec:10.4_IVS-300_Plant_and_Equipment}--\pageref{sec:10.4_IVS-300_Plant_and_Equipment-End});
\item \hyperref[sec:10.5_IVS-400_Real_Property]{МСО~400. Оценка недвижимости}~(раздел~\ref{sec:10.5_IVS-400_Real_Property}~\nameref{sec:10.5_IVS-400_Real_Property}
\vpageref{sec:10.5_IVS-400_Real_Property}--\pageref{sec:10.5_IVS-400_Real_Property-End});
\item \hyperref[sec:10.6_IVS-410_Development_Property]{МСО~410. Оценка недвижимости для~целей девелопмента}~(раздел~\ref{sec:10.6_IVS-410_Development_Property}~\nameref{sec:10.6_IVS-410_Development_Property}
\vpageref{sec:10.6_IVS-410_Development_Property}--\pageref{sec:10.6_IVS-410_Development_Property-End});
\item \hyperref[sec:10.7_IVS-500_Financial_Instruments]{Оценка финансовых активов}~(раздел~\ref{sec:10.7_IVS-500_Financial_Instruments}~\nameref{sec:10.7_IVS-500_Financial_Instruments}
\vpageref{sec:10.7_IVS-500_Financial_Instruments}--\pageref{sec:10.7_IVS-500_Financial_Instruments-End}).
\end{itemize}
Полный перечень \textbf{Практических руководств} включает в~себя:
\begin{itemize}
\item \hyperref[sec:5.1_VPGA-1_Valuation_for_financial_statements]{ПР~1. Оценка для~целей финансовой отчётности}~(раздел~\ref{sec:5.1_VPGA-1_Valuation_for_financial_statements}~\nameref{sec:5.1_VPGA-1_Valuation_for_financial_statements}
\vpageref{sec:5.1_VPGA-1_Valuation_for_financial_statements}--\pageref{sec:5.1_VPGA-1_Valuation_for_financial_statements-End});
\item \hyperref[sec:5.2_VPGA-2_Valuation_for_secure_lending]{ПР~2. Оценка для~целей залогового кредитования}~(раздел~\ref{sec:5.2_VPGA-2_Valuation_for_secure_lending}~\nameref{sec:5.2_VPGA-2_Valuation_for_secure_lending}
\vpageref{sec:5.2_VPGA-2_Valuation_for_secure_lending}--\pageref{sec:5.2_VPGA-2_Valuation_for_secure_lending-End});
\item \hyperref[sec:5.3_VPGA-3_Valuation_of_businesses]{ПР~3.~Оценка бизнеса и~долей в~нём}~(раздел~\ref{sec:5.3_VPGA-3_Valuation_of_businesses}~\nameref{sec:5.3_VPGA-3_Valuation_of_businesses}
\vpageref{sec:5.3_VPGA-3_Valuation_of_businesses}--\pageref{sec:5.3_VPGA-3_Valuation_of_businesses-End});
\item \hyperref[sec:5.4_VPGA-4_Valuation_of_trade_properties]{ПР~4.~Оценка специализированных активов}~(раздел~\ref{sec:5.4_VPGA-4_Valuation_of_trade_properties}~\nameref{sec:5.4_VPGA-4_Valuation_of_trade_properties}
\vpageref{sec:5.4_VPGA-4_Valuation_of_trade_properties}--\pageref{sec:5.4_VPGA-4_Valuation_of_trade_properties-End});
\item \hyperref[sec:5.5_VPGA-5-Valuation_of_plant_and_equipment]{ПР~5.~Оценка машин и~оборудования}~(раздел~\ref{sec:5.5_VPGA-5-Valuation_of_plant_and_equipment}~\nameref{sec:5.5_VPGA-5-Valuation_of_plant_and_equipment}
\vpageref{sec:5.5_VPGA-5-Valuation_of_plant_and_equipment}--\pageref{sec:5.5_VPGA-5-Valuation_of_plant_and_equipment-End});
\item \hyperref[sec:5.6_VPGA-6_Valuation_of_intangible_assets]{ПР~6.~Оценка прав на~нематериальные активы}~(раздел~\ref{sec:5.6_VPGA-6_Valuation_of_intangible_assets}~\nameref{sec:5.6_VPGA-6_Valuation_of_intangible_assets}
\vpageref{sec:5.6_VPGA-6_Valuation_of_intangible_assets}--\pageref{sec:5.6_VPGA-6_Valuation_of_intangible_assets-End});
\item \hyperref[sec:5.7_VPGA-7_Valuation_of_personal_propetry]{ПР~7.~Оценка личной собственности}~(раздел~\ref{sec:5.7_VPGA-7_Valuation_of_personal_propetry}~\nameref{sec:5.7_VPGA-7_Valuation_of_personal_propetry}
\vpageref{sec:5.7_VPGA-7_Valuation_of_personal_propetry}--\pageref{sec:5.7_VPGA-7_Valuation_of_personal_propetry-End});
\item \hyperref[sec:5.8_VPGA-8_Valuation_of_real_property]{ПР~8.~Оценка недвижимости}~(раздел~\ref{sec:5.8_VPGA-8_Valuation_of_real_property}~\nameref{sec:5.8_VPGA-8_Valuation_of_real_property}
\vpageref{sec:5.8_VPGA-8_Valuation_of_real_property}--\pageref{sec:5.8_VPGA-8_Valuation_of_real_property-End});
\item \hyperref[sec:5.9_VPGA-9_Identification_of_portfolios]{ПР~9.~Выявление портелей, коллекций и~групп активов}~(раздел~\ref{sec:5.9_VPGA-9_Identification_of_portfolios}~\nameref{sec:5.9_VPGA-9_Identification_of_portfolios}
\vpageref{sec:5.9_VPGA-9_Identification_of_portfolios}--\pageref{sec:5.9_VPGA-9_Identification_of_portfolios-End});
\item \hyperref[sec:5.10_VPGA-10_Matters_for_uncertainty]{ПР~10. Причины возникновения существенной неопределённости в~\textsl{оценке}}~(раздел~\ref{sec:5.10_VPGA-10_Matters_for_uncertainty}~\nameref{sec:5.10_VPGA-10_Matters_for_uncertainty}
\vpageref{sec:5.10_VPGA-10_Matters_for_uncertainty}--\pageref{sec:5.10_VPGA-10_Matters_for_uncertainty-End}).\label{5.0.1-End}
\end{itemize}

\section{ПР~1.~Оценка для~целей финансовой отчётности\label{sec:5.1_VPGA-1_Valuation_for_financial_statements}}

\textbf{Данное руководство носит рекомендательный характер и~не~содержит
обязательных требований. Однако там, где~это~уместно, оно~содержит
отсылки к~соответствующему обязательному материалу, содержащемуся
в~других разделах настоящих Всемирных стандартов, а~также \href{https://www.rics.org/globalassets/rics-website/media/upholding-professional-standards/sector-standards/valuation/international-valuation-standards-rics2.pdf}{Международных стандартов оценки}~\cite{IVS-2020},
реализованные в~виде перекрёстных ссылок. Данные ссылки предназначены
для~помощи }\textbf{\textsl{членам RICS}}\textbf{ и~не~меняют статус
данного практического руководства. }\textbf{\textsl{Членам RICS}}\textbf{
следует помнить следующее:}
\begin{itemize}
\item \textbf{данное руководство не~может охватить все~возможные варианты,
вследствие чего }\textbf{\emph{оценщикам}}\textbf{ при~формировании
своих }\textbf{\emph{суждений о~стоимости}}\textbf{ всегда необходимо
учитывать факты и~обстоятельства, имеющие место в~рамках отдельных
заданий по~}\textbf{\textsl{оценке}}\textbf{;}
\item \textbf{следует внимательно относиться к~тому факту, что~в~ряде
юрисдикций могут существовать особые требования, не~предусмотренные
данным руководством.}
\end{itemize}

\subsection{Область применения\label{subsec:5.1.1_Scope}}

\stepcounter{SubSecCounter}

\thesubsection.\theSubSecCounter.\label{5.1.1.1} Приведённое ниже
руководство содержит дополнительные комментарии, касающиеся оценки
имущества, активов и~обязательств, выполняемой в~целях~её~включения
в~\textsl{финансовую отчётность}.\label{5.1.1.1-End}

\stepcounter{SubSecCounter}

\thesubsection.\theSubSecCounter.\label{5.1.1.2} Необходимо проявлять
особую осмотрительность в~вопросе строгого соответствия \textsl{оценок},
выполняемых для~целей \textsl{финансовой отчётности}, принятым в~организации
стандартам \textsl{финансовой отчётности}. \emph{Оценщикам} настоятельно
рекомендуется с~самого начала уточнить какие стандарты отчётности
применяются заказчиком.\label{5.1.1.2-End}

\stepcounter{SubSecCounter}

\thesubsection.\theSubSecCounter.\label{5.1.1.3} В~то~время как
в~данный момент~\href{http://docs.cntd.ru/document/420332842/}{Международные стандарты финансовой отчетности}
(\href{https://www.ifrs.org/issued-standards/list-of-standards/}{IFRS})~\cite{MSFO-all,IFRS-all}
находят всё~более широкое применение, в~отдельных юрисдикциях могут
применяться и~другие стандарты \textsl{финансовой отчетности}.\label{5.1.1.3-End}

\stepcounter{SubSecCounter}

\thesubsection.\theSubSecCounter.\label{5.1.1.4} Во~всех случаях
\emph{\uline{оценщикам}} следует помнить о~том, что~стандарты
\textsl{финансовой отчётности} "--- будь~то \href{http://docs.cntd.ru/document/420332842/}{МСФО}~\cite{MSFO-all}~(\href{https://www.ifrs.org/issued-standards/list-of-standards/}{IFRS}~\cite{IFRS-all})
либо иные стандарты "--- продолжают развиваться, вследствие и~по~причине
чего \uline{им} следует использовать ту~версию стандартов, которая
действует на~дату составления \textsl{финансовой отчётности} для~которой
осуществляется \textsl{оценка}.\label{5.1.1.4-End}\label{subsec:5.1.1_Scope-End}

\subsection{Проведение оценки согласно МСФО\label{subsec:5.1.2_Valuations_under_IFRS}}

\stepcounter{SubSecCounter}

\thesubsection.\theSubSecCounter.\label{5.1.2.1} В~случае когда
заказчик \textsl{оценки} применяет \href{http://docs.cntd.ru/document/420332842/}{МСФО}~\cite{MSFO-all},
определяемым \textsl{видом стоимости} будет \textsl{справедливая стоимость}
(см.~подраздел~\ref{subsec:4.4.6_Fair_value}~\nameref{subsec:4.4.6_Fair_value}
\vpageref{subsec:4.4.6_Fair_value}--\pageref{subsec:4.4.6_Fair_value-End}),
а~также будут применяться положения \href{http://docs.cntd.ru/document/420334241}{МСФО 13 «Оценка справедливой стоимости»}~\cite{MSFO-13}
(\href{http://eifrs.ifrs.org/eifrs/bnstandards/en/IFRS13.pdf}{IFRS 13 Fair Value Measurement}~\cite{IFRS-13}).
Важно, чтобы \emph{оценщик} был знаком с~его~содержанием, особенно
в~части, касающейся вопросов \emph{раскрытия информации}.\label{5.1.2.1-End}\label{subsec:5.1.2_Valuations_under_IFRS-End}

\subsection{Оценка стоимости активов государственного сектора в~соответствии
с~Международными стандартами учёта в~государственном секторе с~\label{subsec:5.1.3_Valuations-under-IPSAS}}

\stepcounter{SubSecCounter}

\thesubsection.\theSubSecCounter.\label{5.1.3.1} В~случае, когда
активы государственного сектора должны быть включены в~\textsl{финансовую
отчётность}, соответствующую \href{https://minfin.gov.ru/ru/perfomance/budget/bu_gs/sfo/}{МСУГС}~\cite{MSUGS}
(\href{https://www.ifac.org/ipsasb/standards-pronouncements}{IPSAS}\href{https://www.ifac.org/ipsasb/standards-pronouncements}{IPSAS})~\cite{IPSAS_2021},
необходимо позаботиться о~том, чтобы использовалась их~версия, применимая
на~дату \textsl{финансовой отчетности}.\label{5.1.3.1-End}\label{subsec:5.1.3_Valuations-under-IPSAS-End}

\subsection{Иные случаи\label{subsec:5.1.4_Other_cases}}

\stepcounter{SubSecCounter}

\thesubsection.\theSubSecCounter.\label{5.1.4.1} Законодательные,
нормативные, бухгалтерские или~иные требования, существующие в~конкретной
юрисдикции, могут потребовать изменения порядка применения данного
руководства в~некоторых странах~(территориях) или~при~определённых
условиях.\label{5.1.4.1-End}\label{subsec:5.1.4_Other_cases-End}

\subsection{Стандарты исполнения\label{subsec:5.1.5_Performance_standards}}

\subsubsection{Общие требования\label{subsubsec:5.1.5.1_Common}}

\stepcounter{SubSubSecCounter}

\thesubsubsection.\theSubSubSecCounter.\label{5.1.5.1.1} Составление
финансовой отчётности как~в~соответствии с~\href{http://docs.cntd.ru/document/420332842/}{МСФО}~\cite{MSFO-all},
так~и~иным образом во~всех случаях должно осуществляться с~надлежащей
тщательностью, что~в~равной мере относится и~к~проведению \textsl{оценки}
активов (обязательств), несущей вспомогательную роль в~этом процессе.
С~учётом конкретных законодательных, нормативных или~иных регуляторных
требований (см.~раздел~\ref{subsec:3.1.4_Compliance_with_jurisdictional_standards}~\nameref{subsec:3.1.4_Compliance_with_jurisdictional_standards}
\vpageref{subsec:3.1.4_Compliance_with_jurisdictional_standards}--\pageref{subsec:3.1.4_Compliance_with_jurisdictional_standards-End}),
применимых к~конкретному случаю оказания услуг по~\textsl{оценке}
в~соответствующей юрисдикции, рекомендуется придерживаться следующих
практических рекомендаций (следует обратить внимание на~то, что~в
некоторых юрисдикциях подобные практики могут являться обязательными
при~проведении \textsl{оценки}). В~случаях, когда подобные практические
рекомендации содержат элементы, носящие обязательный характер для~\textsl{членов
RICS} и~\textsl{оценочных компаний}, приводятся соответствующие перекрёстные
ссылки.\label{5.1.5.1.1-End}\label{subsubsec:5.1.5.1_Common-End}

\subsubsection{Профессионализм и~компетентность\label{subsubsec:5.1.5.2_Professionalism}}

\stepcounter{SubSubSecCounter}

\thesubsubsection.\theSubSubSecCounter.\label{5.1.5.2.1} Обязательные
общие требования, касающиеся этических стандартов (см.~подраздел~\ref{subsec:3.2.1_Professional_and_ethical_standards}~\nameref{subsec:3.2.1_Professional_and_ethical_standards}
\vpageref{subsec:3.2.1_Professional_and_ethical_standards}--\pageref{subsec:3.2.1_Professional_and_ethical_standards-End}),
квалификации и профессиональной компетентности \textsl{членов RICS}
(см.~подраздел~\ref{subsec:3.2.2_Member qualification}~\nameref{subsec:3.2.2_Member qualification}
\vpageref{subsec:3.2.2_Member qualification}--\pageref{subsec:3.2.2_Member qualification-End}),
применяются здесь в~равной степени наряду с~требованиями, обеспечивающими
уверенность в~компетентности как~в~вопросах формирования условий
\textsl{договора на~проведение оценки} (см.~подраздел~\ref{subsec:4.1.3_Terms_of_engagement_scope_of_work}~\nameref{subsec:4.1.3_Terms_of_engagement_scope_of_work}
\vpageref{subsec:4.1.3_Terms_of_engagement_scope_of_work}--\pageref{subsec:4.1.3_Terms_of_engagement_scope_of_work-End}),
так~и~в~вопросах составления \emph{отчёта об~оценке} (см.~подраздел~\ref{subsec:4.3.2_Report_content}~\nameref{subsec:4.3.2_Report_content}
\vpageref{subsec:4.3.2_Report_content}--\pageref{subsec:4.3.2_Report_content-End}).\label{5.1.5.2.1-End}

\stepcounter{SubSubSecCounter}

\thesubsubsection.\theSubSubSecCounter.\label{5.1.5.2.2} Особенно
важным моментом в~контексте целей \textsl{оценки}, включаемой в~\textsl{финансовую
отчётность}, также является проявление профессионального скептицизма
(см.~п.~\ref{3.2.1.5} \vpageref{3.2.1.5}--\pageref{3.2.1.5-End},
п.~\ref{4.2.1.2} \vpageref{4.2.1.2}--\pageref{4.2.1.2-End}),
способного продемонстрировать непредвзятость \emph{оценщика}.\label{5.1.5.2.2-End}

\stepcounter{SubSubSecCounter}

\thesubsubsection.\theSubSubSecCounter.\label{5.1.5.2.3} В~тех~случаях,
когда \textsl{член RICS} не~обладает уровнем знаний и~опыта, необходимым
для~надлежащего выполнения некоторых аспектов оказания услуг по~\textsl{оценке},
он~может "--- с~согласия заказчика, когда это~необходимо или~уместно,
"--- использовать знания и~опыт других лиц (см.п~\ref{3.2.2.4}--\ref{3.2.2.7}
\vpageref{3.2.2.7}--\pageref{3.2.2.7-End}).\label{5.1.5.2.3-End}

\stepcounter{SubSubSecCounter}

\thesubsubsection.\theSubSubSecCounter.\label{5.1.5.2.4} В~случае
проведения \textsl{оценки} с~привлечением субподрядчиков, \textsl{члену
RICS} либо \textsl{оценочной компании} необходимо убедиться в~том,
что~требования \hyperref[sec:3.1_PS1_Compliance_of_written_valuation]{\textbf{СПД~1.~Соответствие настоящим стандартам при~подготовке письменного отчёта об~оценке}}могу
быть выполнены (см.~п.~\ref{3.2.2.7} \vpageref{3.2.2.7}--\pageref{3.2.2.7-End}).
Необходимо задокументировать и~в~явном виде указать в~\emph{отчёте
об~оценке} уровень ответственности, которую принимает на~себя каждая
из~сторон договора субподряда.\label{5.1.5.2.4-End}

\stepcounter{SubSubSecCounter}

\thesubsubsection.\theSubSubSecCounter.\label{5.1.5.2.5} В~той~части
итогового результата, в~которой \textsl{член RICS} не~может либо
не~согласен принимать на~себя ответственность за~результаты, предоставленные
другими специалистами, необходимо хранить отчёт субподрядчика о~выполненной
им~части работы (равно как~и~иные сопутствующие документы), в~.т.\,ч.~сведения
о~том, что~субподрядчик ознакомлен с~формой предоставления результатов
его~работы в~составе итогового \emph{отчёта об~оценке}, выполненного
\textsl{членов RICS}, и~удовлетворён ей (см.~п.~.\ref{4.3.2.11.5}\vpageref{4.3.2.11.5}\pageref{4.3.2.11.5-End}).\label{5.1.5.2.5-End}\label{subsubsec:5.1.5.2_Professionalism-End}

\subsubsection{Условия договора на~проведение оценки\label{subsec:5.1.5.3_Terms_of_engagement}}

\stepcounter{SubSubSecCounter}

\thesubsubsection.\theSubSubSecCounter.\label{5.1.5.3.1} В~контексте
проведения \textsl{оценки} для~целей \textsl{финансовой отчётности}
особенно важно иметь чёткие условия \textsl{договора на~проведения
оценки} (см.~подраздел~\ref{subsec:3.2.7_Terms_of_engagement_Scope_of_work}~\nameref{subsec:3.2.7_Terms_of_engagement_Scope_of_work}
\vpageref{subsec:3.2.7_Terms_of_engagement_Scope_of_work}--\pageref{subsec:3.2.7_Terms_of_engagement_Scope_of_work-End}),
не~оставляющие сомнений в~части того, что~входит в~периметр работы
\emph{оценщика}, независимо от~комплексности и~иных параметров задания,
включающие в~себя способ воспроизведения результатов \textsl{оценки}
(см.~п.~\ref{4.3.2.10.2}--\ref{4.3.2.10.14} \vpageref{4.3.2.10.2}--\pageref{4.3.2.10.14-End}).
\textsl{Членам RICS} необходимо помнить о~том, что~любые изменения
\emph{задания на~оценку}, необходимость которых возникает либо согласовывается
по~ходу выполнения работы, должны быть доведены до~сведения заказчика
и~надлежащим образом зафиксированы документально до~момента выпуска
\emph{отчёта об~оценке} (см.~п.~\ref{4.1.1.6} \vpageref{4.1.1.6}--\pageref{4.1.1.6-End}).\label{5.1.5.3.1-End}

\stepcounter{SubSubSecCounter}

\thesubsubsection.\theSubSubSecCounter.\label{5.1.5.3.2} В~дополнение
к~исчерпывающему перечню вопросов, рассмотренных в~\hyperref[sec:4.1_VPS1_Terms_of_engagement_Scope_of_work]{\textbf{СПО~1.~Условия договора на~проведение оценки (Задания на~оценку)}}
\vpageref{sec:4.1_VPS1_Terms_of_engagement_Scope_of_work}--\pageref{sec:4.1_VPS1_Terms_of_engagement_Scope_of_work-End},
важно убедиться в~том, что~чётко определены обязанности заказчика
в~той мере, в~какой они~оказывают влияние на~проведение \textsl{оценки}
и~составление \emph{отчёта об~оценке} "--- данное условие может
включать, но~не~обязательно ограничивается указанием характера сведений,
которую заказчик соглашается предоставить. Также ожидается, что~условия
\textsl{договора на~проведение оценки} конкретизируют сроки, в~течение
которых \emph{отчёт} будет подготовлен и~выпущен.\label{5.1.5.3.2-End}\label{subsec:5.1.5.3_Terms_of_engagement-End}

\subsubsection{Источники и~проверка информации и~данных\label{subsubsec:5.1.5.4_Information}}

\stepcounter{SubSubSecCounter}

\thesubsubsection.\theSubSubSecCounter.\label{5.1.5.4.1} В~параграфах~\ref{subsubsec:4.1.3.10_Nature_and_sources_of_information}
\vpageref{subsubsec:4.1.3.10_Nature_and_sources_of_information}--\pageref{subsubsec:4.1.3.10_Nature_and_sources_of_information-End}
и~\ref{subsubsec:4.3.2.8_Nature_and_source_of_the_information} \vpageref{subsubsec:4.3.2.8_Nature_and_source_of_the_information}--\pageref{subsubsec:4.3.2.8_Nature_and_source_of_the_information-End}
рассматриваются вопросы характера и~источников информации и~данных,
на~которые возможно опираться при~проведении \textsl{оценки}.\label{5.1.5.4.1-End}

\stepcounter{SubSubSecCounter}

\thesubsubsection.\theSubSubSecCounter.\label{5.1.5.4.2} \uuline{Исходные
документы} могут включать информацию и~данные, предоставленные \uline{организацией-заказчиком},
и~могут содержать финансовую информацию из~источников, отличных
от~\uline{её} аудированной \textsl{финансовой отчётности}. Также
\uuline{они}~могут включать сведения, подготовленные сторонними
консультантами или~специалистами, привлечёнными этой организацией.
В~любом случае сведения, предоставленные руководством отчитывающейся
организации либо посредством него, должны рассматриваться так~же
объективно, как~и~сведения из~других источников, при~этом при~формировании
мнения об~их достоверности следует применять профессиональный скептицизм
(см.~п.~\ref{4.2.1.8} \vpageref{4.2.1.8}--\vpageref{4.2.1.8-End}).
В~случае обсуждения вопросов, касающихся исходной информации и~данных
с~заказчиком следует применять положения п.\ref{3.2.3.11}--\ref{3.2.3.15}
\vpageref{3.2.3.11}--\vpageref{3.2.3.15-End}, которые распространяют
своё действие также и~на иные случаи опроса руководителей организации-заказчика.\label{5.1.5.4.2-End}\label{subsubsec:5.1.5.4_Information-End}

\subsubsection{Работа с~документами и~ведение записей\label{subsec:5.1.5.4_Documentation}}

\stepcounter{SubSubSecCounter}

\thesubsubsection.\theSubSubSecCounter.\label{5.1.5.5.1} Большое
внимание необходимо уделять документации, включающей исходные и~аналитические
материалы, а~также любые другие соответствующие документы или~рабочие
записи такие как~записи об~\textsl{осмотрах} и~исследованиях (см.~подраздел~\ref{subsec:4.2.3_Valuation_records}~\nameref{subsec:4.2.3_Valuation_records}
\vpageref{subsec:4.2.3_Valuation_records}--\pageref{subsec:4.2.3_Valuation_records-End}),
используемой для~формирования \emph{суждения о~стоимости} и~обеспечения
его~доказательной силы. Данное требование обусловлено тем, что~руководство
хозяйственного общества, подготавливающего \textsl{финансовую отчётность},
полагается на\emph{~отчёт об~оценке} в~процессе подготовки этой
отчётности, и~несёт за~неё~ответственность в~части её~полного
и~строгого соответствия законодательным, нормативным и аудиторским
требованиям. В~случае затрагивания общественных интересов \textsl{членам
RICS} следует обратить внимание на~секцию~\ref{subsubsec:3.2.5.1_Disclosure_requirements}
\vpageref{subsubsec:3.2.5.1_Disclosure_requirements}--\pageref{subsubsec:3.2.5.1_Disclosure_requirements-End},
а~в~случае наличия заинтересованности какой-либо \emph{третьей стороны}
"--- на~секцию~\ref{subsubsec:3.2.5.2_Reliance_by_third_parties}
\vpageref{subsubsec:3.2.5.2_Reliance_by_third_parties}--\pageref{subsubsec:3.2.5.2_Reliance_by_third_parties-End}
(в~особенности на~п.~\vref{3.2.5.2.2}--\pageref{3.2.5.2.2-End})\label{5.1.5.5.1-End}\label{subsec:5.1.5.4_Documentation-End}

\subsubsection{Отчёты об~оценке\label{subsec:5.1.5.6_Reports}}

\stepcounter{SubSubSecCounter}

\thesubsubsection.\theSubSubSecCounter.\label{5.1.5.6.1} В~\emph{отчётах
об~оценке}, подготавливаемых для~целей \textsl{финансовой отчётности},
\guillemotleft в~чёткой и~точной форме должны быть изложены выводы
по~результатам проведения \textsl{оценки}, при~этом повествование
не~должно содержать каких-либо двусмысленных либо вводящих в~заблуждение
выводов, а~также формировать ложное восприятия суждений \emph{оценщика}\guillemotright{}
"--- см.~\hyperref[sec:4.3_VPS3_Valuation_reports]{\textbf{СПО~3.~Требования к~содержанию отчёта об~оценке}}
\vpageref{sec:4.3_VPS3_Valuation_reports}--\pageref{sec:4.3_VPS3_Valuation_reports-End},
содержащий полный перечень обязательных требований, направленных на~достижение
этой цели. Нижеследующие пункты \hyperref[5.1.5.6.2]{5.1.5.6.2}--\hyperref[5.1.5.6.5]{5.1.5.6.5}
\vpageref{5.1.5.6.2}--\pageref{5.1.5.6.5-End} содержат дополнительные
указания по~некоторым аспектам в~контексте проведения \textsl{оценки}
для~целей \textsl{финансовой отчётности}. \label{5.1.5.6.1-End}

\stepcounter{SubSubSecCounter}

\thesubsubsection.\theSubSubSecCounter.\label{5.1.5.6.2} Полнота
описания использованных \emph{подхода(ов) и~метода(ов) оценки} и~доказательств
всегда должна быть пропорциональна задаче (см.~п.~\ref{4.3.2.12.2}
\vpageref{4.3.2.12.2}--\pageref{4.3.2.12.2-End}), но~в~конкретном
контексте \textsl{оценки} для~целей \textsl{финансовой отчётности}
предполагается более подробное описание, облегчающее понимание заказчиком
и~другими предполагаемыми пользователями. Это~способствует обеспечению
надлежащего обоснования управленческих решений в~части полноты и~содержания
\textsl{финансовой отчётности} в~отношении соответствующих активов
(обязательств), а~также гарантирует, что~руководство хозяйственного
общества сможет, в~свою очередь, составить такую \textsl{отчётность},
которая не~будет допускать её~двусмысленного толкования либо вводить
в~заблуждение.\label{5.1.5.6.2-End}

\stepcounter{SubSubSecCounter}

\thesubsubsection.\theSubSubSecCounter.\label{5.1.5.6.3} \textsl{Членам
RICS} следует проявлять особенную осмотрительность в~отношении \uline{информации
и~данных}, которые, как~представляется, не~согласуются с~результатами
\textsl{оценки} и~противоречат им, а~также включать в~\emph{отчёт
об~оценке} сведения о~том, каким образом \uline{они}~были учтены.\label{5.1.5.6.3-End}

\stepcounter{SubSubSecCounter}

\thesubsubsection.\theSubSubSecCounter.\label{5.1.5.6.4}~В~то~время
как~в~\hyperref[sec:4.1_VPS1_Terms_of_engagement_Scope_of_work]{\textbf{СПО~1.~Условия договора на~проведение оценки (Задания на~оценку)}}
и~\hyperref[sec:4.3_VPS3_Valuation_reports]{\textbf{СПО~3.~Требования к~содержанию отчёта об~оценке}}
говорится о~\textsl{дате оценки} и~результатах \textsl{оценки},
в~том~смысле, в~котором эти~выделенные в~тексте \textsl{наклонным
шрифтом} термины определены в~\hyperref[chap:2_Glossary]{Глоссарии RICS},
\emph{оценщики} могут столкнуться с~тем, что~заказчики оперируют
терминами \guillemotleft дата измерения\guillemotright{} и~\guillemotleft сумма
оценки\guillemotright , принятыми в~стандартах бухгалтерского учёта.
Как~говорится в~п.\ref{9.4.1.1} \vpageref{9.4.1.1}--\pageref{9.4.1.1-End},
заказчики вместо термина \textsl{вид~стоимости} иногда могут использовать
термин \guillemotleft стандарт стоимости\guillemotright .\label{5.1.5.6.4-End}

\stepcounter{SubSubSecCounter}

\thesubsubsection.\theSubSubSecCounter.\label{5.1.5.6.5} Если в~\emph{отчёте
об оценке} упоминаются события, произошедшие после \textsl{даты оценки},
(см.~п.~\ref{4.3.2.13.7} \vpageref{4.3.2.13.7}--\pageref{4.3.2.13.7-End}),
следует обеспечить применение актуальной \textsl{даты оценки}.\label{5.1.5.6.5-End}\label{subsec:5.1.5.6_Reports-End}\label{subsec:5.1.5_Performance_standards-End}\label{sec:5.1_VPGA-1_Valuation_for_financial_statements-End}

\newpage

\section{ПР~2. Оценка прав на~имущество в~целях залогового кредитования\label{sec:5.2_VPGA-2_Valuation_for_secure_lending}}

\textbf{Данное руководство носит рекомендательный характер и~не~содержит
обязательных требований. Однако там, где~это~уместно, оно~содержит
отсылки к~соответствующему обязательному материалу, содержащемуся
в~других разделах настоящих Всемирных стандартов, а~также \href{https://www.rics.org/globalassets/rics-website/media/upholding-professional-standards/sector-standards/valuation/international-valuation-standards-rics2.pdf}{Международных стандартов оценки}~\cite{IVS-2020},
реализованные в~виде перекрёстных ссылок. Данные ссылки предназначены
для~помощи }\textbf{\textsl{членам RICS}}\textbf{ и~не~меняют статус
данного практического руководства. }\textbf{\textsl{Членам RICS}}\textbf{
необходимо помнить следующее:}
\begin{itemize}
\item \textbf{данное руководство не~может охватить все~возможные варианты,
вследствие чего }\textbf{\emph{оценщикам}}\textbf{ при~формировании
своих }\textbf{\emph{суждений о~стоимости}}\textbf{ всегда следует
учитывать факты и~обстоятельства, имеющие место в~рамках отдельных
заданий по~}\textbf{\textsl{оценке}}\textbf{;}
\item \textbf{следует внимательно относиться к~тому факту, что~в~ряде
юрисдикций могут существовать особые требования, не~предусмотренные
данным руководством.}
\end{itemize}

\subsection{Область применения\label{subsec:5.2.1_Scope}}

\stepcounter{SubSecCounter}

\thesubsection.\theSubSecCounter.\label{5.2.1.1} Данное практическое
руководство содержит дополнительные комментарии, касающиеся проведения
\textsl{оценки} прав на~недвижимое имущество и~иные материальные
активы в~целях залогового кредитования.\label{5.2.1.1-End}

\stepcounter{SubSecCounter}

\thesubsection.\theSubSecCounter.\label{5.2.1.2} В~тексте данного
руководства приводятся перекрёстные ссылки на~обязательные требования
настоящих \emph{Всемирных стандартов}. Хотя остальные требования носят
рекомендательный характер в~большинстве юрисдикций, \emph{оценщикам}
всегда следует обращаться к~соответствующим \href{https://www.isurv.com/info/1342/rics_national_or_jurisdictional_valuation_standards}{национальным дополнениям стандартов}~\cite{RICS:National-Standards}
(см.~подраздел~\ref{subsec:3.1.4_Compliance_with_jurisdictional_standards}~\nameref{subsec:3.1.4_Compliance_with_jurisdictional_standards}
\vpageref{subsec:3.1.4_Compliance_with_jurisdictional_standards}--\pageref{subsec:3.1.4_Compliance_with_jurisdictional_standards-End}),
содержащим дополнительные обязательные требования.\label{subsec:5.2.1_Scope-End}

\subsection{Справочная информация\label{subsec:5.2.2_Background}}

\stepcounter{SubSecCounter}

\thesubsection.\theSubSecCounter.\label{5.2.2.1} Наиболее распространёнными
примерами залогового обеспечения в~отношении прав на недвижимое имущество,
когда, скорее всего, потребуется консультация \emph{оценщика}, являются:
\begin{enumerate}
\item недвижимость используется либо будет использоваться самим собственником;
\item недвижимость является либо будет являться инвестиционным объектом;
\item недвижимость полностью оборудована и~подготовлена для~ведения определённого
вида деятельности и~оценивается с~точки зрения её~коммерческого
потенциала;
\item недвижимость является либо будет являться объектом девелопмента либо
реконструкции.
\end{enumerate}
Каждый из~этих вариантов рассматривается подробнее далее в~секции~\ref{5.2.6.2}~\vpageref{5.2.6.2}--\pageref{5.2.6.2-End}.\label{5.2.2.1-End}

\stepcounter{SubSecCounter}

\thesubsection.\theSubSecCounter.\label{5.2.2.2} В~настоящем руководстве
рассматриваются следующие вопросы, относящиеся к~вопросам \textsl{оценки}
для~целей залогового кредитования:
\begin{enumerate}
\item принятие задания в~работу и~\emph{раскрытие информации};
\item независимость, объективность и~\href{https://en.wikipedia.org/wiki/Conflict_of_interest}{конфликты интересов}~\cite{COI};
\item \textsl{вид стоимости} и~\textsl{специальные допущения};
\item подготовка \emph{отчёта об~оценке} и~\emph{раскрытие информации}.\label{5.2.2.2-End}
\end{enumerate}
\stepcounter{SubSecCounter}

\thesubsection.\theSubSecCounter.\label{5.2.2.3} В~большинстве
юрисдикций существует широкий спектр активов, предлагаемых в~качестве
залога, и~целый ряд доступных кредитных продуктов, поэтому в~каждом
конкретном случае потребуется свой подход. Поэтому \emph{оценщик}
и~кредитор могут согласовать соответствующие изменения при~условии
соблюдения требований подраздела~\ref{subsec:3.1.4_Compliance_with_jurisdictional_standards}~\nameref{subsec:3.1.4_Compliance_with_jurisdictional_standards}.
Первостепенной задачей является обеспечение \emph{оценщиком} понимания
потребностей и~целей кредитора, включая условия предполагаемого кредита,
а~кредитор, в~свою очередь, должен понимать предоставляемые рекомендации.
Данные принципы применимы как~к~объектам недвижимости, так~и~к~другим
материальным активам.\label{5.2.2.3-End}\label{subsec:5.2.2_Background-End}

\subsection{Независимость, объективность и~конфликты интересов\label{subsec:5.2.3_Independence_objectivity_conflict}}

\stepcounter{SubSecCounter}

\thesubsection.\theSubSecCounter.\label{5.2.3.1} \textsl{Членам
RICS} необходимо помнить о~том, что, согласно требованиям подраздела~\ref{subsec:3.2.3_Independence_objectivity_confidentiality}~\nameref{subsec:3.2.3_Independence_objectivity_confidentiality},
они~всегда должны действовать честно, независимо и~объективно, избегать
\emph{\href{https://en.wikipedia.org/wiki/Conflict_of_interest}{конфликтов интересов}~\cite{COI}}
и~любых действий или~ситуаций, несовместимых с~их~профессиональными
обязательствами. Кроме того, \textsl{они}~также должны заявлять о~любых
потенциальных \emph{\href{https://en.wikipedia.org/wiki/Conflict_of_interest}{конфликтах интересов}~\cite{COI}}
"--- личных или~профессиональных "--- всем заинтересованным сторонам.
Необходимо понимать, что~данные требования особенно актуальны и~важны
в~контексте залогового кредитования.\label{5.2.3.1-End}

\stepcounter{SubSecCounter}

\thesubsection.\theSubSecCounter.\label{5.2.3.2} По~запросу или~в~случае
необходимости \emph{оценщики}, соблюдающие требованиями независимости
и~объективности, согласно требованиям подраздела~\ref{subsec:3.2.3_Independence_objectivity_confidentiality}~\nameref{subsec:3.2.3_Independence_objectivity_confidentiality}
\vpageref{subsec:3.2.3_Independence_objectivity_confidentiality}--\pageref{subsec:3.2.3_Independence_objectivity_confidentiality-End},
могут подтвердить, что~они действуют в~качестве \guillemotleft независимых
оценщиков\guillemotright , при~условии соблюдения требований нижеследующего
пункта~\ref{5.2.3.3}.\label{5.2.3.2-End}

\stepcounter{SubSecCounter}

\thesubsection.\theSubSecCounter.\label{5.2.3.3} Кредитор может
установить дополнительные требования независимости \textsl{оценки}
для~целей залогового кредитования. Также могут существовать особые
критерии, применяемые в~отдельных юрисдикциях. В~отсутствие каких-либо
уточнений считается, что~дополнительные критерии включают положение
о~том, что~\emph{оценщик} не~имел никакого предыдущего, текущего
или~предполагаемого участия в~сделках с~заёмщиком или~потенциальным
заёмщиком либо активом, подлежащим \emph{оценке} равно как~и~любой
другой стороной, связанной со~сделкой, для~которой требуется кредитование.
Под~\guillemotleft предшествующим участием в~сделках\guillemotright{}
понимается сотрудничество в~течение 24~(двадцати четырёх) месяцев,
предшествующих дате подписания либо согласования окончательных условий
\textsl{договора на~проведение оценки} (в~зависимости от~того,
что~наступит раньше), однако в~отдельных юрисдикциях может быть
законодательно установлен или~принят обычаем более длительный период.\label{5.2.3.3-End}

\stepcounter{SubSecCounter}

\thesubsection.\theSubSecCounter.\label{5.2.3.4} Любая предыдущая
или~текущая связь с~заёмщиком либо оцениваемым имуществом~(активом)
должна быть раскрыта перед кредитором до~принятия работы. \emph{Раскрытие
информации} также предполагает сообщение о~любом предполагаемом сотрудничестве
в~будущем. Под~\guillemotleft заёмщиком\guillemotright{} понимается
также потенциальный заёмщик либо любая другая сторона, вовлечённая
в~сделку, для~которой требуется кредит. Примеры такого вовлечения,
которое может привести к~конфликту интересов, включают ситуации,
когда \emph{оценщик} или~\textsl{оценочная компания}:
\begin{itemize}
\item имеет давние профессиональные отношения с~заёмщиком или~владельцем
имущества или~активов;
\item предлагает сделку кредитору либо~заёмщику, получая за~это~вознаграждение;
\item имеет финансовую заинтересованность в~отношении актива либо заёмщика;
\item действует в~интересах собственника имущества~(актива) в~другой
сделке, связанной с~рассматриваемой;
\item действует либо действовал в~интересах заёмщика при~приобретении
имущества~(актива);
\item имеет поручение осуществлять продажи либо сдачу в~аренду актива либо
объекта недвижимости после завершения его~строительства;
\item в~недавнем прошлом действовал в~рамках рыночной сделки с~данным
имуществом (активом);
\item предоставлял платные профессиональные консультации, касающиеся имущества
(актива), нынешним или прежним владельцам либо их~кредиторам;
\item предоставляет либо предоставлял в~прошлом консультации, касающиеся
девелопмента объекта недвижимости его~нынешним либо прежним собственникам.\label{5.2.3.4-End}
\end{itemize}
\stepcounter{SubSecCounter}

\thesubsection.\theSubSecCounter.\label{5.2.3.5} В~связи с~вышеизложенным
в~п.~\ref{5.2.3.4}, \uline{оценщикам} необходимо не~забывать
о~своей обязанности рассмотреть вопрос о~том, является~ли любое
предыдущее, текущее или~ожидаемое участие в~сделках с~имуществом~(активом),
наличие соответствующих финансовых интересов, наличие деловых либо
личных связей с~заинтересованными сторонами препятствием для~\uline{их}~независимости
и~объективности. Существенное значение могут иметь такие вопросы
как: величина финансовых интересов, проистекающих из~отношений с~одной
из~сторон сделки, возможность извлечения существенной выгоды \emph{оценщиком}
либо \textsl{оценочной компанией} при~определённом результате \textsl{оценки}
и~размер вознаграждения за~проведение \textsl{оценки}, уплачиваемого
одной из~сторон, в~контексте его~доли в~общей выручке.\label{5.2.3.5-End}

\stepcounter{SubSecCounter}

\thesubsection.\theSubSecCounter.\label{5.2.3.6} В~случае, когда
\emph{оценщик} приходит к~выводу о~том, что~существующая заинтересованность
влечёт неизбежное противоречие с~его~обязанностями перед заказчиком,
ему следует отказаться от~выполнения данной работы.\label{5.2.3.6-End}

\stepcounter{SubSecCounter}

\thesubsection.\theSubSecCounter.\label{5.2.3.7} Если \emph{оценщик}
и~заказчик приходят к~мнению о~том, что~существует возможность
избежать возникновение какого-либо \href{https://en.wikipedia.org/wiki/Conflict_of_interest}{конфликта интересов}~\cite{COI}
путём принятия соответствующих управленческих мер, эти~меры должны
быть зафиксированы в~письменном виде, включены в~условия \textsl{договора
на~проведение оценки} и~упомянуты в~\emph{отчёте}.\label{5.2.3.7-End}

\stepcounter{SubSecCounter}

\thesubsection.\theSubSecCounter.\label{5.2.3.8} Несмотря на~то,
что~\emph{оценщик} при~рассмотрении вопроса о~том, создаёт~ли
недавняя, текущая или~ожидаемая заинтересованность \emph{\href{https://en.wikipedia.org/wiki/Conflict_of_interest}{конфликт интересов}~\cite{COI},}
может принять во~внимание мнение потенциального заказчика, профессиональной
обязанностью \emph{оценщика} остаётся самостоятельное принятие решения
о~том, принимать или~не~принимать работу с~учётом принципов, установленных
\href{https://www.rics.org/ssa/upholding-professional-standards/standards-of-conduct/rules-of-conduct/}{Правилами поведения RICS}~\cite{RICS:Conduct,RICS:Conduct-Firms,RICS:Conduct-Members}.
В~случае принятия работы в~условиях раскрытой информации о~наличии
существенной заинтересованности RICS может запросить у~\emph{оценщика}
обоснование данного решения. В~случае непредставления надлежащего
и~убедительного обоснования RICS вправе принять меры дисциплинарного
характера.\label{5.2.3.8-End}

\stepcounter{SubSecCounter}

\thesubsection.\theSubSecCounter.\label{5.2.3.9} \textsl{Членам
RICS} следует придерживаться дополнительных требований, установленных
всемирным стандартом RICS \href{https://www.rics.org/globalassets/rics-website/media/upholding-professional-standards/standards-of-conduct/conflicts-of-interest/conflicts_of_interest_global_1st-edition_dec_2017_revisions_pgguidance_2017_rw.pdf}{Конфликт интересов}~\cite{RICS:Rule:Conflicts-of-interest}.\label{5.2.3.9-End}\label{subsec:5.2.3_Independence_objectivity_conflict-End}

\subsection{Принятие задания в~работу и~раскрытие информации\label{subsec:5.2.4_Instructions_disclosures}}

\stepcounter{SubSecCounter}

\thesubsection.\theSubSecCounter.\label{5.2.4.1}~\emph{Оценщикам}
следует помнить о~том, что~условия \textsl{договора на~проведение
оценки} должны отвечать минимальным требованиям, установленным подразделом~\ref{subsec:4.1.3_Terms_of_engagement_scope_of_work}~\nameref{subsec:4.1.3_Terms_of_engagement_scope_of_work}
\vpageref{subsec:4.1.3_Terms_of_engagement_scope_of_work}--\pageref{subsec:4.1.3_Terms_of_engagement_scope_of_work-End}.
В~случае, если кредитор предъявляет дополнительные или~альтернативные
требования, их~необходимо подтвердить в~письменном виде, особое
внимание следует уделить согласованию и~фиксации любых \textsl{специальных
допущений}, которые должны быть сделаны.\label{5.2.4.1-End}

\stepcounter{SubSecCounter}

\thesubsection.\theSubSecCounter.\label{5.2.4.2} В~ряде случаев
заказчиком \textsl{оценки} для~залогового кредитования может выступать
сторона, не~являющаяся кредитором "--- например, потенциальный заёмщик
либо брокер. В~случае, если такая сторона не~знает, кто~является
потенциальным кредитором либо не~желает его~раскрывать, в~условиях
\textsl{договора на~проведение оценки} необходимо сделать заявление
о~том, что~\textsl{оценка} может быть неприемлема для~кредитора.
Это~следует из~того факта, что~некоторые кредиторы полагают, что~\textsl{оценка},
проведённая заёмщиком или~агентом, не~является достаточно независимой,
либо может быть связано с~тем, что~конкретный кредитор предъявляет
специальные требования к~\emph{отчёту}.\label{5.2.4.2-End}

\stepcounter{SubSecCounter}

\thesubsection.\theSubSecCounter.\label{5.2.4.3}\emph{ Оценщику}
следует выяснить, не~совершались~ли в~последнее время сделки с~оцениваемым
объектом, а~также имеет~ли место факт согласования его~предварительной
цены. В~случае наличия таких сведений, следует провести дополнительные
исследования, например, в~части того, каким образом имущество было
представлено на~рынке, имели~ли место какие-либо стимулирующие меры
(например скидка), каково значение цены сделки либо предварительно
согласованной цены, являются~ли они~наилучшими ценами из~всех возможных.\label{5.2.4.3-End}

\stepcounter{SubSecCounter}

\thesubsection.\theSubSecCounter.\label{5.2.4.4} \emph{Оценщику}
следует запросить сведения о~параметрах кредита, предусмотренных
кредитором.\label{5.2.4.4-End}

\stepcounter{SubSecCounter}

\thesubsection.\theSubSecCounter.\label{5.2.4.5} \emph{Оценщикам}
следует помнить о~том, что~они~обязаны обеспечить \emph{раскрытие
информации} в~соответствии с~требованиями, предусмотренными п.~\ref{4.1.3.1}
\vpageref{4.1.3.1}--\pageref{4.1.3.1-End}, а~также с~учётом положений
подраздела~\ref{subsec:5.2.6_Reporting_and_disclosures}~\nameref{subsec:5.2.6_Reporting_and_disclosures}
\vpageref{subsec:5.2.6_Reporting_and_disclosures}--\pageref{subsec:5.2.6_Reporting_and_disclosures-End}.\label{5.2.4.5-End}\label{subsec:5.2.4_Instructions_disclosures-End}

\subsection{Виды стоимости и~специальные допущения\label{subsec:5.2.5_Basis_of_value}}

\stepcounter{SubSecCounter}

\thesubsection.\theSubSecCounter.\label{5.2.5.1} \textsl{Рыночная
стоимость} представляет собой \textsl{вид стоимости}, широко используемый
для~\textsl{оценки} или~экспертизы, проводимых в~целях залогового
кредитования. Однако в~ряде юрисдикций может допускаться либо требоваться
применение иных \textsl{видов стоимости}, например, в~силу требований
законодательных или~нормативных актов, может применяться \guillemotleft залоговая
стоимость\guillemotright . Подобные альтернативные \textsl{виды стоимости}
могут предполагать (что на~практике бывает довольно часто) применение
предписанных \emph{подходов~к~оценке} либо свой набор \textsl{допущений},
вследствие и~по~причине чего их~применение может привести к~значению
стоимости объекта, выступающего в~качестве залога, существенно отличающемуся
от~его~\textsl{рыночной стоимости} в~том~её~понимании, которое
предусмотрено п.~\ref{9.4.2.1.1.1} \vpageref{9.4.2.1.1.1}--\pageref{9.4.2.1.1.1-End},
и~воспроизведено в~подразделе~\ref{subsec:4.4.3_Market-value}~\nameref{subsec:4.4.3_Market-value}
\vpageref{subsec:4.4.3_Market-value}--\pageref{subsec:4.4.3_Market-value-End}.
Иногда такой подход можно описать как~\guillemotleft\textsl{оценка}
по~методике\guillemotright , что~должно позволить отделить его~от~традиционной
\guillemotleft\textsl{оценки} по~рынку\guillemotright . Результат
подобной \textsl{оценки} можно использовать исключительно в~тех~конкретных
целях, для~которой \textsl{она}~была проведена. Это~объясняется
тем, что~зачастую подобные \textsl{виды стоимости} и~методики представляют
собой инструмент оценки рисков (либо являются его~частью), в~котором
риск рассматривается в~длительной перспективе и~с~точки зрения
кредитора. Хотя \emph{оценщики} вправе давать консультации, используя
подобные альтернативные \emph{виды стоимости}, крайне важно всегда
чётко указывать конкретный \textsl{вид стоимости} и~приводить соответствующие
комментарии.\label{5.2.5.1-End}

\stepcounter{SubSecCounter}

\thesubsection.\theSubSecCounter.\label{5.2.5.2} Любые \textsl{специальные
допущения} (см.~подраздел~\ref{subsec:4.4.8_Special_assumptions}~\nameref{subsec:4.4.8_Special_assumptions}
\vpageref{subsec:4.4.8_Special_assumptions}--\pageref{subsec:4.4.8_Special_assumptions-End}),
введённые в~процессе определения заявленной стоимости, должны быть
заранее согласованы с~кредитором в~письменном виде и~упомянуты
в~\emph{отчёте об~оценке}.\label{5.2.5.2-End}

\stepcounter{SubSecCounter}

\thesubsection.\theSubSecCounter.\label{5.2.5.3} Ниже приводится
перечень типичных ситуаций, возникающих при~\textsl{оценке} стоимости
для~целей залогового обеспечения, когда могут быть уместны \textsl{специальные
допущения}:
\begin{itemize}
\item в~отношении \emph{объекта оценки} было получено разрешение на~строительство;
\item имело место изменение физических свойств \emph{объекта оценки}, например,
новое строительство или~реконструкция;
\item заключён новый договор аренды на~определённых условиях либо завершён
пересмотр арендной платы в~рамках действующего договора;
\item существует \textsl{специальный покупатель}, которым может быть в~т.\,ч.~и~потенциальный
заёмщик;
\item следует игнорировать наличие ограничения, препятствующего выведению
объекта на~рынок либо его~экспозиции там;
\item получен новый статус в~части экономического либо экологического зонирования
территории;
\item объект подвергается негативному воздействию природного, антропогенного
либо природоохранного характера, ограничивающему возможности по~его~использованию;
\item следует уменьшить значение существующей на~\textsl{дату оценки} рыночной
волатильности;
\item следует не~принимать во~внимание договор либо договоры аренды, заключённые
между аффилированными сторонами.
\end{itemize}
Вышеприведённый перечень не~является исчерпывающим, и~необходимые
\textsl{специальные допущения} будут зависеть от~обстоятельств, при~которых
требуется проведение \textsl{оценки}, а~также характера оцениваемого
имущества.\label{5.2.5.3-End}

\stepcounter{SubSecCounter}

\thesubsection.\theSubSecCounter.\label{5.2.5.4} Любая \textsl{оценка}
для~целей залогового кредитования, выполненная с~применением \textsl{специального
допущения}, должна сопровождаться комментарием о~любой существенной
разнице между стоимостями, получаемыми с~учётом и~без~учёта этого
\textsl{специального допущения}.\label{5.2.5.4-End}\label{subsec:5.2.5_Basis_of_value-End}

\subsection{Подготовка отчёта об~оценке и~раскрытие информации\label{subsec:5.2.6_Reporting_and_disclosures}}

\stepcounter{SubSecCounter}

\thesubsection.\theSubSecCounter.\label{5.2.6.1} В~дополнение к~вопросам,
требующим изложения в~\emph{отчёте об~оценке} согласно требованиям,
установленным в~подразделе~\ref{subsec:4.3.2_Report_content}~\nameref{subsec:4.3.2_Report_content}
\vpageref{subsec:4.3.2_Report_content}--\pageref{subsec:4.3.2_Report_content-End},
при~проведении \textsl{оценки} для~целей залогового кредитования
рассмотрению часто подлежат также следующие вопросы.
\begin{itemize}
\item \emph{Раскрытие информации} о~любой заинтересованности, в~т.\,ч.~проистекающей
из~факта прежнего сотрудничества, (см.~подраздел~\ref{subsec:5.2.4_Instructions_disclosures}~\nameref{subsec:5.2.4_Instructions_disclosures}
\vpageref{subsec:5.2.4_Instructions_disclosures}--\pageref{subsec:5.2.4_Instructions_disclosures-End}),
описанной в~первоначальных условиях \textsl{договора на~проведение
оценки} либо дополнительных соглашениях к~нему либо о~всех согласованных
мерах, направленных на~предотвращение \href{https://en.wikipedia.org/wiki/Conflict_of_interest}{конфликта интересов}~\cite{COI}.
В~случае отсутствия подобной заинтересованности об~этом также следует
сообщить в~\emph{отчёте}.
\item Описание применённого \emph{метода оценки}, сопровождаемое (если это~уместно
или~требуется) выполненными расчётами (следует принять во~внимание,
что~в~некоторых юрисдикциях сами методы или~их~совокупность будут
предписаны в~той или~иной степени).
\item В~случае, если недавно имела место сделка с~\emph{объектом оценки},
либо имеются данные о~его~предварительно согласованной цене, следует
привести сведения о~степени, в~которой эти~данные были приняты
в~качестве базы для~определения стоимости. Если запрос не~позволил
получить какие-либо сведения на~сей~счёт, \emph{оценщик} делает
соответствующее заявление в~\emph{отчёте}, сопровождая его~просьбой
о~том, что, если такие сведения появятся до~того, как~состоится
сделка по~предоставлению кредита, их~необходимо предоставить \emph{оценщику}
для~дальнейшего рассмотрения вопроса стоимости объекта.
\item Комментарий о~пригодности имущества в~качестве залога для~целей
ипотеки с~учётом срока и~условий предполагаемого кредита. Если условия
неизвестны, комментарий ограничивается общим описанием ликвидности
и~оборотоспособности объекта.
\item Сведения о~каких-либо обстоятельствах, способных оказать влияние
на~цену, которые \emph{оценщику} следует иметь ввиду. Они~также
должны быть доведены до~сведения кредитора, включая, в~обязательном
порядке, данные о~степени их~влияния.
\item Обстоятельства, потенциально противоречащие предпосылкам либо \textsl{допущениям},
лежащим в~основе определяемого \textsl{вида стоимости}, и~их~влияние
должны быть изложены и~объяснены.
\item Потенциал объекта и~привлекательность альтернативных вариантов его~использования
либо любые прогнозируемые изменения в~текущем статусе или~виде использования.
\item Потенциальный арендный спрос на~объект.
\item \href{https://meganorm.ru/Data2/1/4293740/4293740894.pdf}{Аварийное состояние}~\cite{STO-2015}
либо наличие на~объекте вредных или~опасных материалов.
\item Комментарии, касающиеся природоохранного либо экономического статуса.
\item Комментарии по~вопросам экологии, например, таким как~риск наводнения,
прежнее загрязнение либо ненадлежащая эксплуатация.
\item Прошлые, текущие и~будущие тенденции, а~также любые колебания местного
рынка и~(или) спроса на~недвижимость данной категории.
\item Текущие оборотоспособность и~ликвидность прав на~объект, а~также
вероятность того, что~они~будут сохранять устойчивость в~течение
всего срока кредитования.
\item Подробные сведения о~любых сопоставимых сделках, использованных в~качестве
базы для~\textsl{оценки}, и~их~значимости для~неё.
\item Иные моменты, выявленные в~ходе направления стандартных запросов,
способные оказать существенное влияние на~стоимость, определённую
в~\emph{отчёте об~оценке}.
\item Если недвижимость является или~должна стать объектом девелопмента
либо реконструкции, направленных на~создание жилых объектов, следует
описать влияние стимулирующих мер~в~адрес покупателей.
\end{itemize}
Вопросы \emph{\href{https://en.wikipedia.org/wiki/Sustainability}{устойчивого развития}~\cite{Wiki:sustainability}}
(см.~раздел~\ref{sec:5.8_VPGA-8_Valuation_of_real_property}~\nameref{sec:5.8_VPGA-8_Valuation_of_real_property}
\vpageref{sec:5.8_VPGA-8_Valuation_of_real_property}--\pageref{sec:5.8_VPGA-8_Valuation_of_real_property-End})
приобретают всё~большее значение и~влияние на~рынок "--- при~\textsl{оценке}
стоимости для~целей залогового обеспечения по~кредиту всегда следует
учитывать их~значимость в~контексте конкретной работы.\label{5.2.6.1-End}

\stepcounter{SubSecCounter}

\thesubsection.\theSubSecCounter.\label{5.2.6.2} В~нижеследующих
параграфах описаны вопросы, рассмотрение которых в~\emph{отчёте}
может быть целесообразно при~\textsl{оценке} различных категорий
недвижимого имущества, перечисленных в~пункте~\ref{5.2.2.1} \vpageref{5.2.2.1}--\pageref{5.2.2.1-End}.
\begin{enumerate}
\item Недвижимость используется либо будет использоваться самим собственником. 
\begin{enumerate}
\item Ниже приводится перечень типовых \textsl{специальных допущений}, использование
которых может быть уместно при~\textsl{оценке} данной категории объектов
недвижимости:
\end{enumerate}
\begin{itemize}
\item было либо будет получено разрешение на~развитие и~застройку территории,
в~т.\,ч.~предполагающее изменение вида разрешённого использования;
\item строительство здания либо иного предполагаемого объекта было завершено
в~соответствии с~планом и~проектной документацией;
\item имеются все~необходимые лицензии и~разрешения;
\item объект недвижимости был подвергнут определённым изменениям (например,
был выполнен демонтаж оборудования или~приспособлений);
\item объект недвижимости свободен, тогда как~в~реальности по~состоянию
на~\textsl{дату оценки} он~используется.
\end{itemize}
\item Объект недвижимости является либо будет являться инвестиционным объектом.
\begin{enumerate}
\item В~тексте \emph{отчёта об~оценке} дополнительно следует отразить
следующее:
\begin{itemize}
\item перечень и~основные условия действующих договоров аренды, с~указанием
того, читал~ли их~\emph{оценщик}, а~также источники использованных
данных;
\item сведения об~арендном доходе и~соответствующие комментарии, а~также
сравнительный анализ фактических и~рыночных ставок, в~случае, если
объект включает в~себя составные части, которые могут быть сданы
в~аренду по~отдельности, следует привести сведения по~каждой из~них;
\item \textsl{допущение} о~способности арендатора исполнять условия договора
в~случае отсутствия доступных сведений либо комментарии о~качестве,
пригодности и~способности арендатора исполнять условия договора с~точки
зрения восприятия этих аспектов рынком;
\item комментарии относительно возможности сохранения арендного дохода в~течение
срока действия кредита, а~также любых потенциальных рисков в~этой
части, с~акцентированием на~прекращение и~заключение договоров,
ожидаемых рыночных тенденций "--- при~этом может потребоваться рассмотрение
этих вопросов в~широком контексте \emph{\href{https://en.wikipedia.org/wiki/Sustainability}{устойчивого развития}~\cite{Wiki:sustainability};}
\item комментарии о~возможности проведения перепланировки или~реконструкции
по~окончании срока договора~(договоров) аренды.
\end{itemize}
\item Ниже приводится перечень типовых \textsl{специальных допущений}, использование
которых может быть уместно при~\textsl{оценке} данной категории объектов
недвижимости:
\begin{itemize}
\item установлен либо согласован иной размер величины арендной платы, например
в~результате пересмотра условий аренды;
\item предполагается, что~объект свободен и~может быть сдан в~аренду
несмотря на~наличие договоров аренды в~реальности;
\item заключение договора аренды на~определённых условиях успешно завершено.
\end{itemize}
\end{enumerate}
\item Недвижимость полностью оборудована и~подготовлена для~ведения определённого
вида деятельности и~оценивается с~точки зрения её~коммерческого
потенциала.
\begin{enumerate}
\item Закрытие бизнеса может оказать существенное влияние на~\textsl{рыночную
стоимость}. Вследствие и~по~причине чего \emph{оценщик} обязан сообщить
о~влиянии этого события либо отдельно либо в~сочетании с~одним
или~несколькими нижеследующими \textsl{специальными допущениями}:
\begin{itemize}
\item бизнес закрылся, а~объект недвижимости свободен;
\item товарные запасы исчерпаны либо были вывезены;
\item лицензии, согласования, сертификаты и~(или) разрешения утратили силу
либо находятся под~такой угрозой;
\item документы учёта и~коммерческие документы недоступны потенциальному
покупателю.
\end{itemize}
\item Ниже приводится перечень типовых \textsl{специальных допущений}, использование
которых может быть уместно при~оценке данной категории объектов недвижимости:
\begin{itemize}
\item допущения, касающиеся результатов коммерческой деятельности;
\item прогнозы показателей коммерческой деятельности, существенно отличающиеся
от~текущих рыночных ожиданий.
\end{itemize}
\end{enumerate}
\item Объект недвижимости является либо будет являться объектом девелопмента
либо реконструкции.
\begin{enumerate}
\item В \emph{отчёте} должен быть приведены следующие дополнительные сведения:
\begin{itemize}
\item комментарии, касающиеся затрат и~поставок по~контрактам;
\item комментарии по~вопросам жизнеспособности предлагаемого проекта;
\item в~случае, если \textsl{оценка} основана на~использовании \emph{метода
остатка}, следует приводить анализ чувствительности к~каждому использованному
\textsl{допущению};
\item влияние на~стоимость превышения лимита расходов либо задержек в~реализации
проекта;
\item поскольку проведение перепланировки либо реконструкции может оказывать
влияние на~стоимость объекта недвижимости вследствие и~по~причине
неудобства либо невозможности его~использования в~этот период, следует
привести соответствующие комментарии относительно их~продолжительности.
\end{itemize}
\item Ниже приводится перечень типовых \textsl{специальных допущений}, использование
которых может быть уместно при~\textsl{оценке} данной категории объектов
недвижимости:
\begin{itemize}
\item описанные работы были выполнены надлежащим образом в~соответствии
со~всеми установленными требованиями законодательства;
\item завершённый строительством объект был~сдан в~аренду на~определённых
условиях;
\item ранее согласованный договор купли-продажа либо аренды не~был заключён.
\end{itemize}
\item В~случае, когда требуется проведение \textsl{оценки}, основанной
на~\textsl{специальном допущении} о~том, что работы были завершены
на~текущую \textsl{дату оценки}, определение стоимости должно осуществляться
на~основе текущих рыночных условий. В~случае, когда \textsl{оценка}
осуществляется на~основе \textsl{специального допущения} о~том,
что~работы будут завершены в~будущем на~дату, являющуюся \textsl{датой
оценки}, \emph{оценщику} следует применять положения секций~\ref{subsubsec:4.1.3.11_Assumptions_and_special_assumptions}
\vpageref{subsubsec:4.1.3.11_Assumptions_and_special_assumptions}--\pageref{subsubsec:4.1.3.11_Assumptions_and_special_assumptions-End},
а~также \ref{subsubsec:4.3.2.9_Assumptions_and_special_assumptions}\vpageref{subsubsec:4.3.2.9_Assumptions_and_special_assumptions}--\pageref{subsubsec:4.3.2.9_Assumptions_and_special_assumptions-End}.\label{5.2.6.2-End}
\end{enumerate}
\stepcounter{SubSecCounter}

\thesubsection.\theSubSecCounter.\label{5.2.6.3} Хорошей практикой
является приложение к~\emph{отчёту} \textsl{условий договора на~проведение
оценки} и~письменных поручений заказчика, а~также наличие ссылок
на~них в~тексте отчёта.\label{5.2.6.3-End}\label{subsec:5.2.6_Reporting_and_disclosures-End}\label{sec:5.2_VPGA-2_Valuation_for_secure_lending-End}
\end{enumerate}
\newpage

\section{ПР~3.~Оценка бизнеса и~долей в~нём\label{sec:5.3_VPGA-3_Valuation_of_businesses}}

\textbf{Данное руководство носит рекомендательный характер и~не~содержит
обязательных требований. Однако там, где~это~уместно, оно~содержит
отсылки к~соответствующему обязательному материалу, содержащемуся
в~других разделах настоящих Всемирных стандартов, а~также \href{https://www.rics.org/globalassets/rics-website/media/upholding-professional-standards/sector-standards/valuation/international-valuation-standards-rics2.pdf}{Международных стандартов оценки}~\cite{IVS-2020},
реализованные в~виде перекрёстных ссылок. Данные ссылки предназначены
для~помощи }\textbf{\textsl{членам RICS}}\textbf{ и~не~меняют статус
данного практического руководства. }\textbf{\textsl{Членам RICS}}\textbf{
необходимо помнить следующее:}
\begin{itemize}
\item \textbf{данное руководство не~может охватить все~возможные варианты,
вследствие чего }\textbf{\emph{оценщикам}}\textbf{ при~формировании
своих }\textbf{\emph{суждений о~стоимости}}\textbf{ всегда следует
учитывать факты и~обстоятельства, имеющие место в~рамках отдельных
заданий по~}\textbf{\textsl{оценке}}\textbf{;}
\item \textbf{следует внимательно относиться к~тому факту, что~в~ряде
юрисдикций могут существовать особые требования, не~предусмотренные
данным руководством.}
\end{itemize}

\subsection{Область применения\label{subsec:5.3.1_Scope}}

\stepcounter{SubSecCounter}

\thesubsection.\theSubSecCounter.\label{5.3.1.1} Данное практическое
руководство содержит дополнительные комментарии по~вопросам \textsl{оценки}
бизнеса и~долей в~нём, а~также рекомендации по~практическому применению
\hyperref[sec:10.1_IVS-200_Businesses]{\textbf{МСО~200.~Оценка бизнеса и~долей в~нём}}
(см.~раздел~\ref{sec:10.1_IVS-200_Businesses}~\nameref{sec:10.1_IVS-200_Businesses}
\vpageref{sec:10.1_IVS-200_Businesses}--\pageref{sec:10.1_IVS-200_Businesses-End}).
В~тексте руководства приводятся перекрёстные ссылки на~обязательные
требования настоящих \emph{Всемирных стандартов}.\label{5.3.1.1-End}\label{subsec:5.3.1_Scope-End}

\subsection{Введение\label{subsec:5.3.2_Introduction}}

\stepcounter{SubSecCounter}

\thesubsection.\theSubSecCounter.\label{5.3.2.1} В~целях настоящего
руководства под~\guillemotleft частичным правом\guillemotright{} понимается
набор прав на~материальный либо не~материальный актив меньший чем~\href{https://studme.org/52885/ekonomika/pravo_sobstvennosti_sisteme_imuschestvennyh_prav}{полный набор прав}~\cite{Pravo_sobstv},
например \href{https://ru.wikipedia.org/wiki/\%D0\%9F\%D0\%BE\%D0\%BB\%D1\%8C\%D0\%B7\%D0\%BE\%D0\%B2\%D0\%B0\%D0\%BD\%D0\%B8\%D0\%B5}{право пользования}~\cite{Pravo_polyz}
без~\href{https://ru.wikipedia.org/wiki/\%D0\%A0\%D0\%B0\%D1\%81\%D0\%BF\%D0\%BE\%D1\%80\%D1\%8F\%D0\%B6\%D0\%B5\%D0\%BD\%D0\%B8\%D0\%B5_\%D0\%B8\%D0\%BC\%D1\%83\%D1\%89\%D0\%B5\%D1\%81\%D1\%82\%D0\%B2\%D0\%BE\%D0\%BC}{права распоряжения}~\cite{Pravo_rasp}
имуществом либо активом. Под \guillemotleft дробной долей\guillemotright{}
понимается владение процентной долей в~праве либо правах на~материальный
или~нематериальный актив (будь~то право на~весь актив или частичная
доля в~нём), например, совместное с~другими сторонами владение активом.\label{5.3.2.1-End}

\stepcounter{SubSecCounter}

\thesubsection.\theSubSecCounter.\label{5.3.2.2} \hyperref[sec:10.1_IVS-200_Businesses]{МСО~200}
содержит следующее определение: \guillemotleft бизнес "--- коммерческая,
производственная, сервисная либо инвестиционная деятельность\guillemotright .
Областью применения данного руководства является \textsl{оценка} бизнеса
целиком "--- хозяйственных обществ, индивидуальных предпринимателей,\footnote{Прим. пер.: в~российской правовой системе \textsl{оценка} бизнеса
индивидуального предпринимателя как~такового не~имеет смысла, поскольку
такой бизнес неотделим от~личности физического лица, зарегистрированного
в~качестве индивидуального предпринимателя. Можно говорить лишь об~\textsl{оценке}
имущественного комплекса, принадлежащего индивидуальному предпринимателю.} товариществ, партнёрств, кооперативов (в~т.\,ч.~товариществ с~ограниченной
ответственностью, а~также долей в~них, например акций, долей в~хозяйственных
обществах, паёв членов товариществ и~кооперативов.\label{5.3.2.2-End}

\stepcounter{SubSecCounter}

\thesubsection.\theSubSecCounter.\label{5.3.2.3} Данное руководство
не~распространяет своё действие на~\textsl{оценку} \textsl{нематериальных
активов} (которая рассматривается в~разделе~\ref{sec:5.6_VPGA-6_Valuation_of_intangible_assets}~\nameref{sec:5.6_VPGA-6_Valuation_of_intangible_assets}
\vpageref{sec:5.6_VPGA-6_Valuation_of_intangible_assets}--\pageref{sec:5.6_VPGA-6_Valuation_of_intangible_assets-End}),
\textsl{машин и~оборудования} (см.~раздел~\ref{sec:5.5_VPGA-5-Valuation_of_plant_and_equipment}~\nameref{sec:5.5_VPGA-5-Valuation_of_plant_and_equipment}
\vpageref{sec:5.5_VPGA-5-Valuation_of_plant_and_equipment}--\pageref{sec:5.5_VPGA-5-Valuation_of_plant_and_equipment-End}),
земли либо иных материальных активов, являющихся составной частью
оцениваемого бизнеса. При~этом \emph{оценщику} бизнеса зачастую приходится
полагаться на~\textsl{оценку} таких активов, выполненную другими
специалистами, например \textsl{оценку} недвижимости либо прав на~месторождения
полезных ископаемых.\label{5.3.2.3-End}

\stepcounter{SubSecCounter}

\thesubsection.\theSubSecCounter.\label{5.3.2.4} \textsl{Оценка}
финансовых активов, заёмного капитала, долговых обязательств, опционов,
варрантов, конвертируемых финансовых инструментов и~финансовых инструментов
с~фиксированной доходностью может являться частью процесса \textsl{оценки}
бизнеса.\label{5.3.2.4-End}

\stepcounter{SubSecCounter}

\thesubsection.\theSubSecCounter.\label{5.3.2.5} В~целях соблюдения
требований, установленных подразделом~\ref{subsec:3.2.2_Member qualification}~\nameref{subsec:3.2.2_Member qualification}
\vpageref{subsec:3.2.2_Member qualification}--\pageref{subsec:3.2.2_Member qualification-End},
необходимо чтобы \emph{оценщик} регулярно выполнял \textsl{оценки}
бизнеса, поскольку ему~необходимо обладать практическими знаниями
факторов, влияющих на~инвестиции в~конкретные виды недвижимости,
активов, бизнеса или~долей в~нём.\label{5.3.2.5-End}\label{subsec:5.3.2_Introduction-End}

\subsection{Задание на~оценку и~условия договора на~проведение оценки\label{subsec:5.3.3_Scope_of_work}}

\stepcounter{SubSecCounter}

\thesubsection.\theSubSecCounter.\label{5.3.3.1} Познания заказчиков
в~сфере \textsl{оценки} могут существенно различаться. Некоторые
из~них~будут иметь глубокое понимание \textsl{оценки} бизнеса, тогда
как~другие даже не~будут знакомы с~терминами и~понятиями, используемыми
\emph{оценщиками} бизнеса.\label{5.3.3.1-End}

\stepcounter{SubSecCounter}

\thesubsection.\theSubSecCounter.\label{5.3.3.2} Важно, чтобы \emph{задание
на~оценку} и~\textsl{условия договора на~проведение оценки} были
понятны и~согласованы между \emph{оценщиком} и~заказчиком до~начала
выполнения работы. Актив или~обязательство, конкретная доля в~активе
или~обязательстве либо подлежащие \textsl{оценке} права на~них должны
быть зафиксированы в~письменном виде. Их~описание должно содержать
следующие сведения:
\begin{itemize}
\item организационно-правовая форма бизнеса;
\item размер оцениваемой доли в~активе и~её~статус;
\item состав \emph{объекта оценки}, а~также выведенные за~её~периметр
активы и~обязательства;
\item тип~оцениваемых акций либо вид~оцениваемой доли.\label{5.3.3.2-End}
\end{itemize}
\stepcounter{SubSecCounter}

\thesubsection.\theSubSecCounter.\label{5.3.3.3} Любые принятые
допущения должны быть ясно изложены в~соответствии с~требованиями
подразделов~\ref{subsec:4.4.7_Assumptions}~\nameref{subsec:4.4.7_Assumptions},
\ref{subsec:4.4.8_Special_assumptions}~\nameref{subsec:4.4.8_Special_assumptions}
\vpageref{subsec:4.4.7_Assumptions}--\pageref{subsec:4.4.7_Assumptions-End},
\pageref{subsec:4.4.8_Special_assumptions}--\pageref{subsec:4.4.8_Special_assumptions-End}
соответственно. Например, \emph{оценщику} следует привести сведения
о~том, использует~ли он~\textsl{допущение} о~том, что~владельцы
акций либо долей намереваются продать либо сохранить владение ими,
а~также о~том, что~определённые активы или~обязательства исключены
из~периметра \textsl{оценки}.\label{5.3.3.3-End}

\stepcounter{SubSecCounter}

\thesubsection.\theSubSecCounter.\label{5.3.3.4} Возможны ситуации,
при~которых право на~оцениваемый актив является общим с~другими
сторонами, т.\,е.~имеет место факт совместного пользования либо
владения "--- в~таких случаях данное обстоятельство должно быть
чётко отражено в~отчёте.\label{5.3.3.4-End}

\stepcounter{SubSecCounter}

\thesubsection.\theSubSecCounter.\label{5.3.3.5} \emph{Оценщики}
могут захотеть разработать типовые формы заявок и~\textsl{договоров
на~проведение оценки}, подходящие для~всех видов работ. В~случае,
если \textsl{оценка} должна соответствовать требованиям настоящих
\emph{Всемирных стандартов}, \emph{оценщику} будет необходимо разработать
\textsl{договор}, удовлетворяющий требованиям подраздела~\ref{subsec:3.2.7_Terms_of_engagement_Scope_of_work}~\nameref{subsec:3.2.7_Terms_of_engagement_Scope_of_work}
\vpageref{subsec:3.2.7_Terms_of_engagement_Scope_of_work}--\pageref{subsec:3.2.7_Terms_of_engagement_Scope_of_work-End},
а~также раздела~\ref{sec:4.1_VPS1_Terms_of_engagement_Scope_of_work}~\nameref{sec:4.1_VPS1_Terms_of_engagement_Scope_of_work}
\vpageref{sec:4.1_VPS1_Terms_of_engagement_Scope_of_work}--\pageref{sec:4.1_VPS1_Terms_of_engagement_Scope_of_work-End},
при~необходимости адаптированных для~\textsl{оценки} бизнеса.\label{5.3.3.5-End}\label{subsec:5.3.3_Scope_of_work-End}

\subsection{Бизнес и~доли в~нём\label{subsec:5.3.4_Businesses_and_business_interests}}

\stepcounter{SubSecCounter}

\thesubsection.\theSubSecCounter.\label{5.3.4.1} \textsl{Оценка}
бизнеса может осуществляться как~в~отношении всей деятельности компании,
так~и~её~части. До~начала проведения \textsl{оценки} важно провести
разграничение между полной стоимостью компании, стоимостью ограниченного
набора прав~(см.~п.~\ref{5.3.2.1} \vpageref{5.3.2.1}--\pageref{5.3.2.1-End})
и~стоимостью конкретных активов и~обязательств компании, а~также
принять во~внимание назначение \textsl{оценки} (например для~налогового
планирования или~внутренних управленческих целей).\label{5.3.4.1-End}

\stepcounter{SubSecCounter}

\thesubsection.\theSubSecCounter.\label{5.3.4.2} Следует определить
\emph{цель проведения оценки} и~предполагаемое использование её~результатов.
\emph{Целью проведения оценки} может являться формирование \emph{суждения
о~стоимости} определённого \textsl{вида}, например \textsl{рыночной}
или~\textsl{справедливой}. \emph{Предполагаемое использование результатов
оценки} может отражать вид либо тип~деятельности, для~которых она~была
проведена, например, составление \textsl{финансовой отчётности}.\label{5.3.4.2-End}

\stepcounter{SubSecCounter}

\thesubsection.\theSubSecCounter.\label{5.3.4.3} В~случае, когда
\textsl{оценке} подлежат отдельные активы, подразделения либо обязательства,
которые могут быть отчуждены на~рынке, их~стоимость, по~возможности,
следует определять на~основе их~собственной \textsl{рыночной стоимости},
а~не~путём отнесения на~них~пропорциональной доли стоимости всего
бизнеса.\label{5.3.4.3-End}

\stepcounter{SubSecCounter}

\thesubsection.\theSubSecCounter.\label{5.3.4.4} При~проведении
\textsl{оценки} бизнеса либо доли в~нём \emph{оценщику} следует рассмотреть,
в~том~числе, вопрос о~том, может~ли стоимость, полученная в~результате
ликвидации предприятия, оказаться выше стоимости, полученной в~случае
продолжения его~деятельности, и, в~случае положительного ответа
на~этот вопрос, рассмотреть перспективу определения стоимости бизнеса
на~основе варианта его~ликвидации с~учётом доли~(долей) в~нём.\label{5.3.4.4-End}

\stepcounter{SubSecCounter}

\thesubsection.\theSubSecCounter.\label{5.3.4.5} Независимо от~организационно-правовой
формы организации бизнеса "--- будь~то единоличная собственность,
товарищество либо корпоративная форма "--- права, привилегии и~условия,
связанные с~этой формой, должны быть учтены при~\textsl{оценке}.
Оцениваться могут как~права на~весь бизнес, так~и~на~долю в~нём
либо акции, также может потребоваться проведение разграничения между
номинальным юридическим и~фактическим бенефициарным владением, а~также
между правами и~обязанностями, следующими из~существующего права
в~силу закона либо обычая, и~правами и~обязанностями, следующими
из~корпоративного соглашения между нынешними акционерами. Права собственников,
как~правило, закреплены в~юридически значимых документах, имеющих
обязательную силу, таких как~устав, учредительный договор, деловые
меморандумы, локальные нормативные акты, договоры об~учреждении партнёрств
и~иные документы партнёрств, товариществ и~кооперативов, а~также
корпоративные соглашения акционеров.\label{5.3.4.5-End}

\stepcounter{SubSecCounter}

\thesubsection.\theSubSecCounter.\label{5.3.4.6} Документы, упомянутые
выше в~п.~\ref{5.3.4.5} \vpageref{5.3.4.5}--\pageref{5.3.4.5-End},
могут содержать ограничения на~передачу прав на~доли в~бизнесе,
а~также предписывать определённый \textsl{вид стоимости}, который
должен быть использован при~их передаче. Важно различать права и~обязанности,
следующие из~существующего оцениваемого права. Например, соответствующие
документы могут предполагать проведение \textsl{оценки} на~основе
пропорционального выделения стоимости доли из~общей стоимости предприятия
без~учёта характера доли, следующего из~её размера. В~таких случаях
\emph{оценщику} следует соблюдать соответствующие требования и~в~отношении
любых иных видов прав. \hyperref[sec:10.1_IVS-200_Businesses]{МСО~200}
содержит дополнительные комментарии относительно прав собственности
на~бизнес.\label{5.3.4.6-End}

\stepcounter{SubSecCounter}

\thesubsection.\theSubSecCounter.\label{5.3.4.7} Неконтрольная доля
может стоить меньше, чем~контрольная, при~этом наиболее крупная
доля в~капитале предприятия не~обязательно является контрольной.
Вопросы контроля над~предприятием путём голосования и~иные права
собственников долей устанавливаются документами, перечисленными в~п.~\ref{5.3.4.5}
\vpageref{5.3.4.5}--\pageref{5.3.4.5-End}, которые в~определённых
случаях могут наделять контролирующими правами либо правом вето даже
миноритарных собственников. В~бизнесе часто существуют различные
\href{https://investor100.ru/chto-takoe-klassyi-aktsiy-a-b-s/}{классы акций}~\cite{Klassy_aktsiy},
каждый из~которых предоставляет их~держателям различные права.\label{5.3.4.7-End}

\stepcounter{SubSecCounter}

\thesubsection.\theSubSecCounter.\label{5.3.4.8} Поскольку может
существовать множество причин проведения \textsl{оценки} бизнеса,
\emph{оценщику} необходимо понять, каково предполагаемое использование
результатов проводимой им~конкретной \textsl{оценки}. Примерами причин
проведения \textsl{оценки} являются: составление \textsl{финансовой
отчётности}, вопросы налогообложения, государственные заказы, сделки
и~размещение акций, предоставление суждения об~обоснованности стоимостных
параметров сделок, закупочных цен и~иных операций, кредитные соглашения,
банкротство и~конкурсное управление, управление знаниями, анализ
портфеля активов.\footnote{Прим.~пер: данный перечень не~является исчерпывающим. }
\emph{Цель и~назначение (предполагаемое использование) оценки} предопределяют
применение тех или~иных \textsl{видов стоимости}, часть из~которых
установлена нормами \href{https://ru.wikipedia.org/wiki/\%D0\%A1\%D1\%82\%D0\%B0\%D1\%82\%D1\%83\%D1\%82\%D0\%BD\%D0\%BE\%D0\%B5_\%D0\%BF\%D1\%80\%D0\%B0\%D0\%B2\%D0\%BE}{статутного}~\cite{Wiki:statute_law_rus}
либо \href{https://ru.wikipedia.org/wiki/\%D0\%9E\%D0\%B1\%D1\%89\%D0\%B5\%D0\%B5_\%D0\%BF\%D1\%80\%D0\%B0\%D0\%B2\%D0\%BE}{прецедентного}~\cite{Wiki:common_law_rus}
права,\footnote{Прим.~пер.: указание на~эти~два вида источника права является применимым
только для~территорий, на~которых действует \href{https://ru.wikipedia.org/wiki/\%D0\%90\%D0\%BD\%D0\%B3\%D0\%BB\%D0\%BE\%D1\%81\%D0\%B0\%D0\%BA\%D1\%81\%D0\%BE\%D0\%BD\%D1\%81\%D0\%BA\%D0\%B0\%D1\%8F_\%D0\%BF\%D1\%80\%D0\%B0\%D0\%B2\%D0\%BE\%D0\%B2\%D0\%B0\%D1\%8F_\%D1\%81\%D0\%B5\%D0\%BC\%D1\%8C\%D1\%8F}{англо-американское право}~\cite{Wiki:anglo-american_law},
для~территорий, на~которых действует \href{https://ru.wikipedia.org/wiki/\%D0\%A0\%D0\%BE\%D0\%BC\%D0\%B0\%D0\%BD\%D0\%BE-\%D0\%B3\%D0\%B5\%D1\%80\%D0\%BC\%D0\%B0\%D0\%BD\%D1\%81\%D0\%BA\%D0\%B0\%D1\%8F_\%D0\%BF\%D1\%80\%D0\%B0\%D0\%B2\%D0\%BE\%D0\%B2\%D0\%B0\%D1\%8F_\%D1\%81\%D0\%B5\%D0\%BC\%D1\%8C\%D1\%8F}{романо-германское}~\cite{Wiki:roman-german_law}
либо, например, \href{https://ru.wikipedia.org/wiki/\%D0\%A1\%D0\%BA\%D0\%B0\%D0\%BD\%D0\%B4\%D0\%B8\%D0\%BD\%D0\%B0\%D0\%B2\%D1\%81\%D0\%BA\%D0\%B0\%D1\%8F_\%D0\%BF\%D1\%80\%D0\%B0\%D0\%B2\%D0\%BE\%D0\%B2\%D0\%B0\%D1\%8F_\%D1\%81\%D0\%B8\%D1\%81\%D1\%82\%D0\%B5\%D0\%BC\%D0\%B0}{скандинавское право}~\cite{Wiki:scandinavian_law}\cite{Wiki:scandinavian_law},
подобное разделение не~соответствует реалиям правовой системы.} часть "--- международными и~национальными стандартами осуществления
\emph{оценочной деятельности}.\label{5.3.4.8-End}

\stepcounter{SubSecCounter}

\thesubsection.\theSubSecCounter.\label{5.3.4.9} Наиболее распространёнными
\textsl{видами стоимости} в~подобных \textsl{оценках} являются: \textsl{справедливая
стоимость}, \emph{справедливая рыночная стоимость}, \textsl{рыночная
стоимость}, \emph{стоимость на~открытом рынке}, \textsl{инвестиционная
стоимость}, \emph{стоимость во~владении} и~\emph{чистая стоимость
реализации}. Важно изучить точное определение и~описание применимого
\textsl{вида стоимости}, которое может содержаться, например, в~корпоративном
соглашении акционеров, законодательстве либо иных нормативных актах.
В~вопросах применения \textsl{видов стоимости}, не~предусмотренных
данными \emph{Всемирными стандартами}, \emph{оценщикам} следует руководствоваться
положениями подразделов~\ref{subsec:3.1.3_Compliance_with_IS}~\nameref{subsec:3.1.3_Compliance_with_IS}
\vpageref{subsec:3.1.3_Compliance_with_IS}--\pageref{subsec:3.1.3_Compliance_with_IS-End},
\ref{subsec:3.1.4_Compliance_with_jurisdictional_standards}~\nameref{subsec:3.1.4_Compliance_with_jurisdictional_standards}
\vpageref{subsec:3.1.4_Compliance_with_jurisdictional_standards}--\pageref{subsec:3.1.4_Compliance_with_jurisdictional_standards-End},
\ref{subsec:3.1.7_Regulation_monitoring}~\nameref{subsec:3.1.7_Regulation_monitoring}
\vpageref{subsec:3.1.7_Regulation_monitoring}--\pageref{subsec:3.1.7_Regulation_monitoring-End}.\label{5.3.4.9-End}

\stepcounter{SubSecCounter}

\thesubsection.\theSubSecCounter.\label{5.3.4.10} В~зависимости
от~правил и~практики проведения \textsl{оценки} отдельных \textsl{видов
стоимости}, результаты \textsl{оценки} одного и~того~же актива могут
существенно отличаться. Например, вследствие правил \textsl{оценки}
для~целей налогообложения, налоговый орган может рассматривать её~не~так,
как~это~делал~бы участник судебного разбирательства, участник сделки
по~слиянию либо \textsl{специальный покупатель}.\label{5.3.4.10-End}

\stepcounter{SubSecCounter}

\thesubsection.\theSubSecCounter.\label{5.3.4.11} Несмотря на~то,
что~\emph{оценщик} должен учитывать будущие доходы, которые могут
быть получены от~бизнеса, а~также иные, часто теоретические, аспекты
\textsl{оценки} (в~частности, вопросы налогообложения), в~конечном
итоге всё~же оценивается бизнес, который реально существует или~может
существовать в~существующих рыночных условиях на~\textsl{дату оценки}.
Вследствие и~по~причине этого \emph{оценщику} необходимо учитывать
будущие ожидания деятельности бизнеса. Подобные ожидания частично
могут быть основаны на~исторических показателях деятельности, частично
"--- на~условно достижимых. Во~втором случае речь идёт об~ожиданиях
участников рынка, выявленных \emph{оценщиком} после соответствующих
исследований деятельности конкретного бизнеса и~прогнозов развития
соответствующей отрасли, а~также обсуждении ожиданий перспектив деятельности
компании с~её~менеджментом.\label{5.3.4.11-End}

\stepcounter{SubSecCounter}

\thesubsection.\theSubSecCounter.\label{5.3.4.12} Поскольку концепция
\textsl{оценки} основывается на~той~прибыли, которую покупатель
сможет извлечь из~владения предприятием, как~правило, она~рассчитывается
после вычета затрат на~управление. Следовательно, при~определении
прибыли, являющейся базой стоимости предприятия, в~тех~случаях,
когда оно~не~несёт фактических управленческих расходов, \emph{оценщику}
необходимо осуществить вычет условных управленческих расходов, определённых
на~основе данных открытого рынка.\label{5.3.4.12-End}

\stepcounter{SubSecCounter}

\thesubsection.\theSubSecCounter.\label{5.3.4.13} Во~многих случаях
от~\emph{оценщика} может потребоваться применение более одного \emph{метода
оценки}, особенно в~тех~случаях, когда информации или~данных недостаточно
для~того, чтобы \emph{оценщик} мог полагаться только на~какой-то
один из~них. В~подобных ситуациях \emph{оценщик} может использовать
дополнительные \emph{методы}, позволяющие придти к~итоговому \emph{суждению
о~стоимости}, приводя при~этом сведения о~том, почему предпочтение
было отдано какой-либо одной или~нескольким методологиям. \emph{Оценщику}
следует рассмотреть возможность применения каждого из~\emph{подходов
к~оценке} и~указать причины, по~которым тот~или~иной из~них
не~был реализован.\label{5.3.4.13-End}\label{subsec:5.3.4_Businesses_and_business_interests-End}

\subsection{Информация и~данные\label{subsec:5.3.5_Information}}

\stepcounter{SubSecCounter}

\thesubsection.\theSubSecCounter.\label{5.3.5.1} \textsl{Оценка}
бизнеса часто бывает основана на~сведениях, полученных от~его~собственников,
их~консультантов и~представителей. \emph{Оценщик} должен указать,
на~какие сведения он~полагался, а~также обосновать необходимость
принятия и~использования без~проверки \uline{сведений}, предоставленных
заказчиком либо его~представителем. Некоторые из~\uline{них}
могут быть полностью или~частично опровергнуты, при~этом данный
факт подлежит отражению в~\emph{отчёте об~оценке}. Несмотря на~то,
что~стоимость может в~значительной степени зависеть от~будущих
ожиданий, исторические данные могут помочь определить, что~следует
ожидать в~будущем.\label{5.3.5.1-End}

\stepcounter{SubSecCounter}

\thesubsection.\theSubSecCounter.\label{5.3.5.2} \emph{Оценщику}
следует быть в~курсе всех соответствующих экономических событий и~отраслевых
тенденций, в~контексте которых проводится \textsl{оценка}, например,
развития политической обстановки, планов правительства, ожидаемых
инфляции и~процентных ставок, а~также деловой активности на~рынке.
Такие факторы могут по-разному влиять на~предприятия в~различных
отраслях.\label{5.3.5.2-End}

\stepcounter{SubSecCounter}

\thesubsection.\theSubSecCounter.\label{5.3.5.3} \textsl{Оценка}
долей бизнеса отражает его~финансовое состояние по~состоянию на~\textsl{дату
оценки}. Также необходимо понять характер активов и~обязательств:
\emph{оценщику} следует рассмотреть вопрос о~том, какие из~них~используются
для~получения дохода, а~какие являются избыточными для~текущей
деятельности по~состоянию на~\textsl{дату оценки}. При~необходимости,
\emph{оценщик} также должен учесть забалансовые активы и~обязательства.\label{5.3.5.3-End}\label{subsec:5.3.5_Information-End}

\subsection{Оценочные исследования\label{subsec:5.3.6_Valuation-investigation}}

\stepcounter{SubSecCounter}

\thesubsection.\theSubSecCounter.\label{5.3.6.1} Минимальным требованием
является требование о~том, что~\emph{оценщики} могут приступать
к~проведению \textsl{оценки} бизнеса только после детального изучения
и~понимания оцениваемого предприятия, его~деятельности и~активов,
в~т.\,ч.~анализа исторических сведений об~их функционировании.
Помимо этого \emph{им}~также потребуется всестороннее понимание структуры
управления и~персонала, состояния отрасли, общих экономических перспектив
и~политических факторов. В~дополнение к~вышеперечисленному, необходимо
принимать во~внимание такие аспекты как~права миноритарных акционеров.
По~этим причинам, \emph{оценщикам}, занимающимся \textsl{оценкой}
бизнеса, следует обладать набором специфических знаний.\label{6.3.6.1-End}

\stepcounter{SubSecCounter}

\thesubsection.\theSubSecCounter.\label{5.3.6.2} Типовой перечень
данных, необходимых для~обеспечения понимания \emph{оценщиком} деятельности
хозяйствующего субъекта, включает в~себя следующее:
\begin{itemize}
\item последняя \textsl{финансовая отчётность}, а~также детали текущих
и~прежних планов и~прогнозов;
\item описание и~история бизнеса~(актива), включая вопросы их~правовой
защиты;
\item информация о~бизнесе или~активе и~сопутствующей принадлежащей этому
бизнесу интеллектуальной собственности и~правах на~нематериальные
активы такие как, например, маркетинговые и~технологические ноу-хау,
права на~результаты исследований и~разработок, техническая документация,
чертежи и~справочные руководства, а~также любые лицензии, согласования,
одобрения, разрешения на~ведение деятельности, и~т.\,д.
\item учредительные документы (устав, учредительный договор), акционерные
соглашения, договоры о~подписке на~ценные бумаги, иные дополнительные
соглашения;
\item точное описание деятельности компании, а~также её~аффилированных
и~дочерних структур;
\item описание прав держателей \href{https://investor100.ru/chto-takoe-klassyi-aktsiy-a-b-s/}{акций каждого класса}~\cite{Klassy_aktsiy},
а~также прав держателей долговых обязательств, обеспеченных залогом;
\item прежние \emph{отчёты об~оценке};
\item продаваемые, поддерживаемые и~развиваемые бизнесом продукты и~нематериальные
активы;
\item описание рынка~(рынков) и~конкурентной среды, барьеров для~входа
на~эти~рынки, деловые и~маркетинговые планы, заключение \href{https://ru.wikipedia.org/wiki/Due_diligence}{сводной инвестиционной экспертизы}
(т.\,н. \href{https://en.wikipedia.org/wiki/Due_diligence}{«due diligence»})~\cite{Wiki:due_diligence-rus,Wiki:due_diligence-eng};
\item сведения об~участии в~стратегических альянсах и~совместных предприятиях;
\item сведения о~возможности передачи либо уступки прав на~нематериальные
активы или~выплаты по~лицензионным соглашениям;
\item сведениях об~основных покупателях\textbf{ }и~поставщиках;
\item цели предприятия, ожидаемые в~отрасли события и~тенденции, а~также
сведения о~том, как~образом они~могут оказать влияние на~бизнес~(актив);
\item учётная политика предприятия;
\item анализ сильных и~слабых сторон, возможностей и~угроз (\href{https://ru.wikipedia.org/wiki/SWOT-\%D0\%B0\%D0\%BD\%D0\%B0\%D0\%BB\%D0\%B8\%D0\%B7}{SWOT-анализ})~\cite{Wiki:SWOT-rus};
\item ключевые особенности положения на~рынке: наличие монопольного либо
доминирующего положения, доля~рынка);
\item планируемые ключевые инвестиции в~основные средства;
\item положение конкурентов;
\item сезонные либо циклические тенденции;
\item изменения в~технике и~технологиях, оказывающие влияние на~бизнес~(актив);
\item риски, связанные с~поставками сырья и~материалов, в~т.\,ч.~контрактные
риски поставщиков;
\item сведения о~\href{https://ru.wikipedia.org/wiki/\%D0\%A1\%D0\%BB\%D0\%B8\%D1\%8F\%D0\%BD\%D0\%B8\%D1\%8F_\%D0\%B8_\%D0\%BF\%D0\%BE\%D0\%B3\%D0\%BB\%D0\%BE\%D1\%89\%D0\%B5\%D0\%BD\%D0\%B8\%D1\%8F}{сделках слияния и поглощения}~\cite{Wiki:M&A-rus},
имевших место незадолго до~\textsl{даты оценки}, и~их~параметрах;
\item сведения о~событиях либо изменениях в~\uline{бизнесе}, произошедших
после последней отчётной даты таких как, например, принятие управленческих
решений или~бюджетов, изменение прогнозов и~т.\,п., имеющих существенное
значение для~\uline{его} \textsl{оценки};
\item сведения о~наличии предложения о~покупке бизнеса либо о~переговорах
по~вопросу \href{https://ru.wikipedia.org/wiki/\%D0\%9F\%D0\%B5\%D1\%80\%D0\%B2\%D0\%B8\%D1\%87\%D0\%BD\%D0\%BE\%D0\%B5_\%D0\%BF\%D1\%83\%D0\%B1\%D0\%BB\%D0\%B8\%D1\%87\%D0\%BD\%D0\%BE\%D0\%B5_\%D0\%BF\%D1\%80\%D0\%B5\%D0\%B4\%D0\%BB\%D0\%BE\%D0\%B6\%D0\%B5\%D0\%BD\%D0\%B8\%D0\%B5}{первичного публичного предложения}
~\cite{Wiki:IPO-rus} акций компании с~банками или~иными организаторами
данного процесса;
\item сведения об~управлении исследованиями и~разработками, например,
\href{https://ru.wikipedia.org/wiki/\%D0\%A1\%D0\%BE\%D0\%B3\%D0\%BB\%D0\%B0\%D1\%88\%D0\%B5\%D0\%BD\%D0\%B8\%D0\%B5_\%D0\%BE_\%D0\%BD\%D0\%B5\%D1\%80\%D0\%B0\%D0\%B7\%D0\%B3\%D0\%BB\%D0\%B0\%D1\%88\%D0\%B5\%D0\%BD\%D0\%B8\%D0\%B8}{соглашениях о неразглашении}~\cite{Wiki:NDA-rus},
субподрядчиках, системах обучения и~мотивации сотрудников;
\item \textsl{оценки} активов, являющихся составной частью оцениваемого
бизнеса.\label{5.3.6.2-End}
\end{itemize}
\stepcounter{SubSecCounter}

\thesubsection.\theSubSecCounter.\label{5.3.6.3} Существенная часть
сведений, на~основе которых проводится \textsl{оценка} бизнеса, бывает
предоставлена самим заказчиком, при~этом её~проверка не~представляется
возможной. В~этом случае данное обстоятельство должно быть в~явном
виде отражено в~\emph{отчёте}. То~же самое может относиться к~сведениям,
полученным от~других \emph{специалистов-оценщиков}, из~комментариев
и~иных информированных источников, как~это~описано выше в~п.~\ref{5.3.5.1}
\vpageref{5.3.5.1}--\pageref{5.3.5.1-End}, при~этом также возникает
необходимость отражения данного обстоятельства в~\emph{отчёте}.\label{5.3.6.3-End}\label{subsec:5.3.6_Valuation-investigation-End}

\subsection{Подходы к~оценке и~её~методы\label{subsec:5.3.7_Valuation_approaches}}

\stepcounter{SubSubSecCounter}

\thesubsection.\theSubSubSecCounter.\label{5.3.7.0.1} В~широком
смысле теория \textsl{оценки} признаёт четыре подхода к~оценке акций
и~бизнеса:
\begin{itemize}
\item \textsl{рыночный подход} (известный также как~подход, основанный
на~прямом рыночном сравнении или~сравнительный подход);
\item \textsl{доходный подход};
\item \textsl{затратный подход};
\item метод чистых активов.\footnote{Прим.~пер.: в~российской теории \textsl{оценки} данный метод не~образует
самостоятельный подход, являясь одним из~методов \textsl{затратного
подхода}.}\label{5.3.7.0.1-End}
\end{itemize}
\stepcounter{SubSubSecCounter}

\thesubsection.\theSubSubSecCounter.\label{5.3.7.0.2} В~то~время
как~\textsl{рыночный}~(сравнительный) и~\textsl{доходный подходы}
могут быть использованы для~\textsl{оценки} любого бизнеса или~доли
в~нём, \textsl{затратный подход} как~правило не~применяется, за~исключением
случаев, когда прибыль и~денежные потоки не~могут быть определены
с~достаточной степенью надёжности, например, в~случае \textsl{оценки}
стартапов и~компаний, находящихся на~ранней стадии развития.\label{5.3.7.0.2-End}

\stepcounter{SubSubSecCounter}

\thesubsection.\theSubSubSecCounter.\label{5.3.7.0.3} Альтернативным
вариантом \textsl{оценки} бизнеса и~долей в~нём, является метод
чистых активов, основанный на~переоценке базовых активов, если таковая
является необходимой. Данный подход применим в~случае \textsl{оценки}
холдингов, инвестиционных компаний, а~также фондов, владеющих акциями
публичных компаний.\label{5.3.7.0.3-End}

\stepcounter{SubSubSecCounter}

\thesubsection.\theSubSubSecCounter.\label{5.3.7.0.4} По~возможности
рекомендуется привлекать участников рынка, которые могут дать основанное
на~известных только им~сведениях представление о~сделках и~рыночных
условиях, имеющих решающее значение для~правильного использования
данных в~анализе.\label{5.3.7.0.4-End}

\subsubsection{Рыночный~(сравнительный) подход в~оценке бизнеса\label{subsubsec:5.3.7.1_Market_approach}}

\stepcounter{SubSubSecCounter}

\thesubsection.\theSubSubSecCounter.\label{5.3.7.1.1}

\textsl{Рыночный~(сравнительный) подход} основан на~определении
стоимости актива путём сопоставления данных о~продажах либо предложениях
активов, аналогичных либо схожих с~оцениваемым, с~данными об~оцениваемом
предприятии.\label{5.3.7.1.1-End}

\stepcounter{SubSubSecCounter}

\thesubsection.\theSubSubSecCounter.\label{5.3.7.1.2} Двумя основными
методами \textsl{оценки} в~рамках \textsl{рыночного~(сравнительного)
подхода} являются: \guillemotleft метод рыночных мультипликаторов\guillemotright{}
и~\guillemotleft метод сопоставимых сделок\guillemotright . Оба~они
основаны на~данных, полученных из~трёх основных источников:
\begin{itemize}
\item фондовые рынки;
\item рынок сделок по~поглощению, объектами которых являются целые предприятия;
\item предыдущие сделки с~акциями~(долями) предприятий или~предложения
об~их~продаже.\label{5.3.7.1.2-End}
\end{itemize}
\stepcounter{SubSubSecCounter}

\thesubsection.\theSubSubSecCounter.\label{5.3.7.1.3} Метод рыночных
мультипликаторов основан на~сравнении \emph{объекта оценки} с~аналогичными
компаниями и~активами, торгующимися на~бирже. При~применении данного
метода оценочные мультипликаторы определяются на~основе исторических
и~операционных данных сопоставимых \uline{компаний}. \uline{Они}~отбираются,
по~возможности, из~той~же отрасли либо отрасли, на~которую влияют
те~же экономические факторы, что~и~на~рассматриваемое предприятие,
при~этом \uline{их}~отбор осуществляется как~на~качественной,
так~и~на~количественной основе. Затем эти~мультипликаторы корректируются
с~учётом сильных и~слабых сторон \emph{объекта оценки} относительно
выбранных компаний и~применяются к~соответствующим операционным
данным оцениваемой компании для~получения показателя её~стоимости.
Внесение корректировок, подтверждённых рыночными информацией и~данными,
представленными в~\emph{отчёте}, необходимо для~отражения различных
свойств или~характеристик, присущих конкретному оцениваемому бизнесу.
Примерами таких вопросов, требующих внесения корректировок, являются
различия в~ожидаемом риске и~прогнозах будущей деятельности, а~также
различия в~правах, следующих из~размера оцениваемой доли, включая
вопросы степени контроля, оборотоспособности и~ликвидности.\label{5.3.7.1.3-End}

\stepcounter{SubSubSecCounter}

\thesubsection.\theSubSubSecCounter.\label{5.3.7.1.4} Метод сопоставимых
сделок основан на~использовании оценочных мультипликаторов, рассчитанных
на~основе данных о~сделках с~акциями~(долями) оцениваемой компании,
имевших место в~прошлом и~(или) на~основе данных о~сделках с~акциями~(долями)
других компаний той~же либо смежной отрасли. Затем данные мультипликаторы
корректируются и~применяются к~показателям операционной деятельности
оцениваемой компании для~получения показателя её~стоимости.\label{5.3.7.1.4-End}

\stepcounter{SubSubSecCounter}

\thesubsection.\theSubSubSecCounter.\label{5.3.7.1.5} В~некоторых
отраслях предприятия покупаются и~продаются на~основе сложившейся
рыночной практики либо эмпирических правил, часто (хотя и~не~исключительно)
основанных на~мультипликаторах или~процентах от~оборота и~не~связанных
с~прибылью. Если такие правила существуют, и~имеется подтверждение
того, что~продавцы и~покупатели реального рынка полагаются на~них,
\emph{оценщику} бизнеса следует принимать их~во~внимание. Однако
представляется разумным осуществить перепроверку результатов, полученных
на~основе применения подобной рыночной практики, с~помощью одного
или~нескольких других методов. Необходимо убедиться в~том, что~\guillemotleft сложившаяся
рыночная практика\guillemotright{} не~была вытеснена с~течением времени
под~действием изменившихся обстоятельств.\label{5.3.7.1.5-End}\label{subsubsec:5.3.7.1_Market_approach-End}

\subsubsection{Доходный подход\label{subsubsec:5.3.7.2_Income_approach}}

\stepcounter{SubSubSecCounter}

\thesubsection.\theSubSubSecCounter.\label{5.3.7.2.1} \textsl{Доходный
подход} имеет ряд~реализаций, но~в~целом всегда основывается на~доходах,
которые актив, вероятно, будет приносить в~течение оставшегося срока
полезного использования либо определённого периода фиксированной продолжительности.
\emph{Оценка} бизнеса, выполненная \textsl{доходным подходом}, основывается
как~на~исторических так~и~на~прогнозных показателях деятельности
предприятия. В~случае их~недоступности вместо этого возможно использование
\emph{метода прямой капитализации} прибыли за~один период.\label{5.3.7.2.1-End}

\stepcounter{SubSubSecCounter}

\thesubsection.\theSubSubSecCounter.\label{5.3.7.2.2} \emph{Метод
прямой капитализации}, как~правило, предполагает проведение \textsl{оценки}
на~основе капитализации прибыли за~один период. Во~всех случаях
требуется глубокое понимание бухгалтерской и~экономической прибыли,
различий между ними, принципов её~учёта, отражённого в~исторической
\textsl{финансовой отчётности}, а~также того, каким образом она~прогнозируется.
Далее определяется нормализованная прибыль после налогообложения,
которая, при~необходимости, корректируется для~отражения разницы
между фактическими исторической прибылью и~денежными потоками и~теми,
которые, как~можно ожидать, получил~бы покупатель бизнеса на~\textsl{дату
оценки}.\label{5.3.7.2.2-End}

\stepcounter{SubSubSecCounter}

\thesubsection.\theSubSubSecCounter.\label{5.3.7.2.3} Дополнительные
корректировки могут включать пересчёт затрат~и~результатов сделок,
осуществлённых на~нерыночных условиях, либо затрат, понесённых в~пользу
аффилированных структур, необходимый для~приведения соответствующих
показателей к~рыночным условиям, а~также внесение поправок, необходимых
для~исключения влияния событий, вызывающих возникновение разовых
доходов либо расходов. Примерами последних являются разовые сокращения
персонала либо прибыли и~убытки, нетипичные для~обычной операционной
деятельности. При~этом необходимо обеспечить применение сопоставимой
базы для~амортизации и~объёма запасов.\label{5.3.7.2.3-End}

\stepcounter{SubSubSecCounter}

\thesubsection.\theSubSubSecCounter.\label{5.3.7.2.4} Далее с~использованием
мультипликатора отношения цены к~чистой прибыли~(\href{https://en.wikipedia.org/wiki/Price\%E2\%80\%93earnings_ratio}{P/E}~\cite{Wiki:P/E_ratio})
осуществляется её~капитализация. Также возможно использование нормализованной
прибыли до~налогообложения и~соответствующего ей~мультипликатора.
Широкое распространение получила капитализация \href{https://www.audit-it.ru/finanaliz/terms/performance/ebit.html}{прибыли до вычета процентов и налогов (EBIT)}~\cite{EBIT,EBIT&EBITDA}
либо \href{https://www.audit-it.ru/finanaliz/terms/performance/ebitda.html}{прибыли до вычета процентов, налогов и амортизации (EBITDA)}~\cite{EBITDA,EBIT&EBITDA}
с~использованием соответствующих им~мультипликаторов. При~этом
необходимо внимательно проводить разграничение между:
\begin{itemize}
\item \emph{стоимостью предприятия} \emph{(стоимостью инвестированного капитала)},
учитывающей его~задолженность и~избыточные активы, способные влиять
на~его~цену с~точки зрения~покупателя;
\item \emph{стоимостью акционерного~(собственного) капитала}.\footnote{Прим.~пер.: в~целом, можно сказать о~том, что~\emph{стоимость
инвестированного капитала} отражает его~\guillemotleft мгновенную\guillemotright{}
стоимость, тогда как~\emph{стоимость акционерного~(собственного)
капитала} в~определённой степени характеризует будущее состояние
стоимости. Ознакомиться подробнее с~различиями между этими понятиями
можно, например, в~\href{https://ru.moneynx.com/enterprise-value-vs-equity-value}{данной статье}~\cite{Equity&Enterprise}.
Формулы, описывающие взаимосвязь между ними, можно найти по~\href{http://kvalexam.ru/wiki/3.8._\%D0\%A0\%D0\%B0\%D1\%81\%D1\%87\%D0\%B5\%D1\%82_\%D1\%81\%D1\%82\%D0\%BE\%D0\%B8\%D0\%BC\%D0\%BE\%D1\%81\%D1\%82\%D0\%B8_\%D0\%B8\%D0\%BD\%D0\%B2\%D0\%B5\%D1\%81\%D1\%82\%D0\%B8\%D1\%80\%D0\%BE\%D0\%B2\%D0\%B0\%D0\%BD\%D0\%BD\%D0\%BE\%D0\%B3\%D0\%BE_\%D0\%B8_\%D1\%81\%D0\%BE\%D0\%B1\%D1\%81\%D1\%82\%D0\%B2\%D0\%B5\%D0\%BD\%D0\%BD\%D0\%BE\%D0\%B3\%D0\%BE_\%D0\%BA\%D0\%B0\%D0\%BF\%D0\%B8\%D1\%82\%D0\%B0\%D0\%BB\%D0\%B0}{ссылке}~\cite{Equity_value_formulas}.}\label{5.3.7.2.4-End}
\end{itemize}
\stepcounter{SubSubSecCounter}

\thesubsection.\theSubSubSecCounter.\label{5.3.7.2.5} \emph{Стоимость
предприятия (инвестированного капитала)} часто определяется путём
капитализации прибыли либо иного денежного потока до~затрат на~обслуживание
долга с~использование ставки капитализации либо дисконтирования,
представляющих собой \href{https://ru.wikipedia.org/wiki/\%D0\%A1\%D1\%80\%D0\%B5\%D0\%B4\%D0\%BD\%D0\%B5\%D0\%B2\%D0\%B7\%D0\%B2\%D0\%B5\%D1\%88\%D0\%B5\%D0\%BD\%D0\%BD\%D0\%B0\%D1\%8F_\%D1\%81\%D1\%82\%D0\%BE\%D0\%B8\%D0\%BC\%D0\%BE\%D1\%81\%D1\%82\%D1\%8C_\%D0\%BA\%D0\%B0\%D0\%BF\%D0\%B8\%D1\%82\%D0\%B0\%D0\%BB\%D0\%B0}{средневзвешенную стоимость капитала}~(\href{https://www.audit-it.ru/articles/finance/a106/648167.html}{WACC})~\cite{Wiki:WACC_rus,WACC_rus}
для~компании с~сопоставимым соотношением собственных и~заёмных
средств. \emph{Стоимость акционерного~(собственного) капитала} "---
это~\emph{стоимость инвестированного капитала} за~вычетом \textsl{рыночной
стоимости} чистого долга. Также она~может быть установлена путём
использования денежного потока на~собственный капитал.\label{5.3.7.2.5-End}

\stepcounter{SubSubSecCounter}

\thesubsection.\theSubSubSecCounter.\label{5.3.7.2.6} \emph{Методы
оценки}, основанные на~понятии приведённой стоимости, предполагают
её~расчёт на~базе текущей стоимости будущих денежных потоков, генерируемых
в~течение определённого периода времени активом, группой активов
либо предприятием в~целом. К~таким денежным потокам относятся прибыль,
экономия затрат, налоговые вычеты, выручка от~реализации актива.
В~процессе~\emph{оценки} бизнеса его~стоимость определяется путём
дисконтирования ожидаемых денежных потоков, рассчитываемых, в~случае
необходимости, с~учётом их~роста и~инфляции, с~использованием
ставки требуемой доходности. Требуемая норма доходности включает в~себя
безрисковую ставку инвестирования, ожидаемую инфляцию и~риски, связанные
с~конкретными инвестицией и~рынком. Как~правило, ставка дисконтирования
выбирается на~основе ставок доходности альтернативных инвестиций
аналогичного типа и~качества на~\textsl{дату оценки}. Термины, подобные
\guillemotleft норме доходности\guillemotright , могут иметь разное
значение в~понимании разных людей, вследствие и~по~причине чего
\emph{оценщикам} следует приводить их~определения в~тексте \emph{отчёта}.\label{5.3.7.2.6-End}

\stepcounter{SubSubSecCounter}

\thesubsection.\theSubSubSecCounter.\label{5.3.7.2.7} При~\textsl{оценке}
стоимости \emph{инвестированного} либо \emph{акционерного капитала}
необходимо учитывать стоимость избыточных либо неиспользуемых активов,
находящихся в~собственности предприятия.\label{5.3.7.2.7-End}

\stepcounter{SubSubSecCounter}

\thesubsection.\theSubSubSecCounter.\label{5.3.7.2.8} Применение
методов \textsl{доходного подхода}, основанных на~\textsl{виде стоимости},
образуемом дивидендными выплатами, оправдано, в~первую очередь, для~оценки
акций в~составе миноритарных пакетов. В~таких случаях проведение
\textsl{оценки} бизнеса осуществляется на~основе моделей первоначальной
доходности и~дисконтирования будущих дивидендов путём прогнозирования
их~размера и~стабильности, а~также расчёта требуемой нормы доходности.\label{5.3.7.2.8-End}\label{subsubsec:5.3.7.2_Income_approach-End}

\subsubsection{Затратный подход\label{subsubsec:5.3.7.3_Cost_approach}}

\stepcounter{SubSubSecCounter}

\thesubsection.\theSubSubSecCounter.\label{5.3.7.3.1} \textsl{Затратный
подход} определяет стоимость актива на~основе затрат, необходимых
для~его~создания либо замены другим аналогичным активом, исходя
из~предпосылки о~том, что~покупатель не~заплатит за~актив больше
той~цены, в~которую обойдётся получение актива равной полезности.
Данный подход часто используется для~\textsl{оценки} инвестиционных
компаний либо компаний в~капиталоёмких отраслях. Однако, всё~же
следует признать, что, чаще всего, данный подход вообще не~используется
при~\textsl{оценке} бизнеса за~исключение случаев, когда был~сделан
и~отражён в~\emph{отчёте} вывод о~невозможности применения других
подходов.\label{5.3.7.3.1-End}

\stepcounter{SubSubSecCounter}

\thesubsection.\theSubSubSecCounter.\label{5.3.7.3.2} При~\textsl{оценке}
стоимости бизнеса также следует учитывать факторы устаревания, технического
обслуживания и~стоимости денег во~времени. Если оцениваемый актив
уступает современному аналогу в~силу возраста либо устаревания, от~оценщика
может потребоваться применение корректировок к~стоимости предполагаемого
современного аналога, получив, таким образом, стоимость замещения
с~учётом износа и~устареваний.\label{5.3.7.3.2-End}\label{subsubsec:5.3.7.3_Cost_approach-End}

\subsubsection{Метод чистых активов\label{subsubsec:5.3.7.4_Asset-based_approach}}

\stepcounter{SubSubSecCounter}

\thesubsection.\theSubSubSecCounter.\label{5.3.7.4.1} Метод чистых
активов предполагает проведение \textsl{оценки} бизнеса на~основе
сложения стоимостей входящих в~него отдельных активов и~обязательств.
Применение данного метода как~правило оправдано в~случаях \textsl{оценки}
бизнеса в~сфере инвестиций в~недвижимость, а~также инвестиционных
фондов, размещающих средства в~акциях. Как~правило, данный метод
не~является предпочтительным при~оценке компаний, ведущих собственную
операционную деятельность, за~исключением случаев, когда такие компании
получают недостаточную прибыль от~своих основных средств либо помимо
операционной деятельности ведут значительную инвестиционную деятельность,
либо обладают существенным объёмом денежных средств и~их~эквивалентов.
При~определении стоимости чистых активов, приходящейся на~акцию,
могут вноситься повышающие либо понижающие корректировки.\label{5.3.7.4.1-End}

\stepcounter{SubSubSecCounter}

\thesubsection.\theSubSubSecCounter.\label{5.3.7.4.2} Исходные данные
и~\textsl{допущения}, используемые в~процессе \textsl{оценки}, могут
основываться как~на~фактических данных, так~и~на~предположениях.
\textsl{Рыночный~(сравнительный) подход}, как~правило, основан на~фактических
исходных данных, таких как~цены, имевшие место при~продаже аналогичных
активов или~предприятий либо фактические выручка или~прибыль. Предполагаемые
исходные данные могут включать прогнозы или~ожидания касательно денежных
потоков. При~проведении \textsl{оценки} \textsl{затратным подходом}
фактические исходные данные могут включать фактическую себестоимость
актива, в~то~время как~предполагаемые исходные данные могут учитывать
его~расчётную себестоимость и~другие факторы, такие как, например,
отношение к~риску других рыночных агентов.\label{5.3.7.4.2-End}

\stepcounter{SubSubSecCounter}

\thesubsection.\theSubSubSecCounter.\label{5.3.7.4.3} По~общему
правилу следует избегать \guillemotleft сборки стоимости\guillemotright{}
путём простого суммирования стоимостей составляющих частей. Соответственно,
при~\textsl{оценке} совокупности различных активов или~составных
частей предприятия \emph{оценщику} необходимо избегать получения итоговой
стоимости одним только сложением стоимостей отдельных активов или~составных
частей бизнеса.\label{5.3.7.4.3-End}\label{subsubsec:5.3.7.4_Asset-based_approach-End}\label{subsec:5.3.7_Valuation_approaches-End}

\subsection{Составление отчёта об~оценке стоимости предприятия~(бизнеса)\label{subsec:5.3.8_Reports}}

\stepcounter{SubSecCounter}

\thesubsection.\theSubSecCounter.\label{5.3.8.1} В~случае, когда
требуется проведение \textsl{оценки}, соответствующей требованиям
настоящих \emph{Всемирных стандартов}, \emph{оценщик} обязан подготовить
\emph{отчёт об~оценке} в~соответствии с~минимальными требованиями,
изложенными в~разделе~\ref{sec:4.3_VPS3_Valuation_reports}~\nameref{sec:4.3_VPS3_Valuation_reports}
\vpageref{sec:4.3_VPS3_Valuation_reports}--\pageref{sec:4.3_VPS3_Valuation_reports-End}.
Как~правило, \emph{отчёт} содержит краткий вводный раздел или~резюме,
в~котором описывается \emph{задание на~оценку}, излагаются основные
выводы и~приводятся ссылки на~разделы \emph{отчёта}, содержащие
детали \textsl{оценки}. Повествование в~\emph{\uline{отчёте}}
должно быть построено на~переходе от~общего к~частному, логически
связанных описаниях исходных данных и~анализа, содержащего все~необходимые
рассуждения, приводящие к~сделанным в~\uline{нём}~выводам.\label{5.3.8.1-End}

\stepcounter{SubSecCounter}

\thesubsection.\theSubSecCounter.\label{5.3.8.2} Большинство \emph{отчётов
об~оценке} бизнеса содержат следующие разделы, порядок которых может
отличаться от~приведённого ниже
\begin{itemize}
\item введение;
\item \emph{цель оценки} и~\textsl{вид определяемой стоимости};
\item \textsl{допущения} и~\textsl{специальные допущения};
\item описание \emph{объекта оценки};
\item описание и~история оцениваемого предприятия~(бизнеса);
\item бухгалтерская отчётность и~учётная политика предприятия;
\item анализ \textsl{финансовой отчётности};
\item анализ деловых и~маркетинговых планов и~прогноз развития;
\item выводы по~результатам анализа сопоставимых компаний и~сделок;
\item описание отрасли~(отраслей), в~которой~(которых) осуществляет свою
деятельность оцениваемая компания, общей экономической ситуации, доходности
и~рисков;
\item экологические ограничения;
\item \emph{методы оценки} и~выводы о~стоимости;
\item оговорки, отказы от~ответственности и~ограничения применимости результатов
\textsl{оценки}.\label{5.3.8.2-End}
\end{itemize}
\stepcounter{SubSecCounter}

\thesubsection.\theSubSecCounter.\label{5.3.8.3} Некоторые \emph{отчёты}
содержат отдельный раздел, содержащий общее описание методологии \textsl{оценки},
который часто следует за~введением. В~случаях, когда общенациональные,
региональные либо экономические данные имеют значение для~\textsl{оценки}
бизнеса или~актива, каждой категории таких данных может быть посвящён
отдельный раздел.\label{5.3.8.3-End}

\stepcounter{SubSecCounter}

\thesubsection.\theSubSecCounter.\label{5.3.8.4} Там~где~это~необходимо,
фактические данные либо ссылки на~них~приводятся в~тексте \emph{отчёта}
либо в~приложениях к~нему. В~случае проведения \textsl{оценки}
в~целях судебного разбирательства, \emph{отчёт} должен соответствовать
требованиям, установленным национальным и~региональным законодательством,
и~содержать все~необходимые \emph{раскрытия информации}, включая
заявление о~квалификации эксперта и~подписку об~истинности его~выводов.\label{5.3.8.4-End}\label{subsec:5.3.8_Reports-End}

\subsection{Конфиденциальность\label{subsec:5.3.9_Confidentiality}}

\stepcounter{SubSecCounter}

\thesubsection.\theSubSecCounter.\label{5.3.9.1} Сведения в~отношении
многих бизнес-активов являются конфиденциальными. \emph{Оценщики}
должны приложить максимум усилий для~сохранения этой конфиденциальности,
особенно в~части, касающейся сведений, полученных в~отношении сопоставимых
активов. По~требованию заказчика \emph{оценщики} бизнеса заключают
с~ним \href{https://ru.wikipedia.org/wiki/\%D0\%A1\%D0\%BE\%D0\%B3\%D0\%BB\%D0\%B0\%D1\%88\%D0\%B5\%D0\%BD\%D0\%B8\%D0\%B5_\%D0\%BE_\%D0\%BD\%D0\%B5\%D1\%80\%D0\%B0\%D0\%B7\%D0\%B3\%D0\%BB\%D0\%B0\%D1\%88\%D0\%B5\%D0\%BD\%D0\%B8\%D0\%B8}{соглашения о неразглашении}~\cite{Wiki:NDA-rus}
либо аналогичные по~смыслу соглашения.\label{5.3.9.1-End}\label{subsec:5.3.9_Confidentiality-End}\label{sec:5.3_VPGA-3_Valuation_of_businesses-End}

\newpage

\section{ПР~4. Оценка специализированной недвижимости \label{sec:5.4_VPGA-4_Valuation_of_trade_properties}}

\textbf{Данное руководство носит рекомендательный характер и~не~содержит
обязательных требований. Однако там, где~это~уместно, оно~содержит
отсылки к~соответствующему обязательному материалу, содержащемуся
в~других разделах настоящих Всемирных стандартов, а~также \href{https://www.rics.org/globalassets/rics-website/media/upholding-professional-standards/sector-standards/valuation/international-valuation-standards-rics2.pdf}{Международных стандартов оценки}~\cite{IVS-2020},
реализованные в~виде перекрёстных ссылок. Данные ссылки предназначены
для~помощи }\textbf{\textsl{членам RICS}}\textbf{ и~не~меняют статус
данного практического руководства. }\textbf{\textsl{Членам RICS}}\textbf{
необходимо помнить следующее:}
\begin{itemize}
\item \textbf{данное руководство не~может охватить все~возможные варианты,
вследствие чего }\textbf{\emph{оценщикам}}\textbf{ при~формировании
своих }\textbf{\emph{суждений о~стоимости}}\textbf{ всегда следует
учитывать факты и~обстоятельства, имеющие место в~рамках отдельных
заданий по~}\textbf{\textsl{оценке}}\textbf{;}
\item \textbf{следует внимательно относиться к~тому факту, что~в~ряде
юрисдикций могут существовать особые требования, не~предусмотренные
данным руководством.}
\end{itemize}

\subsection{Основная информация\label{subsec:5.4.1_Background}}

\stepcounter{SubSecCounter}

\thesubsection.\theSubSecCounter.\label{5.4.1.1} Некоторые объекты
специализированной коммерческой недвижимости оцениваются методами
\textsl{доходного подхода}. В~приведённом ниже руководстве изложены
основные принципы данных \emph{методов оценки} без~излишней конкретизации
и~деталей, которые могут варьироваться в~зависимости от~свойств
оцениваемого имущества.\label{5.4.1.1-End}

\stepcounter{SubSecCounter}

\thesubsection.\theSubSecCounter.\label{5.4.1.2} Данное руководство
распространяет своё действие только на~\textsl{оценку} отдельных
объектов недвижимости, определение стоимости которых основано на~их~коммерческом
потенциале т.\,е.~способности генерировать прибыль для~их~собственника.\label{5.4.1.2-End}

\stepcounter{SubSecCounter}

\thesubsection.\theSubSecCounter.\label{5.4.1.3} Сделки купли-продажи
некоторых объектов недвижимости основаны в~первую очередь на~их
коммерческом потенциале. Примерами таких объектов являются гостиницы,
пабы и~бары, рестораны, ночные клубы, заправочные станции, дома престарелых,
казино, кинотеатры и~театры, а~также различные другие виды недвижимости,
предназначенной для~досуга и~развлечений. Главной особенностью данного
типа недвижимости является то, что~она~была спроектирована либо
позднее приспособлена для~конкретного вида деятельности, и~обусловленное
этим отсутствие гибкости обычно означает, что~стоимость такого имущества
неразрывно связана с~доходами, которые владелец может получить от~его~использования
по~назначению. В~этом заключается её~отличие от~типовой недвижимости,
которая может быть использована для~различных видов деятельности,
например, стандартной офисной, промышленной или~торговой.\label{5.4.1.3-End}

\stepcounter{SubSecCounter}

\thesubsection.\theSubSecCounter.\label{5.4.1.4} Вышеприведённый
в~п.~\ref{5.4.1.3} \vpageref{5.4.1.3}--\pageref{5.4.1.3-End}
перечень таких объектов не~является закрытым. Другие виды недвижимости
(такие как, например, автостоянки, парки, стоянки для~жилых трейлеров,
крематории и~т.\,д.) также лучше всего могут быть оценены методами
\textsl{доходного подхода} на~основе их~коммерческого потенциала.
Отнесение недвижимости к~категории специализированной является предметом
\emph{суждения оценщика}, основанном на~свойствах и~фактическом
использовании объекта недвижимости, а~также текущем состоянии рынка
с~учётом перспектив его~изменения.\label{5.4.1.4-End}

\stepcounter{SubSecCounter}

\thesubsection.\theSubSecCounter.\label{5.4.1.5} \emph{Оценщики},
занимающиеся \textsl{оценкой} специализированной недвижимости, как~правило,
специализируются на~конкретном рынке. Понимание деятельности, для~ведения
которой предназначен объект недвижимости, как~на~операционном уровне,
так~и~на~уровне положения дел в~отрасли является основополагающим
условием понимания сделок на~рынке и~необходимого анализа.\label{5.4.1.5-End}

\stepcounter{SubSecCounter}

\thesubsection.\theSubSecCounter.\label{5.4.1.6} Сведения, необходимые
для~сравнительного анализа, могут быть получены не~только на~основании
данных о~совершённых сделках, но~и~из~широкого перечня источников.
Также сведения, касающиеся параметров доходности, могут быть получены
от~предприятий, работающих в~той~же отрасли. \emph{Оценщику} следует
особо отметить в~\emph{отчёте} то~обстоятельство, что~\textsl{оценка}
основана на~коммерческом потенциале \emph{объекта оценки}, и~привести
сведения о~фактически достигнутой прибыли. В~случае изменение показателей
деятельности и~прибыли объекта, его~стоимость также может измениться
(см.~секции~\ref{subsubsec:4.3.2.8_Nature_and_source_of_the_information},
\ref{subsubsec:4.3.2.15_Commentary_ on_any_material_uncertainty}
\vpageref{subsubsec:4.3.2.8_Nature_and_source_of_the_information}--\pageref{subsubsec:4.3.2.8_Nature_and_source_of_the_information-End},
\pageref{subsubsec:4.3.2.15_Commentary_ on_any_material_uncertainty}--\pageref{subsubsec:4.3.2.15_Commentary_ on_any_material_uncertainty-End}).\label{5.4.1.6-End} 

\stepcounter{SubSecCounter}

\thesubsection.\theSubSecCounter.\label{5.4.1.7} Данное руководство
основано на~предпосылке о~продолжении текущего варианта использования
\emph{объекта оценки}. Однако, если становится очевидным, что~альтернативный
вариант использования приведёт к~увеличению стоимости объекта, в~\emph{отчёте}
следует привести соответствующий комментарий. В~случае указания в~\emph{отчёте}
стоимости, основанной на~таком альтернативном использовании, оно~должно
сопровождаться комментарием о~том, что~при~проведении такой \textsl{оценки}
не~учитывались затраты на~закрытие существующего бизнеса, простои
и~любые иные затраты, связанные с изменением профиля деятельности.\label{5.4.1.7-End}\label{subsec:5.4.1_Background-End}

\subsection{Термины\label{subsec:5.4.2_Terms_used}}

\stepcounter{SubSubSecCounter}

\thesubsubsection.\theSubSubSecCounter.\label{5.4.2.0.1} Значение
терминов, используемых в~настоящем практическом руководстве, может
отличать от~их значения в~других сферах профессиональной деятельности.\label{5.4.2.0.1-End}

\subsubsection{Нормализованная чистая прибыль\label{subsubsec:5.4.2.1_Adjusted_net_profit}}

\stepcounter{SubSubSecCounter}

\thesubsubsection.\theSubSubSecCounter.\emph{\label{5.4.2.1.1} Нормализованная
чистая прибыль} представляет собой определённое \emph{оценщиком} значение
прибыли от~текущей деятельности предприятия. Это~вид прибыли, отражаемый
в~отчётности после исключения нетипичных и~разовых расходов, расходов
на~обслуживание долга и~амортизации, относящихся к~самому объекту,
а~также, при~необходимости, к~арендной плате. Данный показатель
отражает результаты деятельности собственника оцениваемого объекта
недвижимости и~позволяет ему~определить значение \emph{справедливой
устойчивой операционной прибыли}~(\href{https://www.adamsandco.com/accounts-appraisal-methodolgy/}{FMOP}~\cite{FMOP}).\label{5.4.2.1.1-End}
\label{subsubsec:5.4.2.1_Adjusted_net_profit-End}

\subsubsection{Прибыль до~уплаты процентов, налога на~прибыль, износа и~амортизации~(EBITDA)\label{subsubsec:5.4.2.2_EBITDA}}

\stepcounter{SubSubSecCounter}

\thesubsubsection.\theSubSubSecCounter.\label{5.4.2.2.1} Данный
показатель отражает фактические результаты деятельности предприятия
и~может отличаться от~определённого оценщиком значения \emph{справедливой
устойчивой операционной прибыли}~(\href{https://www.adamsandco.com/accounts-appraisal-methodolgy/}{FMOP}~\cite{FMOP}).\label{5.4.2.2.1-End}\label{subsubsec:5.4.2.2_EBITDA-End}

\subsubsection{Справедливая устойчивая операционная прибыль~(FMOP)\label{subsubsec:5.4.2.3_FMOP}}

\stepcounter{SubSubSecCounter}

\thesubsubsection.\theSubSubSecCounter.\label{5.4.2.3.1} \emph{Справедливая
устойчивая операционная }\emph{\uline{прибыль}} "--- это~величина
прибыли, определённой до~амортизации и~финансовых расходов (уплаты
процентов по~кредитам), связанных с~самим активом либо арендной
платой в~случае аренды, \uline{которую} разумно эффективный оператор~(REO)
ожидал~бы получить от~\emph{справедливого устойчивого оборота} (FMT)
на~основе оценки восприятия рынком потенциального дохода от~объекта
недвижимости. Данный показатель должен отражать все~операционные
расходы и~издержки REO, а~также ежегодное резервирование средств
на~такие периодические затраты как~текущий или~капитальный ремонт,
реконструкция и~(или) обновление оборудования и~инвентаря, необходимых
для~продолжения текущей деятельности.\footnote{Прим. пер.~данный показатель соответствует принятому в~русской оценочной
практике показателю чистого операционного дохода~(ЧОД)}\label{5.4.2.3.1-End}\label{subsubsec:5.4.2.3_FMOP-End}

\subsubsection{Справедливый устойчивый оборот~(FMT)\label{subsubsec:5.4.2.4_FMT}}

\stepcounter{SubSubSecCounter}

\thesubsubsection.\theSubSubSecCounter.\label{5.4.2.4.1} \emph{Справедливый
устойчивый оборот} "--- это~величина выручки, которую может ожидать
REO при~условии \textsl{допущения} о~том, что~объект недвижимости
был~отремонтирован и~оснащён надлежащим образом, имеет соответствующую
отделку и~обслуживается в~соответствии с~действующими нормами и~правилами.\label{5.4.2.4.1-End}\label{subsubsec:5.4.2.4_FMT-End}

\subsubsection{Рыночная арендная плата\label{subsubsec:5.4.2.5_Market_rent}}

\stepcounter{SubSubSecCounter}

\thesubsubsection.\theSubSubSecCounter.\label{5.4.2.5.1} \emph{Рыночная
арендная плата} "--- расчётная величина денежной суммы, за~которую
недвижимое имущество может быть передано в~рамках сделки по~его
аренде, заключённой на~\textsl{дату оценки} на~рыночных условиях
между заинтересованными арендодателем и~арендатором после надлежащего
изучения рынка на~взаимоприемлемых условиях в~случае, когда обе~стороны
являются независимыми, проявляют должную осмотрительность и~осведомлённость,
действуют разумно, в~своих интересах и~без~принуждения. При~определении
\textsl{рыночной арендной платы} следует приводить \guillemotleft взаимоприемлемые
условия\guillemotright{} предполагаемого договора аренды.\label{5.4.2.5.1-End}\label{subsubsec:5.4.2.5_Market_rent-End}

\subsubsection{Рыночная стоимость\label{subsubsec:5.4.2.6_Market_value}}

\stepcounter{SubSubSecCounter}

\thesubsubsection.\theSubSubSecCounter.\label{5.4.2.6.1} \emph{Рыночная
стоимость} "--- расчётная величина денежной суммы, в~обмен на~которую
актив или~обязательство могут быть переданы в~рамках сделки, заключённой
на~\textsl{дату оценки} на~рыночных условиях между заинтересованными
продавцом и~покупателем после надлежащего изучения рынка на~взаимоприемлемых
условиях в~случае, когда обе стороны являются независимыми, проявляют
должную осмотрительность и~осведомлённость, действуют разумно, в~своих
интересах и~без~принуждения.\label{5.4.2.6.1-End}\label{subsubsec:5.4.2.6_Market_value-End}

\subsubsection{Компания "--- оператор объекта недвижимости\label{subsubsec:5.4.2.7_Operational_entity}}

\stepcounter{SubSubSecCounter}

\thesubsubsection.\theSubSubSecCounter.\label{5.4.2.7.1} \emph{Компания
"--- оператор объекта недвижимости}, как~правило, представляет собой
структуру, обладающую:
\begin{itemize}
\item правами на~землю и~здания;
\item оснащение, включающее оборудование, приспособления, мебель и~оснастку;
\item нематериальные активы, образуемые воспринимаемым рынком коммерческим
потенциалом в~совокупности с~возможностью получения лицензий, согласований,
сертификатов и~разрешений либо продления срока действия уже~существующий;
\end{itemize}
Расходные материалы и~товарные запасы в~данном случае, как~правило,
не~принимаются к~учёту.\label{5.4.2.7.1-End}\label{subsubsec:5.4.2.7_Operational_entity-End}

\subsubsection{Персональный гудвилл текущей компании-оператора\label{subsubsec:5.4.2.8_Personal_goodwill}}

\stepcounter{SubSubSecCounter}

\thesubsubsection.\theSubSubSecCounter.\label{5.4.2.8.1} \emph{Персональный
гудвилл текущей компании-оператора} "--- это~ценность, образуемая
прибылью, полученной сверх ожиданий рынка, которая исчезнет при~продаже
объекта \textsl{специализированной недвижимости}, вследствие существования
финансовых факторов, относящихся непосредственно и~исключительно
к~текущему оператору бизнеса, таких как~налогообложение, применяемые
правила амортизации, расходы на~обслуживание долга, величина собственного
капитала, инвестированного в~бизнес.\label{5.4.2.8.1-End}\label{subsubsec:5.4.2.8_Personal_goodwill-End}

\subsubsection{Разумно эффективный оператор~(REO)\label{subsubsec:5.4.2.9_REO}}

\stepcounter{SubSubSecCounter}

\thesubsubsection.\theSubSubSecCounter.\label{5.4.2.9.1} \emph{Концепция
разумно эффективного оператора~(REO)} отражает предположение \emph{оценщика}
о~том, что~рыночные агенты являются компетентными действующими эффективно
операторами бизнеса, осуществляемого в~помещениях объекта недвижимости.
Она~предполагает учёт коммерческого потенциала объекта, а~не текущие
показатели действующего собственника, а~также исключение фактора
его~\emph{персонального гудвила}.\label{5.4.2.9.1-End}\label{subsubsec:5.4.2.9_REO-End}

\subsubsection{Капитал арендатора\label{subsubsec:5.4.2.10_Tenant=002019s_capital}}

\stepcounter{SubSubSecCounter}

\thesubsubsection.\theSubSubSecCounter.\label{5.4.2.10.1} К~данной
категории могут относиться, например, все~расходные материалы, инвентарь,
запасы и~оборотный капитал.\label{5.4.2.10.1-End}\label{subsubsec:5.4.2.10_Tenant=002019s_capital-End}

\subsubsection{Специализированный объект недвижимости\label{subsubsec:5.4.2.11_Trade_related_property}}

\stepcounter{SubSubSecCounter}

\thesubsubsection.\theSubSubSecCounter.\label{5.4.2.11.1} Это~любой
вид~недвижимости, спроектированной либо впоследствии приспособленной
для~определённого вида бизнеса, стоимость которой отражает коммерческий
потенциал этого бизнеса.\label{5.4.2.11.1-End}\label{subsubsec:5.4.2.11_Trade_related_property-End}

\subsubsection{Коммерческий потенциал\label{subsubsec:5.4.2.12_Trading_potential}}

\stepcounter{SubSubSecCounter}

\thesubsubsection.\theSubSubSecCounter.\label{5.4.2.12.1} Для~целей
\textsl{оценки} под~коммерческим потенциалом понимается будущая \uline{прибыль},
которую REO ожидает получить от~владения объектом недвижимости. \uline{Она}~может
быть выше либо ниже исторических значений. Коммерческий потенциал
определяется влиянием ряда факторов, присущих данному объекту, таких~как~местоположение,
конструкция и~тип, степень приспособленности для~ведения конкретной
\uuline{деятельности} и~\uuline{её}~исторические показатели
в~контексте преобладающих на~рынке условий.\label{5.4.2.12.1-End}\label{subsubsec:5.4.2.12_Trading_potential-End}\label{subsec:5.4.2_Terms_used-End}

\subsection{Методы доходного подхода в~оценке специализированной недвижимости\label{subsec:5.4.3_Profits_methods}}

\stepcounter{SubSecCounter}

\thesubsection.\theSubSecCounter.\label{5.4.3.1} Применение методов
\textsl{доходного подхода} применительно к~\textsl{оценке} специализированной
недвижимости включает следующие этапы.
\begin{enumerate}
\item \label{5.4.3.1.1}Определение FMT, который REO может получить от~использования
объекта недвижимости.\label{5.4.3.1.1-End}
\item \label{5.4.3.1.2}Определение потенциальной валовой прибыли при~соответствующем
FMT (при необходимости).\label{5.4.3.1.2-End}
\item \label{5.4.3.1.3}Определение FMOP. Затраты и~вычеты, используемые
при~этом, должны должны отражать те, которые ожидает REO, т.\,е.~наиболее
вероятный покупатель или~оператор объекта недвижимости, если~бы
последний~был предложен к~покупке на~рынке.\label{5.4.3.1.3-End}
\item \label{5.4.3.1.4}Далее в~зависимости от~обстоятельств возможны
следующие шаги.
\begin{enumerate}
\item \textsl{\label{5.4.3.1.4.a}Рыночная стоимость} определяется путём
капитализации FMOP по~ставке капитализации, отражающей премии за~риск,
присущий данной недвижимости, и~её~коммерческий потенциал. Для~этого
необходимо проведение сравнительного анализа сделок, совершённых с~сопоставимыми
объектами недвижимости.\label{5.4.3.1.4.a-End}
\item \label{5.4.3.1.4.b}При~\textsl{оценке} \textsl{рыночной стоимости}
\emph{оценщик} может решить, что~новый оператор объекта недвижимости
может провести перепланировку либо улучшение с~целью повышения его~коммерческого
потенциала. Учёт этого обстоятельства осуществляется на~стадии определения
FMT на~\hyperref[5.4.3.1.1]{этапе~1}. В~этом случае, из~результата
\textsl{оценки}, полученного на~\hyperref[5.4.3.1.4]{этапе~4},
вычитаются затраты на~такие улучшения и~перепланировки, а~также
проводится учёт фактора задержки получения FMT. Аналогичным образом,
в~случае, если для~достижения REO целевого значения FMT на~объекте
необходимо проведение ремонта и~(или) отделочных работ, величина
соответствующих затрат вычитается из~результата, полученного на~\hyperref[5.4.3.1.4.a]{шаге~<<a>>} .\label{5.4.3.1.4.b-End}
\item \label{5.4.3.1.4.c}При~\textsl{оценке} \textsl{рыночной арендной
платы} будь~то для~заключения нового договора аренды, пересмотра
условий действующего либо определения её~обоснованного текущего уровня
(в~частности, при~проведении \hyperref[subsec:5.4.9_Valuation_for_investment]{инвестиционной \textsl{оценки}})
FMOP следует уменьшать на~размер дохода арендатора, приходящегося
на~его~собственный капитал, инвестированный в~операционный бизнес,
"--- например, стоимость инвентаря, товарных запасов и~оборотных
средств. Остаток такой операции вычитания называется \guillemotleft остаточной
распределяемой прибылью\guillemotright . Она~распределяется между
арендодателем и~арендатором с~учётом соответствующих рисков и~премий
за~них, при~этом доля арендодателя представляет собой величину годовой
арендной платы.\label{5.4.3.1.4.c-End}\label{5.4.3.1.4-End}\label{5.4.3.1-End}
\end{enumerate}
\end{enumerate}
\stepcounter{SubSecCounter}

\thesubsection.\theSubSecCounter.\label{5.4.3.2} Может быть уместно
и~применение расширенных либо более детализированных методов \textsl{доходного
подхода}, в~особенности в~случае проведения \textsl{оценки} крупных
либо сложных объектов \textsl{специализированной недвижимости}. Например
возможно применение метода дисконтированных денежных потоков с~использованием
потоков, представляющих собой различные виды прибыли. Использование
детализированных моделей \textsl{оценки} помогает в~анализе и~обобщении
исторических и~текущих показателей деятельности, а~также в~прогнозировании,
которое может показать увеличение или~уменьшение показателей деятельности
относительно фактических. Это~может позволить понять реалистичный
для~REO либо вероятного покупателя уровень FMT и~FMOP.\label{5.4.3.2-End}

\stepcounter{SubSecCounter}

\thesubsection.\theSubSecCounter.\label{5.4.3.3} Поскольку для~успешной
работы важно наличие практических навыков и~понимание факторов, существующих
на~рынке, важно, чтобы \emph{оценщик} регулярно практиковал в~том~его~сегменте,
к~которому относится оцениваемый объект \textsl{специализированной
недвижимости}.\label{5.4.3.3-End}

\stepcounter{SubSecCounter}

\thesubsection.\theSubSecCounter.\label{5.4.3.4} При~проведении
\textsl{\uline{оценки}} \textsl{специализированных объектов недвижимости}
\emph{оценщику} следует учитывать итоговый результат различных \uline{её}~этапов.
Также она~должна быть основана на~имеющихся у~\emph{оценщика} общем
профессиональном опыте и~знаниях конкретного рынка.\label{5.4.3.4-End}\label{subsec:5.4.3_Profits_methods-End}

\subsection{Специальные допущения \label{subsec:5.4.4_Valuation_special_assumptions}}

\stepcounter{SubSecCounter}

\thesubsection.\theSubSecCounter.\label{5.4.4.1} Применительно к~\textsl{специализированной
недвижимости}, как~правило, осуществляется \textsl{оценка} \textsl{рыночной
стоимости} либо \textsl{рыночной арендной платы}, однако \emph{оценщиков}
часто просят провести \emph{оценку}, основанную на~\textsl{специальных
допущениях}.

Типовыми \textsl{специальными допущениями} являются:
\begin{enumerate}
\item \label{5.4.4.1-1}осуществление коммерческой деятельности прекращено,
связанная с~ней~коммерческая информация недоступна потенциальным
покупателям либо арендаторам;\label{5.4.4.1-1-End}
\item в~дополнение к~перечисленному выше в п.~\hyperref[5.4.4.1-1]{1}
также был вывезен инвентарь, необходимый для~ведения соответствующей
деятельности;
\item предприятие полностью готово к~началу операционной деятельности,
но~сама деятельности ещё~не~началась (т.\,н.~\guillemotleft\textsl{оценка}
первого дня\guillemotright );
\item \textsl{оценка} проводится на~основе прогнозных показателей, принимаемых
в~качестве подтверждённых, что~является целесообразным при~её~проведении
в~целях рассмотрения вопросов девелопмента.\label{5.4.4.1-End}\label{subsec:5.4.4_Valuation_special_assumptions-End}
\end{enumerate}

\subsection{Методы оценки объектов недвижимости полностью оснащённых для~ведения
деятельности\label{subsec:5.4.5_Valuation_fully_equiped}}

\stepcounter{SubSecCounter}

\thesubsection.\theSubSecCounter.\label{5.4.5.1} \textsl{Оценка}
\textsl{объекта специализированной недвижимости}, полностью оснащённого
для~ведения определённой деятельности, обязательно основывается на~предположении
о~том, что~сделка с~ним будет представлять собой сдачу в~аренду
либо продажу совместно со~всем инвентарём, лицензиями и~иными активами,
необходимыми для~продолжения этой деятельности.\label{5.4.5.1-End}

\stepcounter{SubSecCounter}

\thesubsection.\theSubSecCounter.\label{5.4.5.2} Однако, необходимо
проявлять осторожность, поскольку вышеуказанное предположение не~обязательно
означает, что~в~периметр \textsl{оценки} следует включать весь инвентарь.
Например, некоторое оборудование может принадлежать \textsl{третьим
лицам}, вследствие и~по~причине чего оно~не~будет включаться в~состав
оцениваемого права. В~\emph{отчёте} следует приводить все~\textsl{допущения},
касающиеся инвентаря, включаемого в~периметр \textsl{оценки}.\label{5.4.5.2-End}

\stepcounter{SubSecCounter}

\thesubsection.\theSubSecCounter.\label{5.4.5.3} Возможна ситуация,
когда некоторые основные средства, необходимые для~ведения операционной
деятельности, используются на~основании прав, отдельных от~прав
на~сам~объект недвижимости (земля либо здания) либо на~основании
договора финансовой аренды~(лизинга), либо на~основании иных соглашений,
предполагающих отдельную оплату. В~таких случаях может потребоваться
\textsl{допущение} о~том, что~в~случае \uline{продажи компании
"--- оператора объекта недвижимости} владельцы таких активов либо
иные получатели платежей за~них согласятся передать их~в~рамках
\uline{такой сделки}. В случаи отсутствия уверенности в~возможности
введения такого \textsl{допущения} \emph{оценщик} обязан тщательно
изучить влияние отсутствия таких активов на~тех, кто~приобретёт
компанию-оператора либо арендует объект, и~привести необходимые комментарии
в~тексте \emph{отчёта об~оценке}.\label{5.4.5.3-End}

\stepcounter{SubSecCounter}

\thesubsection.\theSubSecCounter.\label{5.4.5.4} В~случаях, когда
\textsl{специализированная недвижимость} продаётся либо сдаётся в~аренду
полностью укомплектованной для~ведения соответствующей операционной
деятельности, её~покупателю либо оператору, как~правило, предстоит
обновить лицензии и~иные установленные законом разрешения, а~также
получить выгоды от~действующих после передачи прав на~них. Если
\emph{оценщик} исходит из~другого \textsl{допущения}, его~следует
в~явном виде изложить в~тексте \emph{отчёта} в~качестве \textsl{специального
допущения}.\label{5.4.5.4-End}

\stepcounter{SubSecCounter}

\thesubsection.\theSubSecCounter.\label{5.4.5.5} В~случае невозможности
проверки \uline{лицензий, согласований, сертификатов и~разрешений},
касающихся оцениваемого объекта \textsl{специализированной недвижимости}
или~каких-либо иных сведений, в~\emph{отчёте} должны быть приведены
сделанные в~этой части \textsl{допущения}, а~также рекомендация
проверки \uline{их}~наличия со~стороны консультантов заказчика
в~области права.\label{5.4.5.5-End}\label{subsec:5.4.5_Valuation_fully_equiped-End}

\subsection{Определение коммерческого потенциала\label{subsec:5.4.6_Assessing_the_trading_potential}}

\stepcounter{SubSecCounter}

\thesubsection.\theSubSecCounter.\label{5.4.6.1} Существует различие
между \textsl{рыночной стоимостью} \textsl{специализированной недвижимости}
и~её~\textsl{инвестиционной стоимостью}~\textsl{(ценностью)} для~конкретного
пользователя~(оператора). \textsl{Ценность} для~конкретного оператора
образуется текущей и~потенциальной чистой прибылью, получаемой от~операционной
деятельности в~выбранном формате. Хотя нынешний оператор может быть
одним из~потенциальных покупателей объекта, для~формирования \emph{суждения
о~его~стоимости} \emph{оценщику} будет необходимо понять требования
и~достижимую прибыль других потенциальных покупателей, а~также динамику
открытого рынка.\label{5.4.6.1-End}

\stepcounter{SubSecCounter}

\thesubsection.\theSubSecCounter.\label{5.4.6.2} Объекты \textsl{специализированной
недвижимости} рассматриваются в~качестве~отдельных предприятий и,
как~правило, оцениваются исходя из~\textsl{допущения} о~продолжении
текущей коммерческой деятельности.\label{5.4.6.2-End}

\stepcounter{SubSecCounter}

\thesubsection.\theSubSecCounter.\label{5.4.6.3} При~\textsl{оценке}
будущего коммерческого потенциала \emph{оценщику} следует исключить
любые выручку и~затраты, возникающие в~связи с~особыми обстоятельствами,
навыками, опытом, репутацией, товарным знаком (знаком обслуживания),
связанными с~конкретным текущим оператором объекта. Однако \emph{оценщику}
следует отразить дополнительный коммерческий потенциал, который может
быть обеспечен со~стороны REO в~случае владения им~этим объектом
недвижимости на~\textsl{дату оценки}.\label{5.4.6.3-End}

\stepcounter{SubSecCounter}

\thesubsection.\theSubSecCounter.\label{5.4.6.4} Необходимо сопоставить
фактические показатели деятельности с~теми, которые имеют место у~подобных
объектов \textsl{специализированной недвижимости}, в~которых осуществляется
схожая деятельность. Для~этого \emph{оценщику} следует обладать должным
пониманием коммерческого потенциала данного вида недвижимости, а~также
принципов сравнения относящихся к~нему объектов. Используя данные
открытого рынка о~сделках со~схожими объектами \textsl{специализированной
недвижимости} и~показателях их~деятельности, \emph{оценщику} следует
определить, обеспечивает~ли деятельность, фактически осуществляемая
на~оцениваемом объекте, получение денежного потока, являющегося \hyperref[5.4.2.4]{\emph{справедливым устойчивым оборотом(FMT)}}
в~текущих рыночных условиях. В~случае наличия соответствующих данных,
фактические показатели деятельности, осуществляемой на~\emph{объекте
оценки} и~аналогичных объектах, могут быть скорректированы с~целью
учёта особенностей REO.\label{5.4.6.4-End}

\stepcounter{SubSecCounter}

\thesubsection.\theSubSecCounter.\label{5.4.6.5} Для~многих предприятий,
ведущих деятельность, связанную с~объектами \textsl{специализированной
недвижимости}, передача бизнеса будет осуществляться путём продажи
права собственности либо права аренды. Данные о~таких сделках могут
использоваться в~целях \textsl{оценки} объекта \textsl{специализированной
недвижимости} при~условии, что~\emph{оценщик} способен исключить
из~цены сделки те~её~компоненты, которые не~относятся к~самому
объекту недвижимости, например, товарные запасы, расходные материалы,
обязательства и~нематериальные активы (такие как~товарные знаки
и~знаки обслуживания, контрактные права и~обязательства) в~том
объёме, в~котором они~не~перейдут к~REO.\label{5.4.6.5-End}

\stepcounter{SubSecCounter}

\thesubsection.\theSubSecCounter.\label{5.4.6.6} Изменения конкурентной
среды могут оказать существенное влияние на~прибыльность и, как~следствие,
стоимость. \emph{Оценщику} следует принимать во~внимание влияние
текущего и~ожидаемого в~будущем уровня конкуренции. В~случае, когда
ожидаются существенные изменения относительно существующей ситуации,
\emph{оценщику} следует ясно описать это~в~\emph{отчёте} и~привести
комментарии касательно того, какое влияние данное событие может оказать
на~доходность и~стоимость объекта.\label{5.4.6.6-End}

\stepcounter{SubSecCounter}

\thesubsection.\theSubSecCounter.\label{5.4.6.7} Внешние факторы
такие как, например, строительство новой дороги или~изменения в~соответствующем
законодательстве также могут повлиять на~коммерческий потенциал и,
следовательно, на~стоимость \textsl{специализированной недвижимости}.\label{5.4.6.7-End}

\stepcounter{SubSecCounter}

\thesubsection.\theSubSecCounter.\label{5.4.6.8} В~случае необходимости
отражения затрат покупателя (как~правило такое происходит в~случаях
проведения~\hyperref[subsec:5.4.9_Valuation_for_investment]{\textsl{оценки} для~инвестиционных целей})
следует использовать обычный \textsl{сравнительный подход} и~привести
в~\emph{отчёте} соответствующий комментарий.\label{5.4.6.8-End}

\stepcounter{SubSecCounter}

\thesubsection.\theSubSecCounter.\label{5.4.6.9} В~случае проведения
\emph{оценки} объекта \textsl{специализированной недвижимости}, основанной
на~предположении о~продолжении текущей операционной деятельности,
в~\emph{отчёте об~оценке} необходимо привести следующую формулировку:

\guillemotleft$\ldots$\textsl{рыночная стоимость} \textsl{(рыночная
арендная плата)} объекта недвижимости как~полностью оснащённого для~ведения
операционной деятельности, основанная на~его коммерческом потенциале,
с~учётом всех согласованных и~(или) \textsl{специальных допущений}
{[}которые должны быть ясно изложены в~\emph{отчёте}{]} составляет
{[}{]}$\ldots$\guillemotright .\label{5.4.6.9-End}\label{subsec:5.4.6_Assessing_the_trading_potential-End}

\subsection{Оценка неиспользуемых объектов недвижимости\label{subsec:5.4.7_Valuation_non_trading_property}}

\stepcounter{SubSecCounter}

\thesubsection.\theSubSecCounter.\label{5.4.7.1} Процесс \textsl{оценки}
неиспользуемой на~\textsl{дату оценки} \textsl{специализированной
недвижимости} аналогичен описанному выше в~подразделе~\ref{subsec:5.4.5_Valuation_fully_equiped}~\nameref{subsec:5.4.5_Valuation_fully_equiped}
\vpageref{subsec:5.4.5_Valuation_fully_equiped}--\pageref{subsec:5.4.5_Valuation_fully_equiped-End},
однако если подобный простой вызван прекращением прежней операционной
деятельности либо тем, что~объект является новым и~не~имеет истории
использования, следует использовать набор других \textsl{допущений}.
Например в~случаях, когда с~неиспользуемого объекта был вывезен
весь необходимый для~его~функционирования инвентарь или~его б\'{о}льшая
часть, либо новый объект ещё~не~был оснащён таким инвентарём, \textsl{оценка}
всё~равно может проводиться на~основе коммерческого потенциала объекта
так, как~будто он~используется по~назначению.\label{5.4.7.1-End}

\stepcounter{SubSecCounter}

\thesubsection.\theSubSecCounter.\label{5.4.7.2} Прекращение операционной
деятельности на~объекте, а~также полный либо частичный вывоз необходимого
для~её~осуществления инвентаря скорее всего окажут влияние на~стоимость
этого объекта. Вследствие и~по~причине этого представляется целесообразным
определить стоимость, основанную как~на~одном либо нескольких \textsl{специальных
допущениях}, так~и~на~существующем положении дел. Подобный подход
часто бывает необходим при~проведении \textsl{оценки} объектов \textsl{специализированной
недвижимости} для~целей залогового кредитования. Разница между описанными
выше стоимостями может, например, отражать влияние фактора затрат
и~времени, необходимых на~оснащение объекта, получение новых лицензий,
наём персонала и~выход на~уровень выручки, соответствующий FMT.\label{5.4.7.2-End}

\stepcounter{SubSecCounter}

\thesubsection.\theSubSecCounter.\label{5.4.7.3} В~случае проведения
\textsl{оценки} неиспользуемого объекта \textsl{специализированной
недвижимости} в~\emph{отчёте об~оценке} необходимо привести следующую
формулировку:

\guillemotleft$\ldots$\textsl{рыночная стоимость} \textsl{(рыночная
арендная плата)} неиспользуемого объекта недвижимости, основанная
на~его коммерческом потенциале, с~учётом всех согласованных \textsl{специальных
допущений} {[}которые должны быть ясно изложены в~\emph{отчёте}{]}
составляет {[}{]}$\ldots$\guillemotright .\label{5.4.7.3-End}\label{subsec:5.4.7_Valuation_non_trading_property-End}

\subsection{Распределение стоимости\label{subsec:5.4.8_Apportionment}}

\stepcounter{SubSecCounter}

\thesubsection.\theSubSecCounter.\label{5.4.8.1} \emph{Оценщику}
может потребоваться, либо его~могут попросить осуществить индикативное
распределение результата \textsl{оценки} для:
\begin{itemize}
\item сравнительного анализа;
\item включения в~\textsl{финансовую отчётность} в~соответствии с~применимыми
стандартами бухгалтерского учёта;
\item целей залогового кредитования;
\item целей налогообложения.\label{5.4.8.1-End}
\end{itemize}
\stepcounter{SubSecCounter}

\thesubsection.\theSubSecCounter.\label{5.4.8.2} Подобное распределение
часто проводится с~целью распределения \textsl{рыночной стоимости}
между:
\begin{itemize}
\item землёй и~зданиями на~основе их~коммерческого потенциала;
\item оборудованием и~иным инвентарём, необходимыми для~осуществления
операционной деятельности.\label{5.4.8.2-End}
\end{itemize}
\stepcounter{SubSecCounter}

\thesubsection.\theSubSecCounter.\label{5.4.8.3} При~рассмотрении
вопроса о~распределении \uline{цены сделки}, особенно если продажа
осуществляется путём передачи акций либо доли в~компании, \emph{оценщику}
следует проявлять осмотрительность, поскольку \uline{она}~может,
помимо перечисленного выше в~п.\ref{5.4.8.2} \vpageref{5.4.8.2}--\pageref{5.4.8.2-End},
отражать стоимость:
\begin{itemize}
\item товарных запасов, расходных материалов и~денежных средств;
\item нематериальных активов;
\item обязательств, таких как~необходимость перечисления заработной платы,
уплаты налогов, погашения кредитов и~т.\,д.\label{5.4.8.3-End}
\end{itemize}
\stepcounter{SubSecCounter}

\thesubsection.\theSubSecCounter.\label{5.4.8.4} Распределение стоимости
в~целях налогообложения осуществляется в~соответствии с~требованиями
применимого законодательства и~не~является предметом данного практического
руководства.\label{5.4.8.4-End}\label{subsec:5.4.8_Apportionment-End}

\subsection{Оценка для~инвестиционных целей\label{subsec:5.4.9_Valuation_for_investment}}

\stepcounter{SubSecCounter}

\thesubsection.\theSubSecCounter.\label{5.4.9.1} Основные принципы
инвестиционной \textsl{оценки} \textsl{специализированной недвижимости}
аналогичны принципам \textsl{оценки} любой другой категории недвижимого
имущества. В~случае проведения такой \textsl{оценки} для~группы
либо портфеля активов следует также применять положения раздела~\ref{sec:5.9_VPGA-9_Identification_of_portfolios}~\nameref{sec:5.9_VPGA-9_Identification_of_portfolios}
\vpageref{sec:5.9_VPGA-9_Identification_of_portfolios}--\pageref{sec:5.9_VPGA-9_Identification_of_portfolios-End}.\label{5.4.9.1-End}

\stepcounter{SubSecCounter}

\thesubsection.\theSubSecCounter.\label{5.4.9.2} В~случае проведения
\textsl{оценки} объектов \textsl{специализированной недвижимости}
для~целей инвестирования в~них необходимо провести расчёт \hyperref[subsubsec:5.4.2.4_FMT]{FMT}
и~\hyperref[subsubsec:5.4.2.3_FMOP]{FMOP} в~порядке, предусмотренном
секцией~\ref{5.4.3.1} \vpageref{5.4.3.1}--\pageref{5.4.3.1-End}.
Для~определения стабильности денежного потока и~потенциала его~роста
также необходимо определить величину \textsl{рыночной арендной платы}
за~объект. Величина арендной платы и~порядок её~пересмотра определяются
условиями действующего либо предполагаемого договора аренды.\label{5.4.9.2-End}

\stepcounter{SubSecCounter}

\thesubsection.\theSubSecCounter.\label{5.4.9.3} \emph{Коэффициент
капитализации}, используемый при~\textsl{оценке} для~целей инвестирования,
отличается от~того, который используется при~\textsl{оценке} незанятых
объектов недвижимости. Норма доходности инвестиций, как~правило,
определяется на~основе данных о~рыночных сделках с~аналогичными
объектами \textsl{специализированной недвижимости}. Очевидно, что~в~связи
с~различиями характеристик объектов \textsl{специализированной недвижимости},
а~также большим разнообразием условий аренды, большое значение имеет
тщательный анализ сопоставимых сделок.\label{5.4.9.3-End}

\stepcounter{SubSecCounter}

\thesubsection.\theSubSecCounter.\label{5.4.9.4} Как~правило, \emph{оценщику}
следует включать в~стоимость \emph{единого объекта недвижимости},
образуемого земельным участком и~зданием~(зданиями), оборудование
и~элементы благоустройства, принадлежащие арендодателю, при~этом
инвентарь, необходимый для~ведения операционной деятельности, чаще
всего находится в~собственности арендатора. В~таких условиях \emph{оценщику}
следует подчеркнуть важность такого инвентаря для~коммерческого потенциала
и~стоимости объекта.\label{5.4.9.4-End}\label{subsec:5.4.9_Valuation_for_investment-End}\label{sec:5.4_VPGA-4_Valuation_of_trade_properties-End}

\newpage

\section{ПР~5. Оценка машин и~оборудования\label{sec:5.5_VPGA-5-Valuation_of_plant_and_equipment}}

\textbf{Данное руководство носит рекомендательный характер и~не~содержит
обязательных требований. Однако там, где~это~уместно, оно~содержит
отсылки к~соответствующему обязательному материалу, содержащемуся
в~других разделах настоящих Всемирных стандартов, а~также \href{https://www.rics.org/globalassets/rics-website/media/upholding-professional-standards/sector-standards/valuation/international-valuation-standards-rics2.pdf}{Международных стандартов оценки}~\cite{IVS-2020},
реализованные в~виде перекрёстных ссылок. Данные ссылки предназначены
для~помощи }\textbf{\textsl{членам RICS}}\textbf{ и~не~меняют статус
данного практического руководства. }\textbf{\textsl{Членам RICS}}\textbf{
необходимо помнить следующее:}
\begin{itemize}
\item \textbf{данное руководство не~может охватить все~возможные варианты,
вследствие чего }\textbf{\emph{оценщикам}}\textbf{ при~формировании
своих }\textbf{\emph{суждений о~стоимости}}\textbf{ всегда следует
учитывать факты и~обстоятельства, имеющие место в~рамках отдельных
заданий по~}\textbf{\textsl{оценке}}\textbf{;}
\item \textbf{следует внимательно относиться к~тому факту, что~в~ряде
юрисдикций могут существовать особые требования, не~предусмотренные
данным руководством.}
\end{itemize}

\subsection{Область применения\label{subsec:5.5.1_Scope}}

\stepcounter{SubSecCounter}

\thesubsection.\theSubSecCounter.\label{5.5.1.1} Данное практическое
руководство содержит дополнительные комментарии по~вопросам \textsl{оценки}
\textsl{машин и~оборудования}, а~также~рекомендации по~практическому
применению \hyperref[sec:10.4_IVS-300_Plant_and_Equipment]{МСО~300.~Оценка машин и~оборудования}~(см.~раздел~\ref{sec:10.4_IVS-300_Plant_and_Equipment}~\nameref{sec:10.4_IVS-300_Plant_and_Equipment}
\vpageref{sec:10.4_IVS-300_Plant_and_Equipment}--\pageref{sec:10.4_IVS-300_Plant_and_Equipment-End}).
В~тексте руководства приводятся перекрёстные ссылки на~обязательные
требования настоящих \emph{Всемирных стандартов}.\label{5.5.1.1-End}\label{subsec:5.5.1_Scope-End}

\subsection{Основная информация\label{subsec:5.5.2_Background}}

\stepcounter{SubSecCounter}

\thesubsection.\theSubSecCounter.\label{5.5.2.1} \textsl{Машины
и~оборудование} образуют значительную часть такого глобального класса
активов как~основные средства, обладая при~этом особыми свойствами,
отличающими их~от~большей части типов недвижимого имущества. Данное
обстоятельство определяет применимые способы классификации, расчёта
стоимости и~описания \textsl{машин и~оборудования} в~\emph{отчёте
об~оценке}. \textsl{Машины и~оборудование} могут как~иметь физическую
связь с~объектами недвижимости полностью либо частично, так~и~быть
пригодными к~перемещению внутри него либо к~полной релокации. Обесценение
некоторых классов \textsl{машин и~оборудования} может происходить
быстрее либо менее равномерно относительно объектов недвижимости вследствие
стремительных изменений технологий в~отдельных секторах экономики.
\textsl{Машины и~оборудование} зачастую оцениваются как~операционная
единица совместно с~другими активами, а~также в~связи с~более
широкой \textsl{оценкой} стоимости коммерческого предприятия. Из~этого
следует, что~хотя природа \textsl{машин и~оборудования} отличается
от~других классов материальных активов, \emph{оценщики} при~подготовке
\emph{отчётов} обязаны обеспечить взаимную связь и~последовательность
в~отношении описания всех классов активов в~той~степени, в~которой
это~возможно. \textsl{Оценка} \textsl{машин и~оборудования} может
проводиться в~т.\,ч.~в~целях лизинга либо залогового обеспечения,
что~потребует дополнительного рассмотрения аспектов \textsl{рыночной
стоимости}, включая концепции \guillemotleft продажи на~месте \guillemotleft как
есть\guillemotright\guillemotright{} и~\guillemotleft продажи после
демонтажа (полного либо частичного)\guillemotright .\label{5.5.2.1-End}

\stepcounter{SubSecCounter}

\thesubsection.\theSubSecCounter.\label{5.5.2.2} В~целом, \textsl{машины
и~оборудование} можно разделить на~три категории.
\begin{itemize}
\item \textbf{Установки}. Данная категория включает в~себя промышленные
активы, функционирование которых тесно связано с~другими элементами.
В~неё~включаются объекты промышленной инфраструктуры, системы жизнеобеспечения
зданий и~поставки коммунальных ресурсов, здания и~сооружения специального
назначения, а~также станки и~оборудование, образующие отдельные
технологические комплексы.
\item \textbf{Машины}. Отдельные единицы настроенной техники, их~группы,
парки, системы (включая автомобильный и~железнодорожный транспорт,
а~также морские и~воздушные суда), используемые по~месту монтажа
либо дистанционно управляемые в~промышленных или~коммерческих процессах,
торговле либо в~связи с~иными потребностями бизнеса. Иными словами,
машина "--- техническое устройство, используемое для~выполнения
определённых процессов.
\item \textbf{Оборудование} "--- всеобъемлющий термин для~обозначения
других активов таких как~прочее оборудование, инструменты, приспособления
и~оснастка, мебель и~предметы интерьера, торговые принадлежности
и~инвентарь, прочие элементы оснащения, ручной инструмент и~расходные
материалы, имеющие вспомогательный характер для~деятельности предприятия
либо имущественного комплекса.\label{5.5.2.2-End}
\end{itemize}
\stepcounter{SubSecCounter}

\thesubsection.\theSubSecCounter.\label{5.5.2.3} Границы между этими
категориями не~всегда могут быть легко определены, а~критерии отнесения
к~одной из~них могут варьироваться в~зависимости от~конкретного
сегмента рынка, на~котором они~используются, \textsl{цели проведения
оценки}, а~также применимых международных и~национальных практик
бухгалтерского учёта. В~частности понятие \guillemotleft личное имущество\guillemotright{}
в~некоторых юрисдикциях используется применительно к~\textsl{машинам
и~оборудованию} равно как~и~к~иным материальным активам, не~являющимся
недвижимостью, тогда как~в~других "--- дополнительно для~описания
предметов искусства и~антиквариата.\label{5.5.2.3-End}

\stepcounter{SubSecCounter}

\thesubsection.\theSubSecCounter.\label{5.5.2.4} Согласно общему
правилу, активы, предназначенные в~первую очередь для~обеспечения
жизнедеятельности зданий и~работающих в~них~людей и~поставки коммунальных
ресурсов, следует оценивать как~часть этих зданий, если в~обычных
условиях они~включаются в~сделку по~отчуждению зданий и~(или)
учитываются на~балансе в~их~составе. Однако отход от~этого правила
практически наверняка будет иметь место в~случаях проведения \textsl{оценки}
для~постановки на~баланс либо для~целей налогообложения. В~таких
ситуациях заказчик может запросить проведение отдельной \textsl{оценки}
для~обособленных объектов инженерной инфраструктуры здания либо взаимосвязанных
единиц оборудования.\label{5.5.2.4-End}

\stepcounter{SubSecCounter}

\thesubsection.\theSubSecCounter.\label{5.5.2.5} При~проведении
\textsl{оценки} для~целей \textsl{финансовой отчётности} группировка
активов на~балансе предприятия, как~правило, может служить некоторым
ориентиром в~вопросе того, какие единицы, относящиеся к~классу \textsl{машин
и~оборудования}, можно оценивать отдельно. Однако, во~многих случаях
\emph{оценщику} следует уточнить у~заказчика и~его~консультантов,
какие единицы следует включать в~периметр \textsl{оценки} \textsl{машин
и~оборудования}. Может потребоваться дополнительное рассмотрение
в~отношении корректировок, затрагивающих такие вопросы как~показатели
рынка, утилизация активов, \textsl{оценка} методами \textsl{доходного
подхода}, учёт износов и~устаревания. Настоятельно рекомендуется
проведение соответствующих консультаций на~этапе заключения \textsl{договора
на~проведение оценки}.\label{5.5.2.5-End}

\stepcounter{SubSecCounter}

\thesubsection.\theSubSecCounter.\label{5.5.2.6} В~случае привлечения
разных \emph{оценщиков} для~проведения \textsl{оценки} бизнеса, недвижимости
и~оборудования необходимо обеспечить их~тесное взаимодействие во~избежание
невключения каких-либо единиц в~периметр \textsl{оценки} либо, наоборот,
их~двойного учёта. Аналогичным образом, взаимодействие между \emph{оценщиками}
бизнеса и~оборудования требуется в~тех случаях, когда \textsl{оценка}
последнего осуществляется \textsl{доходным подходом} либо на~основе
выделения его~стоимости из~стоимости всего имущественного комплекса.\label{5.5.2.6-End}\label{subsec:5.5.2_Background-End}

\subsection{Машины и~оборудование, стоимость которых, как~правило, учитывается
в~составе объекта недвижимости\label{subsec:5.5.3_Property_interest}}

\stepcounter{SubSecCounter}

\thesubsection.\theSubSecCounter.\label{5.5.3.1} Данная категория
включает:
\begin{itemize}
\item инженерную и~коммунальную инфраструктуру здания: элементы систем
газоснабжения, водоснабжения и~водоотведения, электроснабжения, противопожарной
защиты и~безопасности;
\item оборудование, необходимое для~отопления и~горячего водоснабжения,
кондиционирования воздуха, не~являющееся неотъемлемой частью какого-либо
технологического процесса;
\item конструкции и~приспособления, не~являющиеся неотъемлемой частью
технологического оборудования, например, дымовые трубы, корпуса установок
и~рельсовые пути.\label{5.5.3.1-End}
\end{itemize}
\stepcounter{SubSecCounter}

\thesubsection.\theSubSecCounter.\label{5.5.3.2} В~ряде случаев
предметы, обычно оцениваемые в~составе недвижимого имущества, являются
объектами прав \textsl{третьих лиц}, например в~случае наличия соответствующих
финансовых соглашений либо договора лизинга (см.~подраздел~\ref{subsec:5.5.5_Subject_to_finance}~\nameref{subsec:5.5.5_Subject_to_finance}
\vpageref{subsec:5.5.5_Subject_to_finance}--\pageref{subsec:5.5.5_Subject_to_finance-End}).
\emph{Оценщику} следует проявлять особенную осторожность в~подобных
случаях и~обратиться за~консультацией к~заказчику и~его~консультантам
относительно порядка \textsl{оценки} таких активов, который может
варьироваться в~зависимости от~их~правового статуса и~юрисдикции.
В~таких случаях при~проведении \textsl{оценки} и~её~описании в~\emph{отчёте}
может потребоваться введение \textsl{специальных допущений}, которые
должны быть зафиксированы в~письменном виде на~стадии заключения
\textsl{договора на~проведение оценки}.\label{5.5.3.2-End}\label{subsec:5.5.3_Property_interest-End}

\subsection{Машины и~оборудование, оцениваемые отдельно\label{subsec:5.5.4_Separately_valued}}

\stepcounter{SubSecCounter}

\thesubsection.\theSubSecCounter.\label{5.5.4.1} \textsl{Машины
и~оборудование}, оцениваемые отдельно от~недвижимого имущества,
могут быть разделены на~широкие категории. Перечень активов, относящихся
к~\emph{основным средствам}, часто определяется стандартами бухгалтерского
учёта, применимыми на~соответствующей территории либо в~определённой
стране. В~зависимости от~\textsl{цели оценки} может потребоваться
классификация активов по~этим категориям и~отдельная \textsl{оценка}
для~каждой из них.\label{5.5.4.1-End}

\stepcounter{SubSecCounter}

\thesubsection.\theSubSecCounter.\label{5.5.4.2} Ниже приводятся
примеры \emph{основных средств}:
\begin{itemize}
\item технологические либо производственные установки и~оборудование;
\item объекты транспортной инфраструктуры;
\item разведанные запасы полезных ископаемых, шахты и~металлы;
\item автотранспорт, железнодорожный подвижной состав, морские и~воздушные
суда;
\item компьютерное оборудование, оргтехника и~офисная мебель;
\item передвижные установки;
\item общезаводское оборудование;
\item вспомогательное оборудование.\footnote{Прим. пер.: в~ориг. англ. тексте присутствует ещё~один пункт \guillemotleft\foreignlanguage{english}{health,
education and defense}\guillemotright , что, по~мнению переводчика,
является опечаткой и~не~подлежит переводу.}\label{5.5.4.2-End}
\end{itemize}
\stepcounter{SubSecCounter}

\thesubsection.\theSubSecCounter.\label{5.5.4.3} Стоимость ряда
активов, имеющих \guillemotleft пограничный\guillemotright{} статус,
но~всё~же не~относящихся к~основным средствам, определяется \emph{оценщиками}
\textsl{машин и~оборудования}. К~таким активам, например, относятся:
\begin{itemize}
\item компьютерное и~промышленное программное обеспечение, лицензии и~соглашения;
\item запасные части и~расходные материалы;
\item запасы сырья и~материалов;
\item изделия, предназначенные для~производства определённой продукции
(например, пресс-формы, оснастка и~штампы);
\item объекты незавершённого производства.\label{5.5.4.3-End}
\end{itemize}
\stepcounter{SubSecCounter}

\thesubsection.\theSubSecCounter.\label{5.5.4.4} Хотя \textsl{нематериальные
активы} не~подпадают под~определение \textsl{машин и~оборудования},
активы, относящиеся к~данным двум классам, часто используются совместно,
что~может оказать влияние на~их совместную и~(или) индивидуальную
стоимость. В~подобных случаях \emph{оценщику} следует установить
соответствующие \textsl{допущения} до~момента составления \emph{отчёта},
а~желательно "--- на~стадии согласования условий \textsl{договора
на~проведение оценки}. \emph{Оценщикам} следует иметь ввиду, что~определение
\textsl{нематериальных активов} может отличаться в~зависимости от~законодательства,
местной практики и~стандартов бухгалтерского учёта. Особое внимание
следует уделять случаям, когда оцениваемые \textsl{машины и~оборудование}
образуют единое целое (либо тесно связаны) с~\textsl{нематериальными
активами}, основной деятельностью предприятия, лицензиями, программным
обеспечением, согласованиями, денежными потоками, роялти и~иными
видами интеллектуальной собственности. В~таких ситуациях может потребоваться
комбинация \textsl{затратного}, \textsl{доходного} и~\textsl{сравнительного
подходов}.\label{5.5.4.4-End}\label{subsec:5.5.4_Separately_valued-End}

\subsection{Машины и~оборудование, являющиеся объектами договоров финансирования,
лизинга и~залога\label{subsec:5.5.5_Subject_to_finance}}

\stepcounter{SubSecCounter}

\thesubsection.\theSubSecCounter.\label{5.5.5.1} \textsl{Машины
и~оборудование} часто являются объектами договоров лизинга либо иного
финансирования. Подобные соглашения, в~которых актив является предметом
договора либо обеспечением, варьируются от~рядовых договоров рассрочки
либо лизинга до~сложных трансграничных схем финансирования. Следовательно,
\emph{оценщикам} необходимо установить предпосылки \textsl{оценки},
а~также \textsl{специальные допущения} в~момент заключения д\textsl{оговора
на~проведение оценки} либо согласовать и~зафиксировать их~в~письменном
виде в~процессе работы. В~частности, необходимо определить, кто~является
заинтересованной стороной сделки, принимать во~внимание условия договора
лизинга либо иного финансового соглашения и~общие рамки коммерческих
условий, при~этом для~разъяснения подобных моментов \emph{оценщику},
возможно, потребуется обратиться к~другим консультантам.\label{5.5.5.1-End}

\stepcounter{SubSecCounter}

\thesubsection.\theSubSecCounter.\label{5.5.5.2} Национальные и~\href{https://www.ifrs.org/issued-standards/list-of-standards/}{Международные}~\cite{IFRS-all}
стандарты \textsl{финансовой отчётности} и~правила, устанавливаемые
регулятором в~сфере кредитования, касающиеся учёта активов, являющихся
объектами договоров лизинга либо иного финансирования, подлежать регулярному
пересмотру и~изменению. \emph{Оценщикам} следует ясно установить
основу и~объём предполагаемой работы, исходя из~требований таких
стандартов и~правил, с~целью обеспечения соответствия итоговой \textsl{оценки}
конкретным обстоятельствам и~её~практической применимости.\label{5.5.5.2-End}

\stepcounter{SubSecCounter}

\thesubsection.\theSubSecCounter.\label{5.5.5.3} Некоторые недавние
изменения в~правилах бухгалтерского учёта требуют оценки \textsl{машин
и~оборудования} как~элемента более широкого финансового инструмента,
в~т.\,ч.~приведения мнения относительно распределения будущей остаточной
стоимости. Аналогичным образом, учёт резервов на~будущие неплатежи
по~кредитам потребует \textsl{оценки} будущей \textsl{рыночной стоимости}
\textsl{машин и~оборудования} в~качестве элемента общих финансовых
резервов. Из~этого следует, что~от~\emph{оценщика} потребуется
тесное взаимодействие с~заказчиками, аудиторами и~другими специалистами
по~\textsl{оценке}, обладающими дополнительной квалификацией, и,
как минимум, \emph{оценщики} должны чётко формулировать условия \textsl{договора
на~проведение оценки}.\label{5.5.5.3-End}\label{subsec:5.5.5_Subject_to_finance-End}

\subsection{Иные существенные соображения\label{subsec:5.5.6_Material_considerations}}

\stepcounter{SubSecCounter}

\thesubsection.\theSubSecCounter.\label{5.5.6.1} При~определении
\textsl{рыночной стоимости} \textsl{машин и~оборудования} подразделом~\ref{subsec:4.4.3_Market-value}\nameref{subsec:4.4.3_Market-value}\vpageref{subsec:4.4.3_Market-value}\pageref{subsec:4.4.3_Market-value-End}
установлено требование приводить указание на~то, предполагается~ли
продолжение их~эксплуатации по~месту текущего нахождения либо их~демонтаж
с~последующим перемещением (целиком либо поштучно). В~зависимости
от~\emph{цели проведения оценки} также могут потребоваться дополнительные
\textsl{допущения}. 

Примеры:
\begin{itemize}
\item каким образом активы будут предложены на~продажу (например: как~единый
комплекс либо отдельными единицами);
\item предполагаемый способ продажи;
\item вопросы защиты окружающей среды и~экологические ограничения;
\item любые ограничения на~способ продажи (например условиями договора
лизинга исключается возможность продажи через аукцион);
\item кто: продавец или~покупатель несёт расходы на~вывод оборудования
из~эксплуатации и~его демонтаж;
\item учитываются~ли затраты на~монтаж оборудования после его~перемещения
на~новое место, и, если да, кто~их~несёт.\label{5.5.6.1-End}
\end{itemize}
\stepcounter{SubSecCounter}

\thesubsection.\theSubSecCounter.\label{5.5.6.2} В~случае продажи
оборудования отдельно от~объекта недвижимости, в~котором оно~размещено,
возможно существование ограничения времени, доступного для~экспозиции,
продажи и~демонтажа "--- например, в~случае истечения срока аренды
либо наступления более раннего события (такого как~банкротство).
Если \emph{оценщик} полагает, что~такой период времени является заведомо
недостаточным для~надлежащего маркетинга, являющегося концептуальной
основой \textsl{рыночной стоимости}, может потребоваться введения
соответствующего \textsl{специального допущения} в~рамках \emph{отчёта
об~оценке}. Тем~не~менее \emph{оценщик} всегда должен приводить
в~первую очередь значение эталонной \textsl{рыночной стоимости} и~лишь
затем давать коммерческие рекомендации касательно вероятности цены
продажи и~широкого круга обстоятельств. \emph{Оценщику} не~следует
описывать такую ситуацию как~\guillemotleft вынужденная продажа\guillemotright{}
(см.~п.~\ref{4.4.9.5}--\ref{4.4.9.9-End} \vpageref{4.4.9.5}--\pageref{4.4.9.9-End}),
если только такое обозначение не~требуется в~соответствии с~требованиями
законодательства юрисдикции, в~которой составляется \emph{отчёт}.
Хотя от~\emph{оценщиков} часто требуется проведение \textsl{оценок}
для~вынужденной продажи либо ликвидации (см.~п.~\ref{9.4.2.1.6.1}
\vpageref{9.4.2.1.6.1}--\pageref{9.4.2.1.6.1-End}), данные~термины
подлежат широкому толкованию, а~их~значение также варьируется в~зависимости
от~юрисдикции. Следовательно, базовая реализация заключается в~использовании
\textsl{рыночной стоимости}, опирающейся на~обычные строго обоснованные
разумные рыночные \textsl{допущения}, а~также \textsl{специальные
допущения}, оказывающие дополнительное влияние на~её~значение.\label{5.5.6.2-End}

\stepcounter{SubSecCounter}

\thesubsection.\theSubSecCounter.\label{5.5.6.3} В~тех случаях,
когда, по~мнению \emph{оценщика}, на~\textsl{дату оценки} отсутствуют
какие-либо ограничения, но~заказчику требуется консультация в~части
влияния на~стоимость ограниченного срока экспозиции, указание \textsl{рыночной
стоимости} возможно при~условии наличия \textsl{специального допущения},
описывающего конкретное сокращение срока и~его~причины, при~условии,
что~в~\emph{отчёте}, в~первую очередь, приводится значение \textsl{рыночной
стоимости} без~учёта такого ограничения. Это~особенно важно в~ситуациях,
связанных с~залоговым кредитованием, обращением взыскания на~имущество
и~процедурами, осуществляемыми при~банкротстве.\label{5.5.6.3-End}

\stepcounter{SubSecCounter}

\thesubsection.\theSubSecCounter.\label{5.5.6.4} Многие требования,
касающиеся вопросов \textsl{осмотра}, установленные разделом \ref{sec:4.2_VPS2_Inspections_investigations_and_records}~\nameref{sec:4.2_VPS2_Inspections_investigations_and_records}
\vpageref{sec:4.2_VPS2_Inspections_investigations_and_records}--\pageref{sec:4.2_VPS2_Inspections_investigations_and_records-End},
могут быть легко адаптированы для~целей \textsl{оценки} \textsl{машин
и~оборудования}. Для~проведения \textsl{оценки} \emph{оценщику}
сначала необходимо установить такие свойства объекта как тип, спецификация,
производительность мощность и назначение, затем рассмотреть вопросы
возраста, эффективности, состояния, функционального и~экономического
устаревания, а~также установить предполагаемый общий и~оставшийся
срок экономической жизни. При~проведении \textsl{оценки} стоимости
\textsl{машин и~оборудования} на~месте их~текущего расположения,
\emph{оценщику} следует дать конкретное описание методологии и~доводов,
в~т.\,ч. привести разъяснение в~части того, основана~ли их~стоимость
(полностью либо частично) на~потенциале будущих доходов. При~проведении
\textsl{оценки} \textsl{рыночной стоимости}, основанной на~предпосылке
о~том, что~оборудование остаётся на~своём прежнем месте, \emph{оценщику}
также следует приводить конкретные обоснования методов и~доводов.
В~частности следует ответить на~вопросы: включает~ли в~таких обстоятельствах
\textsl{рыночная стоимость} премию за~отсутствие необходимости оплачивать
демонтаж; учтено~ли внешнее~(экономическое) устаревание.\label{5.5.6.4-End}

\stepcounter{SubSecCounter}

\thesubsection.\theSubSecCounter.\label{5.5.6.5} Как~и~в~случае
с~активами других классов и~с~учётом широкого и~сложного спектра
существующих \textsl{машин~и~оборудования}, а~также разнообразия
отраслей, в~которых они~используются, \emph{оценщик}, как~правило,
сталкивается с~нецелесообразностью, а~порой и~невозможностью учёта
влияния каждого фактора, имеющего значение для~стоимости объекта
и~его~\textsl{оценки}. Вследствие и~по~причине этого объём исследований,
выполняемых \emph{оценщиком}, а~также любые \textsl{допущения}, отражаемые
в~\textsl{оценке}, подлежат согласованию с~заказчиком на~этапе
заключения \textsl{договора на~проведение оценки} (в~той~мере,
в~которой эти~аспекты возможно было предвидеть), а~также последующему
включению в~текст \emph{отчёта об~оценке}. Выполнение данного требования
особенно важно в~случаях использования значения \textsl{рыночной
стоимости} для~целей залогового кредитования либо иного финансирования,
а~также для~целей совершения сделки с~оцениваемыми \textsl{машинами
и~оборудованием}.\label{5.5.6.5-End}

\stepcounter{SubSecCounter}

\thesubsection.\theSubSecCounter.\label{5.5.6.6} Аналогичным образом
будут возникать ситуации, при~которых факторы, влияющие на~стоимость
активов других классов (например земли и~зданий), также будут влиять
и~на~стоимость \textsl{машин и~оборудования}. В~качестве примеров
можно привести ситуации, когда недвижимое имущество используется на~основании
краткосрочного договора аренды, если при~этом есть планы по~его
перепланировке либо если имеет место загрязнение земли и~зданий,
что~потребует обеззараживания оборудования перед его~демонтажем
и~перемещением.\label{5.5.6.6-End}\label{subsec:5.5.6_Material_considerations-End}

\subsection{Вопросы регулирования производственной деятельности\label{subsec:5.5.7_Regulatory_measures}}

\stepcounter{SubSecCounter}

\thesubsection.\theSubSecCounter.\label{5.5.7.1} Производственная
деятельность часто регулируется специальным законодательством и~нормативными
актами. Несоблюдение этих законодательных требований может привести
к~приостановке права использования соответствующих \textsl{машин
и~оборудования}. Многие нормативные требования специфичны для~рассматриваемого
оборудования и~технологического процесса, а~также основаны на~более
широких национальных и~местных законодательных и~нормативных актах.
Хотя от~\emph{оценщиков} не~требуется обладание экспертными знаниями
в~вопросах нормативно-технического регулирования, им~всё~же следует
обладать общими знаниями в~данной области в~части, касающейся оцениваемого
типа оборудования. Также ожидается, что~\emph{оценщик} будет в~состоянии
сформулировать и~обсудить аспекты влияния такого регулирования на~\textsl{оценку}.\label{5.5.7.1-End}

\stepcounter{SubSecCounter}

\thesubsection.\theSubSecCounter.\label{5.5.7.2} При~наличии сомнений
относительно соблюдения требований каких-либо нормативных актов, влияющих
на~стоимость \textsl{машин и~оборудования}, \emph{оценщик} должен
обсудить этот вопрос с~заказчиком и~любыми связанными с~ним консультантами
и~сослаться на~результаты такого обсуждения в~\emph{отчёте}. Это~можно
сделать либо путём введения \textsl{допущения} о~соблюдении требований,
либо путём приведения ссылки на~обязательства по~их~соблюдению
со~стороны заказчика, приведённые им~самим либо его~консультантами.\label{5.5.7.2-End}\label{subsec:5.5.7_Regulatory_measures-End}\label{sec:5.5_VPGA-5-Valuation_of_plant_and_equipment-End}

\newpage

\section{ПР~6. Оценка нематериальных активов\label{sec:5.6_VPGA-6_Valuation_of_intangible_assets}}

\textbf{Данное руководство носит рекомендательный характер и~не~содержит
обязательных требований. Однако там, где~это~уместно, оно~содержит
отсылки к~соответствующему обязательному материалу, содержащемуся
в~других разделах настоящих Всемирных стандартов, а~также \href{https://www.rics.org/globalassets/rics-website/media/upholding-professional-standards/sector-standards/valuation/international-valuation-standards-rics2.pdf}{Международных стандартов оценки}~\cite{IVS-2020},
реализованные в~виде перекрёстных ссылок. Данные ссылки предназначены
для~помощи }\textbf{\textsl{членам RICS}}\textbf{ и~не~меняют статус
данного практического руководства. }\textbf{\textsl{Членам RICS}}\textbf{
необходимо помнить следующее:}
\begin{itemize}
\item \textbf{данное руководство не~может охватить все~возможные варианты,
вследствие чего }\textbf{\emph{оценщикам}}\textbf{ при~формировании
своих }\textbf{\emph{суждений о~стоимости}}\textbf{ всегда следует
учитывать факты и~обстоятельства, имеющие место в~рамках отдельных
заданий по~}\textbf{\textsl{оценке}}\textbf{;}
\item \textbf{следует внимательно относиться к~тому факту, что~в~ряде
юрисдикций могут существовать особые требования, не~предусмотренные
данным руководством.}
\end{itemize}

\subsection{Область применения\label{subsec:5.6.1_Scope}}

\stepcounter{SubSecCounter}

\thesubsection.\theSubSecCounter.\label{5.6.1.1} Данное практическое
руководство содержит дополнительные комментарии по~вопросам \textsl{оценки}
прав на~\textsl{нематериальные активы}, а~также~рекомендации по~практическому
применению \hyperref[sec:10.2_IVS-210_Intangible_Assets]{МСО~210. Оценка прав на~нематериальные активы}~(см.~раздел~\ref{sec:10.2_IVS-210_Intangible_Assets}~\nameref{sec:10.2_IVS-210_Intangible_Assets}
\vpageref{sec:10.2_IVS-210_Intangible_Assets}--\pageref{sec:10.2_IVS-210_Intangible_Assets-End}).
В~тексте руководства приводятся перекрёстные ссылки на~обязательные
требования настоящих \emph{Всемирных стандартов}.

Оно~охватывает вопросы \textsl{оценки} прав на~\textsl{НМА} при~их~непосредственной
продаже, а~также при~их~передаче в~рамках приобретения, слияния
и~продажи предприятий или~долей в~них.\label{5.6.1.1-End}\label{subsec:5.6.1_Scope-End}

\subsection{Введение\label{subsec:5.6.2_Introduction}}

\stepcounter{SubSecCounter}

\thesubsection.\theSubSecCounter.\label{5.6.2.1} Нематериальный
актив представляет собой нефинансовый актив, в~основе которого лежит
приносимый им~экономический эффект. Он~не~имеет физической формы,
но~приносит своему владельцу определённые права и~(или) экономические
выгоды. Таким образом, это~актив, который может быть отделён от~хозяйствующего
субъекта либо выделен из~него и~продан, передан в~т.\,ч.~на~основании
лицензионного договора, сдан в~аренду или~передан иным образом отдельно
или~в~связи с~каким-либо активом, обязательством или~договором.
Неидентифицируемые нематериальные \uline{активы}, возникающие из~контрактных
либо иных прав, \uline{которые} могут быть отделимы либо неотделимы
от~предприятия либо иных прав и~обязательств, как~правило обозначаются
термином \guillemotleft гудвил\guillemotright .\label{5.6.2.1-End}

\stepcounter{SubSecCounter}

\thesubsection.\theSubSecCounter.\label{5.6.2.2} Идентифицируемые
\textsl{нематериальные активы} включают:
\begin{itemize}
\item маркетинговые активы: как~правило, в~первую очередь, используются
для~маркетинга и~продвижения товаров и~услуг компании: фирменные
наименования, бренды, товарные знаки и~знаки обслуживания, фирменные
стили, доменные имена в~ИТС \guillemotleft Интернет\guillemotright ,
логотипы в~СМИ, соглашения об~ограничении конкуренции;
\item активы, связанные с~покупателями и~поставщиками: возникают из~отношений
с~ними либо знаний о~них и~используются в~разработке, управлении
закупочной деятельностью, управлении обслуживанием клиентов компании
(база данных клиентов, заказы либо заявки на~производство, договоры
с~клиентами и~связанные с~ними отношения, внедоговорные отношения
с~покупателями);
\item активы, связанные с~творчеством: возникают в~результате творческой
деятельности в~форме продуктов либо услуг, имеющих правовую защиту
в~силу закона либо договора, и~являющихся объектами авторского права,
и~приносят выгоду своим обладателям в~т.\,ч.~за~счёт получения
авторских отчислений (пьесы, оперы, балеты, книги, журналы, газеты,
музыкальные произведения, картины, фотографии, видеозаписи, фильмы,
телевизионные программы);
\item промышленная интеллектуальная собственность: имеет в~своей основе
ценность, образуемую технологическими инновациями и~достижениями,
и~может возникать из~внедоговорных прав на~использование технологии
либо иметь защиту в~силу закона или~договора (защищённые патентами
технологии, программное обеспечение, незапатентованные технологии,
базы данных, коммерческие тайны, текущие исследования и~разработки,
производственные процессы и~ноу-хау).\label{5.6.2.2-End}
\end{itemize}
\stepcounter{SubSecCounter}

\thesubsection.\theSubSecCounter.\label{5.6.2.3} \textsl{Нематериальные
активы} могут носить как~договорной, так~и~недоговорной характер.
Активы, возникающие в~силу договора, представляют собой ценность,
образуемую правом, проистекающим из~договора (лицензионные соглашения;
соглашения о~выплате роялти; соглашения о~приостановке деятельности;
договоры на~рекламу, строительство, управление, оказание услуг или~поставки;
договоры аренды; разрешения на~строительство; договоры франшизы;
права на~трансляцию и~вещание; права на~использование чего-либо
за~исключением случаев, когда они~прямо относятся к~материальным
активам либо рассматриваются в~качестве таковых; договоры на~обслуживание;
трудовые договоры).\label{5.6.2.3-End}

\stepcounter{SubSecCounter}

\thesubsection.\theSubSecCounter.\label{5.6.2.4} Основным \textsl{нематериальным
активом} является \textsl{гудвил}, определяемый как~любая будущая
экономическая выгода, возникающая вследствие владения бизнесом или~долей
в~нём либо группой активов, и~неотделимая от~них. Выгоды, которые
могут являться частью \textsl{гудвила}, включают синергетический эффект,
возникающий вследствие объединения предприятий и~зависят от~специфики
этих предприятий. В~качестве примеров можно привести:
\begin{itemize}
\item экономия, возникающая вследствие эффекта масштаба, не~отражённая
в~стоимости других активов;
\item возможности роста такие как, например, освоение новых рынков;
\item организационный капитал, например, преимущества, возникающие вследствие
объединения.
\end{itemize}
Иногда \textsl{гудвил} определяют как~величину стоимости, остающейся
после вычета стоимостей всех отделимых и~идентифицируемых активов
из~общей стоимости бизнеса. Данное определение часто используется
в~бухгалтерских целях.\label{5.6.2.4-End}

\stepcounter{SubSecCounter}

\thesubsection.\theSubSecCounter.\label{5.6.2.5} \textsl{Нематериальные
активы} отличаются друг от~друга по~таким характеристикам как~принадлежность,
назначение, положение на~рынке и~имидж. Например, бренды модной
женской обуви могут характеризоваться использованием определённых
цветов и~стилей, а~также ценой. Кроме того, хотя \textsl{нематериальные
активы}, относящиеся к~одному классу, неизбежно будут иметь схожие
характеристики, также будут существовать и~аспекты, отличающие их~от~других
подобных активов.\label{5.6.2.5-End}

\stepcounter{SubSecCounter}

\thesubsection.\theSubSecCounter.\label{5.6.2.6} Необходимо, чтобы
\emph{оценщик} регулярно практиковался в~области \textsl{оценки}
\textsl{нематериальных активов}, поскольку обладание практическим
пониманием факторов, оказывающих влияние на~конкретный актив, является
крайне важным (см.~подраздел~\ref{subsec:3.2.2_Member qualification}~\nameref{subsec:3.2.2_Member qualification}
\vpageref{subsec:3.2.2_Member qualification}--\pageref{subsec:3.2.2_Member qualification-End}).\label{5.6.2.6-End}\label{subsec:5.6.2_Introduction-End}

\subsection{Условия договора на~проведение оценки\label{subsec:5.6.3_Terms_of_engagement}}

\stepcounter{SubSecCounter}

\thesubsection.\theSubSecCounter.\label{5.6.3.1} Познания заказчиков
в~области \textsl{оценки} могут существенно варьироваться. Некоторые
из~них могут иметь экспертные познания в~области прав на~\textsl{нематериальные
активы} и~их~\textsl{оценки}, другие~же могут даже не~понимать
значение терминов, используемых \emph{оценщиками} \textsl{нематериальных
активов}.\label{5.6.3.1-End}

\stepcounter{SubSecCounter}

\thesubsection.\theSubSecCounter.\label{5.6.3.2} Необходимо, чтобы
условия д\textsl{оговора на~проведение оценки} были понятны заказчику
и~согласованы между ним~и~\emph{оценщиком} до~начала выполнения
задания. Необходимо идентифицировать все~дополнительные и~вспомогательные
активы и~придти к~соглашению о~том, включать~ли их~в~периметр
\textsl{оценки} либо нет. Вспомогательные активы используются совместно
с~оцениваемым активом с~целью генерации денежных потоков. В~случае,
если вспомогательные активы исключаются из~периметра \textsl{оценки},
следует уточнить, предполагается~ли \textsl{оценка} основного актива
как~обособленной единицы.\label{5.6.3.2-End}

\stepcounter{SubSecCounter}

\thesubsection.\theSubSecCounter.\label{5.6.3.3} Возможны ситуации,
когда право на~актив является долевым и~принадлежит нескольким лицам.
В~таких случаях в~отчёте обязательно следует отразить данный факт.\label{5.6.3.3-End}

\stepcounter{SubSecCounter}

\thesubsection.\theSubSecCounter.\label{5.6.3.4} \emph{Оценщики}
могут стремиться разработать типовые формы заявок на~проведение \textsl{оценки},
которые будут являться универсальными для~любых видов работ. В~тех~случаях,
когда \emph{оценка} должна быть выполнена в~соответствии с~требованиями
настоящих \emph{Всемирных стандартов}, условия \textsl{договора на~проведение
оценки} должны соответствовать минимальным требованиями, установленным
подразделом~\ref{subsec:3.2.7_Terms_of_engagement_Scope_of_work}~\nameref{subsec:3.2.7_Terms_of_engagement_Scope_of_work},
а~также разделом~\ref{sec:4.1_VPS1_Terms_of_engagement_Scope_of_work}~\nameref{sec:4.1_VPS1_Terms_of_engagement_Scope_of_work}
\vpageref{subsec:3.2.7_Terms_of_engagement_Scope_of_work}--\pageref{subsec:3.2.7_Terms_of_engagement_Scope_of_work-End}
и~\pageref{sec:4.1_VPS1_Terms_of_engagement_Scope_of_work}--\pageref{sec:4.1_VPS1_Terms_of_engagement_Scope_of_work-End}
соответственно\label{5.6.3.4-End}\label{subsec:5.6.3_Terms_of_engagement-End}

\subsection{Концептуальная основа оценки\label{subsec:5.6.4_Valuation_concepts}}

\stepcounter{SubSecCounter}

\thesubsection.\theSubSecCounter.\label{5.6.4.1} С~учётом разнообразия
возможных \emph{целей проведения оценки} \textsl{нематериальных активов}
необходимо понимать цель привлечения \emph{оценщика} к~конкретной
работе. Примерами \emph{целей проведения оценки} являются: составление
\textsl{финансовой отчётности}, вопросы налогообложения, государственные
заказы, сделки с~акциями и~их размещение, формирование мнения об~обоснованности
ценовых параметров сделок, банковские соглашения, процедуры, осуществляемые
при~банкротстве и~конкурсном управлении, управление знаниями, мониторинг
портфеля активов. От~\emph{цели проведения оценки} зависят применимые
концепции и~методики \textsl{оценки}, некоторые из~которых являются
объектом регулирования со~стороны статутного либо прецедентного права,\footnote{Прим.~пер.: указание на~эти~два вида источника права является применимым
только для~территорий, на~которых действует \href{https://ru.wikipedia.org/wiki/\%D0\%90\%D0\%BD\%D0\%B3\%D0\%BB\%D0\%BE\%D1\%81\%D0\%B0\%D0\%BA\%D1\%81\%D0\%BE\%D0\%BD\%D1\%81\%D0\%BA\%D0\%B0\%D1\%8F_\%D0\%BF\%D1\%80\%D0\%B0\%D0\%B2\%D0\%BE\%D0\%B2\%D0\%B0\%D1\%8F_\%D1\%81\%D0\%B5\%D0\%BC\%D1\%8C\%D1\%8F}{англо-американское право}~\cite{Wiki:anglo-american_law},
для~территорий, на~которых действует \href{https://ru.wikipedia.org/wiki/\%D0\%A0\%D0\%BE\%D0\%BC\%D0\%B0\%D0\%BD\%D0\%BE-\%D0\%B3\%D0\%B5\%D1\%80\%D0\%BC\%D0\%B0\%D0\%BD\%D1\%81\%D0\%BA\%D0\%B0\%D1\%8F_\%D0\%BF\%D1\%80\%D0\%B0\%D0\%B2\%D0\%BE\%D0\%B2\%D0\%B0\%D1\%8F_\%D1\%81\%D0\%B5\%D0\%BC\%D1\%8C\%D1\%8F}{романо-германское}~\cite{Wiki:roman-german_law}
либо, например, \href{https://ru.wikipedia.org/wiki/\%D0\%A1\%D0\%BA\%D0\%B0\%D0\%BD\%D0\%B4\%D0\%B8\%D0\%BD\%D0\%B0\%D0\%B2\%D1\%81\%D0\%BA\%D0\%B0\%D1\%8F_\%D0\%BF\%D1\%80\%D0\%B0\%D0\%B2\%D0\%BE\%D0\%B2\%D0\%B0\%D1\%8F_\%D1\%81\%D0\%B8\%D1\%81\%D1\%82\%D0\%B5\%D0\%BC\%D0\%B0}{скандинавское право}~\cite{Wiki:scandinavian_law}\cite{Wiki:scandinavian_law},
подобное разделение не~соответствует реалиям правовой системы.}тогда как~другие "--- международных и~национальных стандартов в~области
\textsl{оценки}.\label{5.6.4.1-End}

\stepcounter{SubSecCounter}

\thesubsection.\theSubSecCounter.\label{5.6.4.2} Как~правило, при~проведении
подобных \textsl{оценок} используются следующие \textsl{виды стоимости}
(не~все~из~которых признаются \href{https://www.rics.org/globalassets/rics-website/media/upholding-professional-standards/sector-standards/valuation/international-valuation-standards-rics2.pdf}{МСО}~\cite{IVS-2020}
и~настоящими \emph{Всемирными стандартами}): \textsl{справедливая
стоимость}, \emph{справедливая рыночная стоимость}, \textsl{рыночная
стоимость}, \emph{стоимость на~открытом рынке}. При~проведении письменной
\textsl{оценки} \emph{оценщикам} следует придерживаться требований,
установленных подразделом~\ref{subsec:3.1.1_Mandatory_application}~\nameref{subsec:3.1.1_Mandatory_application}
\vpageref{subsec:3.1.1_Mandatory_application}--\pageref{subsec:3.1.1_Mandatory_application-End}.\label{5.6.4.2-End}

\stepcounter{SubSecCounter}

\thesubsection.\theSubSecCounter.\label{5.6.4.3} В~зависимости
от~правил и~практики, присущих~применяемой концепции, выводы относительно
стоимости одного и~того~же актива могут отличаться. Например, вследствие
правил проведения \textsl{оценки} для~целей налогообложения, налоговый
орган может рассматривать оценку иначе, чем~это~делали~бы сторона
судебного процесса, участник сделки по~слиянию или~\textsl{специальный
покупатель}.\label{5.6.4.3-End}

\stepcounter{SubSecCounter}

\thesubsection.\theSubSecCounter.\label{5.6.4.4} За~исключением
случаев, когда имеются надёжные доказательства обратного, презюмируется
непрерывность осуществления деятельности, а~также то, что~в~обозримом
будущем течение срока экономической жизни будет продолжаться. В~некоторых
случаях этот срок будет основываться на~законе или~прямо устанавливаться
им, в~других "--- условиями соглашений и~правил, устанавливающих
правила пользования активом. Однако, при~проведении \textsl{оценки}
для~целей \textsl{финансовой отчётности} может потребоваться рассмотреть
вопрос стоимости актива, подлежащего списанию либо ликвидации.\label{5.6.4.4-End}

\stepcounter{SubSecCounter}

\thesubsection.\theSubSecCounter.\label{5.6.4.5} Во~многих ситуациях
может потребоваться применение более чем~одного \emph{метода оценки},
в~особенности это~касается тех~случаев, когда вследствие недостаточности
информации и~данных, использование только одного \emph{метода} не~обеспечивает
достаточную доказательную силу \textsl{оценки}. В~таких ситуациях
\emph{оценщик} может использовать дополнительные \emph{методы} для~получения
итогового \emph{суждения о~стоимости}, приводя обоснование предпочтений,
отдаваемых какой-либо одной либо~нескольким методикам. \emph{Оценщику}
следует рассмотреть возможность применения каждого из~\emph{подходов
к~оценке}, указав причины, по~которым тот~или~иной из~них~не~был
реализован.\label{5.6.4.5-End}\label{subsec:5.6.4_Valuation_concepts-End}

\subsection{Оценочный аудит\label{subsec:5.6.5_Valuation_due_diligence}}

\stepcounter{SubSecCounter}

\thesubsection.\theSubSecCounter.\label{5.6.5.1} В~соответствии
с~требованиями подраздела~\ref{subsec:3.2.2_Member qualification}~\nameref{subsec:3.2.2_Member qualification}
\vpageref{subsec:3.2.2_Member qualification}--\pageref{subsec:3.2.2_Member qualification-End},
\emph{оценщики} должны обладать необходимой компетенцией в~вопросах
\textsl{оценки} \textsl{нематериальных активов}. Как~минимум, \emph{оценщик}
не~должен приступать к~проведению \textsl{оценки} при~отсутствии
детального знания и~понимания таких вопросов как:
\begin{itemize}
\item существующие права на~актив~(активы);
\item история актива~(активов) и~связанная с~ним~(ними) деятельность;
\item состояние отрасли, в~которой используется \emph{объект оценки}, общие
экономические перспективы и~политические факторы "--- в~той~степени,
в~которой это~необходимо.\label{5.6.5.1-End}
\end{itemize}
\stepcounter{SubSecCounter}

\thesubsection.\theSubSecCounter.\label{5.6.5.2} Типичный перечень
данных, необходимых для~понимания \emph{оценщиком} \emph{объекта
оценки}, может включать:
\begin{itemize}
\item наиболее свежие отчёты о~доходах, связанных с~\emph{объектом оценки},
а~также детали текущих и~прежних прогнозов и~планов;
\item описание и~история \emph{объекта оценки}, включая вопросы его~юридической
защиты и~существующих прав на~него (необходимо раскрыть степень
глубины анализа этих прав);
\item сведения об~активе и~сопроводительной документации к~нему (например,
свидетельствах о~регистрации, заявках на~право использования на~определённой
территории, маркетинговой, технической, исследовательской и~конструкторской
документации, чертежах и~руководствах);
\item соглашения о~залоге;
\item подробные сведения о~конкретных видах деятельности, в~рамках которых
используется \textsl{нематериальный актив};
\item прежние \emph{отчёты об~оценке};
\item сведения о~продукции, выпускаемой, поддерживаемой либо развиваемой
бизнесом посредством использования оцениваемых \textsl{нематериальных
активов};
\item сведения о~наличии разрешения~(разрешений) на~использование \textsl{нематериальных
активов} \emph{третьими лицами} и~планах по~предоставлению таких
разрешений.
\item сведения о~рынке~(рынках), на~котором~(которых) компания осуществляет
деятельность, и~конкурентной среде, барьерах для~входа на~эти~рынки,
деловых и~маркетинговых планах, комплексная экспертиза;
\item сведения о~лицензировании, стратегических альянсах и~детали участия
в~совместных предприятиях;
\item сведения о~том, могут~ли права на~\textsl{нематериальный актив}
быть переданы либо переуступлены, в~т.\,ч.~в~рамках соглашения
о~роялти;
\item сведения об~основных поставщиках и~покупателях;
\item цели, перспективы развития и~тенденции, ожидаемые в~отрасли, а~также
их~возможное влияние на~компанию либо актив;
\item учётная политика;
\item анализ сильных и~слабых сторон, возможностей и~угроз (\href{https://ru.wikipedia.org/wiki/SWOT-\%D0\%B0\%D0\%BD\%D0\%B0\%D0\%BB\%D0\%B8\%D0\%B7}{SWOT-анализ})~\cite{Wiki:SWOT-rus};
\item ключевые факторы положения на~рынке (например, сведения о~монопольном
либо доминирующем положении на~рынке, занимаемой на~нём доле);
\item перспективы крупных капитальных затрат;
\item позиции конкурентов;
\item сезонные и~циклические тенденции;
\item технологические изменения. оказывающие влияние на~актив;
\item уязвимости связанные с~поставками сырья либо риски неисполнения договоров
со~стороны поставщиков;
\item сведения о~том имели~ли место сделки по~слиянию и~поглощению в~отрасли
незадолго до~\textsl{даты оценки}, а~также об~использованных в~них~критериях
принятия соответствующих решений;
\item сведения об~исследованиях и~разработках (в~т.\,ч.~о~соглашениях
о~неразглашении, субподрядчиках, обучении и~поощрении персонала);
\item сведения о~том, имеется~ли перечень всей интеллектуальной собственности,
содержащий в~т.\,ч.~данные о~степени прав на~неё, а~также правах
третьих лиц (при~наличии таковых);
\item результаты исследования рынка лицензирования сопоставимых активов.\label{5.6.5.2-End}
\end{itemize}
\stepcounter{SubSecCounter}

\thesubsection.\theSubSecCounter.\label{5.6.5.3} \emph{Оценщику}
следует, в~той~степени, в~которой это~возможно, проверять факты
и~сведения, использованные при~определении стоимости, и, по~возможности,
сопоставлять исходные данные с~эталонными показателями отрасли.\label{5.6.5.3-End}

\stepcounter{SubSecCounter}

\thesubsection.\theSubSecCounter.\label{5.6.5.4} Б\'{о}льшая часть
исходных данных, на~основе которых \emph{оценщик} сформирует своё
\emph{суждение}, будет предоставлена заказчиком, а~их~проверка чаще
всего окажется невозможной. В~таких случаях данный факт следует ясно
изложить в~\emph{отчёте}.\label{5.6.5.4-End}\label{subsec:5.6.5_Valuation_due_diligence-End}

\subsection{Подходы к~оценке\label{subsec:5.6.6_Valuation_approaches}}

\subsubsection{Общие сведения\label{subsubsec:5.6.6.1_Common}}

\stepcounter{SubSubSecCounter}

\thesubsubsection.\theSubSubSecCounter.\label{5.6.6.1.1} Теория
\textsl{оценки}, в~широком смысле, признаёт три~\emph{подхода к~оценке},
в~т.\,ч.~применительно к~\textsl{нематериальным активам}. Такими
подходами являются \textsl{сравнительный~(рыночный) подход}, также
иногда называемый \textsl{подход прямого рыночной сравнения}, \textsl{доходный
подход}, \textsl{затратный подход}.\label{5.6.6.1.1-End}

\stepcounter{SubSubSecCounter}

\thesubsubsection.\theSubSubSecCounter.\label{5.6.6.1.2} Независимо
от~применяемого \textsl{подхода} \emph{оценщику} необходимо рассчитать
оставшийся срок экономической жизни (полезного использования). Он~может
представлять собой как~конечный отрезок времени, установленный условиями
договора либо практикой отрасли, так~и~носить бессрочный характер.
При~определении срока экономической жизни необходимо учитывать ряд~факторов,
включая юридические, технические, экономические, а~также условия
использования. Предполагаемый срок экономической жизни актива, лицензия
на~который выдана на~определённый срок, может оказаться короче срока
лицензии, если до~окончания её~действия рынке появится более совершенный
конкурирующий продукт. В~таких случаях, \emph{оценщику} необходимо
привести собственные суждения на~сей~счёт.\label{5.6.6.1.2-End}

\subsubsection{Сравнительный подход\label{subsubsec:5.6.6.2_Market_approach}}

\stepcounter{SubSubSecCounter}

\thesubsubsection.\theSubSubSecCounter.\label{5.6.6.2.1} Реализация
\textsl{сравнительного подхода} заключается в~\textsl{оценке} стоимости
актива путём анализа предложений либо недавних продаж по~схожим активам
на~основе актуальных рыночных данных. Однако на~практике редко удаётся
найти данные по~идентичным активам.\label{5.6.6.2.1-End}

\stepcounter{SubSubSecCounter}

\thesubsubsection.\theSubSubSecCounter.\label{5.6.6.2.2} Двумя основными
\emph{методами} \textsl{сравнительного подхода} при~\textsl{оценке}
\textsl{нематериальных активов} являются \guillemotleft метод рыночных
мультипликаторов\guillemotright{} и~\guillemotleft метод сопоставимых
сделок\guillemotright .\label{5.6.6.2.2-End}

\stepcounter{SubSubSecCounter}

\thesubsubsection.\theSubSubSecCounter.\label{5.6.6.2.3} \emph{Метод
рыночных мультипликаторов} основан на~сопоставлении \emph{объекта
оценки} с~обобщёнными рыночными данными такими как, например, отраслевые
ставки роялти. При~применении данного \emph{метода} \uline{ставки
роялти} оцениваются и~корректируются на~основе сильных и~слабых
сторон \emph{объекта оценки} относительно аналогичных активов. Затем
\uline{они}~применяются к~соответствующим операционным данным
оцениваемого актива для~определения его~стоимости. В~полученные
данные, как~правило, вносятся соответствующие корректировки, отражающие
различия в~свойствах или~характеристиках.\label{5.6.6.2.3-End}

\stepcounter{SubSubSecCounter}

\thesubsubsection.\theSubSubSecCounter.\label{5.6.6.2.4} \emph{Метод
сопоставимых сделок} использует данные об~исторических сделках, которые
имели место в~отрасли, в~которой используется \emph{объект оценки},
либо в~смежной. Затем, для~получения представления о~стоимости
объекта, полученные данные корректируются и~применяются к~его~соответствующим
операционным показателям.\label{5.6.6.2.4-End}

\stepcounter{SubSubSecCounter}

\thesubsubsection.\theSubSubSecCounter.\label{5.6.6.2.5} В~некоторых
отраслях покупка и~продажа активов осуществляется на~основе сложившейся
рыночной практики или~эмпирических правил, часто (хотя и~не~исключительно)
основанных на~данных о~выручке либо процентах от~неё, и~не~связанных
напрямую с~величиной получаемой прибыли. Необходимо убедиться в~том,
что~с~течением времени \guillemotleft сложившаяся рыночная практика\guillemotright{}
не~изменилась под~воздействием новых обстоятельств. В~случае существования
таких эмпирических правил \emph{оценщику}, возможно, потребуется учесть
их.\label{5.6.6.2.5-End}\label{subsubsec:5.6.6.2_Market_approach-End}

\subsubsection{Доходный подход\label{subsec:5.6.6.3_Income_approach}}

\stepcounter{SubSubSecCounter}

\thesubsubsection.\theSubSubSecCounter.\label{5.6.6.3.1} \textsl{Доходный
подход} также имеет несколько вариантов реализации. При~его~использовании
(например в~варианте \emph{метода дисконтированных денежных потоков})
стоимость актива равна текущей стоимости будущих выгод от~владения
этим активом. Эти~выгоды могут включать доходы от~его~использования,
снижение затрат, возникновение права на~налоговые вычеты и~выручку
от~его~реализации\label{5.6.6.3.1-End}.

\stepcounter{SubSubSecCounter}

\thesubsubsection.\theSubSubSecCounter.\label{5.6.6.3.2} При~\textsl{оценке}
\textsl{нематериальных активов} величина их~стоимости определяется
путём дисконтирования ожидаемых денежных потоков и~приведения их~к~текущей
стоимости с~использованием ставки доходности, которая учитывает безрисковую
ставку за~использование средств, ожидаемый уровень инфляции, а~также~риски,
связанные с~конкретными инвестициями. Выбранная ставка дисконтирования
обычно основывается на~существующих на~\textsl{дату оценки}~ставках
доходности альтернативных инвестиций аналогичного типа и~качества.\label{5.6.6.3.2-End}

\stepcounter{SubSubSecCounter}

\thesubsubsection.\theSubSubSecCounter. \label{5.6.6.3.3} \textsl{Доходный
подход} также может быть реализован в~форме \emph{метода освобождения
от~роялти}, описанного в~разделе~\ref{sec:10.2_IVS-210_Intangible_Assets}~\nameref{sec:10.2_IVS-210_Intangible_Assets}
\vpageref{sec:10.2_IVS-210_Intangible_Assets}--\pageref{sec:10.2_IVS-210_Intangible_Assets-End}.
Данный метод предполагает определение стоимости \textsl{нематериального
актива} на~основе текущей стоимости гипотетических платежей роялти,
которые были~бы сэкономлены в~результате владения активом относительно
варианта его~использования на~основе лицензии, полученной у~третьей
стороны.\label{5.6.6.3.3-End}

\stepcounter{SubSubSecCounter}

\thesubsubsection.\theSubSubSecCounter.\label{5.6.6.3.4} Следующим
методом является \emph{метод дополнительной прибыли}, основанный на~определении
текущей стоимости выгоды владения активом на~протяжении нескольких
периодов путём расчёта денежных потоков, возникающих вследствие использования
актива, и~вычитания периодического платежа, отражающего справедливую
прибыль за~использование внесённых активов.\label{5.6.6.3.4-End}

\stepcounter{SubSubSecCounter}

\thesubsubsection.\theSubSubSecCounter.\label{5.6.6.3.5} Также распространённой
реализацией \textsl{доходного подхода} при~оценке \textsl{нематериальных
активов} является \emph{метод капитализации прибыли}, применимый для~соответствующего
\textsl{вида стоимости}. Во~всех случаях требуется глубокое понимание
бухгалтерской и~экономической прибыли, учёт её~исторических значений
и~прогнозирование будущих.\label{5.6.6.3.5-End}

\stepcounter{SubSubSecCounter}

\thesubsubsection.\theSubSubSecCounter.\label{5.6.6.3.6} \textsl{Оценка
}прав на~ \textsl{нематериальне активы} и~интеллектуальную собственность
также включает методы определения доходов, непосредственно связанных
с~оцениваемым активом, такие как~\emph{метод разницы в~валовой
прибыли}, \emph{метод дополнительной прибыли} и~\emph{метод освобождения
от роялти}. При~этом требуется глубокое знание исторических и~прогнозных
значений выручки и~прибыли.\label{5.6.6.3.6-End}\label{subsec:5.6.6.3_Income_approach-End}

\subsubsection{Затратный подход\label{subsubsec:5.6.6.4_Cost_approach}}

\stepcounter{SubSubSecCounter}

\thesubsubsection.\theSubSubSecCounter.\label{5.6.6.4.1} \textsl{Затратный
подход} предполагает определение стоимости актива на~основе затрат
на~его~создание либо замену другим аналогичным активом. При~\textsl{оценке}
\textsl{нематериальных активов} следует учитывать устаревание, затраты
на~поддержание полезных свойств и~фактор стоимости денег во~времени.
При~определении \textsl{рыночной стоимости} показатели износа необходимо
подтверждать данными открытого рынка.\label{5.6.6.4.1-End}\label{subsubsec:5.6.6.4_Cost_approach-End}\label{subsec:5.6.6_Valuation_approaches-End}

\subsection{Методы приведённой стоимости\label{subsec:5.6.7_Present_value_techniques}}

\stepcounter{SubSecCounter}

\thesubsection.\theSubSecCounter.\label{5.6.7.1} \emph{Методы приведённой
стоимости} (PVT) предполагают определение стоимости актива на~основе
приведённой стоимости будущих денежных потоков, представляющих собой
собой денежные средства, генерируемые в~течение определенного периода
времени активом, группой активов либо~бизнес-единицей.\label{5.6.7.1-End}

\stepcounter{SubSecCounter}

\thesubsection.\theSubSecCounter.\label{5.6.7.2} При~применении
методов данной группы необходимо рассмотреть следующие вопросы:
\begin{itemize}
\item число лет, в~течение которых будет существовать денежный поток;
\item коэффициент капитализации либо ставка дисконтирования применяются
на~конец периода;
\item величина принятой ставки (ставок) дисконтирования;
\item учтена~ли инфляция в~денежных потоках;
\item какие другие переменные необходимо учитывать в~отношении денежных
потоков в~будущем;
\item коммерческое предназначение актива;
\item начальная и~текущая доходность, внутренняя норма доходности (IRR)
и~терминальная стоимость.\label{5.6.7.2-End}
\end{itemize}
\stepcounter{SubSecCounter}

\thesubsection.\theSubSecCounter.\label{5.6.7.3} При~применении
\emph{методов приведённой стоимости} важно, чтобы сопоставимые рыночные
сделки, отражающие тот~же \textsl{подход к~оценке}, принимались
во~внимание. При~этом следует принимать во~внимание, что~получение
применимых подробных данных о~сделках скорее всего будет затруднено.
Однако сведения о~них помогут определить применимую ставку дисконтирования,
требуемую величину внутренней нормы доходности (IRR) и~общие модели
принятия решений, принятые на~рынке.\label{5.6.7.3-End}

\stepcounter{SubSecCounter}

\thesubsection.\theSubSecCounter.\label{5.6.7.4} В~случае проведения
\textsl{оценки} конкретного \textsl{нематериального актива} \emph{оценщик}
перед проведением детального моделирования денежных потоков должен
определить количественные значения оставшегося срока полезного использования
и~темпов износа, связанных именно с~использованием актива. Как правило,
при анализе оставшегося срока экономической жизни используется минимальное
значение из~нижеследующего перечня:
\begin{itemize}
\item физический срок службы (например, базового материального актива);
\item срок службы, при~котором использование актива (например, базового
материального актива) является целесообразным;
\item технологический срок жизни самого \textsl{нематериального актива};
\item срок, в~течение которого использование самого \textsl{нематериального
актива} является экономические целесообразным;
\item срок юридической жизни.\label{5.6.7.4-End}
\end{itemize}
\stepcounter{SubSecCounter}

\thesubsection.\theSubSecCounter.\label{5.6.7.5} Таким образом,
\textsl{оценка} на~основе \emph{моделей приведённой стоимости} включает
в~себя такие ключевые компоненты как~идентификация конкретной интеллектуальной
собственности и~формирование финансового прогноза доходов от~её~использования,
а~также определение ставки дисконтирования (стоимости капитала).
При~этом необходимо учитывать систематический и~несистематический
риски, а~определение ставки дисконтирования в~её~базовом применении
требует идентификацию и~применение стоимости капитала к~существующим
и~прогнозируемым денежным потокам.\label{5.6.7.5-End}

\stepcounter{SubSecCounter}

\thesubsection.\theSubSecCounter.\label{5.6.7.6} Дисконтирование
следует осуществлять на~основе \href{https://www.audit-it.ru/articles/finance/a106/648167.html}{средневзвешенной стоимости капитала}~(\href{https://ru.wikipedia.org/wiki/\%D0\%A1\%D1\%80\%D0\%B5\%D0\%B4\%D0\%BD\%D0\%B5\%D0\%B2\%D0\%B7\%D0\%B2\%D0\%B5\%D1\%88\%D0\%B5\%D0\%BD\%D0\%BD\%D0\%B0\%D1\%8F_\%D1\%81\%D1\%82\%D0\%BE\%D0\%B8\%D0\%BC\%D0\%BE\%D1\%81\%D1\%82\%D1\%8C_\%D0\%BA\%D0\%B0\%D0\%BF\%D0\%B8\%D1\%82\%D0\%B0\%D0\%BB\%D0\%B0}{WACC})~\cite{WACC_rus,Wiki:WACC_rus}.
Двумя основными составляющими стоимости капитала являются \href{https://allfi.biz/financialmanagement/CostOfCapital/stoimost-zaemnogo-kapitala.php}{стоимость заёмного капитала}
и~\href{https://www.cfin.ru/appraisal/business/special/Emerging_Markets.shtml}{стоимость собственного капитала}~\cite{Price_of_debt_rus,Price_of_equity_rus}.
Для~расчёта соответствующей нормы возврата и~ставки дисконтирования
\emph{оценщик} использует ряд различных методик, включая \href{https://allfi.biz/financialmanagement/RiskAndReturns/model-ocenki-kapitalnyh-aktivov-capm.php}{модель ценообразования капитальных активов}~(\href{https://allfi.biz/financialmanagement/RiskAndReturns/model-ocenki-kapitalnyh-aktivov-capm.php}{CAPM})~\cite{CAPM_rus},
\href{https://allfi.biz/financialmanagement/RiskAndReturns/teorija-arbitrazhnogo-cenoobrazovanija.php}{теорию арбитражного ценообразования}~\cite{Teor_arb_tsen_rus},
а~также гибридные варианты в~зависимости от~конкретных обстоятельств.\label{5.6.7.6-End}

\stepcounter{SubSecCounter}

\thesubsection.\theSubSecCounter.\label{5.6.7.7} У~\emph{оценщиков}
может возникнуть потребность рассмотреть \textsl{нематериальные активы}
с~точки зрения их~лицензирования, например в~варианте продажи либо
покупки такой лицензии или~патента. Многое из~того, что~было рассмотрено
в~данном Практическом руководстве, применимо для~расчёта соответствующей
доходности в~процессе определения ставки роялти. На практике значение
ставки роялти определяется на~основе части либо всех нижеприведённых
сведений:
\begin{itemize}
\item существующие лицензии на~\textsl{нематериальные активы} (\emph{метод
сопоставимых данных});
\item принятая в~отрасли практика лицензирования схожих активов (\textsl{сравнительный
подход});
\item распределение экономических выгод, получаемых вследствие владения
активом, например изобретения, защищённого патентом (иногда называется
\emph{метод доступной прибыли} либо \emph{аналитический метод});
\item практика лицензирования (\emph{эмпирический метод}).\label{5.6.7.7-End}
\end{itemize}
\stepcounter{SubSecCounter}

\thesubsection.\theSubSecCounter.\label{5.6.7.8} При~проведении
\textsl{оценки} прав, проистекающих из~лицензии, следующие вопросы
подлежат рассмотрению:
\begin{itemize}
\item каким образом осуществлялись сделки со~схожими лицензиями;
\item суть \textsl{нематериального актива} и~его~сопровождение;
\item срок действия лицензионного соглашения;
\item исключительность прав, предоставляемых лицензией;
\item особые условия в~специальных случаях;
\item территория действия лицензии;
\item отрасль, в~которой осуществляется лицензирование \textsl{нематериального
актива};
\item наличие каких-либо особых взаимоотношений.
\end{itemize}
Даже если предыдущая практика лицензирования является либо признаётся
сопоставимой, она~может использоваться исключительно в~качестве
ориентира. Любые \textsl{нематериальные активы} уникальны по~своей
природе, вследствие и~по~причине чего для~их~адекватного сравнения
может потребоваться проведение значительного количества корректировок.\label{5.6.7.8-End}

\stepcounter{SubSecCounter}

\thesubsection.\theSubSecCounter.\label{5.6.7.9} \emph{Метод приведённой
стоимости} моделирует процессы, происходящие, например, при~использовании
\hyperref[5.6.6.3.3]{метода освобождения отроялти}
(см.~п.~\ref{5.6.6.3.3} \vpageref{5.6.6.3.3}--\pageref{5.6.6.3.3-End}).\label{5.6.7.9-End}\label{subsec:5.6.7_Present_value_techniques-End}

\subsection{Составление отчёта об~оценке\label{subsec:5.6.8_Reports}}

\stepcounter{SubSecCounter}

\thesubsection.\theSubSecCounter.\label{5.6.8.1} В~тех~случаях,
когда \textsl{оценка} должна соответствовать требованиям настоящих
\emph{Всемирных стандартов}, \emph{оценщик} обязан подготовить \emph{отчёт},
соответствующий минимальным требования к~его~содержанию, установленным
разделом~\ref{sec:4.3_VPS3_Valuation_reports}~\nameref{sec:4.3_VPS3_Valuation_reports}
\vpageref{sec:4.3_VPS3_Valuation_reports}--\pageref{sec:4.3_VPS3_Valuation_reports-End}.
Как~правило, \emph{отчёт} содержит краткий вводный раздел или~резюме,
в~котором описывается \emph{задание на~оценку}, излагаются основные
выводы и~приводятся ссылки на~разделы \emph{отчёта}, содержащие
его~детали. Повествование в~\emph{\uline{отчёте}} должно быть
построено на~переходе от~общего к~частному, логически связанных
описаниях исходных данных и~анализа, содержащего все~необходимые
рассуждения, приводящие к~сделанным в~\uline{нём}~выводам.\label{5.6.8.1-End}

\stepcounter{SubSecCounter}

\thesubsection.\theSubSecCounter.\label{5.6.8.2} В~большинстве
случаев материал \emph{отчёта} может быть легко структурирован по~разделам,
приведённым ниже, при~этом порядок их~расположения может отличаться:
\begin{itemize}
\item введение;
\item \emph{цель оценки} и~\textsl{вид определяемой стоимости};
\item \textsl{допущения }и~\textsl{специальные допущения};
\item \emph{объект оценки};
\item описание актива~(активов) и~его~(их)~истории, описание бизнеса,
в~рамках которого он~(они)~используется~(используются);
\item бухгалтерские документы и~сведения об~учётной политике;
\item \textsl{анализ финансовой отчётности} (когда это~применимо);
\item анализ деловых и~маркетинговых планов и~перспектив;
\item результаты поиска данных о~сопоставимых сделках;
\item описание отрасли, в~которой используется оцениваемый актив;
\item вопросы экономического и~экологического окружения, доходность и~оценка
риска;
\item \emph{методы оценки} и~сделанные на~их~основе выводы;
\item предостережения, вопросы ограничения ответственности и~пределов применимости
результатов \textsl{оценки}.\label{5.6.8.2-End}
\end{itemize}
\stepcounter{SubSecCounter}

\thesubsection.\theSubSecCounter.\label{5.6.8.3} Некоторые \emph{отчёты}
также содержат обособленный раздел, в~котором приводится общее описание
методологии \textsl{оценки}, часто следующий за~введением. Если общенациональные,
региональные и~экономические данные имеют значение для~компании
и~(или) оцениваемого активов~(активов), то~для~каждой их~категории
может быть предусмотрен свой раздел.\label{5.6.8.3-End}

\stepcounter{SubSecCounter}

\thesubsection.\theSubSecCounter.\label{5.6.8.4} В~соответствующих
случаях фактические информация и~данные либо~их источники должны
быть приведены в~основной части \emph{отчёта} либо приложениях к~нему.
В~случаях, когда \emph{отчёт} эксперта-оценщика необходим для~целей
судебного разбирательства, \emph{он}~должен соответствовать требованиям,
предъявляемым в~местной юрисдикции, в~т.\,ч.~содержать все~соответствующие
сведения, включая заявление о~квалификации эксперта и~\guillemotleft подписку
об~истинности содержания заключения\guillemotright .\label{5.6.8.4-End}\label{subsec:5.6.8_Reports-End}

\subsection{Конфиденциальность\label{subsec:5.6.9_Confidentiality}}

\stepcounter{SubSecCounter}

\thesubsection.\theSubSecCounter.\label{5.6.9.1} Сведения о~многих
\textsl{нематериальных активах} имеют конфиденциальный характер. \emph{Оценщики}
обязаны прилагать максимум усилий для~сохранения такой конфиденциальности,
в~т.\,ч.~в~части сведений, полученных в~отношении сопоставимых
активов. \emph{Оценщикам} \textsl{нематериальных активов} следует
подписывать соглашения о~неразглашении либо иные аналогичные соглашения
в~случае поступления соответствующего требования со~стороны заказчика.\label{5.6.9.1-End}\label{subsec:5.6.9_Confidentiality-End}\label{sec:5.6_VPGA-6_Valuation_of_intangible_assets-End}

\newpage

\section{ПР~7. Оценка личной собственности, в~т.\,ч.~предметов искусства
и~антиквариата\label{sec:5.7_VPGA-7_Valuation_of_personal_propetry}}

\textbf{Данное руководство носит рекомендательный характер и~не~содержит
обязательных требований. Однако там, где~это~уместно, оно~содержит
отсылки к~соответствующему обязательному материалу, содержащемуся
в~других разделах настоящих Всемирных стандартов, а~также \href{https://www.rics.org/globalassets/rics-website/media/upholding-professional-standards/sector-standards/valuation/international-valuation-standards-rics2.pdf}{Международных стандартов оценки}~\cite{IVS-2020},
реализованные в~виде перекрёстных ссылок. Данные ссылки предназначены
для~помощи }\textbf{\textsl{членам RICS}}\textbf{ и~не~меняют статус
данного практического руководства. }\textbf{\textsl{Членам RICS}}\textbf{
необходимо помнить следующее:}
\begin{itemize}
\item \textbf{данное руководство не~может охватить все~возможные варианты,
вследствие чего }\textbf{\emph{оценщикам}}\textbf{ при~формировании
своих }\textbf{\emph{суждений о~стоимости}}\textbf{ всегда следует
учитывать факты и~обстоятельства, имеющие место в~рамках отдельных
заданий по~}\textbf{\textsl{оценке}}\textbf{;}
\item \textbf{следует внимательно относиться к~тому факту, что~в~ряде
юрисдикций могут существовать особые требования, не~предусмотренные
данным руководством.}
\end{itemize}

\subsection{Введение и~область применения\label{subsec:5.7.1_Introduction_and_scope}}

\stepcounter{SubSecCounter}

\thesubsection.\theSubSecCounter.\label{5.7.1.1} Данное практическое
руководство содержит дополнительные комментарии по~применению \href{https://www.rics.org/globalassets/rics-website/media/upholding-professional-standards/sector-standards/valuation/international-valuation-standards-rics2.pdf}{Международных стандартов оценки}~\cite{IVS-2020},
а~также разделов~\ref{sec:4.1_VPS1_Terms_of_engagement_Scope_of_work}--\ref{sec:4.5_VPS5_Valuation_approaches_and_methods}
применительно к~\textsl{оценке} объектов личной собственности, т.\,е.~активов
и~обязательств, перечисленных ниже в~п.~\ref{5.7.1.2} \vpageref{5.7.1.2}--\pageref{5.7.1.2-End}.\label{5.7.1.1-End}

\stepcounter{SubSecCounter}

\thesubsection.\theSubSecCounter.\label{5.7.1.2} В~целях данного
практического руководства под~\guillemotleft личной собственностью\guillemotright{}
понимаются активы либо обязательства, не~имеющие постоянной связи
с~землёй либо зданиями. Данная категория:
\begin{itemize}
\item \textbf{включает} (но~не~ограничивается данным перечнем) предметы
изобразительного и~декоративного искусства, антиквариат, картины,
драгоценные камни и~ювелирные изделия, предметы коллекционирования,
предметы интерьера и~мебель, а~также другие предметы общего назначения;
\item \textbf{не~включает} предметы, предназначенные для~ведения коммерческой
деятельности, \textsl{машины и~оборудование}, коммерческие предприятия
и~доли в~них, \textsl{нематериальные активы}.
\end{itemize}
Оценка личной собственности может потребоваться в~различных ситуациях
и~для~различных целей, включая следующие, но~не~ограничиваясь
ими:
\begin{itemize}
\item страховая защита;
\item повреждение либо утрата вследствие пожара, воздействия воды либо иных
причин;
\item налогообложение (благотворительные взносы, налог на~дарение, налог
на~имущество, вред от~несчастных случаев);
\item \textsl{финансовая отчётность};
\item коммерческие операции;
\item судебные разбирательства, в~т.\,ч.~связанные с~мошенничеством;
\item планирование наследства, справедливое распределение активов и~утверждение
завещания;
\item брачные договоры;
\item раздел имущества при~расторжении брака;
\item ликвидация хозяйственного общества;
\item консультации по~вопросам приобретения либо отчуждения имущества в~инвестиционных
целях либо личного потребления;
\item залоговое обеспечения кредита;
\item банкротство;
\item \textsl{оценка} стоимости товарно-материальных запасов.\label{5.7.1.2-End}
\end{itemize}
\stepcounter{SubSecCounter}

\thesubsection.\theSubSecCounter.\label{5.7.1.3} Вышеуказанный перечень
не~является закрытым в~силу существования национальной либо региональной
специфики. При~этом законодательные требования конкретной юрисдикции
имеют приоритет. Это~особенно актуально в~тех~ситуациях, когда
\textsl{оценка} проводится в~целях определения налоговых обязательств,
включая случаи наследования, либо для~целей бухгалтерского учёта.\label{5.7.1.3-End}

\stepcounter{SubSecCounter}

\thesubsection.\theSubSecCounter.\label{5.7.1.4} Необходимо чётко
понимать \emph{цель оценки}, поскольку она~зачастую предопределяет
\textsl{вид}~определяемой в~\emph{отчёте} \textsl{стоимости}. См.~раздел~\ref{sec:4.1_VPS1_Terms_of_engagement_Scope_of_work}~\nameref{sec:4.1_VPS1_Terms_of_engagement_Scope_of_work}
\vpageref{sec:4.1_VPS1_Terms_of_engagement_Scope_of_work}--\pageref{sec:4.1_VPS1_Terms_of_engagement_Scope_of_work-End}.\label{5.7.1.4-End}\label{subsec:5.7.1_Introduction_and_scope-End}

\subsection{Условия договора на~проведение оценки\label{subsec:5.7.2_Terms_of_engagement}}

\stepcounter{SubSecCounter}

\thesubsection.\theSubSecCounter.\label{5.7.2.1} Для~правильного
формулирования задач, стоящих перед \emph{оценщиком}, и, соответственно,
его~обязанностей, ему~следует установить, кто~является заказчиком,
а~также иных лиц, полагающихся на~\textsl{оценку} (т.\,е.~предполагаемых
пользователей \textsl{оценки}), статус и~роль заказчика, обеспечивая
таким образом её~информативность и~отсутствие введения в~заблуждение.\label{5.7.2.1-End}

\stepcounter{SubSecCounter}

\thesubsection.\theSubSecCounter.\label{5.7.2.2} Как~правило, условия
\textsl{договора на~проведение оценки,} в~т.\,ч.~в~части минимальных
требований к~нему, установленных разделом~\ref{sec:4.1_VPS1_Terms_of_engagement_Scope_of_work}~\nameref{sec:4.1_VPS1_Terms_of_engagement_Scope_of_work}
\vpageref{sec:4.1_VPS1_Terms_of_engagement_Scope_of_work}--\pageref{sec:4.1_VPS1_Terms_of_engagement_Scope_of_work-End}
согласовываются \emph{оценщиком} и~заказчиком до~начала оказания
услуг. В~случае необходимости начать работу до~полного согласования
условий \textsl{договора} и~его~подписания, все~вопросы, касающиеся
этих условий, должны быть доведены до~сведения заказчика и~документальное
оформлены до~момента выпуска \emph{отчёта} (см.~раздел~\ref{sec:4.1_VPS1_Terms_of_engagement_Scope_of_work}~\nameref{sec:4.1_VPS1_Terms_of_engagement_Scope_of_work}
\vpageref{sec:4.1_VPS1_Terms_of_engagement_Scope_of_work}--\pageref{sec:4.1_VPS1_Terms_of_engagement_Scope_of_work-End}).\label{5.7.2.2-End}

\stepcounter{SubSecCounter}

\thesubsection.\theSubSecCounter.\label{5.7.2.3} В~процессе согласования
условий \textsl{договора на~проведение оценки} \emph{оценщик} обязан
довести до~сведения клиента возможность влияния на~стоимость \emph{объекта
оценки} иных значимых факторов (например, его~происхождения либо
последствий \textsl{оценки} объекта в~составе группы активов либо
коллекции вместо его~\textsl{оценки} в~качестве~обособленной единицы).
Невыполнение этого требования может привести к~введению заказчика
в~заблуждение, что~является нарушением требований раздела~\ref{sec:4.3_VPS3_Valuation_reports}~\nameref{sec:4.3_VPS3_Valuation_reports}
\vpageref{sec:4.3_VPS3_Valuation_reports}--\pageref{sec:4.3_VPS3_Valuation_reports-End}.\label{5.7.2.3-End}\label{subsec:5.7.2_Terms_of_engagement-End}

\subsection{Определение сегмента рынка\label{subsec:5.7.3_Identifying_the_market}}

\stepcounter{SubSecCounter}

\thesubsection.\theSubSecCounter.\label{5.7.3.1} \textsl{Оценка}
основывается на~понимании рынка, к~которому относится \emph{объект
оценки}. \emph{Оценщики} обязаны оценить характер и~состояние рынка,
обеспечивающего контекст для~их~исследований и~выводов о стоимости.
Соображения, которые \emph{оценщик} должен принять во~внимание включают
уровень активности рынка, степень уверенности его~участников и~существующие
на~\textsl{дату оценки} тенденции. Также \emph{оценщикам} необходимо
учитывать правовой статус \emph{объекта оценки}, применимые нормативные
требования и~любые обстоятельства, способные оказать влияние на~поведение
участников рынка.\label{5.7.3.1-End}

\stepcounter{SubSecCounter}

\thesubsection.\theSubSecCounter.\label{5.7.3.2} \emph{Оценщикам}
\textsl{личного имущества} необходимо понимать, что~оцениваемый актив
может быть продан на~разных рынках, каждый из~которых формирует
собственные данные о~продажах. В~частности стоимость актива на~оптовом
рынке, розничном рынке и~аукционе может отличаться. \emph{Оценщику}
следует определить и~проанализировать рынок, наиболее подходящий
с~точки зрения типа актива и~\emph{цели оценки}. Следует признать,
что~\textsl{оценка}, проводимая для~целей консультирования по~вопросам
продажи между предприятиями, занимающимися продажей определённых видов
активов, может отличаться от~\textsl{оценки}, проводимой в~отношении
сделки между предприятием и~физическим лицом.\label{5.7.3.2-End}

\stepcounter{SubSecCounter}

\thesubsection.\theSubSecCounter.\label{5.7.3.3} При~определении
рынка \textsl{оценщикам} \textsl{личной собственности} необходимо
иметь ввиду, что~способ продажи может повлиять на~итоговую цену.
Например, онлайн-аукционы и~другие формы электронной коммерции ослабили
многие транзакционные ограничения, расширив круг потенциальных покупателей
некоторых видов товаров. Однако \textsl{оценщикам} также следует учитывать,
что~качество данных и~отсутствие точных сведений о~размере комиссионного
вознаграждения онлайн-платформы и~расходов на~продажу, связанных
с~некоторыми онлайн-платформами, если они~не~связаны с~продажами
в~оффлайне, могут сделать данные о~продажах ненадёжным источником
сопоставимых сведений.\label{5.7.3.3-End}

\stepcounter{SubSecCounter}

\thesubsection.\theSubSecCounter.\label{5.7.3.4} Объекты \textsl{личной
собственности} часто хранятся в~форме коллекций, которые при~разделении
могут стоить существенно больше или~меньше в~расчёте на~одну единицу
относительно варианта существования в~виде коллекции. \emph{Оценщику}
необходимо определить влияние коллективного хранения активов на~их
\textsl{оценку} и~дать соответствующую консультацию.\label{5.7.3.4-End}\label{subsec:5.7.3_Identifying_the_market-End}

\subsection{Осмотр, исследования и~анализ\label{subsec:5.7.4_Inspection_research_analysis}}

\stepcounter{SubSecCounter}

\thesubsection.\theSubSecCounter.\label{5.7.4.1} \emph{Оценщикам}
\textsl{личного имущества} следует собирать, проверять и~анализировать
соответствующие данные о~продажах, анализировать соответствующий
общеэкономический и~рыночный контекст, а~также учитывать любую дополнительные
сведения, необходимые для~получения реалистичных выводов о~стоимости.
Раздел~\ref{sec:4.2_VPS2_Inspections_investigations_and_records}~\nameref{sec:4.2_VPS2_Inspections_investigations_and_records}
\vpageref{sec:4.2_VPS2_Inspections_investigations_and_records}--\pageref{sec:4.2_VPS2_Inspections_investigations_and_records-End}
устанавливает требования к~проведению исследований.\label{5.7.4.1-End}

\stepcounter{SubSecCounter}

\thesubsection.\theSubSecCounter.\label{5.7.4.2} \emph{Оценщики}
личного имущества всегда должны иметь ввиду, что~степень надёжности
данных о~совершённых сделках может быть ограниченной, и~всегда самостоятельно
оценивать надёжность данных, используемых для~информационного обеспечения
анализа. Использованные источники информации и~данных необходимо
оформлять документально. Как~было сказано выше в~п.~\ref{5.7.3.3}
\vpageref{5.7.3.3}--\pageref{5.7.3.3-End}, \emph{оценщикам} следует
проявлять особую осторожность при~использовании данных, полученных
с~онлайн-платформ и~интернет-источников.\label{5.7.4.2-End}

\stepcounter{SubSecCounter}

\thesubsection.\theSubSecCounter.\label{5.7.4.3} \emph{Оценщику}
необходимо принимать во~внимание любые ограничения, препятствующие
проведению \textsl{осмотра}, исследованиям и~(или) анализу. При~наличии
таких ограничений \emph{оценщику} может потребоваться ввести \textsl{допущения}
либо \textsl{специальные допущения}. Требования, предъявляемые к~допущениям
и~специальным допущениям, установлены разделом~\ref{sec:4.4_VPS4_Bases_of_value}~\nameref{sec:4.4_VPS4_Bases_of_value}
\vpageref{sec:4.4_VPS4_Bases_of_value}--\pageref{sec:4.4_VPS4_Bases_of_value-End}.
Любые \textsl{допущения} подлежат обязательному обсуждению и~согласованию
с~заказчиком до~завершения проведения \textsl{оценки} и~должны
быть зафиксированы в\ письменном виде как~в~условиях \textsl{договора
на~проведение оценки}, так~и~в~\emph{отчёте}.\label{5.7.4.3-End}

\stepcounter{SubSecCounter}

\thesubsection.\theSubSecCounter.\label{5.7.4.4} \emph{Оценщик}
обязан рассмотреть общеэкономические и~рыночные данные, такие как~спрос
и~предложение на~конкретном рынке и~его~динамика. В~случае наличия
неопределённости в~используемых данных и~состоянии рынка, \emph{оценщику}
следует руководствоваться положениями раздела~\ref{sec:4.3_VPS3_Valuation_reports}~\nameref{sec:4.3_VPS3_Valuation_reports}
\vpageref{sec:4.3_VPS3_Valuation_reports}--\pageref{sec:4.3_VPS3_Valuation_reports-End}.\label{5.7.4.4-End}

\stepcounter{SubSecCounter}

\thesubsection.\theSubSecCounter.\label{5.7.4.5} \emph{Оценщик}
обязан обеспечить надлежащий уровень должной осмотрительности в~отношении
установления происхождения \emph{объекта оценки}, поскольку оно~может
оказать очень существенное влияние на~его~стоимость. Для~установления
происхождения, помимо собственных экспертных знаний в~данной области,
могут потребоваться архивные исследования и~(или) экспертиза. В~случае,
когда происхождение вызывает сомнения, \emph{оценщику} до~завершения
работы по~\textsl{оценке} следует провести консультацию с~заказчиком
для~того, чтобы определить объём исследования, которое должно быть
проведено, и~любые последствия, которые оно~может иметь в~т.\,ч.~в~части
его~оплаты или~привлечения третьих лиц для~его~осуществления.\label{5.7.4.5-End}

\stepcounter{SubSecCounter}

\thesubsection.\theSubSecCounter.\label{5.7.4.6} В~тех~случаях,
когда \emph{оценщику} необходимо проконсультироваться со~специалистами
либо профессионалами, осуществляющими деятельность в~отрасли, ему~следует,
в~той~мере, в~какой это~необходимо для~целей \textsl{оценки},
убедиться, что~специалист или~профессионал имеет соответствующую
квалификацию и~оказывает качественные услуги.\label{5.7.4.6-End}\label{subsec:5.7.4_Inspection_research_analysis-End}

\subsection{Процесс оценки\label{subsec:5.7.5_Valuation}}

\subsubsection{Подходы к~оценке и~их~реализация\label{subsubsec:5.7.5.1_Valuation_approaches}}

\stepcounter{SubSubSecCounter}

\thesubsubsection.\theSubSubSecCounter.\label{5.7.5.1.1} В~соответствии
с~п.~\ref{9.4.2.1.1.1} \vpageref{9.4.2.1.1.1}--\pageref{9.4.2.1.1.1-End}
для~\textsl{оценки} \textsl{личной собственности} применяются три~подхода:
\begin{itemize}
\item \emph{подход сравнения продаж};
\item \textsl{затратный подход};
\item \textsl{доходный подход}.\label{5.7.5.1.1-End}
\end{itemize}

\paragraph{Подход сравнения продаж\label{par:5.7.5.1.1_Sales_comparison_approach}}

\stepcounter{ParCounter}

\theparagraph.\theParCounter.\label{5.7.5.1.1.1} Данный подход позволяет
определить стоимость путём сравнения \emph{объекта оценки} с~аналогичными
активами, в~отношении сделок с~которыми имеются данные. Данный подход
наиболее распространён при~\textsl{оценке} \textsl{личной собственности}.
В~случае его~применения \emph{оценщику} необходимо тщательно анализировать
данные о~сопоставимых продажах в~соответствии с~требованиями, установленными
подразделом~\ref{subsec:5.7.4_Inspection_research_analysis}~\nameref{subsec:5.7.4_Inspection_research_analysis}
\vpageref{subsec:5.7.4_Inspection_research_analysis}--\pageref{subsec:5.7.4_Inspection_research_analysis-End}.
В~случае наличия возможности проведения \textsl{оценки} на~основе
сопоставимых данных о~продажах применение данного подхода, как~правило,
является предпочтительным.\label{5.7.5.1.1.1-End}\label{par:5.7.5.1.1_Sales_comparison_approach-End}

\paragraph{The cost approach\label{par:5.7.5.1.2_Cost_approach}}

\stepcounter{ParCounter}

\theparagraph.\theParCounter.\label{5.7.5.1.2.1} Данный подход позволяет
определить стоимость на~основе текущей стоимости предполагаемых затрат
на~воспроизводство либо создание объекта равного качества, полезности
и~ликвидности. Данный подход допускает замену \href{https://ru.wikipedia.org/wiki/\%D0\%9A\%D0\%BE\%D0\%BF\%D0\%B8\%D1\%8F_(\%D0\%B8\%D1\%81\%D0\%BA\%D1\%83\%D1\%81\%D1\%81\%D1\%82\%D0\%B2\%D0\%BE)}{копией}~\cite{Wiki:copy_art_rus}
либо замену, произведённую иным способом, например, \href{https://ru.wikipedia.org/wiki/\%D0\%A4\%D0\%B0\%D0\%BA\%D1\%81\%D0\%B8\%D0\%BC\%D0\%B8\%D0\%BB\%D0\%B5}{факсимиле}~\cite{Wiki:facsimile_art_rus}.
\begin{itemize}
\item \href{https://ru.wikipedia.org/wiki/\%D0\%9A\%D0\%BE\%D0\%BF\%D0\%B8\%D1\%8F_(\%D0\%B8\%D1\%81\%D0\%BA\%D1\%83\%D1\%81\%D1\%81\%D1\%82\%D0\%B2\%D0\%BE)}{Копия}~\cite{Wiki:copy_art_rus}
"--- это~общий термин, используемый в~случаях, когда воспроизведение
оригинального предмета осуществляется как~можно ближе к~оригиналу
с~точки зрения сути, качества, материалов и~их~возраста, а~также
технологии производства. В~случае, когда \href{https://ru.wikipedia.org/wiki/\%D0\%9A\%D0\%BE\%D0\%BF\%D0\%B8\%D1\%8F_(\%D0\%B8\%D1\%81\%D0\%BA\%D1\%83\%D1\%81\%D1\%81\%D1\%82\%D0\%B2\%D0\%BE)}{копия}~\cite{Wiki:copy_art_rus}
изготовлена автором оригинального произведения, она~называется \href{https://ru.wikipedia.org/wiki/\%D0\%9A\%D0\%BE\%D0\%BF\%D0\%B8\%D1\%8F_(\%D0\%B8\%D1\%81\%D0\%BA\%D1\%83\%D1\%81\%D1\%81\%D1\%82\%D0\%B2\%D0\%BE)}{репликой}~\cite{Wiki:copy_art_rus}.
\item \href{https://ru.wikipedia.org/wiki/\%D0\%A4\%D0\%B0\%D0\%BA\%D1\%81\%D0\%B8\%D0\%BC\%D0\%B8\%D0\%BB\%D0\%B5}{Факсимиле}~\cite{Wiki:facsimile_art_rus}
"--- точная копия оригинального предмета, созданная из~материалов
максимально близкого характера, качества и~возраста, с~использованием
техники или~методов изготовления оригинального периода.
\end{itemize}
Все~эти~концепции (т.\,е.~копия, реплика или~факсимиле) обычно
применяются только для~целей страхования, когда невозможно определить
стоимость другим методом. При~применении \textsl{затратного подхода}
\emph{оценщик} должен проанализировать актуальные и~применимые данные
о~затратах для~определения стоимости замещения. \emph{Оценщику}
следует знать, что~характер воспроизведения (копия, реплика или~факсимиле)
оказывает существенное влияние на~итоговую стоимость, и~соответствующим
образом корректировать свою \textsl{оценку}.\label{5.7.5.1.2.1-End}\label{par:5.7.5.1.2_Cost_approach-End}

\paragraph{Доходный подход\label{par:5.7.5.1.3_Income_approach}}

\stepcounter{ParCounter}

\theparagraph.\theParCounter.\label{5.7.5.1.3.1} Данный подход позволяет
определить стоимость \emph{объекта оценки} путём расчёта ожидаемых
финансовых выгод (например, потока доходов). При~применении данного
подхода \emph{оценщик} должен проанализировать актуальные и~применимые
данные для~определения дохода на~соответствующем сегменте рынка.
\emph{Оценщикам} следует основывать прогнозы ожидаемых финансовых
выгод на~анализе прошлых и~текущих данных, тенденций, а~также факторов
конкуренции. Несмотря на~возможность ситуаций, при~которых личное
имущество сдается владельцем в~аренду другой стороне, тем~не~менее,
в~большинстве случаев произведения искусства, антиквариат и~имущество
личного пользования не~приносят доход, вследствие и~по~причине
чего вряд~ли могут оцениваться методами, применимыми для~инвестиционных
активов. Однако в~ряде случаев, особенно когда они~расположены внутри
другого приносящего доход актива, такого как~недвижимость, имеющая
историческую ценность, их~присутствие там~может увеличить стоимость
всего комплекса и, следовательно, подлежит рассмотрению в~таком контексте.\label{5.7.5.1.3.1-End}\label{par:5.7.5.1.3_Income_approach-End}

\paragraph{Выводы, являющиеся общими для~всех подходов\label{par:5.7.5.1.4_Common_conclusions}}

\stepcounter{ParCounter}

\theparagraph.\theParCounter.\label{5.7.5.1.4.1} При~использовании
любого подхода \emph{оценщику} следует применять взвешенное и~информированное
мнение для~обобщения собранных данных, их~дальнейшего анализа и~последующего
формирования выводов о~стоимости.\label{5.7.5.1.4.1-End}

\stepcounter{ParCounter}

\theparagraph.\theParCounter.\label{5.7.5.1.4.2} Все~сделанные
\emph{оценщиком} выводы должны иметь разумные обоснования и~иметь
чёткое подтверждение соответствующими фактами, включая доказательства
происхождения объекта. В~случае применения более чем~одного \textsl{подхода
к~оценке} \emph{оценщику} следует включить в~анализ ~\uline{результаты}
каждого из~них, а~затем провести \uline{их}~согласование.\label{5.7.5.1.4.2-End}

\stepcounter{ParCounter}

\theparagraph.\theParCounter.\label{5.7.5.1.4.3} RICS не~предписывает
\emph{оценщикам} использование каких-либо конкретных \emph{методов}
и~методик \textsl{оценки}. Однако им~следует быть готовыми обосновать
причины выбора \emph{подхода~(подходов)} и~\emph{метода}~(методов).\label{5.7.5.1.4.3-End}\label{subsubsec:5.7.5.1_Valuation_approaches-End}

\subsubsection{Иные соображения, касающиеся оценки личной собственности\label{subsubsec:5.7.5.2_Other_valuation_considerations}}

\stepcounter{SubSubSecCounter}

\thesubsubsection.\theSubSubSecCounter.\label{5.7.5.2.1} В~дополнение
к~требованиям, установленным разделом~\ref{sec:4.3_VPS3_Valuation_reports}~\nameref{sec:4.3_VPS3_Valuation_reports}
\vpageref{sec:4.3_VPS3_Valuation_reports}--\pageref{sec:4.3_VPS3_Valuation_reports-End},
в~\emph{отчётах об~оценке} \textsl{личной собственности} должны
быть рассмотрены следующие вопросы.
\begin{itemize}
\item Объём информации и~данных, которые должны быть доведены до~сведения
заказчика \textsl{оценки} и~других её~предполагаемых пользователей.
\emph{Оценщику} следует принимать во~внимание то~обстоятельство,
что~знания заказчиков в~области \textsl{оценки} могут различаться,
вследствие и~по~причине чего ему~следует сообщать им~те~информацию
и~данные, которые будут понятны всем пользователям\emph{ отчёта}.
\item Оцениваемые права: возможны ситуации, при~которых оцениваемые права
на~личное имущество принадлежат нескольким лицам, в~таких случаях
данное обстоятельство должно быть ясно изложено в~тексте \emph{отчёта}.
\item Характеристики, необходимые для~идентификации объекта, в~т.\,ч.
такие как~художник либо создатель, материал либо носитель, размер,
название, место происхождения, стиль, возраст, происхождение либо
история создания, состояние, заключения экспертов, история экспозиции
на~выставках, упоминания в~литературе.
\item \textsl{Вид определяемой стоимости}, например, \textsl{рыночная стоимость},
\emph{стоимость замещения} и~т.\,д., а~также источник его~определения.
\item Любые особые условия \textsl{договора на~проведение оценки} и~(или)
регуляторные либо законодательные требования.
\item Наличие ограничений, обременений, прав арендаторов, обязательств,
договоров, способных оказать влияние на~\textsl{оценку} либо полноту
прав на~оцениваемый предмет.
\item Степень проверяемости \uline{сведений}, предоставленных третьими
лицами, а~также уверенности в~\uline{них}.
\item Связь объекта с~какими-либо иными материальными либо нематериальным
активами, способная оказать влияние на~его \textsl{оценку}.
\item Значение отдельных активов в~составе общего \emph{задания на~оценку},
включающего множество объектов с~широким диапазоном стоимости.
\item Анализ прежних сделок с~оцениваемым имуществом в~случаях, когда
это~уместно.
\item Степень влияния общеэкономических факторов и~факторов рынка, к~которому
относится \emph{объект оценки}, на~уровень определённости выводов
о~стоимости.\label{5.7.5.2.1-End}\label{subsubsec:5.7.5.2_Other_valuation_considerations-End}\label{subsec:5.7.5_Valuation-End}
\end{itemize}

\subsection{Содержание отчёта об~оценке\label{subsec:5.7.6_Reports}}

\stepcounter{SubSecCounter}

\thesubsection.\theSubSecCounter.\label{5.7.6.1} \emph{Оценщик}
обязан обеспечить ясность и~точность \emph{отчёта об~оценке}, а~также
отсутствие в~нём элементов, допускающих двоякое толкование либо вводящих
в~заблуждение. При~его~подготовке \emph{оценщик} обязан следовать
принципам независимости, честности и~объективности (см.~раздел~\ref{sec:3.2_PS2_Ethics_competency_objectivity}~\nameref{sec:3.2_PS2_Ethics_competency_objectivity}
\vpageref{sec:3.2_PS2_Ethics_competency_objectivity}--\pageref{sec:3.2_PS2_Ethics_competency_objectivity-End}).\label{5.7.6.1-End}

\stepcounter{SubSecCounter}

\thesubsection.\theSubSecCounter.\label{5.7.6.2} \emph{Оценщику}
следует обеспечить соответствие содержания \emph{отчёта} минимальным
требованиям к~нему, установленным разделом~\ref{sec:4.3_VPS3_Valuation_reports}\nameref{sec:4.3_VPS3_Valuation_reports}\vpageref{sec:4.3_VPS3_Valuation_reports}\pageref{sec:4.3_VPS3_Valuation_reports-End},
а~также обеспечить рассмотрение в~нём~вопросов, перечисленных выше
в~п.~\ref{5.7.5.2.1} \vpageref{5.7.5.2.1}--\pageref{5.7.5.2.1-End}.
Кроме того, если в~процессе подготовки \textsl{оценки} \emph{оценщик}
консультировался со~специалистом, профессионалом либо организацией,
практикующими в~определённом сегменте рынка, во~всех случаях необходимо
приводить соответствующие ссылки на~них, а~также указать характер
и~степень влияния их~вклада в~результаты такой \textsl{оценки}
(см.~п.~\ref{5.7.4.6} \vpageref{5.7.4.6}--\pageref{5.7.4.6-End}).\label{5.7.6.2-End}

\stepcounter{SubSecCounter}

\thesubsection.\theSubSecCounter.\label{5.7.6.3} Реализованная в~\emph{отчёте}
степень детализации описаний и~исследований, должна адекватно отвечать
потребностям заказчика и~иных предполагаемых пользователей, характеру
оцениваемого имущества и~предполагаемому использованию \textsl{оценки}.
Применяемая в~\emph{отчёте} терминология должна быть понятной всем
его~предполагаемым пользователям.\label{5.7.6.3-End}

\stepcounter{SubSecCounter}

\thesubsection.\theSubSecCounter.\label{5.7.6.4} \emph{Оценщик}
обязан описать ограничения и~(или) условия, касающиеся \textsl{осмотра},
исследований и~анализа, а~также объяснить их~влияние на~сделанные
выводы.\label{5.7.6.4-End}

\stepcounter{SubSecCounter}

\thesubsection.\theSubSecCounter.\label{5.7.6.5} Также в~\emph{отчёте}
в~явном виде должны содержаться сведения о~\emph{цели проведения
оценки} (например, справедливое распределение), \textsl{виде определяемой
стоимости} (например, \textsl{рыночная стоимость}), а~также о~рынке
и~способе совершения условной либо реальной сделки с~оцениваемым
активом.\label{5.7.6.5-End}

\stepcounter{SubSecCounter}

\thesubsection.\theSubSecCounter.\label{5.7.6.6} При~необходимости
\emph{оценщик} приводит сведения о~соответствии \emph{отчёта} специальным
требованиям заказчика, регуляторов либо применимых законов и~подзаконных
актов.\label{5.7.6.6-End}

\stepcounter{SubSecCounter}

\thesubsection.\theSubSecCounter.\label{5.7.6.7} \emph{Оценщик}
приводит краткое описание проведённого исследования и~данных, использованных
при~анализе. Также необходимо привести сведения об~использованном
\emph{подходе}~(\emph{подходах}) и~обоснование выбора одного либо
нескольких из~них, включая объяснение причин отказа от~использования
других. В~случае применения нескольких \emph{подходов}, данное обстоятельство
необходимо раскрыть в~тексте \emph{отчёта}, а~также включить в~него
описание согласования полученных результатов.\label{5.7.6.7-End}

\stepcounter{SubSecCounter}

\thesubsection.\theSubSecCounter.\label{5.7.6.8} При~проведении
\textsl{оценки}, основанной на~каких-либо \textsl{допущениях} или~\textsl{специальных
допущениях} (например при~определении общей стоимости группы активов),
они~должны быть рассмотрены в~\emph{отчёте} совместно с~описанием
влияния, оказываемого ими~на~стоимость, если таковое имеет место.
В~частности, в~тех случаях, когда \emph{оценщик} не~смог установить
происхождение \emph{объекта оценки} с~достаточной степенью уверенности,
данное обстоятельство подлежит включению в~\emph{отчёт} совместно
со всеми прочими сделанными \textsl{допущениями}.\label{5.7.6.8-End}

\stepcounter{SubSecCounter}

\thesubsection.\theSubSecCounter.\label{5.7.6.9} \emph{Оценщик}
обязан привести комментарии в~отношении любых вопросов, оказывающих
влияние на~определённость \emph{оценки}. Степень детализации таких
комментариев может варьироваться в~зависимости от~\emph{цели проведения
оценки}, а~также знаний её~предполагаемого пользователя.\label{5.7.6.9-End}

\stepcounter{SubSecCounter}

\thesubsection.\theSubSecCounter.\label{5.7.6.10} Приводимые в~\emph{отчёте}
фотографии должны быть уместными и~использоваться в~соответствии
с~требованиями \textsl{договора на~проведение оценки}. В~случае
внесения в~них каких-либо изменений, данное обстоятельство подлежит
отражению в~тексте \emph{отчёта об~оценке}.\label{5.7.6.10-End}\label{subsec:5.7.6_Reports-End}\label{sec:5.7_VPGA-7_Valuation_of_personal_propetry-End}

\newpage

\section{ПР~8.~Оценка недвижимого имущества и~прав на~него\label{sec:5.8_VPGA-8_Valuation_of_real_property}}

\textbf{Данное руководство носит рекомендательный характер и~не~содержит
обязательных требований. Однако там, где~это~уместно, оно~содержит
отсылки к~соответствующему обязательному материалу, содержащемуся
в~других разделах настоящих Всемирных стандартов, а~также \href{https://www.rics.org/globalassets/rics-website/media/upholding-professional-standards/sector-standards/valuation/international-valuation-standards-rics2.pdf}{Международных стандартов оценки}~\cite{IVS-2020},
реализованные в~виде перекрёстных ссылок. Данные ссылки предназначены
для~помощи }\textbf{\textsl{членам RICS}}\textbf{ и~не~меняют статус
данного практического руководства. }\textbf{\textsl{Членам RICS}}\textbf{
необходимо помнить следующее:}
\begin{itemize}
\item \textbf{данное руководство не~может охватить все~возможные варианты,
вследствие чего }\textbf{\emph{оценщикам}}\textbf{ при~формировании
своих }\textbf{\emph{суждений о~стоимости}}\textbf{ всегда следует
учитывать факты и~обстоятельства, имеющие место в~рамках отдельных
заданий по~}\textbf{\textsl{оценке}}\textbf{;}
\item \textbf{следует внимательно относиться к~тому факту, что~в~ряде
юрисдикций могут существовать особые требования, не~предусмотренные
данным руководством.}
\end{itemize}
Данное практическое руководство содержит дополнительные комментарии,
затрагивающие отдельные специфические темы и~вопросы, возникающие
в~связи с~проведением \textsl{оценки} \textsl{недвижимого имущества},
и~дополняет разделы~\ref{sec:10.5_IVS-400_Real_Property}~\nameref{sec:10.5_IVS-400_Real_Property},
\ref{sec:10.6_IVS-410_Development_Property}~\nameref{sec:10.6_IVS-410_Development_Property}
\vpageref{sec:10.5_IVS-400_Real_Property}--\pageref{sec:10.6_IVS-410_Development_Property-End},
а~также раздел~\ref{sec:4.2_VPS2_Inspections_investigations_and_records}~\nameref{sec:4.2_VPS2_Inspections_investigations_and_records}
\vpageref{sec:4.2_VPS2_Inspections_investigations_and_records}--\pageref{sec:4.2_VPS2_Inspections_investigations_and_records-End}.
Оно~содержит ряд рекомендаций, прямо касающихся \textsl{осмотров}
и~исследований, а~также важный новый материал, затрагивающий вопросы
\textsl{\href{https://en.wikipedia.org/wiki/Sustainability}{устойчивого развития}}~\cite{Wiki:sustainability,Investo:sustainability}
и~защиты окружающей среды "--- факторов, влияние которых на~рынок
\textsl{недвижимого имущества} неуклонно продолжает возрастать.

\subsection{Осмотр\label{subsec:5.8.1_Inspection}}

\stepcounter{SubSecCounter}

\thesubsection.\theSubSecCounter.\label{5.8.1.1} Данный, а~также
следующий (\ref{subsec:5.8.2_Investigations_and_assumptions}~\nameref{subsec:5.8.2_Investigations_and_assumptions}
\vpageref{subsec:5.8.2_Investigations_and_assumptions}--\pageref{subsec:5.8.2_Investigations_and_assumptions-End})
подразделы затрагивают вопросы \textsl{осмотра} и~исследований в~отношении
объектов \textsl{недвижимого имущества}, а~точнее, когда ситуации,
когда \emph{объект оценки} представляет собой право собственности,
контроля, пользования либо занятия земли и~зданий~(см.~п.~\ref{10.5.2.2}
\vpageref{10.5.2.2}--\pageref{10.5.2.2-End}).\label{5.8.1.1-End}

\stepcounter{SubSecCounter} 

\thesubsection.\theSubSecCounter.\label{5.8.1.2} Многие аспекты,
которые потенциально способны либо однозначно будут оказывать влияние
на~восприятие стоимости соответствующего права со~стороны рынка,
могут быть установлены в~явном виде только в~ходе \textsl{осмотра}
объекта недвижимости. Их~перечень приводится ниже.
\begin{enumerate}
\item \label{5.8.1.2-1}Характеристики местоположения и~окружающей местности,
доступность инженерных коммуникаций, служб и~сооружений, оказывающих
влияние на~стоимость.\label{5.8.1.2-1-End}
\item \label{5.8.1.2-2}Характеристики объекта и~его~использования:
\begin{enumerate}
\item размер, площадь и~использование составных элементов;
\item возраст, конструктивная схема и~материал здания или~сооружения;
\item доступность для~сотрудников и~посетителей;
\item наличие дополнительных установок, удобств и~услуг;
\item наличие приспособлений, инвентаря и~улучшений;
\item наличие \textsl{машин и~оборудования}, являющихся по~умолчанию частью
здания (см.~также раздел~\ref{sec:5.5_VPGA-5-Valuation_of_plant_and_equipment}~\nameref{sec:5.5_VPGA-5-Valuation_of_plant_and_equipment}
\vpageref{sec:5.5_VPGA-5-Valuation_of_plant_and_equipment}--\pageref{sec:5.5_VPGA-5-Valuation_of_plant_and_equipment-End});
\item визуально наблюдаемое состояние конструкций и~отделки;
\item наличие на~объекте опасных материалов, включая, но~не~ограничиваясь
теми, оборот которых ограничен, например, химикатов, радиоактивных
материалов, взрывчатых веществ, асбеста, веществ, разрушающих озон,
нефтепродуктов и~т.\,д., либо осуществление на~нём деятельности,
подлежащей регулированию, например деятельности по~обращению с~отходами.\label{5.8.1.2-2-End}
\end{enumerate}
\item Характеристики земельного участка:\label{5.8.1.2-3}
\begin{enumerate}
\item природные угрозы такие~как~нестабильность грунтов, добыча либо извлечение
природных ресурсов, наличие риска затопления по~различным причинам,
включая атмосферные осадки и~разлив рек;
\item угрозы неприродного характера такие~как загрязнение грунта, когда
на~его~поверхности либо в~его~толще находятся опасные вещества,
появившиеся там~вследствие осуществления текущей либо прежней деятельности
(см.~также пп.~\hyperref[5.8.1.2-2]{b} \vpageref{5.8.1.2-2}\pageref{5.8.1.2-2-End}).\label{5.8.1.2-3-End}
\end{enumerate}
\item \label{5.8.1.2-4}Потенциал девелопмента либо редевелопмента:
\begin{enumerate}
\item наличие физических ограничений для~дальнейшего развития, если применимо.\label{5.8.1.2-4-End}\label{5.8.1.2-End}
\end{enumerate}
\end{enumerate}
\stepcounter{SubSecCounter} 

\thesubsection.\theSubSecCounter.\label{5.8.1.3} Иные вопросы, по~которым
соответствующие сведения могут быть получены по~результатам \textsl{осмотра}
либо следующего за~ним~исследования, перечислены ниже.
\begin{enumerate}
\item Наличие улучшений \textsl{недвижимости}, сдаваемой в~аренду: при~\textsl{оценке}
арендованной либо передаваемой обратно по~окончанию срока аренды
\textsl{недвижимости} необходимо внимательно отнестись к~определению
того, что~именно должно быть оценено, поскольку подлежащая \textsl{оценке}
\textsl{недвижимость} в~её~первоначальном состоянии может отличаться
от~той, которая наблюдается при~\textsl{осмотре} и~может быть измерена
(при~необходимости), поскольку ряд улучшений мог быть выполнен силами
арендатора либо за~его счёт, что~означает возможное отсутствие на~них
прав у~арендодателя в~зависимости от~условий договора аренды. В~том
случае, когда \emph{оценщик} не~может провести \textsl{осмотр} арендуемого
объекта, либо, когда вследствие отсутствия соответствующих документов
невозможно подтвердить объём перепланировок или~улучшений, \emph{оценщик}
должен исходить из~указанных \textsl{допущений}.
\item Практика в~области градостроительного контроля и~зонирования, включая
вопросы контроля и~необходимость получения согласований либо разрешений
на~расширение или~изменение использования, включая застройку, отличается
в~разных странах и~территориях. Объём конкретных запросов, которые
уместно и~необходимо сделать в~отдельных случаях, определяется знаниями
\emph{оценщика} относительно соответствующего рынка, характером и~масштабом
имущества, а~также \emph{целью оценки}.
\item Там, где~это уместно, данные о~любых существенных расходах, а~также
эксплуатационных расходах, в~т.\,ч.~об~уровне их~возмещения со~стороны
арендатора, при~этом энергоэффективность может быть одним факторов,
имеющих значение при~рассмотрении вопросов \textsl{\href{https://en.wikipedia.org/wiki/Sustainability}{устойчивого развития}}~\cite{Wiki:sustainability,Investo:sustainability}
(см.~далее подсекцию~\vref{par:5.8.2.5.3_Sustainability}--\pageref{par:5.8.2.5.3_Sustainability-End}).\label{5.8.1.3-End}\label{subsec:5.8.1_Inspection-End}
\end{enumerate}

\subsection{Исследования и~допущения\label{subsec:5.8.2_Investigations_and_assumptions}}

\stepcounter{SubSubSecCounter} 

\thesubsubsection.\theSubSubSecCounter.\label{5.8.2.0.1} Рассмотренные
в~данном подразделе аспекты являются общими для~многих \textsl{оценок},
связанных с~\textsl{недвижимостью}, и, в~основном, затрагивают вопросы
уместной глубины исследований либо характера \textsl{допущений}, которые
могут быть обоснованно сделаны. Приведённое ниже руководство не~может
охватить все~обстоятельства "--- при~выполнении конкретных заданий
всегда необходимо применять знания, опыт и~критическое мышление \emph{оценщика},
а~в~некоторых случаях соответствующие ограничения будут установлены
заказчиком либо обсуждены с~ним~и~согласованы в~условиях \textsl{договора
на~проведение оценки}. Аналогичным образом, актуальность и~уместность
\textsl{допущений} можно оценивать только в~контексте каждого конкретного
случая "--- изложенное ниже ни~в~коей мере не~является предписанием.\label{5.8.2.0.1-End}

\subsubsection{Права на~оцениваемый объект\label{subsubsec:5.8.2.1_Title}}

\stepcounter{SubSubSecCounter} 

\thesubsubsection.\theSubSubSecCounter.\label{5.8.2.1.1} \emph{Оценщику}
необходимо обладать сведениями об~основных деталях прав на~\emph{оцениваемый
объект}. Такие сведения могут быть предоставлены в~различных формах,
таких как: справка, подготовленная заказчиком либо третьей стороной;
копии соответствующих документов; текущий подробный отчёт о~праве
собственности, подготовленный юристами заказчика "--- данный перечень
не~является исчерпывающим.

\emph{Оценщик} обязан привести перечень использованных информации
и~данных и, при~необходимости, привести перечень введённых \textsl{допущений}.
Например, в~случае отсутствия документов по~аренде, \emph{оценщику}
может потребоваться сделать \textsl{допущение} о~том, что~её~условия,
указанные в~\emph{отчёте}, соответствуют условиям фактической аренды.
Однако, в~случае если была предоставлена гарантия наличия соответствующего
права, \emph{оценщик} может обоснованно полагаться на~правильность
этих сведений "--- при~этом, в~конечном счёте, вопрос прав является
юридическим, и~в~соответствующих случаях \emph{оценщик} может отдельно
отметить, что~правовое положение объекта подлежит обязательной проверке
со~стороны юридических консультантов заказчика. \emph{Оценщику} не~следует
принимать на~себя ответственность либо обязательства по~ установлению
наличия, проверке и~интерпретации права собственности заказчика на~имущество
или~актив.\label{5.8.2.1.1-End}\label{subsubsec:5.8.2.1_Title-End}

\subsubsection{Состояние зданий\label{subsubsec:5.8.2.2_Condition_of_buildings}}

\stepcounter{SubSubSecCounter} 

\thesubsubsection.\theSubSubSecCounter.\label{5.8.2.2.1} Даже если
\emph{оценщик} обладает соответствующей компетенцией, как~правило
он~не~проводит обследование здания для~установления деталей любых
дефектов или~факта непригодности здания. Однако, для~\emph{оценщика}
также неправильно игнорировать очевидные дефекты, которые могут повлиять
на~стоимость, если только на~этот счёт не~было согласовано \textsl{специальное
допущение}. Поэтому \emph{оценщику} необходимо указать на~то~обстоятельство,
что~\textsl{осмотр} не~является полноценным обследованием. Кроме
того, должны быть определены пределы ответственности \emph{оценщика}
за~исследование и~описание конструкций либо любых дефектов. Там,
где~это~уместно, следует также указать, что~будет сделано \textsl{допущение}
о~том, что~здание~(здания) в~целом находится~(находятся) в~\href{https://docs.cntd.ru/document/1200100941}{работоспособном состоянии}~\cite{GOST:obsledovanie_zdanij},
указав на~отдельные незначительные дефекты.\label{5.8.2.2.1-End}\label{subsubsec:5.8.2.2_Condition_of_buildings-End}

\subsubsection{Инженерное обеспечение\label{subsec:5.8.2.3_Services}}

\stepcounter{SubSubSecCounter} 

\thesubsubsection.\theSubSubSecCounter.\label{5.8.2.3.1} Наличие
и~производительность инженерных коммуникаций здания и~любых связанных
с~ним~установок и~оборудования часто оказывает значительное влияние
на~стоимость, однако детальное исследование данных вопросов, как~правило,
выходит за~рамки \textsl{оценки}. При~рассмотрении соответствующих
вопросов \emph{оценщику} необходимо определить, какие источники информации
и~данных доступны, и~в~какой степени на~них~можно полагаться
при~проведении \textsl{оценки}. Обычно принято исходить из~того,
что~инженерное оборудование и~коммуникации, равно как~и~любые
связанные с~ними средства управления или~программное обеспечение
находятся в~\href{https://docs.cntd.ru/document/1200100941}{работоспособном состоянии}~\cite{GOST:obsledovanie_zdanij}
или~не~имеют дефектов.\label{5.8.2.3.1-End}\label{subsec:5.8.2.3_Services-End}

\subsubsection{Градостроительное зонирование и~территориальное планирование\label{subsubsec:5.8.2.4_Planning_(zoning)}}

\stepcounter{SubSubSecCounter} 

\thesubsubsection.\theSubSubSecCounter.\label{5.8.2.4.1} При~наличии
сомнений \emph{оценщику} может потребоваться установить, имеются~ли
необходимые разрешительные документы на~текущее использование объекта
и~создание на~нём существующих зданий, либо посоветовать заказчику
обратиться за~соответствующей проверкой, а~также выяснить, существуют~ли
какие-либо планы либо решения органов власти, которые могут положительно
или~отрицательно повлиять на~стоимость. Подобные данные в~общем
виде часто бывают легкодоступны, однако, при~получении конкретных
сведений могут иметь место задержки либо расходы. \emph{Оценщик},
среди прочего, обязан привести сведения о~том, какие исследования
предполагается провести, или~какие \textsl{допущения} будут сделаны,
если проверка информации нецелесообразна в~контексте \textsl{оценки}.\label{5.8.2.4.1-End}\label{subsubsec:5.8.2.4_Planning_(zoning)-End}

\subsubsection{Вопросы защиты окружающей среды\label{subsubsec:5.8.2.5_Environmental_matters}}

\stepcounter{ParCounter}

\theparagraph.\theParCounter.\label{5.8.2.5.0.1} Потенциальные или~фактические
ограничения на~использование \textsl{недвижимого имущества}, вызванные
экологическими факторами, могут быть следствием естественных причин
(например, наводнения), неестественных причин (например, загрязнения)
или~иногда комбинации этих двух факторов (например, просадка грунта
в~результате имевшей место в~прошлом добычи полезных ископаемых).
Несмотря на~значительное разнообразие обстоятельств, ключевым вопросом
всегда остаётся степень влияния выявленных факторов на~стоимость.
Особую осторожность следует проявлять при~определении степени влияния
либо описания экологических факторов, поскольку \emph{оценщики} могут
не~обладать специальными познаниями и~опытом, которые часто требуются.
В~соответствующих случаях \emph{оценщик} может рекомендовать проведение
дополнительных запросов и~(или) получение дополнительных консультаций
специалистов или~экспертов в~отношении экологических вопросов. В~следующих
секциях данные вопросы рассмотрены подробнее.\label{5.8.2.5.0.1-End}

\paragraph{Природные экологические ограничения\label{par:5.8.2.5.1_Natural_environmental_constraint}}

\stepcounter{ParCounter}

\theparagraph.\theParCounter.\label{5.8.2.5.1.1} Некоторые объекты
недвижимости подвержены влиянию экологических факторов, являющихся
неотъемлемой характеристикой либо самого объекта, либо окружающей
местности и~оказывающих влияние на~стоимость прав на~него. Примерами
таких факторов являются проблемы нестабильности грунта (например,
пучение либо усадка глинистых пород, проседание вследствие прежней
либо текущей добычи полезных ископаемых и~т.\,д.), а~также риск
наводнения любого вида.\label{5.8.2.5.1.1-End}

\stepcounter{ParCounter}

\theparagraph.\theParCounter.\label{5.8.2.5.1.2}Хотя вопросы соответствующих
рисков и~их~последствий могут находиться за~пределами непосредственных
знаний и~опыта \emph{оценщика}, наличие или~потенциальное наличие
этих факторов часто может быть установлено в~ходе проведения \textsl{оценки}
путём обычных расспросов или~на~основе знаний о~местности. При~проведении
\textsl{осмотров} может помочь использование соответствующего перечня
контрольных наблюдений из~приложений A--C \href{https://www.rics.org/globalassets/rics-website/media/upholding-professional-standards/sector-standards/land/contamination-environment-sustainability-3rd-edition-rics.pdf}{руководства RICS «Загрязнение, окружающая среда и устойчивое развитие», 3-е издание (2010)}
\cite{RICS:Cont_Env_Sust}. Необходимо учитывать не~только сам~риск
наступления конкретного события, но~и~его~различные последствия.
Например, если объект недавно пострадал в~результате наводнения,
это~может повлиять на~возможность его~страхования в~дальнейшем,
что~должно быть отражено в~\textsl{оценке} в~случае существенности
данного обстоятельства.\label{5.8.2.5.1.2-End}

\stepcounter{ParCounter}

\theparagraph.\theParCounter.\label{5.8.2.5.1.3} \emph{Оценщик}
обязан внимательно описать ограничения объёма проводимых им~исследований,
а~также \textsl{допущения}, касающиеся экологических вопросов, привести
перечень всех~использованных источников информации и~данных.\label{5.8.2.5.1.3-End}\label{par:5.8.2.5.1_Natural_environmental_constraint-End}

\paragraph{Экологические ограничения неприродного характера (загрязнения и~опасные
вещества)\label{par:5.8.2.5.2_Non_natural_constraints}}

\stepcounter{ParCounter}

\theparagraph.\theParCounter.\label{5.8.2.5.2.1} Как~правило, \emph{оценщик}
не~обладает соответствующей квалификацией для~того, чтобы давать
консультации по~вопросам характера загрязнения или~опасных веществ
либо рисках возникновения таковых, а~также о~каких-либо затратах,
связанных с~их~удалением, за~исключением наиболее простых случаев.
Однако в~тех случаях, когда \emph{оценщик} обладает достаточными
знаниями о~местности и~имеет опыт \textsl{оценки} данного типа недвижимости,
от~него можно обоснованно ожидать комментариев относительно возможного
загрязнения и~влияния, которое оно~может оказать на~стоимость \emph{объекта
оценки} и~его привлекательность с~точки зрения рынка.\label{5.8.2.5.2.1-End}

\stepcounter{ParCounter}

\theparagraph.\theParCounter.\label{5.8.2.5.2.2} Само собой, характер
загрязнения либо риски его~возникновения могут быть непосредственно
связаны с~деятельностью, осуществляемой на~\emph{объекте}. Например,
ряд предприятий зависят от~деятельности, связанной с использованием
опасных веществ, либо сами осуществляют деятельность по~обращению
с~отходами, которая с~точки зрения \textsl{третьих лиц} может восприниматься
как~неудобство. Хотя подробные комментарии о~таких \uline{эффектах}
как~правило выходят за~пределы компетенции \emph{оценщика}, \uline{их}~наличие
либо угроза возникновения зачастую могут быть установлены в~ходе
проведения \textsl{оценки} путём обычных расспросов либо на~основании
общих знаний о~местности.\label{5.8.2.5.2.2-End}

\stepcounter{ParCounter}

\theparagraph.\theParCounter.\label{5.8.2.5.2.3} Необходимо указать
пределы проводимых исследований, привести перечень использованных
источников информации и~данных, а~также перечислить все~введённые
\emph{оценщиком} \textsl{допущения}. Любые выявленные сведения о~прежнем
либо текущем использовании можно фиксировать в~\emph{перечне контрольных
наблюдений} из~приложений A--C \href{https://www.rics.org/globalassets/rics-website/media/upholding-professional-standards/sector-standards/land/contamination-environment-sustainability-3rd-edition-rics.pdf}{руководства RICS «Загрязнение, окружающая среда и устойчивое развитие», 3-е издание (2010)}
\cite{RICS:Cont_Env_Sust}.\label{5.8.2.5.2.3-End}\label{par:5.8.2.5.2_Non_natural_constraints-End}

\paragraph{Устойчивое развитие "--- определение его~влияния на~стоимость\label{par:5.8.2.5.3_Sustainability}}

\stepcounter{ParCounter}

\theparagraph.\theParCounter.\label{5.8.2.5.3.1} Хотя этот термин
ещё~не~имеет общепризнанного определения (см.~\hyperref[chap:2_Glossary]{Глоссарий RICS}
\vpageref{chap:2_Glossary}--\pageref{chap:2_Glossary-End}), в~контексте
\textsl{оценки} \textsl{устойчивое развитие} подразумевает под~собой
широкий спектр физических, социальных, экологических и~экономических
факторов, способных оказывать влияние на~стоимость, которые \emph{оценщики}
должны иметь ввиду.\label{5.8.2.5.3.1-End}

\stepcounter{ParCounter}

\theparagraph.\theParCounter.\label{5.8.2.5.3.2} Перечень вопросов,
относящихся к~проблематике \textsl{\href{https://en.wikipedia.org/wiki/Sustainability}{устойчивого развития}~\cite{Wiki:sustainability,Investo:sustainability,RICS:Cont_Env_Sust}}
в~контексте \textsl{оценки}, включает рассмотрение вопросов ключевых
экологических рисков, таких~как~наводнения, вопросы энергоэффективности
и~климата, текущего и~исторического землепользования, а~также вопросы
проектирования, конфигурации, доступной среды, законодательства, управления
и~фискальных соображений, но~не~ограничивается этим. Поскольку
коммерческие рынки, в~частности, становятся всё~более чувствительными
к~вопросам \textsl{\href{https://en.wikipedia.org/wiki/Sustainability}{устойчивого развития}~\cite{Wiki:sustainability,Investo:sustainability,RICS:Cont_Env_Sust}},
последние могут вскоре встать в~один ряд с~традиционными факторами
стоимости, как~с~точки зрения предпочтений арендаторов, так~и~с~точки
зрения поведения покупателей.\label{5.8.2.5.3.2-End}

\stepcounter{ParCounter}

\theparagraph.\theParCounter.\label{5.8.2.5.3.3} Темпы роста значения
факторов \textsl{устойчивого развития} на~стоимость существенно различаются
в~зависимости от~юрисдикции. \emph{Оценщики} должны постоянно стремиться
к~расширению своих знаний для~того чтобы быть способными адекватно
реагировать на~изменения рынка. Роль \emph{оценщиков} заключается
в~\textsl{оценке} стоимости в~свете доказательств, получаемых, как~правило,
путём анализа сопоставимых сделок. Хотя \emph{оценщики} должны отражать
состояние рынка, а~не~управлять им, они~обязаны знать об~особенностях
\textsl{\href{https://en.wikipedia.org/wiki/Sustainability}{устойчивого развития}~\cite{Wiki:sustainability,Investo:sustainability,RICS:Cont_Env_Sust}}
и~последствиях, которые его~аспекты могут иметь для~стоимости объекта
недвижимости в~краткосрочной, среднесрочной и~долгосрочной перспективе.
Среди таких аспектов можно выделить:
\begin{itemize}
\item экологические вопросы (см.~выше~пар.~\ref{par:5.8.2.5.1_Natural_environmental_constraint}
\vpageref{par:5.8.2.5.1_Natural_environmental_constraint}--\pageref{par:5.8.2.5.1_Natural_environmental_constraint-End}),
включая, где~это~применимо, вопросы \href{https://en.wikipedia.org/wiki/Climate_change}{изменения климата}~\cite{Wiki:climate_change}
и~устойчивости к~нему;
\item вопросы планировки и~дизайна, включая использование материалов и~концепций,
которые всё~больше ассоциируются с~понятием \guillemotleft благополучие
человека\guillemotright ;
\item доступность и~адаптируемость, включая доступ и~возможность использования
со~стороны людей с~ограниченными возможностями;
\item вопросы энергоэффективности, наличие систем интеллектуального управления
зданием и~иные вопросы стоимости его~эксплуатации;
\item вопросы налогообложения.\label{5.8.2.5.3.3-End}
\end{itemize}
\stepcounter{ParCounter}

\theparagraph.\theParCounter.\label{5.8.2.5.3.4} Независимо от~\uline{их}~текущего
влияния на~стоимость, в~целях обеспечения сопоставимости в~будущем
\emph{оценщикам} настоятельное рекомендуется выявлять и~собирать
\uline{данные}, связанные с~вопросами \textsl{\href{https://en.wikipedia.org/wiki/Sustainability}{устойчивого развития}~\cite{Wiki:sustainability,Investo:sustainability,RICS:Cont_Env_Sust}}
по~мере \uline{их}~появления.\label{5.8.2.5.3.4-End}

\stepcounter{ParCounter}

\theparagraph.\theParCounter.\label{5.8.2.5.3.5} Характеристики
объекта, связанные с~\emph{\href{https://en.wikipedia.org/wiki/Sustainability}{устойчивым развитием}}~\cite{Investo:sustainability,Wiki:sustainability,RICS:Cont_Env_Sust},
могут оказывать прямое влияние на~определённую \emph{оценщиком} стоимость
только в~тех~случаях, когда существующие рыночные данные подтверждают
наличие такого влияние, либо когда, по~мнению \emph{оценщика}, участники
рынка прямо отражают такие вопросы в~своих предложениях.\label{5.8.2.5.3.5-End}

\stepcounter{ParCounter}

\theparagraph.\theParCounter.\label{5.8.2.5.3.6} \emph{Оценщиков}
часто просят дать дополнительные комментарии и~советы по~вопросам
стратегии. В~таких ситуациях может быть целесообразным проведением
консультации с~заказчиком по~вопросам использования и~применимости
показателей и~контрольных параметров \emph{\href{https://en.wikipedia.org/wiki/Sustainability}{устойчивого развития}}~\cite{Investo:sustainability,Wiki:sustainability},
являющихся применимыми в~каждом конкретном случае. Например, при~подготовке
\textsl{оценок}, в~которых проводится определение \textsl{\hyperref[2.1.1_Investment_value]{testsl{инвестиционной стоимости (ценности)}}}
могут быть надлежащим образом учтены факторы \emph{\href{https://en.wikipedia.org/wiki/Sustainability}{устойчивого развития}}~\cite{Investo:sustainability,Wiki:sustainability},
способные оказать влияние на~принятие инвестиционных решений, даже
если эти~факторы не~имеют прямого подтверждения данными сделок,
совершённых на~рынке.\label{5.8.2.5.3.6-End}

\stepcounter{ParCounter}

\theparagraph.\theParCounter.\label{5.8.2.5.3.7} В~целях обеспечения
соответствия \textsl{оценки} передовой практике, там~где~это уместно,
\emph{оценщикам} рекомендуется:
\begin{itemize}
\item определить, в~какой степени в~настоящее время оцениваемый объект
недвижимости соответствует \uline{критериям} \emph{\href{https://en.wikipedia.org/wiki/Sustainability}{устойчивого развития}}~\cite{Investo:sustainability,Wiki:sustainability},
ожидаемым для~объектов, относящихся к~тому~же сегменту рынка, и~придти
к~обоснованному мнению о~вероятности их~влияния на~его~стоимость,
т.\,е. определить, как~и~каким образом хорошо информированный покупатель
будет учитывать \uline{их}~при~принятии решения об~обоснованности
цены предложения;
\item представить в~\emph{отчёте} описание выявленных характеристик и~атрибутов
объекта недвижимости, связанных с~\textsl{\guillemotleft устойчивостью\guillemotright},
которое может, при~необходимости, включать элементы, не~отражённые
непосредственно в~описании объекта в~контексте ценообразующих факторов;
\item изложить своё видение в~части взаимосвязи между факторами \textsl{\guillemotleft устойчивости\guillemotright}
и~своим итоговым \emph{суждением о~стоимости}, включая комментарий
о~текущих выгодах~(рисках), связанных с~этими характеристиками
\textsl{\guillemotleft устойчивости\guillemotright}, либо об~отсутствии
таковых рисков;
\item выразить мнение о~потенциальном влиянии этих выгод и~(или) рисков
на~стоимость объекта недвижимости с~течением времени.\label{5.8.2.5.3.7-End}
\end{itemize}
\stepcounter{ParCounter}

\theparagraph.\theParCounter.\label{5.8.2.5.3.8} \href{https://www.rics.org/globalassets/rics-website/media/upholding-professional-standards/sector-standards/land/contamination-environment-sustainability-3rd-edition-rics.pdf}{Руководство RICS «Загрязнение, окружающая среда и устойчивое развитие», 3-е издание (2010)}
\cite{RICS:Cont_Env_Sust} содержит рекомендации по~выявлению и~определению
влияния вопросов \textsl{\guillemotleft устойчивости\guillemotright}
на~\textsl{оценку} коммерческой недвижимости.\label{5.8.2.5.3.8-End}\label{par:5.8.2.5.3_Sustainability-End}\label{subsubsec:5.8.2.5_Environmental_matters-End}\label{subsec:5.8.2_Investigations_and_assumptions-End}\label{sec:5.8_VPGA-8_Valuation_of_real_property-End}

\newpage

\section{ПР~9. Выявление портфелей, коллекций и~групп активов и~имущества\label{sec:5.9_VPGA-9_Identification_of_portfolios}}

\textbf{Данное руководство носит рекомендательный характер и~не~содержит
обязательных требований. Однако там, где~это~уместно, оно~содержит
отсылки к~соответствующему обязательному материалу, содержащемуся
в~других разделах настоящих }\textbf{\emph{Всемирных стандартов}}\textbf{,
а~также \href{https://www.rics.org/globalassets/rics-website/media/upholding-professional-standards/sector-standards/valuation/international-valuation-standards-rics2.pdf}{Международных стандартов оценки}~\cite{IVS-2020},
реализованные в~виде перекрёстных ссылок. Данные ссылки предназначены
для~помощи }\textbf{\textsl{членам RICS}}\textbf{ и~не~меняют статус
данного практического руководства. }\textbf{\textsl{Членам RICS}}\textbf{
необходимо помнить следующее:}
\begin{itemize}
\item \textbf{данное руководство не~может охватить все~возможные варианты,
вследствие чего }\textbf{\emph{оценщикам}}\textbf{ при~формировании
своих }\textbf{\emph{суждений о~стоимости}}\textbf{ всегда следует
учитывать факты и~обстоятельства, имеющие место в~рамках отдельных
заданий по~}\textbf{\textsl{оценке}}\textbf{;}
\item \textbf{следует внимательно относиться к~тому факту, что~в~ряде
юрисдикций могут существовать особые требования, не~предусмотренные
данным руководством.}
\end{itemize}

\subsection{Область применения\label{subsec:5.9.1_Scope}}

\stepcounter{SubSecCounter} 

\thesubsection.\theSubSecCounter.\label{5.9.1.1} Данное руководство
содержит дополнительные комментарии по~\uline{вопросам} идентификации
портфелей, коллекций и~групп активов и~имущества и~\uline{их}~изложению
в~\emph{отчёте} в~соответствии с~требованиями раздела~\ref{sec:4.3_VPS3_Valuation_reports}~\nameref{sec:4.3_VPS3_Valuation_reports}
\vpageref{sec:4.3_VPS3_Valuation_reports}--\pageref{sec:4.3_VPS3_Valuation_reports-End}\label{5.9.1.1-End}\label{subsec:5.9.1_Scope-End}

\subsection{Примеры случаев, когда бывает необходимо введение допущения о~единой
группе\label{subsec:5.9.2_Examples_specific_clarification}}

\stepcounter{SubSecCounter} 

\thesubsection.\theSubSecCounter.\label{5.9.2.1} Примерами таких
случаев являются:
\begin{itemize}
\item физически смежные объекты, приобретённые нынешним владельцем по~отдельности
(например, когда застройщик собирает большой участок из~мелких с~целью
его~будущего развития, либо когда инвестор создаёт для~себя стратегическую
долю на~определённой территории);
\item физически отдельные объекты недвижимости, используемые одним и~тем~же
хозяйствующим субъектом, между которыми существует функциональная
зависимость (например, автостоянка, расположенная за~пределами здания,
но~используемая исключительно его~владельцем);
\item ситуация, когда владение рядом отдельных объектов недвижимости либо
активов даёт особую выгоду их~общему владельцу или арендатору вследствие
экономии, возникающей из-за эффекта масштаба в~результате увеличения
доли рынка, или~экономии на~администрировании, или~распределении
ресурсов, как~в~случае с~многоквартирными домами или~гостиницами;
\item ситуация, когда каждый отдельный объект является важным компонентом
деятельности, охватывающей большую с~географической точки зрения
территорию (например, частью национальной или~региональной инфраструктурной
сети, такой как~вышки, предназначенные для~размещения телекоммуникационного
оборудования).\label{5.9.2.1-End}\label{subsec:5.9.2_Examples_specific_clarification-End}
\end{itemize}

\subsection{Цель оценки и~идентификация лотов в~составе портфеля\label{subsec:5.9.3_Purpose_of_valuation}}

\stepcounter{SubSecCounter} 

\thesubsection.\theSubSecCounter.\label{5.9.3.1} \emph{Цель оценки}
может предопределять применимые действия. В~частности, может существовать
требование указывать отдельно стоимость каждого актива. Состав таких
единиц либо активов следует уточнять у~заказчика.\label{5.9.3.1-End}

\stepcounter{SubSecCounter} 

\thesubsection.\theSubSecCounter.\label{5.9.3.2} Запросы на~проведение
\textsl{оценки}, основанной на~искусственном разделении объектов
на лоты, как~правило, следует отклонять. Тем~не~менее в~ряде случаев
нетипичная группировка объектов может приниматься на~основе \textsl{специального
допущения} (см.~\ref{subsec:4.4.9_Reflection_of_market_constraint}~\nameref{subsec:4.4.9_Reflection_of_market_constraint}
\vpageref{subsec:4.4.9_Reflection_of_market_constraint}--\pageref{subsec:4.4.9_Reflection_of_market_constraint-End}).\label{5.9.3.2-End}

\stepcounter{SubSecCounter} 

\thesubsection.\theSubSecCounter.\label{5.9.3.3} После идентификации
в~портфеле лотов, подлежащих обособленной \textsl{оценке}, \emph{оценщику}
необходимо рассмотреть конкретные необходимые \textsl{допущения} либо
\textsl{специальные допущения}, которые должны быть зафиксированы
в~условиях \textsl{договора на~проведение оценки} (см~раздел~\ref{sec:4.1_VPS1_Terms_of_engagement_Scope_of_work}~\nameref{sec:4.1_VPS1_Terms_of_engagement_Scope_of_work}
\vpageref{sec:4.1_VPS1_Terms_of_engagement_Scope_of_work}--\pageref{sec:4.1_VPS1_Terms_of_engagement_Scope_of_work-End}),
а~также в~тексте \emph{отчёта} (см.~\ref{sec:4.3_VPS3_Valuation_reports}~\nameref{sec:4.3_VPS3_Valuation_reports}
\vpageref{sec:4.3_VPS3_Valuation_reports}--\pageref{sec:4.3_VPS3_Valuation_reports-End}).
Примеры ситуаций, когда различные \textsl{допущения} могут оказать
существенное влияние на~\textsl{оценку} портфеля, рассматриваются
в~следующих пунктах.\label{5.9.3.3-End}

\stepcounter{SubSecCounter} 

\thesubsection.\theSubSecCounter.\label{5.9.3.4} В~том~случае,
если весь портфель или~значительное количество входящих в~него объектов
недвижимости будут выставлены на~продажу одновременно, это~может
привести к~избытку предложения на~рынке и, как~следствие, "---
к~снижению стоимости. В~то~же~время, возможность приобрести определённую
группу объектов недвижимости целиком может дать премию к~их~стоимости.
Иными словами, стоимость целого может как~превышать сумму стоимостей
отдельных частей так~и~быть меньше неё.\label{5.9.3.4-End}

\stepcounter{SubSecCounter} 

\thesubsection.\theSubSecCounter.\label{5.9.3.5} В~случаях проведения
\textsl{оценки} в~целях, предполагающих сохранение текущих прав на~портфель
активов, например, для~включения в~\textsl{финансовую отчётность},
внесение скидок, отражающих потенциальный эффект избытка предложения
на~рынке, является неуместным. Необходимо сделать соответствующее
заявление в~\emph{отчёте}.\label{5.9.3.5-End}

\stepcounter{SubSecCounter} 

\thesubsection.\theSubSecCounter.\label{5.9.3.6} В~то~же~время,
если тот~же портфель будет оцениваться в~качестве залога в~целях
залогового кредитования, нельзя игнорировать эффект возможного негативного
влияния на~стоимость отдельных объектов недвижимости, в~случае единовременного
выставления на~рынок всего портфеля. В~таких случаях обычно уместно
привести в~\emph{отчёте} \textsl{допущение} о~том, что~имущество
будет продаваться упорядоченно без~единовременного вывода на~рынок
всей его~совокупности. Однако, в~случае наличия обстоятельств, препятствующих
принятию такого \textsl{допущения} рынком, например при~наличии общедоступных
сведений о~том, что~нынешний собственник имущества испытывает финансовые
трудности, необходимо ввести соответствующее \textsl{специальное допущение}
и~чётко описать его~влияние на~\textsl{оценку} (см.~подраздел~\ref{subsec:4.4.8_Special_assumptions}~\nameref{subsec:4.4.8_Special_assumptions}
\vpageref{subsec:4.4.8_Special_assumptions}--\pageref{subsec:4.4.8_Special_assumptions-End}).\label{5.9.3.6-End}

\stepcounter{SubSecCounter} 

\thesubsection.\theSubSecCounter.\label{5.9.3.7} Аналогичным образом,
в~тех~случаях, когда \emph{оценщик} определяет единую стоимость
для~группы отдельных объектов, в~тексте \emph{отчёта} следует привести
все~\textsl{допущения}, необходимые для~обоснования такого решения.
Если \emph{оценщик} считает, что~такое восприятие портфеля отличается
от~того, которое обязательно принял~бы рынок, такое \textsl{допущение}
становится \textsl{специальным допущением} (см.~подраздел~\ref{subsec:4.4.8_Special_assumptions}~\nameref{subsec:4.4.8_Special_assumptions}
\vpageref{subsec:4.4.8_Special_assumptions}--\pageref{subsec:4.4.8_Special_assumptions-End}).\label{5.9.3.7-End}

\stepcounter{SubSecCounter} 

\thesubsection.\theSubSecCounter.\label{5.9.3.8} Если общая стоимость
объектов в~рамках портфеля будет существенно отличаться в~зависимости
от~того, продаются~ли они~по~отдельности, группами или~единым
лотом, данное обстоятельство в~обязательном порядке подлежит отражению
в~\emph{отчёте}. \textsl{Допущения}, касающиеся способа группировки
активов, в~обязательном порядке подлежат включению в~любую ссылку
на~\emph{отчёт}.\label{5.9.3.8-End}

\stepcounter{SubSecCounter} 

\thesubsection.\theSubSecCounter.\label{5.9.3.9} В~случае \textsl{оценки}
портфеля либо~группы объектов имущества или~активов на~основе \textsl{допущения}
об~их~продаже единым лотом, полученное значение \textsl{рыночной
стоимости} будет относиться целиком ко~всему лоту. Любая пообъектная
разбивка его~\textsl{рыночной стоимости} должна быть чётко описана
с~указанием на~то~обстоятельство, что~полученная путём такого
распределения стоимость отдельного объекта~(актива) не~обязательно
соответствует его~\textsl{рыночной стоимости}.\label{5.9.3.9-End}

\stepcounter{SubSecCounter} 

\thesubsection.\theSubSecCounter.\label{5.9.3.10} И~наоборот, в~случае
указания суммы \textsl{рыночных стоимостей} отдельных объектов или~активов,
образующих портфель, необходимо избегать её~интерпретации в~качестве
\textsl{рыночной стоимости} самого портфеля.\label{5.9.3.10-End}\label{subsec:5.9.3_Purpose_of_valuation-End}\label{sec:5.9_VPGA-9_Identification_of_portfolios-End}

\section{ПР~10.~Причины возникновения существенной неопределённости оценки\label{sec:5.10_VPGA-10_Matters_for_uncertainty}}

\textbf{Данное руководство носит рекомендательный характер и~не~содержит
обязательных требований. Однако там, где~это~уместно, оно~содержит
отсылки к~соответствующему обязательному материалу, содержащемуся
в~других разделах настоящих }\textbf{\emph{Всемирных стандартов}}\textbf{,
а~также \href{https://www.rics.org/globalassets/rics-website/media/upholding-professional-standards/sector-standards/valuation/international-valuation-standards-rics2.pdf}{Международных стандартов оценки}~\cite{IVS-2020},
реализованные в~виде перекрёстных ссылок. Данные ссылки предназначены
для~помощи }\textbf{\textsl{членам RICS}}\textbf{ и~не~меняют статус
данного практического руководства. }\textbf{\textsl{Членам RICS}}\textbf{
необходимо помнить следующее:}
\begin{itemize}
\item \textbf{данное руководство не~может охватить все~возможные варианты,
вследствие чего }\textbf{\emph{оценщикам}}\textbf{ при~формировании
своих }\textbf{\emph{суждений о~стоимости}}\textbf{ всегда следует
учитывать факты и~обстоятельства, имеющие место в~рамках отдельных
заданий по~}\textbf{\textsl{оценке}}\textbf{;}
\item \textbf{следует внимательно относиться к~тому факту, что~в~ряде
юрисдикций могут существовать особые требования, не~предусмотренные
данным руководством.}
\end{itemize}

\subsection{Область применения\label{subsec:5.10.1_Scope}}

\stepcounter{SubSecCounter} 

\thesubsection.\theSubSecCounter.\label{5.10.1.1} Данное руководство
содержит дополнительные комментарии по~вопросам, способным привести
к~существенной неопределённости результатов \textsl{оценки} согласно
секции~\ref{subsubsec:4.3.2.15_Commentary_ on_any_material_uncertainty}
\vpageref{subsubsec:4.3.2.15_Commentary_ on_any_material_uncertainty}--\pageref{subsubsec:4.3.2.15_Commentary_ on_any_material_uncertainty-End}.\label{subsec:5.10.1_Scope-End}

\subsection{Примеры\label{subsec:5.10.2_Examples}}

\stepcounter{SubSecCounter} 

\thesubsection.\theSubSecCounter.\label{5.10.2.1} Невозможно представить
исчерпывающий перечень обстоятельств, при~которых может возникнуть
существенная неопределённость, однако примеры в~п.~\ref{5.10.2.2}--\ref{5.10.2.4}
описывают три наиболее типичные ситуации.\label{5.10.2.1-End}

\stepcounter{SubSecCounter} 

\thesubsection.\theSubSecCounter.\label{5.10.2.2} Оцениваемый актив
либо обязательство обладают очень специфическими характеристиками,
затрудняющими формирование \emph{оценщиком} \emph{суждения о~стоимости}
независимо от~используемого \emph{подхода} или~\emph{метода оценки}.
Например \emph{объект оценки} может относиться к~очень редкому типу
активов~(обязательств) либо вовсе быть уникальным. Аналогичным образом,
количественная оценка того, как~потенциальные покупатели будут воспринимать
возможное значительное изменение свойств объекта, например такого
как~вероятное получение разрешения на~строительство, может сильно
зависеть от~введённых \textsl{специальных допущений}.\label{5.10.2.2-End}

\stepcounter{SubSecCounter} 

\thesubsection.\theSubSecCounter.\label{5.10.2.3} В~случае, когда
доступность информации и~данных для~\emph{оценщика} ограничена заказчиком
либо обстоятельствами \textsl{оценки}, и~эта~проблема не~может
быть решена в~достаточной степени путём введения одного или нескольких
разумных \textsl{допущений}, результаты \textsl{оценки} следует считать
менее определёнными, чем~если~бы подобное ограничение не~имело
места.\label{5.10.2.3-End}

\stepcounter{SubSecCounter} 

\thesubsection.\theSubSecCounter.\label{5.10.2.4} Рынки могут быть
дестабилизированы в~силу воздействия на~них ряда факторов, возникающих
вследствие и~по~причине непредвиденных финансовых, макроэкономических,
юридических, политических или~даже природных событий. Если \textsl{дата
оценки} совпадает с~датой такого события либо непосредственно следует
за~ней, уровень достоверности \textsl{оценки} может быть снижен вследствие
противоречивости или~отсутствия эмпирических данных либо вследствие
того, что~\emph{оценщик} вынужден основывать своё \emph{суждение
о~стоимости} на~таком наборе обстоятельств, который практически
невозможно соотнести с~любыми другими данными. В~таких ситуациях
выполнение требований к~действиям \emph{оценщиков} становится весьма
нетривиальной задачей. Хотя \emph{оценщики} по-прежнему могут выносить
своё суждение, крайне важно, чтобы его~контекст был чётко описан
в~\emph{отчёте}.\label{5.10.2.4-End}\label{subsec:5.10.2_Examples-End}

\subsection{Требования к~содержанию отчёта об~оценке\label{subsec:5.10.3_Reporting}}

\stepcounter{SubSecCounter} 

\thesubsection.\theSubSecCounter.\label{5.10.3.1} Главным требованием
является то, что~\emph{отчёт об~оценке} не~должен вводить в~заблуждение
или создавать ложное впечатление. \emph{Оценщик} должен отдельно обратить
внимание читателя \emph{отчёта} на~любые вопросы, приводящие к~существенной
неопределённости в~\textsl{оценке} на~указанную \textsl{дату оценки},
и~прокомментировать их. Такие \uline{комментарии} не~должны относиться
к~вопросам общего риска будущих изменений рынка либо риска, связанного
с~прогнозированием будущих денежных потоков: оба~эти~фактора могут
и~должны рассматриваться и~отражаться как~часть процесса \textsl{оценки}
(например, \textsl{оценка} инвестиционной недвижимости основана на~существенной
неопределённости будущих денежных потоков, однако она~может быть
подкреплена детальными и~последовательными релевантными данными о~сделках),
но~\uline{должны охватывать} вопросы рисков, являющихся внешними
по~отношению к~процессу \textsl{оценки} конкретного актива~(обязательства).\label{5.10.3.1-End}

\stepcounter{SubSecCounter} 

\thesubsection.\theSubSecCounter.\label{5.10.3.2} В~случае наличия
существенной неопределённости, как правило, она~описывается качественными,
а~не~количественными терминами, содержащими сведения об~уверенности
\emph{оценщика} в~его~\emph{суждении о~стоимости}, выраженными
в~словесной форме. В~действительности, такой подход может быть единственным
реалистичным способом описать неопределённость, вследствие и~по~причине
того, что~сами условия, создающие неопределённость \textsl{оценки},
часто предполагают отсутствие эмпирических данных, необходимых для~обоснования
её~количественной оценки.\label{5.10.3.2-End}

\stepcounter{SubSecCounter} 

\thesubsection.\theSubSecCounter.\label{5.10.3.3} В~большинстве
случаев количественная оценка неопределённости неуместна либо нецелесообразно,
а~любая попытка её~проведения может дать противоречивые результаты.
В~случае использования математических методов измерения неопределённости,
крайне важно, чтобы используемый метод или~модель были адекватно
объяснены в~\emph{отчёте}, включая акцентирование внимания на~ограничениях
полученного значения. В~некоторых ограниченных случаях может быть
целесообразным проведение анализа чувствительности, позволяющего продемонстрировать
влияние, которое чётко оговорённые изменения значений определённых
переменных могут оказать на~итоговый результат \textsl{оценки}, который
при~этом должен сопровождаться соответствующими разъяснениями и~комментариями.
Необходимо понимать, что~любая попытка количественной оценки неопределённости
несёт в~себе риск возникновения впечатления её~точности, способного
ввести читателя \emph{отчёта} в~заблуждение.\label{5.10.3.3-End}

\stepcounter{SubSecCounter} 

\thesubsection.\theSubSecCounter.\label{5.10.3.4} В~других случаях,
когда \emph{оценщик} может обоснованно предполагать, что~при различных,
но~чётко определённых обстоятельствах могут возникнуть разные стоимости,
альтернативный подход заключается в~том, чтобы \emph{оценщик} обсудил
с~заказчиком вариант проведения ряда \textsl{оценок}, основанных
на~использовании \textsl{специальных допущений}, отражающих эти~различные
обстоятельства. Однако \textsl{специальные допущения} могут быть использованы
только тогда, когда они~могут считаться реалистичными, уместными
и~соответствующими обстоятельствам \textsl{оценки}. В~случае возникновения
разных стоимостей в~зависимости от~различных обстоятельств, возможно
формирование отдельных \emph{отчётов об~оценке}, основанных на~соответствующих
\textsl{специальных допущениях}.\label{5.10.3.4-End}

\stepcounter{SubSecCounter} 

\thesubsection.\theSubSecCounter.\label{5.10.3.5} Наличие в~\emph{отчёте}
стандартной оговорки, касающейся неопределённости результатов \textsl{оценки},
как~правило, является неприемлемым. Степень неопределённости результатов
зависит от~обстоятельств конкретной \textsl{оценки}, следовательно,
использование стандартных оговорок может обесценить либо~поставить
под~сомнение авторитетность данной рекомендации. Задача состоит в~том,
чтобы \emph{отчёт} содержал авторитетное и~взвешенное мнение в~отношении
конкретного \emph{объекта оценки}. Вопросы, оказывающие влияние на~степень
уверенности в~результатах \emph{оценки}, следует излагать именно
в~этом контексте.\label{5.10.3.5-End}

\stepcounter{SubSecCounter} 

\thesubsection.\theSubSecCounter.\label{5.10.3.6} В~отсутствие
специального указания на~это, выражение итоговой стоимости в~виде
диапазона значений не~является хорошей практикой и, чаще всего, не~является
приемлемой формой \emph{раскрытия информации}. В~большинстве случаев
требования заказчика и~условий \textsl{договора на~проведение оценки}
предполагают выражение результата \textsl{оценки} в~виде конкретной
цифры. Аналогичным образом, наличие уточняющих фраз таких~как \guillemotleft в~районе\guillemotright{}
перед итоговым значением стоимости, как~правило, не~является уместным
или~адекватным способом выражение влияния существующей неопределённости
без~дополнительных явных комментариев, вследствие и~по~причине
чего следует избегать их~применения. В~тех случаях, когда существование
различных обстоятельств приводит к~различным значениями стоимости,
предпочтительнее указывать каждое из~них совместно с~уместными \textsl{специальными
допущениями}, описывающими такие обстоятельства (см.~п.~\ref{5.10.3.4}
выше).\label{5.10.3.6-End}

\stepcounter{SubSecCounter} 

\thesubsection.\theSubSecCounter.\label{5.10.3.7} Следует иметь
виду, что~стандарты \textsl{финансовой отчётности} могут содержать
и~часто содержат конкретные требования к~\emph{раскрытию информации}
в~отношении неопределённости \textsl{оценки}, хотя сам этот термин
может и~не~использоваться прямо. Соблюдение таких требований является
обязательным в~применимых случаях.\label{5.10.3.7-End}\label{subsec:5.10.3_Reporting-End}\label{sec:5.10_VPGA-10_Matters_for_uncertainty-End}\label{chap:5_Valuation-applications-End}\label{part:I_RVGS/RBG-End}

\setcounter{chapter}{0}

\part{Международные стандарты оценки\label{part:II_IVS}}

Вступили в~силу с~31~января 2020~г. Приняты и~адаптированы к~применению
со~стороны \textsl{членов RICS} посредством соблюдения требований
настоящих \emph{Всемирных стандартов}. Главы~\ref{chap:3_PS}--\ref{chap:5_Valuation_applications}
содержат перекрёстные ссылки на~настоящие стандарты.

\textsl{Членам RICS} необходимо иметь ввиду, что~\href{https://www.ivsc.org/}{Совет по международным стандартам оценки}~\cite{IVSC:site}
оставляет за~собой право на~их~дальнейшее развитие в~любой момент.
Любые последующие поправки данных \emph{Всемирных стандартов} будут
вноситься в~порядке, предусмотренном~п.~\ref{1.2.5.1.2} \vpageref{1.2.5.1.2}--\pageref{1.2.5.1.2-End}.

\newpage

Дальнейший текст МСО приводится на~английском языке без~каких-либо
правок со~стороны переводчика. Возможна автоматическая правка правописания
в~соответствии с~правилами американского варианта английского языка,
а~также замена ''английских'' кавычек на~\guillemotleft французские\guillemotright .
\selectlanguage{english}%

\chapter{Introduction\label{chap:6_Introduction}}

The~International Valuation Standards Council~(IVSC) is~an~independent,
not-for-profit organization committed to~advancing quality in~the~valuation
profession. Our~primary objective is~to~build confidence and~public
trust in~valuation by~producing standards and~securing their universal
adoption and~implementation for~the~valuation of~assets across
the~world. We~believe that International Valuation Standards~(IVS)
are~a~fundamental part of~the~financial system, along with high
levels of~professionalism in~applying them.

Valuations are~widely used and~relied upon in financial and other
markets, whether for inclusion in financial statements, for regulatory
compliance or to support secured lending and transactional activity.
The International Valuation Standards (IVS) are standards for undertaking
valuation assignments using generally recognized concepts and principles
that promote transparency and consistency in valuation practice. The
IVSC also promotes leading practice approaches for the conduct and
competency of professional valuers.

The IVSC Standards Board is the body responsible for setting the IVS.
The Board has autonomy in the development of its agenda and approval
of its publications. In developing the IVS, the Board:
\begin{itemize}
\item follows established due process in the development of any new standard,
including consultation with stakeholders (valuers, users of valuation
services, regulators, valuation professional organizations, etc) and
public exposure of all new standards or material alterations to existing
standards;
\item liaises with other bodies that have a standard-setting function in
the financial markets;
\item conducts outreach activities including round-table discussions with
invited constituents and targeted discussions with specific users
or user groups.
\end{itemize}
The objective of the IVS is to increase the confidence and trust of
users of valuation services by establishing transparent and consistent
valuation practices. A standard will do one or more of the following:
\begin{itemize}
\item identify or develop globally accepted principles and definitions;
\item identify and promulgate considerations for the undertaking of valuation
assignments and the reporting of valuations;
\item identify specific matters that require consideration and methods commonly
used for valuing different types of assets or liabilities.
\end{itemize}
The IVS consist of mandatory requirements that must be followed in
order to state that a valuation was performed in compliance with the
IVS. Certain aspects of the standards do not direct or mandate any
particular course of action, but provide fundamental principles and
concepts that must be considered in undertaking a valuation.

The IVS are arranged as follows:

\textbf{The IVS Framework}

This serves as a preamble to the IVS. The IVS Framework consists of
general principles for valuers following the IVS regarding objectivity,
judgement, competence and acceptable departures from the IVS. 

\textbf{IVS General Standards}

These set forth requirements for the conduct of all valuation assignments
including establishing the terms of a valuation engagement, bases
of value, valuation approaches and methods, and reporting. They are
designed to be applicable to valuations of all types of assets and
for any valuation purpose. 

\textbf{IVS Asset Standards}

The Asset Standards include requirements related to specific types
of assets. These requirements must be followed in conjunction with
the General Standards when performing a valuation of a specific asset
type. The Asset Standards include certain background information on
the characteristics of each asset type that influence value and additional
asset-specific requirements on common valuation approaches and methods
used.

\textbf{What is the Effective Date?} 

This version of International Valuation Standards is published on
31 July 2019, with an effective date of 31 January 2020. The IVSC
permits early adoption from the date of publication. 

\textbf{Future Changes to these Standards} 

The IVSC Standards Board intends to continuously review the IVS and
update or clarify the standards as needed to meet stakeholder and
market needs. The Board has continuing projects that may result in
additional standards being introduced or amendments being made to
the standards in this publication at any time. News on current projects
and any impending or approved changes can be found on the IVSC website
at \href{http://www.ivsc.org}{www.ivsc.org}. 

An FAQ document in relation to International Valuation Standards is
available at \href{http://www.ivsc.org}{www.ivsc.org}.\label{chap:6_Introduction-End}

\chapter{Glossary\label{chap:7_Glossary}}

\section{Overview of Glossary\label{sec:7.1_Gloss_overview}}

\stepcounter{SecCounter} 

\thesection.\theSecCounter.\label{7.1.1} This glossary defines certain
terms used in the International Valuation Standards.\label{7.1.1-End}

\stepcounter{SecCounter} 

\thesection.\theSecCounter.\label{7.1.2} This glossary is only applicable
to the International Valuation Standards and does not attempt to define
basic valuation, accounting or finance terms, as valuers are assumed
to have an understanding of such terms (see definition of “valuer”).\label{7.1.2-End}\label{sec:7.1_Gloss_overview-End}

\section{Defined Terms\label{sec:7.2_Defined_terms}}
\begin{description}
\item [{Asset~or~Assets.}] To assist in the readability of the standards
and to avoid repetition, the words \guillemotleft asset\guillemotright{}
and \guillemotleft assets\guillemotright{} refer generally to items
that might be subject to a valuation engagement. Unless otherwise
specified in the standard, these terms can be considered to mean \guillemotleft asset,
group of assets, liability, group of liabilities, or group of assets
and liabilities\guillemotright .
\item [{Client.}] The word \guillemotleft client\guillemotright{} refers
to the person, persons, or entity for whom the valuation is performed.
This may include external clients (ie, when a valuer is engaged by
a third-party client) as well as internal clients (ie, valuations
performed for an employer). 
\item [{Intended~Use.}] The use(s) of a valuer's reported valuation or
valuation review results, as identified by the valuer based on communication
with the client.
\item [{Intended~User.}] The client and any other party as identified,
by name or type, as users of the valuation or valuation review report
by the valuer based on communication with the client.
\item [{Jurisdiction.}] The word \guillemotleft jurisdiction\guillemotright{}
refers to the legal and regulatory environment in which a valuation
engagement is performed. This generally includes laws and regulations
set by governments (eg, country, state and municipal) and, depending
on the purpose, rules set by certain regulators (eg, banking authorities
and securities regulators). 
\item [{May.}] The word \guillemotleft may\guillemotright{} describes actions
and procedures that valuers have a responsibility to consider. Matters
described in this fashion require the valuer’s attention and understanding.
How and whether the valuer implements these matters in the valuation
engagement will depend on the exercise of professional judgment in
the circumstances consistent with the objectives of the standards. 
\item [{Must.}] The word \guillemotleft must\guillemotright{} indicates
an unconditional responsibility. The valuer must fulfill responsibilities
of this type in all cases in which the circumstances exist to which
the requirement applies. 
\item [{Participant.}] The word \guillemotleft participant\guillemotright{}
refers to the relevant participants pursuant to the basis (or bases)
of value used in a valuation engagement (see IVS 104 Bases of Value).
Different bases of value require valuers to consider different perspectives,
such as those of \guillemotleft market participants\guillemotright{}
(e.\,g, Market Value, IFRS Fair Value) or a particular owner or prospective
buyer (e.\,g, Investment Value). 
\item [{Purpose.}] The word \guillemotleft purpose\guillemotright{} refers
to the reason(s) a valuation is performed. Common purposes include
(but are not limited to) financial reporting, tax reporting, litigation
support, transaction support, and to support secured lending decisions. 
\item [{Should.}] The word \guillemotleft should\guillemotright{} indicates
responsibilities that are presumptive mandatory. The valuer must comply
with requirements of this type unless the valuer demonstrates that
alternative actions which were followed under the circumstances were
sufficient to achieve the objectives of the standards. In the rare
circumstances in which the valuer believes the objectives of the standard
can be met by alternative means, the valuer must document why the
indicated action was not deemed to be necessary and/or appropriate.
If a standard provides that the valuer \guillemotleft should\guillemotright{}
consider an action or procedure, consideration of the action or procedure
is presumptive mandatory, while the action or procedure is not. 
\item [{Significant~and/or~Material.}] Assessing significance and materiality
require professional judgement. However, that judgement should be
made in the following context:
\item [{\begin{itemize}
\item Aspects of a valuation (including inputs, assumptions, special assumptions, and methods and approaches applied) are considered to be significant/material if their application and/or impact on the valuation could reasonably be expected to influence the economic or other decisions of users of the valuation; and judgments about materiality are made in light of the overall valuation engagement and are affected by the size or nature of the subject asset.;
\item As used in these standards, “material/materiality” refers to materiality to the valuation engagement, which may be different from materiality considerations for other purposes, such as financial statements and their audits.
\end{itemize}}]~
\item [{Subject~or~Subject~Asset.}] These terms refer to the asset(s)
valued in a particular valuation engagement. 
\item [{Valuation.}] A \guillemotleft valuation\guillemotright{} refers
to the act or process of determining an estimate of value of an asset
or liability by applying IVS.
\item [{Valuation~Purpose~or~Purpose~of~Valuation.}] See \guillemotleft Purpose\guillemotright .
\item [{Valuation~Reviewer.}] A \guillemotleft valuation reviewer\guillemotright{}
is a professional valuer engaged to review the work of another valuer.
As part of a valuation review, that professional may perform certain
valuation procedures and/or provide an opinion of value. 
\item [{Value~(n).}] The word \guillemotleft value\guillemotright{} refers
to the judgement of the valuer of the estimated amount consistent
with one of the bases of value set out in IVS 104 Bases of Value.
\item [{Valuer.}] A “valuer” is an individual, group of individuals or
a firm who possesses the necessary qualifications, ability and experience
to execute a valuation in an objective, unbiased and competent manner.
In some jurisdictions, licensing is required before one can act as
a valuer. 
\item [{Weight.}] The word “weight” refers to the amount of reliance placed
on a particular indication of value in reaching a conclusion of value
(eg, when a single method is used, it is afforded 100\% weight). 
\item [{Weighting.}] The word “weighting” refers to the process of analysing
and reconciling differing indications of values, typically from different
methods and/or approaches. This process does not include the averaging
of valuations, which is not acceptable.\label{sec:7.2_Defined_terms-End}\label{chap:7_Glossary-End}
\end{description}

\chapter{IVS Framework\label{chap:8_IVS_Framework}}

\section{Compliance with Standards\label{sec:8.1_Compliance_with_standards}}

\stepcounter{SecCounter} 

\thesection.\theSecCounter.\label{8.1.1} When a statement is made
that a valuation will be, or has been, undertaken in accordance with
the IVS, it is implicit that the valuation has been prepared in compliance
with all relevant standards issued by the IVSC.\label{8.1.1-End}\label{sec:8.1_Compliance_with_standards-End}

\section{Assets and Liabilities\label{sec:8.2_Assets_and_Liabilities}}

\stepcounter{SecCounter} 

\thesection.\theSecCounter.\label{8.2.1} The standards can be applied
to the valuation of both assets and liabilities. To assist the legibility
of these standards, the words asset or assets have been defined to
include liability or liabilities and groups of assets, liabilities,
or assets and liabilities, except where it is expressly stated otherwise,
or is clear from the context that liabilities are excluded.\label{8.2.1-End}\label{sec:8.2_Assets_and_Liabilities-End}

\section{Valuer\label{sec:8.3_Valuer}}

\stepcounter{SecCounter} 

\thesection.\theSecCounter.\label{8.3.1} Valuer has been defined
as “an individual, group of individuals, or a firm possessing the
necessary qualifications, ability and experience to undertake a valuation
in an objective, unbiased and competent manner. In some jurisdictions,
licensing is required before one can act as a valuer. Because a valuation
reviewer must also be a valuer, to assist with the legibility of these
standards, the term valuer includes valuation reviewers except where
it is expressly stated otherwise, or is clear from the context that
valuation reviewers are excluded.\label{8.3.1-End}\label{sec:8.3_Valuer-End}

\section{Objectivity\label{sec:8.4_Objectivity}}

\stepcounter{SecCounter} 

\thesection.\theSecCounter.\label{8.4.1} The process of valuation
requires the valuer to make impartial judgements as to the reliability
of inputs and assumptions. For a valuation to be credible, it is important
that those judgements are made in a way that promotes transparency
and minimises the influence of any subjective factors on the process.
Judgement used in a valuation must be applied objectively to avoid
biased analyses, opinions and conclusions.\label{8.4.1-End}

\stepcounter{SecCounter} 

\thesection.\theSecCounter.\label{8.4.2} It is a fundamental expectation
that, when applying these standards, appropriate controls and procedures
are in place to ensure the necessary degree of objectivity in the
valuation process so that the results are free from bias. The IVSC
Code of Ethical Principles for Professional Valuers provides an example
of an appropriate framework for professional conduct.\label{8.4.2-End}\label{sec:8.4_Objectivity-End}

\section{Competence\label{sec:8.5_Competence}}

\stepcounter{SecCounter} 

\thesection.\theSecCounter.\label{8.5.1}Valuations must be prepared
by an individual or firm having the appropriate technical skills,
experience and knowledge of the subject of the valuation, the market(s)
in which it trades and the purpose of the valuation.\label{8.5.1-End}

\stepcounter{SecCounter} 

\thesection.\theSecCounter.\label{8.5.2}If a valuer does not possess
all of the necessary technical skills, experience and knowledge to
perform all aspects of a valuation, it is acceptable for the valuer
to seek assistance from specialists in certain aspects of the overall
assignment, providing this is disclosed in the scope of work (see
IVS 101 Scope of Work) and the report (see IVS 103 Reporting).\label{8.5.2-End}

\stepcounter{SecCounter} 

\thesection.\theSecCounter.\label{8.5.3} The valuer must have the
technical skills, experience and knowledge to understand, interpret
and utilise the work of any specialists.\label{8.5.3-End}\label{sec:8.5_Competence-End}

\section{Departures\label{sec:8.6_Departures}}

\stepcounter{SecCounter} 

\thesection.\theSecCounter.\label{8.6.1} A “departure” is a circumstance
where specific legislative, regulatory or other authoritative requirements
must be followed that differ from some of the requirements within
IVS. Departures are mandatory in that a valuer must comply with legislative,
regulatory and other authoritative requirements appropriate to the
purpose and jurisdiction of the valuation to be in compliance with
IVS. A valuer may still state that the valuation was performed in
accordance with IVS when there are departures in these circumstances.\label{8.6.1-End}

\stepcounter{SecCounter} 

\thesection.\theSecCounter.\label{8.6.2} The requirement to depart
from IVS pursuant to legislative, regulatory or other authoritative
requirements takes precedence over all other IVS requirements.\label{8.6.2-End}

\stepcounter{SecCounter} 

\thesection.\theSecCounter.\label{8.6.3} As required by IVS 101
Scope of Work, para 20.3 (n) and IVS 103 Reporting, para 10.2 the
nature of any departures must be identified (for example, identifying
that the valuation was performed in accordance with IVS and local
tax regulations). If there are any departures that significantly affect
the nature of the procedures performed, inputs and assumptions used,
and/or valuation conclusion(s), a valuer must also disclose the specific
legislative, regulatory or other authoritative requirements and the
significant ways in which they differ from the requirements of IVS
(for example, identifying that the relevant jurisdiction requires
the use of only a market approach in a circumstance where IVS would
indicate that the income approach should be used).\label{8.6.3-End}

\stepcounter{SecCounter} 

\thesection.\theSecCounter.\label{8.6.4} Departure deviations from
IVS that are not the result of legislative, regulatory or other authoritative
requirements are not permitted in valuations performed in accordance
with IVS.\label{8.6.4-End}\label{sec:8.6_Departures-End}\label{chap:8_IVS_Framework-End}

\chapter{General Standards\label{chap:9-General_Standards}}

\section{IVS 101. Scope of Work\label{sec:9.1_IVS-101_Scope_of_work}}

\subsection{Introduction\label{subsec:9.1.1_Introduction}}

\stepcounter{SubSecCounter} 

\thesubsection.\theSubSecCounter.\label{9.1.1.1} A scope of work
(sometimes referred to as terms of engagement) describes the fundamental
terms of a valuation engagement, such as the asset(s) being valued,
the purpose of the valuation and the responsibilities of parties involved
in the valuation.\label{9.1.1.1-End}

\stepcounter{SubSecCounter} 

\thesubsection.\theSubSecCounter.\label{9.1.1.2} This standard is
intended to apply to a wide spectrum of valuation assignments, including: 
\begin{enumerate}
\item valuations performed by valuers for their own employers (“in-house
valuations”);
\item valuations performed by valuers for clients other than their employers
(“third-party valuations”);
\item valuation reviews where the reviewer may not be required to provide
their own opinion of value.\label{9.1.1.2-End}\label{subsec:9.1.1_Introduction-End}
\end{enumerate}

\subsection{General Requirements\label{subsec:9.1.2_General_Requirements}}

\stepcounter{SubSecCounter} 

\thesubsection.\theSubSecCounter.\label{9.1.2.1}All valuation advice
and the work undertaken in its preparation must be appropriate for
the intended purpose.\label{9.1.2.1-End}

\stepcounter{SubSecCounter} 

\thesubsection.\theSubSecCounter.\label{9.1.2.2}A valuer must ensure
that the intended recipient(s) of the valuation advice understand(s)
what is to be provided and any limitations on its use before it is
finalised and reported.\label{9.1.2.2-End}

\stepcounter{SubSecCounter} 

\thesubsection.\theSubSecCounter.\label{9.1.2.3} A valuer must communicate
the scope of work to its client prior to completion of the assignment,
including the following: 
\begin{enumerate}
\item Identity of the valuer: The valuer may be an individual, group of
individuals or a firm. If the valuer has any material connection or
involvement with the subject asset or the other parties to the valuation
assignment, or if there are any other factors that could limit the
valuer’s ability to provide an unbiased and objective valuation, such
factors must be disclosed at the outset. If such disclosure does not
take place, the valuation assignment is not in compliance with IVS.
If the valuer needs to seek material assistance from others in relation
to any aspect of the assignment, the nature of such assistance and
the extent of reliance must be made clear.
\item Identity of the client(s) (if any): Confirmation of those for whom
the valuation assignment is being produced is important when determining
the form and content of the report to ensure that it contains information
relevant to their needs.
\item Identity of other intended users (if any): It is important to understand
whether there are any other intended users of the valuation report,
their identity and their needs, to ensure that the report content
and format meets those users’ needs. 
\item Asset(s) being valued: The subject asset in the valuation assignment
must be clearly identified. 
\item The valuation currency: The currency for the valuation and the final
valuation report or conclusion must be established. For example, a
valuation might be prepared in euros or US dollars. This requirement
is particularly important for valuation assignments involving assets
in multiple countries and/or cash flows in multiple currencies. 
\item Purpose of the valuation: The purpose for which the valuation assignment
is being prepared must be clearly identified as it is important that
valuation advice is not used out of context or for purposes for which
it is not intended. The purpose of the valuation will also typically
influence or determine the basis/bases of value to be used. 
\item Basis/bases of value used: As required by IVS 104 Bases of Value,
the valuation basis must be appropriate for the purpose of the valuation.
The source of the definition of any basis of value used must be cited
or the basis explained. This requirement is not applicable to a valuation
review where no opinion of value is to be provided and the reviewer
is not required to comment on the basis of value used. 
\item Valuation date: The valuation date must be stated. If the valuation
date is different from the date on which the valuation report is issued
or the date on which investigations are to be undertaken or completed
then where appropriate, these dates should be clearly distinguished. 
\item The nature and extent of the valuer’s work and any limitations thereon:
Any limitations or restrictions on the inspection, enquiry and/or
analysis in the valuation assignment must be identified (see IVS Framework,
paras 60.1-60.4) If relevant information is not available because
the conditions of the assignment restrict the investigation, these
restrictions and any necessary assumptions or special assumptions
(see IVS 104 Bases of Value, paras 200.1-200.5) made as a result of
the restriction must be identified. 
\item The nature and sources of information upon which the valuer relies:
The nature and source of any relevant information that is to be relied
upon and the extent of any verification to be undertaken during the
valuation process must be identified. 
\item Significant assumptions and/or special assumptions: All significant
assumptions and special assumptions that are to be made in the conduct
and reporting of the valuation assignment must be identified. 
\item The type of report being prepared: The format of the report, that
is, how the valuation will be communicated, must be described. 
\item Restrictions on use, distribution and publication of the report: Where
it is necessary or desirable to restrict the use of the valuation
or those relying on it, the intended users and restrictions must be
clearly communicated. 
\item That the valuation will be prepared in compliance with IVS and that
the valuer will assess the appropriateness of all significant inputs:
The nature of any departures must be explained, for example, identifying
that the valuation was performed in accordance with IVS and local
tax regulations. See IVS Framework paras 60.1-60.4 relating to departures.\label{9.1.2.3-End} 
\end{enumerate}
\stepcounter{SubSecCounter} 

\thesubsection.\theSubSecCounter.\label{9.1.2.4} Wherever possible,
the scope of work should be established and agreed between parties
to a valuation assignment prior to the valuer beginning work. However,
in certain circumstances, the scope of a valuation engagement may
not be clear at the start of that engagement. In such cases, as the
scope becomes clear, valuers must communicate and agree the scope
of work to their client.\label{9.1.2.4-End} 

\stepcounter{SubSecCounter} 

\thesubsection.\theSubSecCounter.\label{9.1.2.5} A written scope
of work may not be necessary. However, since valuers are responsible
for communicating the scope of work to their client, a written scope
of work should be prepared.\label{9.1.2.5-End}

\stepcounter{SubSecCounter} 

\thesubsection.\theSubSecCounter.\label{9.1.2.6} Some aspects of
the scope of work may be addressed in documents such as standing engagement
instructions, master services agreements or a company’s internal policies
and procedures.\label{9.1.2.6-End}\label{subsec:9.1.2_General_Requirements-End}

\subsection{Changes to Scope of Work\label{subsec:9.1.3_Changes_to_Scope}}

\stepcounter{SubSecCounter} 

\thesubsection.\theSubSecCounter.\label{9.1.3.1} Some of the items
in para 20.3 may not be determinable until the valuation assignment
is in progress, or changes to the scope may become necessary during
the course of the assignment due to additional information becoming
available or matters emerging that require further investigation.
As such, whilst the scope of work may be established at the outset,
it may also be established over time throughout the course of the
assignment. \label{9.1.3.1-End}

\stepcounter{SubSecCounter} 

\thesubsection.\theSubSecCounter.\label{9.1.3.2} In valuation assignments
where the scope of work changes over time, the items in para 20.3
and any changes made over time must be communicated to the client
before the assignment is completed and the valuation report is issued.\label{9.1.3.2-End}\label{subsec:9.1.3_Changes_to_Scope-End}\label{sec:9.1_IVS-101_Scope_of_work-End}

\section{IVS 102 Investigations and Compliance\label{sec:9.2_IVS-102_Investigations}}

\subsection{General Principle\label{subsec:9.2.1_General_Principle}}

\stepcounter{SubSecCounter} 

\thesubsection.\theSubSecCounter.\label{9.2.1.1} To be compliant
with IVS, valuation assignments, including valuation reviews, must
be conducted in accordance with all of the principles set out in IVS
that are appropriate for the purpose and the terms and conditions
set out in the scope of work.\label{9.2.1.1-End}\label{subsec:9.2.1_General_Principle-End}

\subsection{Investigations\label{subsec:9.2.2_Investigations}}

\stepcounter{SubSecCounter} 

\thesubsection.\theSubSecCounter.\label{9.2.2.1} Investigations
made during the course of a valuation assignment must be appropriate
for the purpose of the valuation assignment and the basis(es) of value.
References to a valuation or valuation assignment in this standard
include a valuation review.\label{9.2.2.1-End}

\stepcounter{SubSecCounter} 

\thesubsection.\theSubSecCounter.\label{9.2.2.2} Sufficient evidence
must be assembled by means such as inspection, inquiry, computation
and analysis to ensure that the valuation is properly supported. When
determining the extent of evidence necessary, professional judgment
is required to ensure the information to be obtained is adequate for
the purpose of the valuation.\label{9.2.2.2-End}

\stepcounter{SubSecCounter} 

\thesubsection.\theSubSecCounter.\label{9.2.2.3} Limits may be agreed
on the extent of the valuer’s investigations. Any such limits must
be noted in the scope of work. However, IVS 105 Valuation Approaches
and Methods, para 10.7 requires valuers to perform sufficient analysis
to evaluate all inputs and assumptions and their appropriateness for
the valuation purpose. If limitations on investigations are so substantial
that the valuer cannot sufficiently evaluate the inputs and assumptions,
the valuation engagement must not state that it has been performed
in compliance with IVS.\label{9.2.2.3-End}

\stepcounter{SubSecCounter} 

\thesubsection.\theSubSecCounter.\label{9.2.2.4} When a valuation
assignment involves reliance on information supplied by a party other
than the valuer, consideration should be given as to whether the information
is credible or that the information may otherwise by relied upon without
adversely affecting the credibility of the valuation opinion. Significant
inputs provided to the valuer (eg, by management/owners) should be
considered, investigated and/or corroborated. In cases where credibility
or reliability of information supplied cannot be supported, consideration
should be given as to whether or how such information is used.\label{9.2.2.4-End}

\stepcounter{SubSecCounter} 

\thesubsection.\theSubSecCounter.\label{9.2.2.5} In considering
the credibility and reliability of information provided, valuers should
consider matters such as:
\begin{enumerate}
\item the purpose of the valuation;
\item the significance of the information to the valuation conclusion;
\item the expertise of the source in relation to the subject matter;
\item whether the source is independent of either the subject asset and/or
the recipient of the valuation (see IVS 101 Scope of Work, paras 20.3
(a)).\label{9.2.2.5-End}
\end{enumerate}
\stepcounter{SubSecCounter} 

\thesubsection.\theSubSecCounter.\label{9.2.2.6} The purpose of
the valuation, the basis of value, the extent and limits on the investigations
and any sources of information that may be relied upon are part of
the valuation assignment’s scope of work that must be communicated
to all parties to the valuation assignment (see IVS 101 Scope of Work).\label{9.2.2.6-End}

\stepcounter{SubSecCounter} 

\thesubsection.\theSubSecCounter.\label{9.2.2.7} If, during the
course of an assignment, it becomes clear that the investigations
included in the scope of work will not result in a credible valuation,
or information to be provided by third parties is either unavailable
or inadequate, or limitations on investigations are so substantial
that the valuer cannot sufficiently evaluate the inputs and assumptions,
the valuation assignment will not comply with IVS.\label{9.2.2.7-End}\label{subsec:9.2.2_Investigations-End}

\subsection{Valuation Record\label{subsec:9.2.3_Valuation_Record}}

\stepcounter{SubSecCounter} 

\thesubsection.\theSubSecCounter.\label{9.2.3.1} A record must be
kept of the work performed during the valuation process and the basis
for the work on which the conclusions were reached for a reasonable
period after completion of the assignment, having regard to any relevant
statutory, legal or regulatory requirements. Subject to any such requirements,
this record should include the key inputs, all calculations, investigations
and analyses relevant to the final conclusion, and a copy of any draft
or final report(s) provided to the client.\label{9.2.3.1-End}\label{subsec:9.2.3_Valuation_Record-End}

\subsection{Compliance with Other Standards\label{subsec:9.2.4_Compliance_with_Other}}

\stepcounter{SubSecCounter} 

\thesubsection.\theSubSecCounter.\label{9.2.4.1} As noted in the
IVS Framework, when statutory, legal, regulatory or other authoritative
requirements must be followed that differ from some of the requirements
within IVS, a valuer must follow the statutory, legal, regulatory
or other authoritative requirements (called a “departure”). Such a
valuation has still been performed in overall compliance with IVS.\label{9.2.4.1-End}

\stepcounter{SubSecCounter} 

\thesubsection.\theSubSecCounter.\label{9.2.4.2} Most other sets
of requirements, such as those written by Valuation Professional Organizations,
other professional bodies, or firms’ internal policies and procedures,
will not contradict IVS and, instead, typically impose additional
requirements on valuers. Such standards may be followed in addition
to IVS without being seen as departures as long as all of the requirements
in IVS are fulfilled.\label{9.2.4.2-End}\label{subsec:9.2.4_Compliance_with_Other-End}\label{sec:9.2_IVS-102_Investigations-End}

\section{IVS 103. Reporting\label{sec:9.3_IVS-103_Reporting}}

\subsection{Introduction\label{subsec:9.3.1_Introduction}}

\stepcounter{SubSecCounter} 

\thesubsection.\theSubSecCounter.\label{9.3.1.1} It is essential
that the valuation report communicates the information necessary for
proper understanding of the valuation or valuation review. A report
must provide the intended users with a clear understanding of the
valuation.\label{9.3.1.1-End}

\stepcounter{SubSecCounter} 

\thesubsection.\theSubSecCounter.\label{9.3.1.2} To provide useful
information, the report must set out a clear and accurate description
of the scope of the assignment, its purpose and intended use (including
any limitations on that use) and disclosure of any assumptions, special
assumptions (IVS 104 Bases of Value, para 200.4), significant uncertainty
or limiting conditions that directly affect the valuation.\label{9.3.1.2-End}

\stepcounter{SubSecCounter} 

\thesubsection.\theSubSecCounter.\label{9.3.1.3} This standard applies
to all valuation reports or reports on the outcome of a valuation
review which may range from comprehensive narrative reports to abbreviated
summary reports.\label{9.3.1.3-End}

\stepcounter{SubSecCounter} 

\thesubsection.\theSubSecCounter.\label{9.3.1.4} For certain asset
classes there may be variations from these standards or additional
requirements to be reported upon. These are found in the relevant
IVS Asset Standards.\label{9.3.1.4-End}\label{subsec:9.3.1_Introduction-End}

\subsection{General Requirements\label{subsec:9.3.2_General_Requirements}}

\stepcounter{SubSecCounter} 

\thesubsection.\theSubSecCounter.\label{9.3.2.1} The purpose of
the valuation, the complexity of the asset being valued and the users’
requirements will determine the level of detail appropriate to the
valuation report. The format of the report should be agreed with all
parties as part of establishing a scope of work (see IVS 101 Scope
of Work).\label{9.3.2.1-End}

\stepcounter{SubSecCounter} 

\thesubsection.\theSubSecCounter.\label{9.3.2.2} Compliance with
this standard does not require a particular form or format of report;
however, the report must be sufficient to communicate to the intended
users the scope of the valuation assignment, the work performed and
the conclusions reached.\label{9.3.2.2-End}

\stepcounter{SubSecCounter} 

\thesubsection.\theSubSecCounter.\label{9.3.2.3} The report should
also be sufficient for an appropriately experienced valuation professional
with no prior involvement with the valuation engagement to review
the report and understand the items in paras 30.1 and 40.1, as applicable.\label{9.3.2.3-End}\label{subsec:9.3.2_General_Requirements-End}

\subsection{Valuation Reports\label{subsec:9.3.3_Valuation_Reports}}

\stepcounter{SubSecCounter} 

\thesubsection.\theSubSecCounter.\label{9.3.3.1} Where the report
is the result of an assignment involving the valuation of an asset
or assets, the report must convey the following, at a minimum:
\begin{enumerate}
\item the scope of the work performed, including the elements noted in para
20.3 of IVS 101 Scope of Work, to the extent that each is applicable
to the assignment;
\item the intended use;
\item the approach or approaches adopted;
\item the method or methods applied;
\item the key inputs used;
\item the assumptions made;
\item the conclusion(s) of value and principal reasons for any conclusions
reached;
\item the date of the report (which may differ from the valuation date).\label{9.3.3.1-End}
\end{enumerate}
\stepcounter{SubSecCounter} 

\thesubsection.\theSubSecCounter.\label{9.3.3.2} Some of the above
requirements may be explicitly included in a report or incorporated
into a report through reference to other documents (engagement letters,
scope of work documents, internal policies and procedures, etc).\label{9.3.3.2-End}\label{subsec:9.3.3_Valuation_Reports-End}

\subsection{Valuation Review Reports\label{subsec:9.3.4_Valuation_Review_Reports}}

\stepcounter{SubSecCounter} 

\thesubsection.\theSubSecCounter.\label{9.3.4.1} Where the report
is the result of a valuation review, the report must convey the following,
at a minimum:
\begin{enumerate}
\item the scope of the review performed, including the elements noted in
para 20.3 of IVS 101 Scope of Work to the extent each is applicable
to the assignment;
\item the valuation report being reviewed and the inputs and assumptions
upon which that valuation was based;
\item the reviewer’s conclusions about the work under review, including
supporting reasons;
\item the date of the report (which may differ from the valuation date).\label{9.3.4.1-End}
\end{enumerate}
\stepcounter{SubSecCounter} 

\thesubsection.\theSubSecCounter.\label{9.3.4.2} Some of the above
requirements may be explicitly included in a report or incorporated
into a report through reference to other documents (eg, engagement
letters, scope of work documents, internal policies and procedures,
etc).\label{9.3.4.2-End}\label{subsec:9.3.4_Valuation_Review_Reports-End}\label{sec:9.3_IVS-103_Reporting-End}

\section{IVS 104. Bases of Value\label{sec:9.4_IVS-104_Bases_of_Value}}

Compliance with this mandatory standard requires a valuer to select
the appropriate basis (or bases) of value and follow all applicable
requirements associated with that basis of value, whether those requirements
are included as part of this standard (for IVS-defined bases of value)
or not (for non-IVS-defined bases of value).

\subsection{Introduction\label{subsec:9.4.1_Introduction}}

\stepcounter{SubSecCounter} 

\thesubsection.\theSubSecCounter.\label{9.4.1.1} Bases of value
(sometimes called standards of value) describe the fundamental premises
on which the reported values will be based. It is critical that the
basis (or bases) of value be appropriate to the terms and purpose
of the valuation assignment, as a basis of value may influence or
dictate a valuer’s selection of methods, inputs and assumptions, and
the ultimate opinion of value.\label{9.4.1.1-End}

\stepcounter{SubSecCounter} 

\thesubsection.\theSubSecCounter.\label{9.4.1.2} A valuer may be
required to use bases of value that are defined by statute, regulation,
private contract or other document. Such bases have to be interpreted
and applied accordingly.\label{9.4.1.2-End}

\stepcounter{SubSecCounter} 

\thesubsection.\theSubSecCounter.\label{9.4.1.3} While there are
many different bases of value used in valuations, most have certain
common elements: an assumed transaction, an assumed date of the transaction
and the assumed parties to the transaction.\label{9.4.1.3-End}

\stepcounter{SubSecCounter} 

\thesubsection.\theSubSecCounter.\label{9.4.1.4} Depending on the
basis of value, the assumed transaction could take a number of forms:
\begin{enumerate}
\item a hypothetical transaction;
\item an actual transaction;
\item a purchase (or entry) transaction;
\item a sale (or exit) transaction;
\item a transaction in a particular or hypothetical market with specified
characteristics.\label{9.4.1.4-End}
\end{enumerate}
\stepcounter{SubSecCounter} 

\thesubsection.\theSubSecCounter.\label{9.4.1.5} The assumed date
of a transaction will influence what information and data a valuer
considers in a valuation. Most bases of value prohibit the consideration
of information or market sentiment that would not be known or knowable
with reasonable due diligence on the measurement/valuation date by
participants.\label{9.4.1.5-End}

\stepcounter{SubSecCounter} 

\thesubsection.\theSubSecCounter.\label{9.4.1.6} Most bases of value
reflect assumptions concerning the parties to a transaction and provide
a certain level of description of the parties. In respect to these
parties, they could include one or more actual or assumed characteristics,
such as:
\begin{enumerate}
\item hypothetical;
\item known or specific parties;
\item members of an identified/described group of potential parties;
\item whether the parties are subject to particular conditions or motivations
at the assumed date (eg, duress);
\item an assumed knowledge level.\label{9.4.1.6-End}\label{subsec:9.4.1_Introduction-End}
\end{enumerate}

\subsection{Bases of Value\label{subsec:9.4.2_Bases-of_Value}}

\stepcounter{SubSubSecCounter} 

\thesubsubsection.\theSubSubSecCounter.\label{9.4.2.0.1} In addition
to the IVS-defined bases of value listed below, the IVS have also
provided a non-exhaustive list of other non-IVS-defined bases of value
prescribed by individual jurisdictional law or those recognized and
adopted by international agreement:
\begin{enumerate}
\item IVS-defined bases of value: 
\begin{enumerate}
\item Market Value (section 30);
\item Market Rent (section 40);
\item Equitable Value (section 50);
\item Investment Value/Worth (section 60);
\item Synergistic Value (section 70);
\item Liquidation Value (section 80).
\end{enumerate}
\item Other bases of value (non-exhaustive list): 
\begin{enumerate}
\item Fair Value (International Financial Reporting Standards) (section
90);
\item Fair Market Value (Organization for Economic Co-operation and Development)
(section 100);
\item Fair Market Value (United States Internal Revenue Service) (section
110);
\item Fair Value (Legal/Statutory) (section 120): 
\begin{enumerate}
\item the Model Business Corporation Act;
\item Canadian case law (Manning v Harris Steel Group Inc).\label{9.4.2.0.1-End}
\end{enumerate}
\end{enumerate}
\end{enumerate}
\stepcounter{SubSubSecCounter} 

\thesubsubsection.\theSubSubSecCounter.\label{9.4.2.0.2} Valuers
must choose the relevant basis (or bases) of value according to the
terms and purpose of the valuation assignment. The valuer’s choice
of a basis (or bases) of value should consider instructions and input
received from the client and/or its representatives. However, regardless
of instructions and input provided to the valuer, the valuer should
not use a basis (or bases) of value that is inappropriate for the
intended purpose of the valuation (for example, if instructed to value
for financial reporting purposes under IFRS, compliance with IVS may
require the valuer to use a basis of value that is not defined or
mentioned in the IVS).\label{9.4.2.0.2-End}

\stepcounter{SubSubSecCounter} 

\thesubsubsection.\theSubSubSecCounter.\label{9.4.2.0.3} In accordance
with IVS 101 Scope of Work, the basis of value must be appropriate
for the purpose and the source of the definition of any basis of value
used must be cited or the basis explained.\label{9.4.2.0.3-End}

\stepcounter{SubSubSecCounter} 

\thesubsubsection.\theSubSubSecCounter.\label{9.4.2.0.4} Valuers
are responsible for understanding the regulation, case law and other
interpretive guidance related to all bases of value used.\label{9.4.2.0.4-End}

\stepcounter{SubSubSecCounter} 

\thesubsubsection.\theSubSubSecCounter.\label{9.4.2.0.5} The bases
of value illustrated in sections 90-120 of this standard are defined
by organizations other than the IVSC and the onus is on the valuer
to ensure they are using the relevant definition.\label{9.4.2.0.5-End}

\subsubsection{IVS-Defined Bases of Value\label{subsubsec:9.4.2.1_IVS-Defined_Bases}}

\paragraph{Market Value\label{par:9.4.2.1.1_Market_Value}}

\stepcounter{ParCounter} 

\theparagraph.\theParCounter.\label{9.4.2.1.1.1} Market Value is
the estimated amount for which an asset or liability should exchange
on the valuation date between a willing buyer and a willing seller
in an arm’s length transaction, after proper marketing and where the
parties had each acted knowledgeably, prudently and without compulsion.\label{9.4.2.1.1.1-End}

\stepcounter{ParCounter} 

\theparagraph.\theParCounter.\label{9.4.2.1.1.2} The definition
of Market Value must be applied in accordance with the following conceptual
framework: 
\begin{enumerate}
\item “The estimated amount” refers to a price expressed in terms of money
payable for the asset in an arm’s length market transaction. Market
Value is the most probable price reasonably obtainable in the market
on the valuation date in keeping with the market value definition.
It is the best price reasonably obtainable by the seller and the most
advantageous price reasonably obtainable by the buyer. This estimate
specifically excludes an estimated price inflated or deflated by special
terms or circumstances such as atypical financing, sale and leaseback
arrangements, special considerations or concessions granted by anyone
associated with the sale, or any element of value available only to
a specific owner or purchaser. 
\item “An asset or liability should exchange” refers to the fact that the
value of an asset or liability is an estimated amount rather than
a predetermined amount or actual sale price. It is the price in a
transaction that meets all the elements of the Market Value definition
at the valuation date. 
\item “On the valuation date” requires that the value is time-specific as
of a given date. Because markets and market conditions may change,
the estimated value may be incorrect or inappropriate at another time.
The valuation amount will reflect the market state and circumstances
as at the valuation date, not those at any other date. 
\item “Between a willing buyer” refers to one who is motivated, but not
compelled to buy. This buyer is neither over-eager nor determined
to buy at any price. This buyer is also one who purchases in accordance
with the realities of the current market and with current market expectations,
rather than in relation to an imaginary or hypothetical market that
cannot be demonstrated or anticipated to exist. The assumed buyer
would not pay a higher price than the market requires. The present
owner is included among those who constitute “the market”. 
\item “And a willing seller” is neither an over-eager nor a forced seller
prepared to sell at any price, nor one prepared to hold out for a
price not considered reasonable in the current market. The willing
seller is motivated to sell the asset at market terms for the best
price attainable in the open market after proper marketing, whatever
that price may be. The factual circumstances of the actual owner are
not a part of this consideration because the willing seller is a hypothetical
owner. 
\item “In an arm’s length transaction” is one between parties who do not
have a particular or special relationship, eg, parent and subsidiary
companies or landlord and tenant, that may make the price level uncharacteristic
of the market or inflated. The Market Value transaction is presumed
to be between unrelated parties, each acting independently. 
\item “After proper marketing” means that the asset has been exposed to
the market in the most appropriate manner to effect its disposal at
the best price reasonably obtainable in accordance with the Market
Value definition. The method of sale is deemed to be that most appropriate
to obtain the best price in the market to which the seller has access.
The length of exposure time is not a fixed period but will vary according
to the type of asset and market conditions. The only criterion is
that there must have been sufficient time to allow the asset to be
brought to the attention of an adequate number of market participants.
The exposure period occurs prior to the valuation date. 
\item “Where the parties had each acted knowledgeably, prudently” presumes
that both the willing buyer and the willing seller are reasonably
informed about the nature and characteristics of the asset, its actual
and potential uses, and the state of the market as of the valuation
date. Each is further presumed to use that knowledge prudently to
seek the price that is most favourable for their respective positions
in the transaction. Prudence is assessed by referring to the state
of the market at the valuation date, not with the benefit of hindsight
at some later date. For example, it is not necessarily imprudent for
a seller to sell assets in a market with falling prices at a price
that is lower than previous market levels. In such cases, as is true
for other exchanges in markets with changing prices, the prudent buyer
or seller will act in accordance with the best market information
available at the time. 
\item “And without compulsion” establishes that each party is motivated
to undertake the transaction, but neither is forced or unduly coerced
to complete it.\label{9.4.2.1.1.2-End}
\end{enumerate}
\stepcounter{ParCounter} 

\theparagraph.\theParCounter.\label{9.4.2.1.1.3} The concept of
Market Value presumes a price negotiated in an open and competitive
market where the participants are acting freely. The market for an
asset could be an international market or a local market. The market
could consist of numerous buyers and sellers, or could be one characterised
by a limited number of market participants. The market in which the
asset is presumed exposed for sale is the one in which the asset notionally
being exchanged is normally exchanged.\label{9.4.2.1.1.3-End}

\stepcounter{ParCounter} 

\theparagraph.\theParCounter.\label{9.4.2.1.1.4} The Market Value
of an asset will reflect its highest and best use (see paras 140.1-140.5).
The highest and best use is the use of an asset that maximises its
potential and that is possible, legally permissible and financially
feasible. The highest and best use may be for continuation of an asset’s
existing use or for some alternative use. This is determined by the
use that a market participant would have in mind for the asset when
formulating the price that it would be willing to bid.\label{9.4.2.1.1.4-End}

\stepcounter{ParCounter} 

\theparagraph.\theParCounter.\label{9.4.2.1.1.5} The nature and
source of the valuation inputs must be consistent with the basis of
value, which in turn must have regard to the valuation purpose. For
example, various approaches and methods may be used to arrive at an
opinion of value providing they use market-derived data. The market
approach will, by definition, use market-derived inputs. To indicate
Market Value, the income approach should be applied, using inputs
and assumptions that would be adopted by participants. To indicate
Market Value using the cost approach, the cost of an asset of equal
utility and the appropriate depreciation should be determined by analysis
of market-based costs and depreciation.\label{9.4.2.1.1.5-End}

\stepcounter{ParCounter} 

\theparagraph.\theParCounter.\label{9.4.2.1.1.6} The data available
and the circumstances relating to the market for the asset being valued
must determine which valuation method or methods are most relevant
and appropriate. If based on appropriately analyzed market-derived
data, each approach or method used should provide an indication of
Market Value.\label{9.4.2.1.1.6-End}

\stepcounter{ParCounter} 

\theparagraph.\theParCounter.\label{9.4.2.1.1.7} Market Value does
not reflect attributes of an asset that are of value to a specific
owner or purchaser that are not available to other buyers in the market.
Such advantages may relate to the physical, geographic, economic or
legal characteristics of an asset. Market Value requires the disregard
of any such element of value because, at any given date, it is only
assumed that there is a willing buyer, not a particular willing buyer.\label{9.4.2.1.1.7-End}\label{par:9.4.2.1.1_Market_Value-End}

\paragraph{Market Rent\label{par:9.4.2.1.2_Market_Rent}}

\stepcounter{ParCounter} 

\theparagraph.\theParCounter.\label{9.4.2.1.2.1} Market Rent is
the estimated amount for which an interest in real property should
be leased on the valuation date between a willing lessor and a willing
lessee on appropriate lease terms in an arm’s length transaction,
after proper marketing and where the parties had each acted knowledgeably,
prudently and without compulsion.\label{9.4.2.1.2.1-End}

\stepcounter{ParCounter} 

\theparagraph.\theParCounter.\label{9.4.2.1.2.2} Market Rent may
be used as a basis of value when valuing a lease or an interest created
by a lease. In such cases, it is necessary to consider the contract
rent and, where it is different, the market rent. \label{9.4.2.1.2.2-End}

\stepcounter{ParCounter} 

\theparagraph.\theParCounter.\label{9.4.2.1.2.3} The conceptual
framework supporting the definition of Market Value shown above can
be applied to assist in the interpretation of Market Rent. In particular,
the estimated amount excludes a rent inflated or deflated by special
terms, considerations or concessions. The “appropriate lease terms”
are terms that would typically be agreed in the market for the type
of property on the valuation date between market participants. An
indication of Market Rent should only be provided in conjunction with
an indication of the principal lease terms that have been assumed.\label{9.4.2.1.2.3-End}

\stepcounter{ParCounter} 

\theparagraph.\theParCounter.\label{9.4.2.1.2.4} Contract Rent is
the rent payable under the terms of an actual lease. It may be fixed
for the duration of the lease, or variable. The frequency and basis
of calculating variations in the rent will be set out in the lease
and must be identified and understood in order to establish the total
benefits accruing to the lessor and the liability of the lessee.\label{9.4.2.1.2.4-End}

\stepcounter{ParCounter} 

\theparagraph.\theParCounter.\label{9.4.2.1.2.5} In some circumstances
the Market Rent may have to be assessed based on terms of an existing
lease (eg, for rental determination purposes where the lease terms
are existing and therefore not to be assumed as part of a notional
lease).\label{9.4.2.1.2.5-End}

\stepcounter{ParCounter} 

\theparagraph.\theParCounter.\label{9.4.2.1.2.6} In calculating
Market Rent, the valuer must consider the following: 
\begin{enumerate}
\item in regard to a Market Rent subject to a lease, the terms and conditions
of that lease are the appropriate lease terms unless those terms and
conditions are illegal or contrary to overarching legislation;
\item in regard to a Market Rent that is not subject to a lease, the assumed
terms and conditions are the terms of a notional lease that would
typically be agreed in a market for the type of property on the valuation
date between market participants.\label{9.4.2.1.2.6-End}\label{par:9.4.2.1.2_Market_Rent-End}
\end{enumerate}

\paragraph{Equitable Value\label{par:9.4.2.1.3_Equitable_Value}}

\stepcounter{ParCounter} 

\theparagraph.\theParCounter.\label{9.4.2.1.3.1} Equitable Value
is the estimated price for the transfer of an asset or liability between
identified knowledgeable and willing parties that reflects the respective
interests of those parties.\label{9.4.2.1.3.1-End}

\stepcounter{ParCounter} 

\theparagraph.\theParCounter.\label{9.4.2.1.3.2} Equitable Value
requires the assessment of the price that is fair between two specific,
identified parties considering the respective advantages or disadvantages
that each will gain from the transaction. In contrast, Market Value
requires any advantages or disadvantages that would not be available
to, or incurred by, market participants generally to be disregarded.\label{9.4.2.1.3.2-End}

\stepcounter{ParCounter} 

\theparagraph.\theParCounter.\label{9.4.2.1.3.3} Equitable Value
is a broader concept than Market Value. Although in many cases the
price that is fair between two parties will equate to that obtainable
in the market, there will be cases where the assessment of Equitable
Value will involve taking into account matters that have to be disregarded
in the assessment of Market Value, such as certain elements of Synergistic
Value arising because of the combination of the interests.\label{9.4.2.1.3.3-End}

\stepcounter{ParCounter} 

\theparagraph.\theParCounter.\label{9.4.2.1.3.4} Examples of the
use of Equitable Value include: 
\begin{enumerate}
\item determination of a price that is equitable for a shareholding in a
non- quoted business, where the holdings of two specific parties may
mean that the price that is equitable between them is different from
the price that might be obtainable in the market;
\item determination of a price that would be equitable between a lessor
and a lessee for either the permanent transfer of the leased asset
or the cancellation of the lease liability.\label{9.4.2.1.3.4-End}\label{par:9.4.2.1.3_Equitable_Value-End}
\end{enumerate}

\paragraph{Investment Value/Worth\label{par:9.4.2.1.4_Investment_Value}}

\stepcounter{ParCounter} 

\theparagraph.\theParCounter.\label{9.4.2.1.4.1} Investment Value
is the value of an asset to a particular owner or prospective owner
for individual investment or operational objectives.\label{9.4.2.1.4.1-End}

\stepcounter{ParCounter} 

\theparagraph.\theParCounter.\label{9.4.2.1.4.2} Investment Value
is an entity-specific basis of value. Although the value of an asset
to the owner may be the same as the amount that could be realized
from its sale to another party, this basis of value reflects the benefits
received by an entity from holding the asset and, therefore, does
not involve a presumed exchange. Investment Value reflects the circumstances
and financial objectives of the entity for which the valuation is
being produced. It is often used for measuring investment performance.\label{9.4.2.1.4.2-End}\label{par:9.4.2.1.4_Investment_Value-End}

\paragraph{Synergistic Value\label{par:9.4.2.1.5_Synergistic_Value}}

\stepcounter{ParCounter} 

\theparagraph.\theParCounter.\label{9.4.2.1.5.1} Synergistic Value
is the result of a combination of two or more assets or interests
where the combined value is more than the sum of the separate values.
If the synergies are only available to one specific buyer then Synergistic
Value will differ from Market Value, as the Synergistic Value will
reflect particular attributes of an asset that are only of value to
a specific purchaser. The added value above the aggregate of the respective
interests is often referred to as “marriage value.”\label{9.4.2.1.5.1-End}\label{par:9.4.2.1.4_Synergistic_Value-End}

\paragraph{Liquidation Value\label{par:9.4.2.1.6_Liquidation_Value}}

\stepcounter{ParCounter} 

\theparagraph.\theParCounter.\label{9.4.2.1.6.1} Liquidation Value
is the amount that would be realised when an asset or group of assets
are sold on a piecemeal basis. Liquidation Value should take into
account the costs of getting the assets into saleable condition as
well as those of the disposal activity. Liquidation Value can be determined
under two different premises of value: 
\begin{enumerate}
\item an orderly transaction with a typical marketing period (see section
160);
\item a forced transaction with a shortened marketing period (see section
170).\label{9.4.2.1.6.1-End}
\end{enumerate}
\stepcounter{ParCounter} 

\theparagraph.\theParCounter.\label{9.4.2.1.6.2} A valuer must disclose
which premise of value is assumed.\label{9.4.2.1.6.2-End}\label{par:9.4.2.1.6_Liquidation_Value-End}

End\label{subsubsec:9.4.2.1_IVS-Defined_Bases-End}

\subsubsection{Other Bases of Value\label{subsec:9.4.2.2_Other_Bases}}

\paragraph{Fair Value\label{par:9.4.2.2.1_Fair_Value_IFRS}}

\stepcounter{ParCounter} 

\theparagraph.\theParCounter.\label{9.4.2.2.1.1} IFRS 13 defines
Fair Value as the price that would be received to sell an asset or
paid to transfer a liability in an orderly transaction between market
participants at the measurement date.\label{9.4.2.2.1.1-End}

\stepcounter{ParCounter} 

\theparagraph.\theParCounter.\label{9.4.2.2.1.2} For financial reporting
purposes, over 130 countries require or permit the use of International
Accounting Standards published by the International Accounting Standards
Board. In addition, the Financial Accounting Standards Board in the
United States uses the same definition of Fair Value in Topic 820.\label{9.4.2.2.1.2-End}\label{par:9.4.2.2.1_Fair_Value_IFRS-End}

\paragraph{Fair Market Value\label{par:9.4.2.2.2_Fair_Market_Value_OECD}}

\stepcounter{ParCounter} 

\theparagraph.\theParCounter.\label{9.4.2.2.2.1} The OECD defines
Fair Market Value as the price a willing buyer would pay a willing
seller in a transaction on the open market.\label{9.4.2.2.2.1-End}

\stepcounter{ParCounter} 

\theparagraph.\theParCounter.\label{9.4.2.2.2.2} OECD guidance is
used in many engagements for international tax purposes.\label{9.4.2.2.2.2-End}\label{par:9.4.2.2.2_Fair_Market_Value_OECD-End}

\paragraph{Fair Market Value\label{par:9.4.2.2.3_Fair_Market_Value_US}}

\stepcounter{ParCounter} 

\theparagraph.\theParCounter.\label{9.4.2.2.3.1} For United States
tax purposes, Regulation §20.2031-1 states: “The fair market value
is the price at which the property would change hands between a willing
buyer and a willing seller, neither being under any compulsion to
buy or to sell and both having reasonable knowledge of relevant facts.”\label{9.4.2.2.3.1-End}\label{par:9.4.2.2.3_Fair_Market_Value-End_US}

\paragraph{Fair Value\label{par:9.4.2.2.4_Fair_Value_Diff_Jur}}

\stepcounter{ParCounter} 

\theparagraph.\theParCounter.\label{9.4.2.2.4.1} Many national,
state and local agencies use Fair Value as a basis of value in a legal
context. The definitions can vary significantly and may be the result
of legislative action or those established by courts in prior cases.\label{9.4.2.2.4.1-End}

\stepcounter{ParCounter} 

\theparagraph.\theParCounter.\label{9.4.2.2.4.2} Examples of US
and Canadian definitions of Fair Value are as follows: 
\begin{enumerate}
\item The Model Business Corporation Act (MBCA) is a model set of law prepared
by the Committee on Corporate Laws of the Section of Business Law
of the American Bar Association and is followed by 24 States in the
United States. The definition of Fair Value from the MBCA is the value
of the corporation’s shares determined: 
\begin{enumerate}
\item immediately before the effectuation of the corporate action to which
the shareholder objects;
\item using customary and current valuation concepts and techniques generally
employed for similar businesses in the context of the transaction
requiring appraisal;
\item without discounting for lack of marketability or minority status except,
if appropriate, for amendments to the articles pursuant to section
13.02(a)(5). 
\end{enumerate}
\item In 1986, the Supreme Court of British Columbia in Canada issued a
ruling in Manning v Harris Steel Group Inc. that stated: “Thus, a
‘fair’ value is one which is just and equitable. That terminology
contains within itself the concept of adequate compensation (indemnity),
consistent with the requirements of justice and equity.”\label{9.4.2.2.4.2- End}\label{par:9.4.2.2.4_Fair_Value_Diff_Jur-End}\label{subsec:9.4.2.2_Other_Bases-End}\label{subsec:9.4.2_Bases-of_Value-End}
\end{enumerate}

\subsection{Premise of Value/Assumed Use\label{subsec:9.4.3_Premise_of_Value}}

\stepcounter{SubSecCounter} 

\theSubSubsection.\theSubSubSecCounter.\label{9.4.3.0.1} A Premise
of Value or Assumed Use describes the circumstances of how an asset
or liability is used. Different bases of value may require a particular
Premise of Value or allow the consideration of multiple Premises of
Value. Some common Premises of Value are: 
\begin{enumerate}
\item highest and best use;
\item current use/existing use;
\item orderly liquidation;
\item forced sale.\label{9.4.3.0.1-End}
\end{enumerate}

\subsubsection{Highest and Best Use\label{subsubsec:9.4.3.1_Highest_and_Best_use}}

\stepcounter{SubSecCounter} 

\theSubSubsection.\theSubSubSecCounter.\label{9.4.3.1.1} Highest
and best use is the use, from a participant perspective, that would
produce the highest value for an asset. Although the concept is most
frequently applied to non-financial assets as many financial assets
do not have alternative uses, there may be circumstances where the
highest and best use of financial assets needs to be considered.\label{9.4.3.1.1-End}

\stepcounter{SubSecCounter} 

\theSubSubsection.\theSubSubSecCounter.\label{9.4.3.1.2} The highest
and best use must be physically possible (where applicable), financially
feasible, legally allowed and result in the highest value. If different
from the current use, the costs to convert an asset to its highest
and best use would impact the value.\label{9.4.3.1.2-End}

\stepcounter{SubSecCounter} 

\theSubSubsection.\theSubSubSecCounter.\label{9.4.3.1.3} The highest
and best use for an asset may be its current or existing use when
it is being used optimally. However, highest and best use may differ
from current use or even be an orderly liquidation.\label{9.4.3.1.3-End}

\stepcounter{SubSecCounter} 

\theSubSubsection.\theSubSubSecCounter.\label{9.4.3.1.4} The highest
and best use of an asset valued on a stand-alone basis may be different
from its highest and best use as part of a group of assets, when its
contribution to the overall value of the group must be considered.\label{9.4.3.1.4-End}

\stepcounter{SubSecCounter} 

\theSubSubsection.\theSubSubSecCounter.\label{9.4.3.1.5}The determination
of the highest and best use involves consideration of the following: 
\begin{enumerate}
\item To establish whether a use is physically possible, regard will be
had to what would be considered reasonable by participants;
\item To reflect the requirement to be legally permissible, any legal restrictions
on the use of the asset, e.\,g, town planning/zoning designations,
need to be taken into account as well as the likelihood that these
restrictions will change.
\item The requirement that the use be financially feasible takes into account
whether an alternative use that is physically possible and legally
permissible will generate sufficient return to a typical participant,
after taking into account the costs of conversion to that use, over
and above the return on the existing use.\label{9.4.3.1.5-End}\label{subsubsec:9.4.3.1_Highest_and_Best_use-End}
\end{enumerate}

\subsubsection{Current Use/Existing Use\label{subsubsec:Current-Use}}

\stepcounter{SubSecCounter} 

\theSubSubsection.\theSubSubSecCounter.\label{9.4.3.2.1} Current
use/existing use is the current way an asset, liability, or group
of assets and/or liabilities is used. The current use may be, but
is not necessarily, also the highest and best use.\label{9.4.3.2.1-End}\label{subsubsec:Current-Use-End}

\subsubsection{Orderly Liquidation\label{subsubsec:9.4.3.3_Orderly_Liquidation}}

\stepcounter{SubSecCounter} 

\theSubSubsection.\theSubSubSecCounter.\label{9.4.3.3.1} An orderly
liquidation describes the value of a group of assets that could be
realised in a liquidation sale, given a reasonable period of time
to find a purchaser (or purchasers), with the seller being compelled
to sell on an as-is, where-is basis.\label{9.4.3.3.1-End}

\stepcounter{SubSecCounter} 

\theSubSubsection.\theSubSubSecCounter.\label{9.4.3.3.2} The reasonable
period of time to find a purchaser (or purchasers) may vary by asset
type and market conditions.\label{9.4.3.3.2-End}\label{subsubsec:9.4.3.3_Orderly_Liquidation-End}

\subsubsection{Forced Sale\label{subsubsec:9.4.3.4_Forced_Sale}}

\stepcounter{SubSecCounter} 

\theSubSubsection.\theSubSubSecCounter.\label{9.4.3.4.1} The term
“forced sale” is often used in circumstances where a seller is under
compulsion to sell and that, as a consequence, a proper marketing
period is not possible and buyers may not be able to undertake adequate
due diligence. The price that could be obtained in these circumstances
will depend upon the nature of the pressure on the seller and the
reasons why proper marketing cannot be undertaken. It may also reflect
the consequences for the seller of failing to sell within the period
available. Unless the nature of, and the reason for, the constraints
on the seller are known, the price obtainable in a forced sale cannot
be realistically estimated. The price that a seller will accept in
a forced sale will reflect its particular circumstances, rather than
those of the hypothetical willing seller in the Market Value definition.
A “forced sale” is a description of the situation under which the
exchange takes place, not a distinct basis of value.\label{9.4.3.4.1-End}

\stepcounter{SubSecCounter} 

\theSubSubsection.\theSubSubSecCounter.\label{9.4.3.4.2} If an indication
of the price obtainable under forced sale circumstances is required,
it will be necessary to clearly identify the reasons for the constraint
on the seller, including the consequences of failing to sell in the
specified period by setting out appropriate assumptions. If these
circumstances do not exist at the valuation date, these must be clearly
identified as special assumptions.\label{9.4.3.4.2-End}

\stepcounter{SubSecCounter} 

\theSubSubsection.\theSubSubSecCounter.\label{9.4.3.4.3} A forced
sale typically reflects the most probable price that a specified property
is likely to bring under all of the following conditions: 
\begin{enumerate}
\item consummation of a sale within a short time period;
\item the asset is subjected to market conditions prevailing as of the date
of valuation or assumed timescale within which the transaction is
to be completed;
\item both the buyer and the seller are acting prudently and knowledgeably;
\item the seller is under compulsion to sell;
\item the buyer is typically motivated;
\item both parties are acting in what they consider their best interests;
\item a normal marketing effort is not possible due to the brief exposure
time;
\item payment will be made in cash.\label{9.4.3.4.3-End}
\end{enumerate}
\stepcounter{SubSecCounter} 

\theSubSubsection.\theSubSubSecCounter.\label{9.4.3.4.4} Sales in
an inactive or falling market are not automatically “forced sales”
simply because a seller might hope for a better price if conditions
improved. Unless the seller is compelled to sell by a deadline that
prevents proper marketing, the seller will be a willing seller within
the definition of Market Value (see paras 30.1-30.7).\label{9.4.3.4.4-End} 

\stepcounter{SubSecCounter} 

\theSubSubsection.\theSubSubSecCounter.\label{9.4.3.4.5} While confirmed
“forced sale” transactions would generally be excluded from consideration
in a valuation where the basis of value is Market Value, it can be
difficult to verify that an arm’s length transaction in a market was
a forced sale.\label{9.4.3.4.5-End}\label{subsubsec:9.4.3.4_Forced_Sale-End}\label{subsec:9.4.3_Premise_of_Value-End}

\subsection{Entity-Specific Factors\label{subsec:9.4.4_Entity_Specific_Factors}}

\stepcounter{SubSecCounter} 

\theSubsection.\theSubSecCounter.\label{9.4.4.1} For most bases
of value, the factors that are specific to a particular buyer or seller
and not available to participants generally are excluded from the
inputs used in a market-based valuation. Examples of entity-specific
factors that may not be available to participants include: 
\begin{enumerate}
\item additional value or reduction in value derived from the creation of
a portfolio of similar assets;
\item unique synergies between the asset and other assets owned by the entity;
\item legal rights or restrictions applicable only to the entity;
\item tax benefits or tax burdens unique to the entity;
\item an ability to exploit an asset that is unique to that entity.\label{9.4.4.1-End}
\end{enumerate}
\stepcounter{SubSecCounter} 

\theSubsection.\theSubSecCounter.\label{9.4.4.2} Whether such factors
are specific to the entity, or would be available to others in the
market generally, is determined on a case-by-case basis. For example,
an asset may not normally be transacted as a stand-alone item but
as part of a group of assets. Any synergies with related assets would
transfer to participants along with the transfer of the group and
therefore are not entity specific.\label{9.4.4.2-End}

\stepcounter{SubSecCounter} 

\theSubsection.\theSubSecCounter.\label{9.4.4.3} If the objective
of the basis of value used in a valuation is to determine the value
to a specific owner (such as Investment Value/Worth discussed in paras
60.1 and 60.2), entity-specific factors are reflected in the valuation
of the asset. Situations in which the value to a specific owner may
be required include the following examples: 
\begin{enumerate}
\item supporting investment decisions;
\item reviewing the performance of an asset.\label{9.4.4.3-End}\label{subsec:9.4.4_Entity_Specific_Factors-End}
\end{enumerate}

\subsection{Synergies\label{subsec:9.4.5_Synergies}}

\stepcounter{SubSecCounter} 

\theSubsection.\theSubSecCounter.\label{9.4.5.1} “Synergies” refer
to the benefits associated with combining assets. When synergies are
present, the value of a group of assets and liabilities is greater
than the sum of the values of the individual assets and liabilities
on a stand-alone basis. Synergies typically relate to a reduction
in costs, and/or an increase in revenue, and/or a reduction in risk.\label{9.4.5.1-End}

\stepcounter{SubSecCounter} 

\theSubsection.\theSubSecCounter.\label{9.4.5.2} Whether synergies
should be considered in a valuation depends on the basis of value.
For most bases of value, only those synergies available to other participants
generally will be considered (see discussion of Entity-Specific Factors
in paras 180.1-180.3).\label{9.4.5.2-End}

\stepcounter{SubSecCounter} 

\theSubsection.\theSubSecCounter.\label{9.4.5.3} An assessment of
whether synergies are available to other participants may be based
on the amount of the synergies rather than a specific way to achieve
that synergy.\label{9.4.5.3-End}\label{subsec:9.4.5_Synergies-End}

\subsection{Assumptions and Special Assumptions\label{subsec:9.4.6_Assumptions}}

\stepcounter{SubSecCounter} 

\theSubsection.\theSubSecCounter.\label{9.4.6.1} In addition to
stating the basis of value, it is often necessary to make an assumption
or multiple assumptions to clarify either the state of the asset in
the hypothetical exchange or the circumstances under which the asset
is assumed to be exchanged. Such assumptions can have a significant
impact on value.\label{9.4.6.1-End}

\stepcounter{SubSecCounter} 

\theSubsection.\theSubSecCounter.\label{9.4.6.2} These types of
assumptions generally fall into one of two categories:
\begin{enumerate}
\item assumed facts that are consistent with, or could be consistent with,
those existing at the date of valuation;
\item assumed facts that differ from those existing at the date of valuation.\label{9.4.6.2-End}
\end{enumerate}
\stepcounter{SubSecCounter} 

\theSubsection.\theSubSecCounter.\label{9.4.6.3} Assumptions related
to facts that are consistent with, or could be consistent with, those
existing at the date of valuation may be the result of a limitation
on the extent of the investigations or enquiries undertaken by the
valuer. Examples of such assumptions include, without limitation: 
\begin{enumerate}
\item an assumption that a business is transferred as a complete operational
entity;
\item an assumption that assets employed in a business are transferred without
the business, either individually or as a group;
\item an assumption that an individually valued asset is transferred together
with other complementary assets;
\item an assumption that a holding of shares is transferred either as a
block or individually.\label{9.4.6.3-End}
\end{enumerate}
\stepcounter{SubSecCounter} 

\theSubsection.\theSubSecCounter.\label{9.4.6.4} Where assumed facts
differ from those existing at the date of valuation, it is referred
to as a “special assumption”. Special assumptions are often used to
illustrate the effect of possible changes on the value of an asset.
They are designated as “special” so as to highlight to a valuation
user that the valuation conclusion is contingent upon a change in
the current circumstances or that it reflects a view that would not
be taken by participants generally on the valuation date. Examples
of such assumptions include, without limitation: 
\begin{enumerate}
\item an assumption that a property is freehold with vacant possession;
\item an assumption that a proposed building had actually been completed
on the valuation date;
\item an assumption that a specific contract was in existence on the valuation
date which had not actually been completed;
\item an assumption that a financial instrument is valued using a yield
curve that is different from that which would be used by a participant.\label{9.4.6.4-End}
\end{enumerate}
\stepcounter{SubSecCounter} 

\theSubsection.\theSubSecCounter.\label{9.4.6.5} All assumptions
and special assumptions must be reasonable under the circumstances,
be supported by evidence, and be relevant having regard to the purpose
for which the valuation is required.\label{9.4.6.5-End}\label{subsubsec:9.4.6_Assumptions-End}

\subsection{Transaction Costs\label{subsec:9.4.7_Transaction_Costs}}

\stepcounter{SubSecCounter} 

\theSubsection.\theSubSecCounter.\label{9.4.7.1} Most bases of value
represent the estimated exchange price of an asset without regard
to the seller’s costs of sale or the buyer’s costs of purchase and
without adjustment for any taxes payable by either party as a direct
result of the transaction.\label{9.4.7.1-End}\label{subsec:9.4.7_Transaction_Costs-End}\label{sec:9.4_IVS-104_Bases_of_Value-End}

\section{IVS 105. Valuation Approaches and Methods\label{sec:9.5_IVS-105_Valuation_Approaches}}

\subsection{Introduction\label{subsec:9.5.1_Introduction}}

\stepcounter{SubSecCounter} 

\theSubsection.\theSubSecCounter.\label{9.5.1.1} Consideration must
be given to the relevant and appropriate valuation approaches. The
three approaches described and defined below are the main approaches
used in valuation. They are all based on the economic principles of
price equilibrium, anticipation of benefits or substitution. The principal
valuation approaches are: 
\begin{enumerate}
\item market approach;
\item income approach;
\item cost approach.\label{9.5.1.1-End}
\end{enumerate}
\stepcounter{SubSecCounter} 

\theSubsection.\theSubSecCounter.\label{9.5.1.2} Each of these valuation
approaches includes different, detailed methods of application. \label{9.5.1.2-End}

\stepcounter{SubSecCounter} 

\theSubsection.\theSubSecCounter.\label{9.5.1.3} The goal in selecting
valuation approaches and methods for an asset is to find the most
appropriate method under the particular circumstances. No one method
is suitable in every possible situation. The selection process should
consider, at a minimum: 
\begin{enumerate}
\item the appropriate basis(es) of value and premise(s) of value, determined
by the terms and purpose of the valuation assignment;
\item the respective strengths and weaknesses of the possible valuation
approaches and methods;
\item the appropriateness of each method in view of the nature of the asset,
and the approaches or methods used by participants in the relevant
market;
\item the availability of reliable information needed to apply the method(s).\label{9.5.1.3-End}
\end{enumerate}
\stepcounter{SubSecCounter} 

\theSubsection.\theSubSecCounter.\label{9.5.1.4} Valuers are not
required to use more than one method for the valuation of an asset,
particularly when the valuer has a high degree of confidence in the
accuracy and reliability of a single method, given the facts and circumstances
of the valuation engagement. However, valuers should consider the
use of multiple approaches and methods and more than one valuation
approach or method should be considered and may be used to arrive
at an indication of value, particularly when there are insufficient
factual or observable inputs for a single method to produce a reliable
conclusion. Where more than one approach and method is used, or even
multiple methods within a single approach, the conclusion of value
based on those multiple approaches and/or methods should be reasonable
and the process of analysing and reconciling the differing values
into a single conclusion, without averaging, should be described by
the valuer in the report.\label{9.5.1.4-End}

\stepcounter{SubSecCounter} 

\theSubsection.\theSubSecCounter.\label{9.5.1.5} While this standard
includes discussion of certain methods within the Cost, Market and
Income approaches, it does not provide a comprehensive list of all
possible methods that may be appropriate. It is the valuer’s responsibility
to choose the appropriate method(s) for each valuation engagement.
Compliance with IVS may require the valuer to use a method not defined
or mentioned in the IVS.\label{9.5.1.5-End}

\stepcounter{SubSecCounter} 

\theSubsection.\theSubSecCounter.\label{9.5.1.6} When different
approaches and/or methods result in widely divergent indications of
value, a valuer should perform procedures to understand why the value
indications differ, as it is generally not appropriate to simply weight
two or more divergent indications of value. In such cases, valuers
should reconsider the guidance in para 10.3 to determine whether one
of the approaches/methods provides a better or more reliable indication
of value.\label{9.5.1.6-End}

\stepcounter{SubSecCounter} 

\theSubsection.\theSubSecCounter.\label{9.5.1.7} Valuers should
maximize the use of relevant observable market information in all
three approaches. Regardless of the source of the inputs and assumptions
used in a valuation, a valuer must perform appropriate analysis to
evaluate those inputs and assumptions and their appropriateness for
the valuation purpose.\label{9.5.1.7-End}

\stepcounter{SubSecCounter} 

\theSubsection.\theSubSecCounter.\label{9.5.1.8} Although no one
approach or method is applicable in all circumstances, price information
from an active market is generally considered to be the strongest
evidence of value. Some bases of value may prohibit a valuer from
making subjective adjustments to price information from an active
market. Price information from an inactive market may still be good
evidence of value, but subjective adjustments may be needed.\label{9.5.1.8-End}

\stepcounter{SubSecCounter} 

\theSubsection.\theSubSecCounter.\label{9.5.1.9} In certain circumstances,
the valuer and the client may agree on the valuation approaches, methods
and procedures the valuer will use or the extent of procedures the
valuer will perform. Depending on the limitations placed on the valuer
and procedures performed, such circumstances may result in a valuation
that is not IVS compliant.\label{9.5.1.9-End}

\stepcounter{SubSecCounter} 

\theSubsection.\theSubSecCounter.\label{9.5.1.10} A valuation may
be limited or restricted where the valuer is not able to employ the
valuation approaches, methods and procedures that a reasonable and
informed third party would perform, and it is reasonable to expect
that the effect of the limitation or restriction on the estimate of
value could be material.\label{9.5.1.10-End}\label{subsec:9.5.1_Introduction-End}

\subsection{Market Approach\label{subsec:9.5.2_Market_Approach}}

\stepcounter{SubSecCounter} 

\theSubsection.\theSubSecCounter.\label{9.5.2.1}The market approach
provides an indication of value by comparing the asset with identical
or comparable (that is similar) assets for which price information
is available.\label{9.5.2.1-End}

\stepcounter{SubSecCounter} 

\theSubsection.\theSubSecCounter.\label{9.5.2.2} The market approach
should be applied and afforded significant weight under the following
circumstances: 
\begin{enumerate}
\item the subject asset has recently been sold in a transaction appropriate
for consideration under the basis of value;
\item the subject asset or substantially similar assets are actively publicly
traded;
\item there are frequent and/or recent observable transactions in substantially
similar assets.\label{9.5.2.2-End}
\end{enumerate}
\stepcounter{SubSecCounter} 

\theSubsection.\theSubSecCounter.\label{9.5.2.3} Although the above
circumstances would indicate that the market approach should be applied
and afforded significant weight, when the above criteria are not met,
the following are additional circumstances where the market approach
may be applied and afforded significant weight. When using the market
approach under the following circumstances, a valuer should consider
whether any other approaches can be applied and weighted to corroborate
the value indication from the market approach: 
\begin{enumerate}
\item Transactions involving the subject asset or substantially similar
assets are not recent enough considering the levels of volatility
and activity in the market. 
\item The asset or substantially similar assets are publicly traded, but
not actively. 
\item Information on market transactions is available, but the comparable
assets have significant differences to the subject asset, potentially
requiring subjective adjustments. 
\item Information on recent transactions is not reliable (ie, hearsay, missing
information, synergistic purchaser, not arm’s-length, distressed sale,
etc). 
\item The critical element affecting the value of the asset is the price
it would achieve in the market rather than the cost of reproduction
or its income-producing ability.\label{9.5.2.3-End}
\end{enumerate}
\stepcounter{SubSecCounter} 

\theSubsection.\theSubSecCounter.\label{9.5.2.4} The heterogeneous
nature of many assets means that it is often not possible to find
market evidence of transactions involving identical or similar assets.
Even in circumstances where the market approach is not used, the use
of market-based inputs should be maximised in the application of other
approaches (eg, market-based valuation metrics such as effective yields
and rates of return).\label{9.5.2.4-End}

\stepcounter{SubSecCounter} 

\theSubsection.\theSubSecCounter.\label{9.5.2.5} When comparable
market information does not relate to the exact or substantially the
same asset, the valuer must perform a comparative analysis of qualitative
and quantitative similarities and differences between the comparable
assets and the subject asset. It will often be necessary to make adjustments
based on this comparative analysis. Those adjustments must be reasonable
and valuers must document the reasons for the adjustments and how
they were quantified.\label{9.5.2.5-End}

\stepcounter{SubSecCounter} 

\theSubsection.\theSubSecCounter.\label{9.5.2.6} The market approach
often uses market multiples derived from a set of comparables, each
with different multiples. The selection of the appropriate multiple
within the range requires judgment, considering qualitative and quantitative
factors.\label{9.5.2.6-End}\label{subsec:9.5.2_Market_Approach-End}

\subsection{Market Approach Methods\label{subsec:9.5.3_Market_Approach_Methods}}

\subsubsection{Comparable Transactions Method\label{subsubsec:9.5.3.1_Comparable_Transactions}}

\stepcounter{SubSubSecCounter} 

\theSubSubsection.\theSubSubSecCounter.\label{9.5.3.1.1} The comparable
transactions method, also known as the guideline transactions method,
utilizes information on transactions involving assets that are the
same or similar to the subject asset to arrive at an indication of
value.\label{9.5.3.1.1-End}

\stepcounter{SubSubSecCounter} 

\theSubSubsection.\theSubSubSecCounter.\label{9.5.3.1.2} When the
comparable transactions considered involve the subject asset, this
method is sometimes referred to as the prior transactions method.\label{9.5.3.1.2-End}

\stepcounter{SubSubSecCounter} 

\theSubSubsection.\theSubSubSecCounter.\label{9.5.3.1.3} If few
recent transactions have occurred, the valuer may consider the prices
of identical or similar assets that are listed or offered for sale,
provided the relevance of this information is clearly established,
critically analyzed and documented. This is sometimes referred to
as the comparable listings method and should not be used as the sole
indication of value but can be appropriate for consideration together
with other methods. When considering listings or offers to buy or
sell, the weight afforded to the listings/ offer price should consider
the level of commitment inherent in the price and how long the listing/offer
has been on the market. For example, an offer that represents a binding
commitment to purchase or sell an asset at a given price may be given
more weight than a quoted price without such a binding commitment.\label{9.5.3.1.3-End}

\stepcounter{SubSubSecCounter} 

\theSubSubsection.\theSubSubSecCounter.\label{9.5.3.1.4} The comparable
transaction method can use a variety of different comparable evidence,
also known as units of comparison, which form the basis of the comparison.
For example, a few of the many common units of comparison used for
real property interests include price per square foot (or per square
meter), rent per square foot (or per square meter) and capitalization
rates. A few of the many common units of comparison used in business
valuation include EBITDA (Earnings Before Interest, Tax, Depreciation
and Amortization) multiples, earnings multiples, revenue multiples
and book value multiples. A few of the many common units of comparison
used in financial instrument valuation include metrics such as yields
and interest rate spreads. The units of comparison used by participants
can differ between asset classes and across industries and geographies.\label{9.5.3.1.4-End}

\stepcounter{SubSubSecCounter} 

\theSubSubsection.\theSubSubSecCounter.\label{9.5.3.1.5} A subset
of the comparable transactions method is matrix pricing, which is
principally used to value some types of financial instruments, such
as debt securities, without relying exclusively on quoted prices for
the specific securities, but rather relying on the securities’ relationship
to other benchmark quoted securities and their attributes (i.\,e,
yield).\label{9.5.3.1.5-End}

\stepcounter{SubSubSecCounter} 

\theSubSubsection.\theSubSubSecCounter.\label{9.5.3.1.6} The key
steps in the comparable transactions method are: 
\begin{enumerate}
\item identify the units of comparison that are used by participants in
the relevant market;
\item identify the relevant comparable transactions and calculate the key
valuation metrics for those transactions;
\item perform a consistent comparative analysis of qualitative and quantitative
similarities and differences between the comparable assets and the
subject asset;
\item make necessary adjustments, if any, to the valuation metrics to reflect
differences between the subject asset and the comparable assets (see
para 30.12(d));
\item apply the adjusted valuation metrics to the subject asset;
\item if multiple valuation metrics were used, reconcile the indications
of value.\label{9.5.3.1.6-End}
\end{enumerate}
\stepcounter{SubSubSecCounter} 

\theSubSubsection.\theSubSubSecCounter.\label{9.5.3.1.7} A valuer
should choose comparable transactions within the following context: 
\begin{enumerate}
\item evidence of several transactions is generally preferable to a single
transaction or event;
\item evidence from transactions of very similar assets (ideally identical)
provides a better indication of value than assets where the transaction
prices require significant adjustments;
\item transactions that happen closer to the valuation date are more representative
of the market at that date than older/dated transactions, particularly
in volatile markets;
\item for most bases of value, the transactions should be “arm’s length”
between unrelated parties;
\item sufficient information on the transaction should be available to allow
the valuer to develop a reasonable understanding of the comparable
asset and assess the valuation metrics/comparable evidence;
\item information on the comparable transactions should be from a reliable
and trusted source;
\item actual transactions provide better valuation evidence than intended
transactions.\label{9.5.3.1.7-End}
\end{enumerate}
\stepcounter{SubSubSecCounter} 

\theSubSubsection.\theSubSubSecCounter.\label{9.5.3.1.8} A valuer
should analyse and make adjustments for any material differences between
the comparable transactions and the subject asset. Examples of common
differences that could warrant adjustments may include, but are not
limited to: 
\begin{enumerate}
\item material characteristics (age, size, specifications, etc);
\item relevant restrictions on either the subject asset or the comparable
assets;
\item geographical location (location of the asset and/or location of where
the asset is likely to be transacted/used) and the related economic
and regulatory environments;
\item profitability or profit-making capability of the assets;
\item historical and expected growth;
\item yields/coupon rates;
\item types of collateral;
\item unusual terms in the comparable transactions;
\item differences related to marketability and control characteristics of
the comparable and the subject asset;
\item ownership characteristics (e.\,g., legal form of ownership, amount
percentage held).\label{9.5.3.1.8-End}\label{subsubsec:9.5.3.1_Comparable_Transactions-End}
\end{enumerate}

\subsubsection{Guideline publicly-traded comparable method\label{subsubsec:9.5.3.2_Guideline_publicity}}

\stepcounter{SubSubSecCounter} 

\theSubSubsection.\theSubSubSecCounter.\label{9.5.3.2.1} The guideline
publicly-traded method utilises information on publicly-traded comparables
that are the same or similar to the subject asset to arrive at an
indication of value.\label{9.5.3.2.1-End}

\stepcounter{SubSubSecCounter} 

\theSubSubsection.\theSubSubSecCounter.\label{9.5.3.2.2} This method
is similar to the comparable transactions method. However, there are
several differences due to the comparables being publicly traded,
as follows: 
\begin{enumerate}
\item the valuation metrics/comparable evidence are available as of the
valuation date;
\item detailed information on the comparables are readily available in public
filings;
\item the information contained in public filings is prepared under well-
understood accounting standards.\label{9.5.3.2.2-End}
\end{enumerate}
\stepcounter{SubSubSecCounter} 

\theSubSubsection.\theSubSubSecCounter.\label{9.5.3.2.3} The method
should be used only when the subject asset is sufficiently similar
to the publicly-traded comparables to allow for meaningful comparison.\label{9.5.3.2.3-End}

\stepcounter{SubSubSecCounter} 

\theSubSubsection.\theSubSubSecCounter.\label{9.5.3.2.4} The key
steps in the guideline publicly-traded comparable method are to: 
\begin{enumerate}
\item identify the valuation metrics/comparable evidence that are used by
participants in the relevant market;
\item identify the relevant guideline publicly-traded comparables and calculate
the key valuation metrics for those transactions;
\item perform a consistent comparative analysis of qualitative and quantitative
similarities and differences between the publicly-traded comparables
and the subject asset;
\item make necessary adjustments, if any, to the valuation metrics to reflect
differences between the subject asset and the publicly-traded comparables;
\item apply the adjusted valuation metrics to the subject asset;
\item if multiple valuation metrics were used, weight the indications of
value.\label{9.5.3.2.4-End}
\end{enumerate}
\stepcounter{SubSubSecCounter} 

\theSubSubsection.\theSubSubSecCounter.\label{9.5.3.2.5} A valuer
should choose publicly-traded comparables within the following context: 
\begin{enumerate}
\item consideration of multiple publicly-traded comparables is preferred
to the use of a single comparable;
\item evidence from similar publicly-traded comparables (for example, with
similar market segment, geographic area, size in revenue and/or assets,
growth rates, profit margins, leverage, liquidity and diversification)
provides a better indication of value than comparables that require
significant adjustments;
\item securities that are actively traded provide more meaningful evidence
than thinly-traded securities.\label{9.5.3.2.5-End}
\end{enumerate}
\stepcounter{SubSubSecCounter} 

\theSubSubsection.\theSubSubSecCounter.\label{9.5.3.2.6} A valuer
should analyze and make adjustments for any material differences between
the guideline publicly-traded comparables and the subject asset. Examples
of common differences that could warrant adjustments may include,
but are not limited to: 
\begin{enumerate}
\item material characteristics (age, size, specifications, etc);
\item relevant discounts and premiums (see para 30.17);
\item relevant restrictions on either the subject asset or the comparable
assets;
\item geographical location of the underlying company and the related economic
and regulatory environments;
\item profitability or profit-making capability of the assets;
\item historical and expected growth;
\item differences related to marketability and control characteristics of
the comparable and the subject asset;
\item type of ownership.\label{9.5.3.2.6-End}\label{subsubsec:9.5.3.2_Guideline_publicity-End}
\end{enumerate}

\subsubsection{Other Market Approach Considerations\label{subsubsec:9.5.3.3_Other_Market_Approach}}

\stepcounter{SubSubSecCounter} 

\theSubSubsection.\theSubSubSecCounter.\label{9.5.3.3.1} The following
paragraphs address a non-exhaustive list of certain special considerations
that may form part of a market approach valuation.\label{9.5.3.3.1-End}

\stepcounter{SubSubSecCounter} 

\theSubSubsection.\theSubSubSecCounter.\label{9.5.3.3.2} Anecdotal
or “rule-of-thumb” valuation benchmarks are sometimes considered to
be a market approach. However, value indications derived from the
use of such rules should not be given substantial weight unless it
can be shown that buyers and sellers place significant reliance on
them.\label{9.5.3.3.2-End}

\stepcounter{SubSubSecCounter} 

\theSubSubsection.\theSubSubSecCounter.\label{9.5.3.3.3} In the
market approach, the fundamental basis for making adjustments is to
adjust for differences between the subject asset and the guideline
transactions or publicly-traded securities. Some of the most common
adjustments made in the market approach are known as discounts and
premiums.
\begin{enumerate}
\item Discounts for Lack of Marketability (DLOM) should be applied when
the comparables are deemed to have superior marketability to the subject
asset. A DLOM reflects the concept that when comparing otherwise identical
assets, a readily marketable asset would have a higher value than
an asset with a long marketing period or restrictions on the ability
to sell the asset. For example, publicly-traded securities can be
bought and sold nearly instantaneously while shares in a private company
may require a significant amount of time to identify potential buyers
and complete a transaction. Many bases of value allow the consideration
of restrictions on marketability that are inherent in the subject
asset but prohibit consideration of marketability restrictions that
are specific to a particular owner. DLOMs may be quantified using
any reasonable method, but are typically calculated using option pricing
models, studies that compare the value of publicly-traded shares and
restricted shares in the same company, or studies that compare the
value of shares in a company before and after an initial public offering. 
\item Control Premiums (sometimes referred to as Market Participant Acquisition
Premiums or MPAPs) and Discounts for Lack of Control (DLOC) are applied
to reflect differences between the comparables and the subject asset
with regard to the ability to make decisions and the changes that
can be made as a result of exercising control. All else being equal,
participants would generally prefer to have control over a subject
asset than not. However, participants’ willingness to pay a Control
Premium or DLOC will generally be a factor of whether the ability
to exercise control enhances the economic benefits available to the
owner of the subject asset. Control Premiums and DLOCs may be quantified
using any reasonable method, but are typically calculated based on
either an analysis of the specific cash flow enhancements or reductions
in risk associated with control or by comparing observed prices paid
for controlling interests in publicly-traded securities to the publicly-traded
price before such a transaction is announced. Examples of circumstances
where Control Premiums and DLOC should be considered include where: 
\begin{enumerate}
\item shares of public companies generally do not have the ability to make
decisions related to the operations of the company (they lack control).
As such, when applying the guideline public comparable method to value
a subject asset that reflects a controlling interest, a control premium
may be appropriate;
\item the guideline transactions in the guideline transaction method often
reflect transactions of controlling interests. When using that method
to value a subject asset that reflects a minority interest, a DLOC
may be appropriate. 
\end{enumerate}
\item Blockage discounts are sometimes applied when the subject asset represents
a large block of shares in a publicly-traded security such that an
owner would not be able to quickly sell the block in the public market
without negatively influencing the publicly-traded price. Blockage
discounts may be quantified using any reasonable method but typically
a model is used that considers the length of time over which a participant
could sell the subject shares without negatively impacting the publicly-traded
price (ie, selling a relatively small portion of the security’s typical
daily trading volume each day). Under certain bases of value, particularly
fair value for financial reporting purposes, blockage discounts are
prohibited.\label{9.5.3.3.3-End}\label{subsubsec:9.5.3.3_Other_Market_Approach-End}\label{subsec:9.5.3_Market_Approach_Methods-End}
\end{enumerate}

\subsection{Income Approach\label{subsec:9.5.4_Income_Approach}}

\stepcounter{SubSecCounter} 

\theSubsection.\theSubSecCounter.\label{9.5.4.1}The income approach
provides an indication of value by converting future cash flow to
a single current value. Under the income approach, the value of an
asset is determined by reference to the value of income, cash flow
or cost savings generated by the asset.\label{9.5.4.1-End}

\stepcounter{SubSecCounter} 

\theSubsection.\theSubSecCounter.\label{9.5.4.2}The income approach
should be applied and afforded significant weight under the following
circumstances: 
\begin{enumerate}
\item the income-producing ability of the asset is the critical element
affecting value from a participant perspective;
\item reasonable projections of the amount and timing of future income are
available for the subject asset, but there are few, if any, relevant
market comparables.\label{9.5.4.2-End}
\end{enumerate}
\stepcounter{SubSecCounter} 

\theSubsection.\theSubSecCounter.\label{9.5.4.3} Although the above
circumstances would indicate that the income approach should be applied
and afforded significant weight, the following are additional circumstances
where the income approach may be applied and afforded significant
weight. When using the income approach under the following circumstances,
a valuer should consider whether any other approaches can be applied
and weighted to corroborate the value indication from the income approach: 
\begin{enumerate}
\item the income-producing ability of the subject asset is only one of several
factors affecting value from a participant perspective;
\item there is significant uncertainty regarding the amount and timing of
future income-related to the subject asset;
\item there is a lack of access to information related to the subject asset
(for example, a minority owner may have access to historical financial
statements but not forecasts/budgets);
\item the subject asset has not yet begun generating income, but is projected
to do so.\label{9.5.4.3-End}
\end{enumerate}
\stepcounter{SubSecCounter} 

\theSubsection.\theSubSecCounter.\label{9.5.4.4}A fundamental basis
for the income approach is that investors expect to receive a return
on their investments and that such a return should reflect the perceived
level of risk in the investment.\label{9.5.4.4-End}

\stepcounter{SubSecCounter} 

\theSubsection.\theSubSecCounter.\label{9.5.4.5}Generally, investors
can only expect to be compensated for systematic risk (also known
as “market risk” or “undiversifiable risk”).\label{9.5.4.5-End}\label{subsec:9.5.4_Income_Approach-End}

\subsection{Income Approach Methods\label{subsec:9.5.5_Income_Approach_Methods}}

\stepcounter{SubSubSecCounter} 

\theSubSubsection.\theSubSubSecCounter.\label{9.5.5.0.1}Although
there are many ways to implement the income approach, methods under
the income approach are effectively based on discounting future amounts
of cash flow to present value. They are variations of the Discounted
Cash Flow (DCF) method and the concepts below apply in part or in
full to all income approach methods.\label{9.5.5.0.1-End}

\subsubsection{Discounted Cash Flow (DCF) Method\label{subsubsec:9.5.5.1_DCF_Method} }

\stepcounter{SubSubSecCounter} 

\theSubSubsection.\theSubSubSecCounter.\label{9.5.5.1.1} Under the
DCF method the forecasted cash flow is discounted back to the valuation
date, resulting in a present value of the asset.\label{9.5.5.1.1-End}

\stepcounter{SubSubSecCounter} 

\theSubSubsection.\theSubSubSecCounter.\label{9.5.5.1.2} In some
circumstances for long-lived or indefinite-lived assets, DCF may include
a terminal value which represents the value of the asset at the end
of the explicit projection period. In other circumstances, the value
of an asset may be calculated solely using a terminal value with no
explicit projection period. This is sometimes referred to as an income
capitalization method.\label{9.5.5.1.2-End} 

\stepcounter{SubSubSecCounter} 

\theSubSubsection.\theSubSubSecCounter.\label{9.5.5.1.3} The key
steps in the DCF method are: 
\begin{enumerate}
\item choose the most appropriate type of cash flow for the nature of the
subject asset and the assignment (i.\,e, pre-tax or post-tax, total
cash flows or cash flows to equity, real or nominal, etc);
\item determine the most appropriate explicit period, if any, over which
the cash flow will be forecast;
\item prepare cash flow forecasts for that period;
\item determine whether a terminal value is appropriate for the subject
asset at the end of the explicit forecast period (if any) and then
determine the appropriate terminal value for the nature of the asset;
\item determine the appropriate discount rate;
\item apply the discount rate to the forecasted future cash flow, including
the terminal value, if any.\label{9.5.5.1.3-End}
\end{enumerate}

\paragraph{Type of Cash Flow\label{par:9.5.5.1.1_Cash_Flow_Type}}

\stepcounter{ParCounter} 

\theparagraph.\theParCounter.\label{9.5.5.1.1.1} When selecting
the appropriate type of cash flow for the nature of asset or assignment,
valuers must consider the factors below. In addition, the discount
rate and other inputs must be consistent with the type of cash flow
chosen. 
\begin{enumerate}
\item Cash flow to whole asset or partial interest: Typically cash flow
to the whole asset is used. However, occasionally other levels of
income may be used as well, such as cash flow to equity (after payment
of interest and principle on debt) or dividends (only the cash flow
distributed to equity owners). Cash flow to the whole asset is most
commonly used because an asset should theoretically have a single
value that is independent of how it is financed or whether income
is paid as dividends or reinvested. 
\item The cash flow can be pre-tax or post-tax: The tax rate applied should
be consistent with the basis of value and in many instances would
be a participant tax rate rather than an owner-specific one. 
\item Nominal versus real: Real cash flow does not consider inflation whereas
nominal cash flows include expectations regarding inflation. If expected
cash flow incorporates an expected inflation rate, the discount rate
has to include an adjustment for inflation as well. 
\item Currency: The choice of currency used may have an impact on assumptions
related to inflation and risk. This is particularly true in emerging
markets or in currencies with high inflation rates. The currency in
which the forecast is prepared and related risks are separate and
distinct from risks associated with the country(ies) in which the
asset resides or operates. The type of cash flow contained in the
forecast: For example, a cash flow forecast may represent expected
cash flows, ie, probability-weighted scenarios), most likely cash
flows, contractual cash flows, etc.\label{9.5.5.1.1.1-End}
\end{enumerate}
\stepcounter{ParCounter} 

\theparagraph.\theParCounter.\label{9.5.5.1.1.2} The type of cash
flow chosen should be in accordance with participant’s viewpoints.
For example, cash flows and discount rates for real property are customarily
developed on a pre-tax basis while cash flows and discount rates for
businesses are normally developed on a post-tax basis. Adjusting between
pre-tax and post-tax rates can be complex and prone to error and should
be approached with caution.\label{9.5.5.1.1.2-End}

\stepcounter{ParCounter} 

\theparagraph.\theParCounter.\label{9.5.5.1.1.3} When a valuation
is being developed in a currency (“the valuation currency”) that differs
from the currency used in the cash flow projections (“the functional
currency”), a valuer should use one of the following two currency
translation methods: 
\begin{enumerate}
\item Discount the cash flows in the functional currency using a discount
rate appropriate for that functional currency. Convert the present
value of the cash flows to the valuation currency at the spot rate
on the valuation date. 
\item Use a currency exchange forward curve to translate the functional
currency projections into valuation currency projections and discount
the projections using a discount rate appropriate for the valuation
currency. When a reliable currency exchange forward curve is not available
(for example, due to lack of liquidity in the relevant currency exchange
markets), it may not be possible to use this method and only the method
described in para 50.7(a) can be applied.\label{par:9.5.5.1.1_Cash_Flow_Type-End}
\end{enumerate}

\paragraph{Explicit Forecast Period\label{par:9.5.5.1.2_Forecast_Period}}

\stepcounter{ParCounter} 

\theparagraph.\theParCounter.\label{9.5.5.1.2.1} The selection criteria
will depend upon the purpose of the valuation, the nature of the asset,
the information available and the required bases of value. For an
asset with a short life, it is more likely to be both possible and
relevant to project cash flow over its entire life.\label{9.5.5.1.2.1-End}

\stepcounter{ParCounter} 

\theparagraph.\theParCounter.\label{9.5.5.1.2.2} Valuers should
consider the following factors when selecting the explicit forecast
period: 
\begin{enumerate}
\item the life of the asset;
\item a reasonable period for which reliable data is available on which
to base the projections;
\item the minimum explicit forecast period which should be sufficient for
an asset to achieve a stabilized level of growth and profits, after
which a terminal value can be used;
\item in the valuation of cyclical assets, the explicit forecast period
should generally include an entire cycle, when possible;
\item for finite-lived assets such as most financial instruments, the cash
flows will typically be forecast over the full life of the asset.\label{9.5.5.1.2.2-End}
\end{enumerate}
\stepcounter{ParCounter}

\theparagraph.\theParCounter.\label{9.5.5.1.2.3} In some instances,
particularly when the asset is operating at a stabilized level of
growth and profits at the valuation date, it may not be necessary
to consider an explicit forecast period and a terminal value may form
the only basis for value (sometimes referred to as an income capitalization
method).\label{9.5.5.1.2.3-End}

\stepcounter{ParCounter} 

\theparagraph.\theParCounter.\label{9.5.5.1.2.4} The intended holding
period for one investor should not be the only consideration in selecting
an explicit forecast period and should not impact the value of an
asset. However, the period over which an asset is intended to be held
may be considered in determining the explicit forecast period if the
objective of the valuation is to determine its investment value.\label{9.5.5.1.2.4-End}\label{par:9.5.5.1.2_Forecast_Period-End}

\paragraph{Cash Flow Forecasts\label{par:9.5.5.1.3_Cash_Flow_Forecast}}

\stepcounter{ParCounter} 

\theparagraph.\theParCounter.\label{9.5.5.1.3.1} Cash flow for the
explicit forecast period is constructed using prospective financial
information (PFI) (projected income/inflows and expenditure/outflows).\label{9.5.5.1.3.1-End}

\stepcounter{ParCounter} 

\theparagraph.\theParCounter.\label{9.5.5.1.3.2} As required by
para 50.12, regardless of the source of the PFI (e.\,g, management
forecast), a valuer must perform analysis to evaluate the PFI, the
assumptions underlying the PFI and their appropriateness for the valuation
purpose. The suitability of the PFI and the underlying assumptions
will depend upon the purpose of the valuation and the required bases
of value. For example, cash flow used to determine market value should
reflect PFI that would be anticipated by participants; in contrast,
investment value can be measured using cash flow that is based on
the reasonable forecasts from the perspective of a particular investor.\label{9.5.5.1.3.2-End}

\stepcounter{ParCounter} 

\theparagraph.\theParCounter.\label{9.5.5.1.3.3} The cash flow is
divided into suitable periodic intervals (eg, weekly, monthly, quarterly
or annually) with the choice of interval depending upon the nature
of the asset, the pattern of the cash flow, the data available, and
the length of the forecast period.\label{9.5.5.1.3.3-End}

\stepcounter{ParCounter} 

\theparagraph.\theParCounter.\label{9.5.5.1.3.4} The projected cash
flow should capture the amount and timing of all future cash inflows
and outflows associated with the subject asset from the perspective
appropriate to the basis of value.\label{9.5.5.1.3.4-End}

\stepcounter{ParCounter} 

\theparagraph.\theParCounter.\label{9.5.5.1.3.5} Typically, the
projected cash flow will reflect one of the following: 
\begin{enumerate}
\item contractual or promised cash flow;
\item the single most likely set of cash flow;
\item the probability-weighted expected cash flow;
\item multiple scenarios of possible future cash flow.\label{9.5.5.1.3.5-End}
\end{enumerate}
\stepcounter{ParCounter} 

\theparagraph.\theParCounter.\label{9.5.5.1.3.6} Different types
of cash flow often reflect different levels of risk and may require
different discount rates. For example, probability-weighted expected
cash flows incorporate expectations regarding all possible outcomes
and are not dependent on any particular conditions or events (note
that when a probability-weighted expected cash flow is used, it is
not always necessary for valuers to take into account distributions
of all possible cash flows using complex models and techniques. Rather,
valuers may develop a limited number of discrete scenarios and probabilities
that capture the array of possible cash flows). A single most likely
set of cash flows may be conditional on certain future events and
therefore could reflect different risks and warrant a different discount
rate.\label{9.5.5.1.3.6-End}

\stepcounter{ParCounter} 

\theparagraph.\theParCounter.\label{9.5.5.1.3.7}While valuers often
receive PFI that reflects accounting income and expenses, it is generally
preferable to use cash flow that would be anticipated by participants
as the basis for valuations. For example, accounting non-cash expenses,
such as depreciation and amortization, should be added back, and expected
cash outflows relating to capital expenditures or to changes in working
capital should be deducted in calculating cash flow.\label{9.5.5.1.3.7-End}

\stepcounter{ParCounter} 

\theparagraph.\theParCounter.\label{9.5.5.1.3.8} Valuers must ensure
that seasonality and cyclicality in the subject has been appropriately
considered in the cash flow forecasts.\label{9.5.5.1.3.8-End}\label{par:9.5.5.1.3_Cash_Flow_Forecast-End}

\paragraph{Terminal Value\label{par:9.5.5.1.4_Terminal_Value}}

\stepcounter{ParCounter} 

\theparagraph.\theParCounter.\label{9.5.5.1.4.1} Where the asset
is expected to continue beyond the explicit forecast period, valuers
must estimate the value of the asset at the end of that period. The
terminal value is then discounted back to the valuation date, normally
using the same discount rate as applied to the forecast cash flow.\label{9.5.5.1.4.1-End}

\stepcounter{ParCounter} 

\theparagraph.\theParCounter.\label{9.5.5.1.4.2} The terminal value
should consider: 
\begin{enumerate}
\item whether the asset is deteriorating/finite-lived in nature or indefinite-lived,
as this will influence the method used to calculate a terminal value;
\item whether there is future growth potential for the asset beyond the
explicit forecast period;
\item whether there is a pre-determined fixed capital amount expected to
be received at the end of the explicit forecast period;
\item the expected risk level of the asset at the time the terminal value
is calculated;
\item for cyclical assets, the terminal value should consider the cyclical
nature of the asset and should not be performed in a way that assumes
“peak” or “trough” levels of cash flows in perpetuity;
\item the tax attributes inherent in the asset at the end of the explicit
forecast period (if any) and whether those tax attributes would be
expected to continue into perpetuity.\label{9.5.5.1.4.2-End}
\end{enumerate}
\stepcounter{ParCounter} 

\theparagraph.\theParCounter.\label{9.5.5.1.4.3} Valuers may apply
any reasonable method for calculating a terminal value. While there
are many different approaches to calculating a terminal value, the
three most commonly used methods for calculating a terminal value
are:
\begin{enumerate}
\item Gordon growth model/constant growth model (appropriate only for indefinite-lived
assets);
\item market approach/exit value (appropriate for both deteriorating/finite-lived
assets and indefinite-lived assets);
\item salvage value/disposal cost (appropriate only for deteriorating/ finite-lived
assets). 
\end{enumerate}

\subparagraph{Gordon Growth Model/Constant Growth Model\label{subpar:9.5.5.1.4.1_Gordon}}

\stepcounter{SubParCounter} 

\thesubparagraph.\theSubParCounter.\label{9.5.5.1.4.1.1} The constant
growth model assumes that the asset grows (or declines) at a constant
rate into perpetuity.\label{9.5.5.1.4.1.1-End}\label{subpar:9.5.5.1.4.1_Gordon-End}

\subparagraph{Market Approach/Exit Value\label{subpar:9.5.5.1.4.2_Exit_value} }

\stepcounter{SubParCounter} 

\thesubparagraph.\theSubParCounter.\label{9.5.5.1.4.2.1} The market
approach/exit value method can be performed in a number of ways, but
the ultimate goal is to calculate the value of the asset at the end
of the explicit cash flow forecast.\label{9.5.5.1.4.2.1-End}

\stepcounter{SubParCounter} 

\thesubparagraph.\theSubParCounter.\label{9.5.5.1.4.2.2} Common
ways to calculate the terminal value under this method include application
of a market-evidence based capitalization factor or a market multiple.\label{9.5.5.1.4.2.2-End}

\stepcounter{SubParCounter} 

\thesubparagraph.\theSubParCounter.\label{9.5.5.1.4.2.3} When a
market approach/exit value is used, valuers should comply with the
requirements in the market approach and market approach methods section
of this standard (sections 20 and 30). However, valuers should also
consider the expected market conditions at the end of the explicit
forecast period and make adjustments accordingly.\label{9.5.5.1.4.2.3-End}\label{par:9.5.5.1.4_Terminal_Value-End}

\subparagraph{Salvage Value/Disposal Cost\label{subpar:9.5.5.1.4.3_Salvage_Value}}

\stepcounter{SubParCounter} 

\thesubparagraph.\theSubParCounter.\label{9.5.5.1.4.3.1} The terminal
value of some assets may have little or no relationship to the preceding
cash flow. Examples of such assets include wasting assets such as
a mine or an oil well.\label{9.5.5.1.4.3.1-End}

\stepcounter{SubParCounter} 

\thesubparagraph.\theSubParCounter.\label{9.5.5.1.4.3.2} In such
cases, the terminal value is typically calculated as the salvage value
of the asset, less costs to dispose of the asset. In circumstances
where the costs exceed the salvage value, the terminal value is negative
and referred to as a disposal cost or an asset retirement obligation.\label{9.5.5.1.4.3.2-End}\label{subpar:9.5.5.1.4.3_Salvage_Value-End}

\paragraph{Discount Rate\label{par:9.5.5.1.5_Discount_Rate}}

\stepcounter{ParCounter} 

\theparagraph.\theParCounter.\label{9.5.5.1.5.1} The rate at which
the forecast cash flow is discounted should reflect not only the time
value of money, but also the risks associated with the type of cash
flow and the future operations of the asset.\label{9.5.5.1.5.1-End}

\stepcounter{ParCounter} 

\theparagraph.\theParCounter.\label{9.5.5.1.5.2} The discount rate
must be consistent with the type of cash flow.\label{9.5.5.1.5.2-End}

\stepcounter{ParCounter} 

\theparagraph.\theParCounter.\label{9.5.5.1.5.3} Valuers may use
any reasonable method for developing an appropriate discount rate.
While there are many methods for developing a discount rate or determining
the reasonableness of a discount rate, a non-exhaustive list of common
methods includes: 
\begin{enumerate}
\item a capital asset pricing model (CAPM);
\item a weighted average cost of capital (WACC);
\item observed or inferred rates/yields;
\item a build-up method.\label{9.5.5.1.5.3-End}
\end{enumerate}
\stepcounter{ParCounter} 

\theparagraph.\theParCounter.\label{9.5.5.1.5.4} Valuers should
consider corroborative analyses when assessing the appropriateness
of a discount rate. A non-exhaustive list of common analyses should
include: 
\begin{enumerate}
\item an internal rate of return (IRR);
\item a weighted average return on assets (WARA);
\item value indications from other approaches, such as market approach,
or comparing implied multiples from the income approach with guideline
company market multiples or transaction multiples.\label{9.5.5.1.5.4-End}
\end{enumerate}
\stepcounter{ParCounter} 

\theparagraph.\theParCounter.\label{9.5.5.1.5.5} In developing a
discount rate, a valuer should consider:
\begin{enumerate}
\item the type of asset being valued. For example, discount rates used in
valuing debt would be different to those used when valuing real property
or a business;
\item the rates implicit in comparable transactions in the market;
\item the geographic location of the asset and/or the location of the markets
in which it would trade;
\item the life/term and/or maturity of the asset and the consistency of
inputs. For example, the maturity of the risk-free rate applied will
depend on the circumstances, but a common approach is to match the
maturity of the risk-free rate to the time horizon of the cash flows
being considered;
\item the bases of value being applied;
\item the currency denomination of the projected cash flows.\label{9.5.5.1.5.5-End}
\end{enumerate}
\stepcounter{ParCounter} 

\theparagraph.\theParCounter.\label{9.5.5.1.5.6} In developing a
discount rate, the valuer must:
\begin{enumerate}
\item document the method used for developing the discount rate and support
its use;
\item provide evidence for the derivation of the discount rate, including
the identification of the significant inputs and support for their
derivation or source.\label{9.5.5.1.5.6-End}
\end{enumerate}
\stepcounter{ParCounter} 

\theparagraph.\theParCounter.\label{9.5.5.1.5.7} Valuers must consider
the purpose for which the forecast was prepared and whether the forecast
assumptions are consistent with the basis of value being applied.
If the forecast assumptions are not consistent with the basis of value,
it could be necessary to adjust the forecast or discount rate (see
para 50.38).\label{9.5.5.1.5.7-End}

\stepcounter{ParCounter} 

\theparagraph.\theParCounter.\label{9.5.5.1.5.8} Valuers must consider
the risk of achieving the forecast cash flow of the asset when developing
the discount rate. Specifically, the valuer must evaluate whether
the risk underlying the forecast cash flow assumptions are captured
in the discount rate.\label{9.5.5.1.5.8-End}

\stepcounter{ParCounter} 

\theparagraph.\theParCounter.\label{9.5.5.1.5.9} While there are
many ways to assess the risk of achieving the forecast cash flow,
a non-exhaustive list of common procedures includes: 
\begin{enumerate}
\item Identify the key components of the forecast cash flow and compare
the forecast cash flow key components to: 
\begin{itemize}
\item Historical operating and financial performance of the asset;
\item Historical and expected performance of comparable assets;
\item Historical and expected performance for the industry;
\item Expected near-term and long-term growth rates of the country or region
in which the asset primarily operates.
\end{itemize}
\item Confirm whether the forecast cash flow represents expected cash flows
(i.\,e, probability-weighted scenarios), as opposed to most likely
cash flows (i.\,e, most probable scenario), of the asset, or some
other type of cash flow;
\item If utilizing expected cash flows, consider the relative dispersion
of potential outcomes used to derive the expected cash flows (e.\,g,
higher dispersion may indicate a need for an adjustment to the discount
rate);
\item Compare prior forecasts of the asset to actual results to assess the
accuracy and reliability of managements’ estimates;
\item Consider qualitative factors;
\item Consider the value indications such as those resulting from the market
approach.\label{9.5.5.1.5.9-End}
\end{enumerate}
\stepcounter{ParCounter} 

\theparagraph.\theParCounter.\label{9.5.5.1.5.10} If the valuer
determines that certain risks included in the forecast cash flow for
the asset have not been captured in the discount rate, the valuer
must 1) adjust the forecast, or 2) adjust the discount rate to account
for those risks not already captured.
\begin{enumerate}
\item When adjusting the cash flow forecast, the valuer should provide the
rationale for why the adjustments were necessary, undertake quantitative
procedures to support the adjustments, and document the nature and
amount of the adjustments.
\item When adjusting the discount rate, the valuer should document why it
was not appropriate or possible to adjust the cash flow forecast,
provide the rationale for why such risks are not otherwise captured
in the discount rate, undertake quantitative and qualitative procedures
to support the adjustments, and document the nature and amount of
the adjustment. The use of quantitative procedures does not necessarily
entail quantitative derivation of the adjustment to the discount rate.
A valuer need not conduct an exhaustive quantitative process but should
take into account all the information that is reasonably available.\label{9.5.5.1.5.10-End}
\end{enumerate}
\stepcounter{ParCounter} 

\theparagraph.\theParCounter.\label{9.5.5.1.5.11} In developing
a discount rate, it may be appropriate to consider the impact the
asset’s unit of account has on unsystematic risks and the derivation
of the overall discount rate. For example, the valuer should consider
whether market participants would assess the discount rate for the
asset on a standalone basis, or whether market participants would
assess the asset in the context of a broader portfolio and therefore
consider the potential diversification of unsystematic risks.\label{9.5.5.1.5.11-End}

\stepcounter{ParCounter} 

\theparagraph.\theParCounter.\label{9.5.5.1.5.12} A valuer should
consider the impact of intercompany arrangements and transfer pricing
on the discount rate. For example, it is not uncommon for intercompany
arrangements to specify fixed or guaranteed returns for some businesses
or entities within a larger enterprise, which would lower the risk
of the entity forecasted cash flows and reduce the appropriate discount
rate. However other businesses or entities within the enterprise are
deemed to be residual earners in which both excess return and risk
are allocated, thereby increasing the risk of the entity forecasted
cash flows and the appropriate discount rate.\label{9.5.5.1.5.12-End}\label{par:9.5.5.1.5_Discount_Rate-End}\label{subsec:9.5.5_Income_Approach_Methods-End}

\subsection{Cost Approach\label{subsec:9.5.6_Cost_Approach}}

\stepcounter{SubSecCounter} 

\theSubsection.\theSubSecCounter.\label{9.5.6.1} The cost approach
provides an indication of value using the economic principle that
a buyer will pay no more for an asset than the cost to obtain an asset
of equal utility, whether by purchase or by construction, unless undue
time, inconvenience, risk or other factors are involved. The approach
provides an indication of value by calculating the current replacement
or reproduction cost of an asset and making deductions for physical
deterioration and all other relevant forms of obsolescence.\label{9.5.6.1-End}

\stepcounter{SubSecCounter} 

\theSubsection.\theSubSecCounter.\label{9.5.6.2} The cost approach
should be applied and afforded significant weight under the following
circumstances:
\begin{enumerate}
\item participants would be able to recreate an asset with substantially
the same utility as the subject asset, without regulatory or legal
restrictions, and the asset could be recreated quickly enough that
a participant would not be willing to pay a significant premium for
the ability to use the subject asset immediately;
\item the asset is not directly income-generating and the unique nature
of the asset makes using an income approach or market approach unfeasible;
\item the basis of value being used is fundamentally based on replacement
cost, such as replacement value.\label{9.5.6.2-End}
\end{enumerate}
\stepcounter{SubSecCounter} 

\theSubsection.\theSubSecCounter.\label{9.5.6.3} Although the circumstances
in para 60.2 would indicate that the cost approach should be applied
and afforded significant weight, the following are additional circumstances
where the cost approach may be applied and afforded significant weight.
When using the cost approach under the following circumstances, a
valuer should consider whether any other approaches can be applied
and weighted to corroborate the value indication from the cost approach:
\begin{enumerate}
\item participants might consider recreating an asset of similar utility,
but there are potential legal or regulatory hurdles or significant
time involved in recreating the asset;
\item when the cost approach is being used as a reasonableness check to
other approaches (for example, using the cost approach to confirm
whether a business valued as a going-concern might be more valuable
on a liquidation basis);
\item the asset was recently created, such that there is a high degree of
reliability in the assumptions used in the cost approach.\label{9.5.6.3-End}
\end{enumerate}
\stepcounter{SubSecCounter} 

\theSubsection.\theSubSecCounter.\label{9.5.6.4} The value of a
partially completed asset will generally reflect the costs incurred
to date in the creation of the asset (and whether those costs contributed
to value) and the expectations of participants regarding the value
of the property when complete, but consider the costs and time required
to complete the asset and appropriate adjustments for profit and risk.\label{9.5.6.4-End}

End\label{subsec:9.5.6_Cost_Approach-End}

\subsection{Cost Approach Methods\label{subsec:9.5.7_Cost_Approach_Methods}}

\stepcounter{SubSubSecCounter} 

\theSubSubsection.\theSubSubSecCounter.\label{9.5.7.0.1}Broadly,
there are three cost approach methods:
\begin{enumerate}
\item replacement cost method: a method that indicates value by calculating
the cost of a similar asset offering equivalent utility;
\item reproduction cost method: a method under the cost that indicates value
by calculating the cost to recreating a replica of an asset;
\item summation method: a method that calculates the value of an asset by
the addition of the separate values of its component parts.\label{9.5.7.0.1-End}
\end{enumerate}

\subsubsection{Replacement Cost Method\label{subsubsec:9.5.7.1_Replacement_Cost}}

\stepcounter{SubSubSecCounter} 

\theSubSubsection.\theSubSubSecCounter.\label{9.5.7.1.1} Generally,
replacement cost is the cost that is relevant to determining the price
that a participant would pay as it is based on replicating the utility
of the asset, not the exact physical properties of the asset.\label{9.5.7.1.1-End}

\stepcounter{SubSubSecCounter} 

\theSubSubsection.\theSubSubSecCounter.\label{9.5.7.1.2} Usually
replacement cost is adjusted for physical deterioration and all relevant
forms of obsolescence. After such adjustments, this can be referred
to as depreciated replacement cost.\label{9.5.7.1.2-End}

\stepcounter{SubSubSecCounter} 

\theSubSubsection.\theSubSubSecCounter.\label{9.5.7.1.3} The key
steps in the replacement cost method are:
\begin{enumerate}
\item calculate all of the costs that would be incurred by a typical participant
seeking to create or obtain an asset providing equivalent utility;
\item determine whether there is any deprecation related to physical, functional
and external obsolescence associated with the subject asset;
\item deduct total deprecation from the total costs to arrive at a value
for the subject asset.\label{9.5.7.1.3-End}
\end{enumerate}
\stepcounter{SubSubSecCounter} 

\theSubSubsection.\theSubSubSecCounter.\label{9.5.7.1.4} The replacement
cost is generally that of a modern equivalent asset, which is one
that provides similar function and equivalent utility to the asset
being valued, but which is of a current design and constructed or
made using current cost-effective materials and techniques.\label{9.5.7.1.4-End}\label{subsubsec:9.5.7.1_Replacement_Cost-End}

\subsubsection{Reproduction Cost Method\label{subsubsec:9.5.7.2_Reproduction_Cost}}

\stepcounter{SubSubSecCounter} 

\theSubSubsection.\theSubSubSecCounter.\label{9.5.7.2.1} Reproduction
cost is appropriate in circumstances such as the following:
\begin{enumerate}
\item the cost of a modern equivalent asset is greater than the cost of
recreating a replica of the subject asset;
\item the utility offered by the subject asset could only be provided by
a replica rather than a modern equivalent.\label{9.5.7.2.1-End}
\end{enumerate}
\stepcounter{SubSubSecCounter} 

\theSubSubsection.\theSubSubSecCounter.\label{9.5.7.2.2} The key
steps in the reproduction cost method are:
\begin{enumerate}
\item calculate all of the costs that would be incurred by a typical participant
seeking to create an exact replica of the subject asset;
\item determine whether there is any deprecation related to physical, functional
and external obsolescence associated with the subject asset;
\item deduct total deprecation from the total costs to arrive at a value
for the subject asset.\label{9.5.7.2.2-End}\label{subsubsec:9.5.7.2_Reproduction_Cost-End}
\end{enumerate}

\subsubsection{Summation Method\label{subsubsec:9.5.7.3_Summation_Method}}

\stepcounter{SubSubSecCounter} 

\theSubSubsection.\theSubSubSecCounter.\label{9.5.7.3.1}The summation
method, also referred to as the underlying asset method, is typically
used for investment companies or other types of assets or entities
for which value is primarily a factor of the values of their holdings.\label{9.5.7.3.1-End}

\stepcounter{SubSubSecCounter} 

\theSubSubsection.\theSubSubSecCounter.\label{9.5.7.3.2} The key
steps in the summation method are:
\begin{enumerate}
\item value each of the component assets that are part of the subject asset
using the appropriate valuation approaches and methods;
\item add the value of the component assets together to reach the value
of the subject asset.\label{9.5.7.3.2-End}\label{subsubsec:9.5.7.3_Summation_Method-End}
\end{enumerate}

\subsubsection{Cost Considerations\label{subsubsec:9.5.7.4_Cost_Considerations}}

\stepcounter{SubSubSecCounter} 

\theSubSubsection.\theSubSubSecCounter.\label{9.5.7.4.1} The cost
approach should capture all of the costs that would be incurred by
a typical participant.\label{9.5.7.4.1-End}

\stepcounter{SubSubSecCounter} 

\theSubSubsection.\theSubSubSecCounter.\label{9.5.7.4.2} The cost
elements may differ depending on the type of the asset and should
include the direct and indirect costs that would be required to replace/
recreate the asset as of the valuation date. Some common items to
consider include:
\begin{enumerate}
\item direct costs:
\begin{enumerate}
\item materials;
\item labor.
\end{enumerate}
\item indirect costs:
\begin{enumerate}
\item transport costs;
\item installation costs;
\item professional fees (design, permit, architectural, legal, etc);
\item other fees (commissions, etc);
\item overheads;
\item taxes;
\item finance costs (e.\,g, interest on debt financing);
\item profit margin/entrepreneurial profit to the creator of the asset (e.\,g,
return to investors).\label{9.5.7.4.2-End}
\end{enumerate}
\end{enumerate}
\stepcounter{SubSubSecCounter} 

\theSubSubsection.\theSubSubSecCounter.\label{9.5.7.4.3} An asset
acquired from a third party would presumably reflect their costs associated
with creating the asset as well as some form of profit margin to provide
a return on their investment. As such, under bases of value that assume
a hypothetical transaction, it may be appropriate to include an assumed
profit margin on certain costs which can be expressed as a target
profit, either a lump sum or a percentage return on cost or value.
However, financing costs, if included, may already reflect participants’
required return on capital deployed, so valuers should be cautious
when including both financing costs and profit margins.\label{9.5.7.4.3-End}

\stepcounter{SubSubSecCounter} 

\theSubSubsection.\theSubSubSecCounter.\label{9.5.7.4.4} When costs
are derived from actual, quoted or estimated prices by third party
suppliers or contractors, these costs will already include a third
parties’ desired level of profit.\label{9.5.7.4.4-End}

\stepcounter{SubSubSecCounter} 

\theSubSubsection.\theSubSubSecCounter.\label{9.5.7.4.5} The actual
costs incurred in creating the subject asset (or a comparable reference
asset) may be available and provide a relevant indicator of the cost
of the asset. However, adjustments may need to be made to reflect
the following:
\begin{enumerate}
\item cost fluctuations between the date on which this cost was incurred
and the valuation date;
\item any atypical or exceptional costs, or savings, that are reflected
in the cost data but that would not arise in creating an equivalent.\label{9.5.7.4.5-End}\label{subsubsec:9.5.7.4_Cost_Considerations-End}
\end{enumerate}

\subsubsection{Depreciation/Obsolescence\label{subsubsec:9.5.7.5_Depreciation}}

\stepcounter{SubSubSecCounter} 

\theSubSubsection.\theSubSubSecCounter.\label{9.5.7.5.1} In the
context of the cost approach, “depreciation” refers to adjustments
made to the estimated cost of creating an asset of equal utility to
reflect the impact on value of any obsolescence affecting the subject
asset. This meaning is different from the use of the word in financial
reporting or tax law where it generally refers to a method for systematically
expensing capital expenditure over time.\label{9.5.7.5.1-End}

\stepcounter{SubSubSecCounter} 

\theSubSubsection.\theSubSubSecCounter.\label{9.5.7.5.2} Depreciation
adjustments are normally considered for the following types of obsolescence,
which may be further divided into subcategories when making adjustments:
\begin{enumerate}
\item Physical obsolescence: Any loss of utility due to the physical deterioration
of the asset or its components resulting from its age and usage.
\item Functional obsolescence: Any loss of utility resulting from inefficiencies
in the subject asset compared to its replacement such as its design,
specification or technology being outdated.
\item External or economic obsolescence: Any loss of utility caused by economic
or locational factors external to the asset. This type of obsolescence
can be temporary or permanent.\label{9.5.7.5.2-End}
\end{enumerate}
\stepcounter{SubSubSecCounter} 

\theSubSubsection.\theSubSubSecCounter.\label{9.5.7.5.3} Depreciation/obsolescence
should consider the physical and economic lives of the asset:
\begin{enumerate}
\item The physical life is how long the asset could be used before it would
be worn out or beyond economic repair, assuming routine maintenance
but disregarding any potential for refurbishment or reconstruction.
\item The economic life is how long it is anticipated that the asset could
generate financial returns or provide a non-financial benefit in its
current use. It will be influenced by the degree of functional or
economic obsolescence to which the asset is exposed.\label{9.5.7.5.3-End}
\end{enumerate}
\stepcounter{SubSubSecCounter} 

\theSubSubsection.\theSubSubSecCounter.\label{9.5.7.5.4} Except
for some types of economic or external obsolescence, most types of
obsolescence are measured by making comparisons between the subject
asset and the hypothetical asset on which the estimated replacement
or reproduction cost is based. However, when market evidence of the
effect of obsolescence on value is available, that evidence should
be considered.\label{9.5.7.5.4-End}

\stepcounter{SubSubSecCounter} 

\theSubSubsection.\theSubSubSecCounter.\label{9.5.7.5.5} Physical
obsolescence can be measured in two different ways:
\begin{enumerate}
\item curable physical obsolescence, i.\,e, the cost to fix/cure the obsolescence;
\item incurable physical obsolescence which considers the asset’s age, expected
total and remaining life where the adjustment for physical obsolescence
is equivalent to the proportion of the expected total life consumed.
Total expected life may be expressed in any reasonable way, including
expected life in years, mileage, units produced.\label{9.5.7.5.5-End}
\end{enumerate}
\stepcounter{SubSubSecCounter} 

\theSubSubsection.\theSubSubSecCounter.\label{9.5.7.5.6} There are
two forms of functional obsolescence:
\begin{enumerate}
\item excess capital cost, which can be caused by changes in design, materials
of construction, technology or manufacturing techniques resulting
in the availability of modern equivalent assets with lower capital
costs than the subject asset;
\item excess operating cost, which can be caused by improvements in design
or excess capacity resulting in the availability of modern equivalent
assets with lower operating costs than the subject asset.\label{9.5.7.5.6-End}
\end{enumerate}
\stepcounter{SubSubSecCounter} 

\theSubSubsection.\theSubSubSecCounter.\label{9.5.7.5.7} Economic
obsolescence may arise when external factors affect an individual
asset or all the assets employed in a business and should be deducted
after physical deterioration and functional obsolescence. For real
estate, examples of economic obsolescence include:
\begin{enumerate}
\item adverse changes to demand for the products or services produced by
the asset;
\item oversupply in the market for the asset;
\item a disruption or loss of a supply of labor or raw material;
\item the asset being used by a business that cannot afford to pay a market
rent for the assets and still generate a market rate of return.\label{9.5.7.5.7-End}
\end{enumerate}
\stepcounter{SubSubSecCounter} 

\theSubSubsection.\theSubSubSecCounter.\label{9.5.7.5.8} Cash or
cash equivalents do not suffer obsolescence and are not adjusted.
Marketable assets are not adjusted below their market value determined
using the market approach.\label{9.5.7.5.8-End}\label{subsubsec:9.5.7.5_Depreciation-End}

\subsubsection{Valuation Model\label{subsubsec:9.5.7.6_Valuation_Model}}

\stepcounter{SubSubSecCounter} 

\theSubSubsection.\theSubSubSecCounter.\label{9.5.7.6.1} A valuation
model refers collectively to the quantitative methods, systems, techniques
and qualitative judgments used to estimate and document value.\label{9.5.7.6.1-End}

\stepcounter{SubSubSecCounter} 

\theSubSubsection.\theSubSubSecCounter.\label{9.5.7.6.2} When using
or creating a valuation model, the valuer must:
\begin{enumerate}
\item Keep appropriate records to support the selection or creation of the
model;
\item Understand and ensure the output of the valuation model, the significant
assumptions and limiting conditions are consistent with the basis
and scope of the valuation;
\item Consider the key risks associated with the assumptions made in the
valuation model.\label{9.5.7.6.2-End}
\end{enumerate}
\stepcounter{SubSubSecCounter} 

\theSubSubsection.\theSubSubSecCounter.\label{9.5.7.6.3} Regardless
of the nature of the valuation model, to be IVS compliant, the valuer
must ensure that the valuation complies with all other requirements
contained within IVS.\label{9.5.7.6.3-End}\label{subsubsec:9.5.7.6_Valuation_Model-End}\label{subsec:9.5.7_Cost_Approach_Methods-End}\label{sec:9.5_IVS-105_Valuation_Approaches-End}\label{chap:9-General_Standards-End}

\chapter{Asset Standards\label{chap:10_Asset_Standards}}

\section{IVS 200. Businesses and Business Interests\label{sec:10.1_IVS-200_Businesses}}

\subsection{Overview\label{subsec:10.1.1_Overview}}

\stepcounter{SubSecCounter} 

\theSubsection.\theSubSecCounter.\label{10.1.1.1}The principles
contained in the General Standards apply to valuations of businesses
and business interests. This standard contains additional requirements
that apply to valuations of businesses and business interests.\label{10.1.1.1-End}\label{subsec:10.1.1_Overview-End}

\subsection{Introduction\label{subsec:10.1.2_Introduction}}

\stepcounter{SubSecCounter} 

\theSubsection.\theSubSecCounter.\label{10.1.2.1} The definition
of what constitutes a business may differ depending on the purpose
of a valuation. However, generally a business conducts a commercial,
industrial, service or investment activity. Businesses can take many
forms, such as corporations, partnerships, joint ventures and sole
proprietorships. The value of a business may differ from the sum of
the values of the individual assets or liabilities that make up that
business. When a business value is greater than the sum of the recorded
and unrecorded net tangible and identifiable intangible assets of
the business, the excess value is often referred to as going concern
value or goodwill.\label{10.1.2.1-End}

\stepcounter{SubSecCounter} 

\theSubsection.\theSubSecCounter.\label{10.1.2.2} When valuing individual
assets or liabilities owned by a business, valuers should follow the
applicable standard for that type of asset or liability (IVS 210 Intangible
Assets, IVS 400 Real Property Interests, etc).\label{10.2.2.2-End}

\stepcounter{SubSecCounter} 

\theSubsection.\theSubSecCounter.\label{10.1.2.3} Valuers must establish
whether the valuation is of the entire entity, shares or a shareholding
in the entity (whether a controlling or non-controlling interest),
or a specific business activity of the entity. The type of value being
provided must be appropriate to the purpose of the valuation and communicated
as part of the scope of the engagement (see IVS 101 Scope of Work).
It is especially critical to clearly define the business or business
interest being valued as, even when a valuation is performed on an
entire entity, there may be different levels at which that value could
be expressed. For example: 
\begin{enumerate}
\item Enterprise value: Often described as the total value of the equity
in a business plus the value of its debt or debt-related liabilities,
minus any cash or cash equivalents available to meet those liabilities. 
\item Total invested capital value: The total amount of money currently
invested in a business, regardless of the source, often reflected
as the value of total assets less current liabilities and cash. 
\item Operating Value: The total value of the operations of the business,
excluding the value of any non-operating assets and liabilities. 
\item Equity value: The value of a business to all of its equity shareholders.\label{10.1.2.3-End}
\end{enumerate}
\stepcounter{SubSecCounter} 

\theSubsection.\theSubSecCounter.\label{10.1.2.4} Valuations of
businesses are required for different purposes including acquisitions,
mergers and sales of businesses, taxation, litigation, insolvency
proceedings and financial reporting. Business valuations may also
be needed as an input or step in other valuations such as the valuation
of stock options, particular class(es) of stock, or debt.\label{10.1.2.4-End}\label{subsec:10.1.2_Introduction-End}

\subsection{Bases of Value\label{subsec:10.1.3_Bases_of_Value}}

\stepcounter{SubSecCounter} 

\theSubsection.\theSubSecCounter.\label{10.1.3.1} In accordance
with IVS 104 Bases of Value, a valuer must select the appropriate
basis(es) of value when valuing a business or business interest.\label{10.1.3.1-End}

\stepcounter{SubSecCounter} 

\theSubsection.\theSubSecCounter.\label{10.1.3.2} Often, business
valuations are performed using bases of value defined by entities/organisations
other than the IVSC (some examples of which are mentioned in IVS 104
Bases of Value) and it is the valuer’s responsibility to understand
and follow the regulation, case law and/or other interpretive guidance
related to those bases of value as of the valuation date.\label{10.1.3.2-End}\label{subsec:10.1.3_Bases_of_Value-End}

\subsection{Valuation Approaches and Methods\label{subsec:10.1.4_Valuation_Approaches}}

\stepcounter{SubSubSecCounter} 

\theSubSubsection.\theSubSubSecCounter.\label{10.1.4.0.1} The three
principal valuation approaches described in IVS 105 Valuation Approaches
and Methods may be applied to the valuation of businesses and business
interests.\label{10.1.4.0.1-End}

\stepcounter{SubSubSecCounter} 

\theSubSubsection.\theSubSubSecCounter.\label{10.1.4.0.2} When selecting
an approach and method, in addition to the requirements of this standard,
a valuer must follow the requirements of IVS 105 Valuation Approaches
and Methods, including para 10.3.\label{10.1.4.0.2-End}

\subsubsection{Market Approach\label{subsubsec:10.1.4.1_Market_Approach}}

\stepcounter{SubSubSecCounter} 

\theSubSubsection.\theSubSubSecCounter.\label{10.1.4.1.1}The market
approach is frequently applied in the valuation of businesses and
business interests as these assets often meet the criteria in IVS
105 Valuation Approaches and Methods, para 20.2 or 20.3. When valuing
businesses and business interests under the Market Approach, valuers
should follow the requirements of IVS 105 Valuation Approaches and
Methods, sections 20 and 30.\label{10.1.4.1.1-End}

\stepcounter{SubSubSecCounter} 

\theSubSubsection.\theSubSubSecCounter.\label{10.1.4.1.2} The three
most common sources of data used to value businesses and business
interests using the market approach are:
\begin{enumerate}
\item public stock markets in which ownership interests of similar businesses
are traded;
\item the acquisition market in which entire businesses or controlling interests
in businesses are bought and sold;
\item prior transactions in shares or offers for the ownership of the subject
business.\label{10.1.4.1.2-End}
\end{enumerate}
\stepcounter{SubSubSecCounter} 

\theSubSubsection.\theSubSubSecCounter.\label{10.1.4.1.3} There
must be a reasonable basis for comparison with, and reliance upon,
similar businesses in the market approach. These similar businesses
should be in the same industry as the subject business or in an industry
that responds to the same economic variables. Factors that should
be considered in assessing whether a reasonable basis for comparison
exists include:
\begin{enumerate}
\item similarity to the subject business in terms of qualitative and quantitative
business characteristics;
\item amount and verifiability of data on the similar business;
\item whether the price of the similar business represents an arm’s length
and orderly transaction.\label{10.1.4.1.3-End}
\end{enumerate}
\stepcounter{SubSubSecCounter} 

\theSubSubsection.\theSubSubSecCounter.\label{10.1.4.1.4} When applying
a market multiple, adjustments such as those in para 60.8 may be appropriate
to both the subject company and the comparable companies.\label{10.1.4.1.4-End}

\stepcounter{SubSubSecCounter} 

\theSubSubsection.\theSubSubSecCounter.\label{10.1.4.1.5} Valuers
should follow the requirements of IVS 105 Valuation Approaches and
Methods, paras 30.7-30.8 when selecting and adjusting comparable transactions.\label{10.1.4.1.5-End}

\stepcounter{SubSubSecCounter} 

\theSubSubsection.\theSubSubSecCounter.\label{10.1.4.1.6} Valuers
should follow the requirements of IVS 105 Valuation Approaches and
Methods, paras 30.13-30.14 when selecting and adjusting comparable
public company information.\label{10.1.4.1.6-End}\label{subsubsec:10.1.4.1_Market_Approach-End}

\subsubsection{Income Approach\label{subsubsec:10.1.4.2_Income_Approach}}

\stepcounter{SubSubSecCounter} 

\theSubSubsection.\theSubSubSecCounter.\label{10.1.4.2.1} The income
approach is frequently applied in the valuation of businesses and
business interests as these assets often meet the criteria in IVS
105 Valuation Approaches and Methods, paras 40.2 or 40.3.\label{10.1.4.2.1-End}

\stepcounter{SubSubSecCounter} 

\theSubSubsection.\theSubSubSecCounter.\label{10.1.4.2.2} When the
income approach is applied, valuers should follow the requirements
of IVS 105 Valuation Approaches and Methods, sections 40 and 50.\label{10.1.4.2.2-End}

\stepcounter{SubSubSecCounter} 

\theSubSubsection.\theSubSubSecCounter.\label{10.1.4.2.3} Income
and cash flow related to a business or business interest can be measured
in a variety of ways and may be on a pre-tax or post-tax basis. The
capitalization or discount rate applied must be consistent with the
type of income or cash flow used.\label{10.1.4.2.3-End}

\stepcounter{SubSubSecCounter} 

\theSubSubsection.\theSubSubSecCounter.\label{10.1.4.2.4} The type
of income or cash flow used should be consistent with the type of
interest being valued. For example:
\begin{enumerate}
\item enterprise value is typically derived using cash flows before debt
servicing costs and an appropriate discount rate applicable to enterprise-
level cash flows, such as a weighted-average cost of capital;
\item equity value may be derived using cash flows to equity, that is, after
debt servicing costs and an appropriate discount rate applicable to
equity- level cash flows, such as a cost of equity.\label{10.1.4.2.4-End}
\end{enumerate}
\stepcounter{SubSubSecCounter} 

\theSubSubsection.\theSubSubSecCounter.\label{10.1.4.2.5} The income
approach requires the estimation of a capitalization rate when capitalizing
income or cash flow and a discount rate when discounting cash flow.
In estimating the appropriate rate, factors such as the level of interest
rates, rates of return expected by participants for similar investments
and the risk inherent in the anticipated benefit stream are considered
(see IVS 105 Valuation Approaches and Methods, paras 50.29-50.31).\label{10.1.4.2.5-End}

\stepcounter{SubSubSecCounter} 

\theSubSubsection.\theSubSubSecCounter.\label{10.1.4.2.6} In methods
that employ discounting, expected growth may be explicitly considered
in the forecasted income or cash flow. In capitalization methods,
expected growth is normally reflected in the capitalization rate.
If a forecasted cash flow is expressed in nominal terms, a discount
rate that takes into account the expectation of future price changes
due to inflation or deflation should be used. If a forecasted cash
flow is expressed in real terms, a discount rate that takes no account
of expected price changes due to inflation or deflation should be
used.\label{10.1.4.2.6-End}

\stepcounter{SubSubSecCounter} 

\theSubSubsection.\theSubSubSecCounter.\label{10.1.4.2.7} Under
the income approach, the historical financial statements of a business
entity are often used as guide to estimate the future income or cash
flow of the business. Determining the historical trends over time
through ratio analysis may help provide the necessary information
to assess the risks inherent in the business operations in the context
of the industry and the prospects for future performance.\label{10.1.4.2.7-End}

\stepcounter{SubSubSecCounter} 

\theSubSubsection.\theSubSubSecCounter.\label{10.1.4.2.8} Adjustments
may be appropriate to reflect differences between the actual historic
cash flows and those that would be experienced by a buyer of the business
interest on the valuation date. Examples include:
\begin{enumerate}
\item adjusting revenues and expenses to levels that are reasonably representative
of expected continuing operations;
\item presenting financial data of the subject business and comparison businesses
on a consistent basis;
\item adjusting non-arm’s length transactions (such as contracts with customers
or suppliers) to market rates;
\item adjusting the cost of labor or of items leased or otherwise contracted
from related parties to reflect market prices or rates;
\item reflecting the impact of non-recurring events from historic revenue
and expense items. Examples of non-recurring events include losses
caused by strikes, new plant start-up and weather phenomena. However,
the forecast cash flows should reflect any non-recurring revenues
or expenses that can be reasonably anticipated and past occurrences
may be indicative of similar events in the future;
\item adjusting the inventory accounting to compare with similar businesses,
whose accounts may be kept on a different basis from the subject business,
or to more accurately reflect economic reality.\label{10.1.4.2.8-End}
\end{enumerate}
\stepcounter{SubSubSecCounter} 

\theSubSubsection.\theSubSubSecCounter.\label{10.1.4.2.9} When using
an income approach it may also be necessary to make adjustments to
the valuation to reflect matters that are not captured in either the
cash flow forecasts or the discount rate adopted. Examples may include
adjustments for the marketability of the interest being valued or
whether the interest being valued is a controlling or non-controlling
interest in the business. However, valuers should ensure that adjustments
to the valuation do not reflect factors that were already reflected
in the cash flows or discount rate. For example, whether the interest
being valued is a controlling or non-controlling interest is often
already reflected in the forecasted cash flows.\label{10.1.4.2.9-End}

\stepcounter{SubSubSecCounter} 

\theSubSubsection.\theSubSubSecCounter.\label{10.1.4.2.10} While
many businesses may be valued using a single cash flow scenario, valuers
may also apply multi-scenario or simulation models, particularly when
there is significant uncertainty as to the amount and/or timing of
future cash flows.\label{10.1.4.2.10-End}\label{subsubsec:10.1.4.2_Income_Approach-End}

\subsubsection{Cost Approach\label{subsec:10.1.4.3_Cost_Approach}}

\stepcounter{SubSubSecCounter} 

\theSubSubsection.\theSubSubSecCounter.\label{10.1.4.3.1} The cost
approach cannot normally be applied in the valuation of businesses
and business interests as these assets seldom meet the criteria in
IVS 105 Valuation Approaches and Methods, paras 70.2 or 70.3. However,
the cost approach is sometimes applied in the valuation of businesses,
particularly when:
\begin{enumerate}
\item the business is an early stage or start-up business where profits
and/ or cash flow cannot be reliably determined and comparisons with
other businesses under the market approach is impractical or unreliable;
\item the business is an investment or holding business, in which case the
summation method is as described in IVS 105 Valuation Approaches and
Methods, paras 70.8-70.9;
\item the business does not represent a going concern and/or the value of
its assets in a liquidation may exceed the business’ value as a going
concern.\label{10.1.4.3.1-End}
\end{enumerate}
\stepcounter{SubSubSecCounter} 

\theSubSubsection.\theSubSubSecCounter.\label{10.1.4.3.2} In the
circumstances where a business or business interest is valued using
a cost approach, valuers should follow the requirements of IVS 105
Valuation Approaches and Methods, sections 70 and 80.\label{10.1.4.3.2-End}\label{subsec:10.1.4.3_Cost_Approach-End}\label{subsec:10.1.4_Valuation_Approaches-End}

\subsection{Special Considerations for Businesses and Business Interests\label{subsec:10.1.5_Special_Considerations}}

\stepcounter{SubSubSecCounter} 

\theSubSubsection.\theSubSubSecCounter.\label{10.1.5.0.1} The following
sections address a non-exhaustive list of topics relevant to the valuation
of businesses and business interests:
\begin{enumerate}
\item Ownership Rights (section 90);
\item Business Information (section 100);
\item Economic and Industry Considerations (section 110);
\item Operating and Non-Operating Assets (section 120);
\item Capital Structure Considerations (section 130).\label{10.1.5.0.1-End}
\end{enumerate}

\subsubsection{Ownership Rights\label{subsubsec:10.1.5.1_Ownership_Rights}}

\stepcounter{SubSubSecCounter} 

\theSubSubsection.\theSubSubSecCounter.\label{10.1.5.1.1} The rights,
privileges or conditions that attach to the ownership interest, whether
held in proprietorship, corporate or partnership form, require consideration
in the valuation process. Ownership rights are usually defined within
a jurisdiction by legal documents such as articles of association,
clauses in the memorandum of the business, articles of incorporation,
bylaws, partnership agreements and shareholder agreements (collectively
“corporate documents”). In some situations, it may also be necessary
to distinguish between legal and beneficial ownership.\label{10.1.5.1.1-End}

\stepcounter{SubSubSecCounter} 

\theSubSubsection.\theSubSubSecCounter.\label{10.1.5.1.2} Corporate
documents may contain restrictions on the transfer of the interest
or other provisions relevant to value. For example, corporate documents
may stipulate that the interest should be valued as a pro rata fraction
of the entire issued share capital regardless of whether it is a controlling
or non-controlling interest. In each case, the rights of the interest
being valued and the rights attaching to any other class of interest
need to be considered at the outset.\label{10.1.5.1.2-End}

\stepcounter{SubSubSecCounter} 

\theSubSubsection.\theSubSubSecCounter.\label{10.1.5.1.3} Care should
be taken to distinguish between rights and obligations inherent to
the interest and those that may be applicable only to a particular
shareholder (ie, those contained in an agreement between current shareholders
which may not apply to a potential buyer of the ownership interest).
Depending on the basis(es) of value used, the valuer may be required
to consider only the rights and obligations inherent to the subject
interest or both those rights and considerations inherent to the subject
interest and those that apply to a particular owner.\label{10.1.5.1.3-End}

\stepcounter{SubSubSecCounter} 

\theSubSubsection.\theSubSubSecCounter.\label{10.1.5.1.4} All the
rights and preferences associated with a subject business or business
interest should be considered in a valuation, including: 
\begin{enumerate}
\item if there are multiple classes of stock, the valuation should consider
the rights of each different class, including, but not limited to:
\begin{enumerate}
\item liquidation preferences;
\item voting rights;
\item redemption, conversion and participation provisions;
\item put and/or call rights.
\end{enumerate}
\item When a controlling interest in a business may have a higher value
than a non-controlling interest. Control premiums or discounts for
lack of control may be appropriate depending on the valuation method(s)
applied (see IVS 105 Valuation Approaches and Methods, para 30.17.(b)).
In respect of actual premiums paid in completed transactions, the
valuer should consider whether the synergies and other factors that
caused the acquirer to pay those premiums are applicable to the subject
asset to a comparable degree.\label{10.1.5.1.4-End}\label{subsubsec:10.1.5.1_Ownership_Rights-End}
\end{enumerate}

\subsubsection{Business Information\label{subsubsec:10.1.5.2_Business_Information}}

\stepcounter{SubSubSecCounter} 

\theSubSubsection.\theSubSubSecCounter.\label{10.1.5.2.1} The valuation
of a business entity or interest frequently requires reliance upon
information received from management, representatives of the management
or other experts. As required by IVS 105 Valuation Approaches and
Methods, para 10.7, a valuer must assess the reasonableness of information
received from management, representatives of management or other experts
and evaluate whether it is appropriate to rely on that information
for the valuation purpose. For example, prospective financial information
provided by management may reflect owner-specific synergies that may
not be appropriate when using a basis of value that requires a participant
perspective.\label{10.1.5.2.1-End}

\stepcounter{SubSubSecCounter} 

\theSubSubsection.\theSubSubSecCounter.\label{10.1.5.2.2}Although
the value on a given date reflects the anticipated benefits of future
ownership, the history of a business is useful in that it may give
guidance as to the expectations for the future. Valuers should therefore
consider the business’ historical financial statements as part of
a valuation engagement. To the extent the future performance of the
business is expected to deviate significantly from historical experience,
a valuer must understand why historical performance is not representative
of the future expectations of the business.\label{10.1.5.2.2-End}\label{subsubsec:10.1.5.2_Business_Information-End}

\subsubsection{Economic and Industry Considerations\label{subsubsec:10.1.5.3_Economic_Considerations}}

\stepcounter{SubSubSecCounter} 

\theSubSubsection.\theSubSubSecCounter.\label{10.1.5.3.1} Awareness
of relevant economic developments and specific industry trends is
essential for all valuations. Matters such as political outlook, government
policy, exchange rates, inflation, interest rates and market activity
may affect assets in different locations and/or sectors of the economy
quite differently. These factors can be particularly important in
the valuation of businesses and business interests, as businesses
may have complex structures involving multiple locations and types
of operations. For example, a business may be impacted by economic
and industry factors specific related to:
\begin{enumerate}
\item the registered location of the business headquarters and legal form
of the business;
\item the nature of the business operations and where each aspect of the
business is conducted (ie, manufacturing may be done in a different
location to where research and development is conducted);
\item where the business sells its goods and/or services;
\item the currency(ies) the business uses;
\item where the suppliers of the business are located;
\item what tax and legal jurisdictions the business is subject to.\label{10.1.5.3.1-End}\label{subsubsec:10.1.5.3_Economic_Considerations-End}
\end{enumerate}

\subsubsection{Operating and Non-Operating Assets\label{subsubsec:10.1.5.4_Operating_Assets}}

\stepcounter{SubSubSecCounter} 

\theSubSubsection.\theSubSubSecCounter.\label{10.1.5.4.1} The valuation
of an ownership interest in a business is only relevant in the context
of the financial position of the business at a point in time. It is
important to understand the nature of assets and liabilities of the
business and to determine which items are required for use in the
income-producing operations of the business and which ones are redundant
or “excess” to the business at the valuation date.\label{10.1.5.4.1-End}

\stepcounter{SubSubSecCounter} 

\theSubSubsection.\theSubSubSecCounter.\label{10.1.5.4.2} Most valuation
methods do not capture the value of assets that are not required for
the operation of the business. For example, a business valued using
a multiple of EBITDA would only capture the value the assets utilized
in generating that level of EBITDA. If the business had non-operating
assets or liabilities such as an idle manufacturing plant, the value
of that non-operating plant would not be captured in the value. Depending
on the level of value appropriate for the valuation engagement (see
para 20.3), the value of non-operating assets may need to be separately
determined and added to the operating value of the business.\label{10.1.5.4.2-End}

\stepcounter{SubSubSecCounter} 

\theSubSubsection.\theSubSubSecCounter.\label{10.1.5.4.3} Businesses
may have unrecorded assets and/or liabilities that are not reflected
on the balance sheet. Such assets could include intangible assets,
machinery and equipment that is fully depreciated and legal liabilities/lawsuits.\label{10.1.5.4.3-End}

\stepcounter{SubSubSecCounter} 

\theSubSubsection.\theSubSubSecCounter.\label{10.1.5.4.4} When separately
considering non-operating assets and liabilities, a valuer should
ensure that the income and expenses associated with non-operating
assets are excluded from the cash flow measurements and projections
used in the valuation. For example, if a business has a significant
liability associated with an underfunded pension and that liability
is valued separately, the cash flows used in the valuation of the
business should exclude any “catch-up” payments related to that liability.\label{10.1.5.4.4-End}

\stepcounter{SubSubSecCounter} 

\theSubSubsection.\theSubSubSecCounter.\label{10.1.5.4.5} If the
valuation considers information from publicly-traded businesses, the
publicly-traded stock prices implicitly include the value of non-operating
assets, if any. As such, valuers must consider adjusting information
from publicly-traded businesses to exclude the value, income and expenses
associated with non-operating assets.\label{10.1.5.4.5-End}\label{subsubsec:10.1.5.4_Operating_Assets-End}

\subsubsection{Capital Structure Considerations\label{subsubsec:10.1.5.5_Capital_Structure}}

\stepcounter{SubSubSecCounter} 

\theSubSubsection.\theSubSubSecCounter.\label{10.1.5.5.1} Businesses
are often financed through a combination of debt and equity. However,
in many cases, valuers could be asked to value only equity, particular
class of equity, or some other form of ownership interest. While equity
or a particular class of equity can occasionally be valued directly,
more often the enterprise value of the business is determined and
then that value is allocated between the various classes of debt and
equity.\label{10.1.5.5.1-End}

\stepcounter{SubSubSecCounter} 

\theSubSubsection.\theSubSubSecCounter.\label{10.1.5.5.2} While
there are many ownership interests in an asset which a valuer could
be asked to value, a non-exhaustive list of such interests includes:
\begin{enumerate}
\item bonds;
\item convertible debt;
\item partnership interest;
\item minority interest;
\item common equity;
\item preferred equity;
\item options;
\item warrants.\label{10.1.5.5.2-End}
\end{enumerate}
\stepcounter{SubSubSecCounter} 

\theSubSubsection.\theSubSubSecCounter.\label{10.1.5.5.3} When a
valuer is asked to value only equity, or determine how the business
value as a whole is distributed among the various debt and equity
classes, a valuer must determine and consider the different rights
and preferences associated with each class of debt and equity. Rights
and preferences can broadly be categorised as economic rights or control
rights.

A non-exhaustive list of such rights and preferences may include:
\begin{enumerate}
\item dividend or preferred dividend rights;
\item liquidation preferences;
\item voting rights;
\item redemption rights;
\item conversion rights;
\item participation rights;
\item anti-dilution rights;
\item registration rights;
\item put and/or call rights.\label{10.1.5.5.3-End}
\end{enumerate}
\stepcounter{SubSubSecCounter} 

\theSubSubsection.\theSubSubSecCounter.\label{10.1.5.5.4} For simple
capital structures that include only common stock and simple debt
structures (such as bonds, loans and overdrafts), it may be possible
to estimate the value of all of the common stock within the enterprise
by directly estimating the value of debt, subtracting that value from
the enterprise value, then allocating the residual equity value pro
rata to all of the common stock. This method is not appropriate for
all companies with simple capital structures, for example it may not
be appropriate for distressed or highly leveraged companies.\label{10.1.5.5.4-End}

\stepcounter{SubSubSecCounter} 

\theSubSubsection.\theSubSubSecCounter.\label{10.1.5.5.5} For complex
capital structures, being those that include a form of equity other
than just common stock, valuers may use any reasonable method to determine
the value of equity or a particular class of equity. In such cases,
typically the enterprise value of the business is determined and then
that value is allocated between the various classes of debt and equity.
Three methods that valuers could utilize in such instances are discussed
in this section, including:
\begin{enumerate}
\item current value method (CVM);
\item option pricing method (OPM);
\item probability-weighted expected return method (PWERM).\label{10.1.5.5.5-End}
\end{enumerate}
\stepcounter{SubSubSecCounter} 

\theSubSubsection.\theSubSubSecCounter.\label{10.1.5.5.6} While
the CVM is not forward looking, both the OPM and PWERM estimate values
assuming various future outcomes. The PWERM relies on discrete assumptions
for future events and the OPM estimates the future distribution of
outcomes using a lognormal distribution around the current value.\label{10.1.5.5.6-End}

\stepcounter{SubSubSecCounter} 

\theSubSubsection.\theSubSubSecCounter.\label{10.1.5.5.7} A valuer
should consider any potential differences between a “pre-money” and
“post-money” valuation, particularly for early stage companies with
complex capital structures. For example, an infusion of cash (ie,
“post- money valuation”) for such companies may impact the overall
risk profile of the enterprise as well as the relative value allocation
between share classes.\label{10.1.5.5.7-End}

\stepcounter{SubSubSecCounter} 

\theSubSubsection.\theSubSubSecCounter.\label{10.1.5.5.8}

\paragraph{Current Value Method (CVM)\label{par:10.1.5.5.1_Current_Value}}

\stepcounter{ParCounter} 

\theparagraph.\theParCounter.\label{10.1.5.5.1.1} The current value
method (CVM) allocates the enterprise value to the various debt and
equity securities assuming an immediate sale of the enterprise. Under
the CVM, the obligations to debt holders, or debt equivalent securities,
is first deducted from the enterprise value to calculate residual
equity value (valuers should consider if the enterprise value includes
or excludes cash, and the resulting use of gross or net debt for allocation
purposes). Next, value is allocated to the various series of preferred
stock based on the series’ liquidation preferences or conversion values,
whichever would be greater. Finally, any residual value is allocated
to any common equity, options, and warrants.\label{10.1.5.5.1.1-End}

\stepcounter{ParCounter} 

\theparagraph.\theParCounter.\label{10.1.5.5.1.2} A limitation of
the CVM is that it is not forward looking and fails to consider the
option-like payoffs of many share classes.\label{10.1.5.5.1.2-End}

\stepcounter{ParCounter} 

\theparagraph.\theParCounter.\label{10.1.5.5.1.3} The CVM should
only be used when 1) a liquidity event of the enterprise is imminent,
2) when an enterprise is at such an early stage of its development
that no significant common equity value above the liquidation preference
on any preferred equity has been created, 3) no material progress
has been made on the company’s business plan, or 4) no reasonable
basis exists for estimating the amount and timing of any such value
above the liquidation preference that might be created in the future.\label{10.1.5.5.1.3-End}

\stepcounter{ParCounter} 

\theparagraph.\theParCounter.\label{10.1.5.5.1.4} V aluers should
not assume that the value of debt, or debt-like securities, and its
book value are equal without rationale for the determination.\label{10.1.5.5.1.4-End}\label{par:10.1.5.5.1_Current_Value-End}

\paragraph{Option Pricing Method (OPM)\label{par:10.1.5.5.2_Option_pricing}}

\stepcounter{ParCounter} 

\theparagraph.\theParCounter.\label{10.1.5.5.2.1} The OPM values
the different share classes by treating each share class as an option
on the cash flows from the enterprise. The OPM is often applied to
capital structures in which the payout to different share classes
changes at different levels of total equity value, for instance, where
there are convertible preferred shares, management incentive units,
options, or other classes of shares that have certain liquidation
preferences. The OPM may be performed on the enterprise value, thereby
including any debt in the OPM, or on an equity basis after separate
consideration of the debt.\label{10.1.5.5.2.1-End}

\stepcounter{ParCounter} 

\theparagraph.\theParCounter.\label{10.1.5.5.2.2} The OPM considers
the various terms of the stockholder agreements that would affect
the distributions to each class of equity upon a liquidity event,
including the level of seniority among the securities, dividend policy,
conversion ratios, and cash allocations.\label{10.1.5.5.2.2-End}

\stepcounter{ParCounter} 

\theparagraph.\theParCounter.\label{10.1.5.5.2.3} The starting point
for the OPM is the value of total equity for the asset. The OPM is
then applied to allocate the total equity value among equity securities.\label{10.1.5.5.2.3-End}

\stepcounter{ParCounter} 

\theparagraph.\theParCounter.\label{10.1.5.5.2.4} The OPM (or a
related hybrid method) is suited to circumstances where specific future
liquidity events are difficult to forecast or the company is in an
early stage of development.\label{10.1.5.5.2.4-End}

\stepcounter{ParCounter} 

\theparagraph.\theParCounter.\label{10.1.5.5.2.5} The OPM most frequently
relies on the Black--Scholes option pricing model to determine the
value associated with distributions above certain value thresholds.\label{10.1.5.5.2.5-End}

\stepcounter{ParCounter} 

\theparagraph.\theParCounter.\label{10.1.5.5.2.6} When applying
the OPM, a non-exhaustive list of the steps valuers should perform
includes:
\begin{enumerate}
\item Determine the total equity value of the asset;
\item Identify the liquidation preferences, preferred dividend accruals,
conversion prices, and other features attached to the relevant securities
that influences the cash distribution;
\item Determine the different total equity value points (breakpoints) in
which the liquidation preferences and conversion prices become effective;
\item Determine the inputs to the Black--Scholes model:
\begin{enumerate}
\item determine a reasonable time horizon for the OPM;
\item select a risk-free rate corresponding to the time horizon;
\item determine the appropriate volatility factor for the equity of the
asset;
\item determine the expected dividend yield.
\end{enumerate}
\item Calculate a value for the various call options and determine the value
allocated to each interval between the breakpoints;
\item Determine the relative allocation to each class of shares in each
interval between the calculated breakpoints;
\item Allocate the value between the breakpoints (calculated as the call
options) among the share classes based on the allocation determined
in step (f) and the value determined in step (e);
\item Consider additional adjustments to the share classes as necessary,
consistent with the basis of value. For example, it may be appropriate
to apply discounts or premiums.\label{10.1.5.5.2.6-End}
\end{enumerate}
\stepcounter{ParCounter} 

\theparagraph.\theParCounter.\label{10.1.5.5.2.7} When determining
the appropriate volatility assumption valuers should consider:
\begin{enumerate}
\item the development stage of the asset and the relative impact to the
volatility when compared to that observed by the comparable companies;
\item the relative financial leverage of the asset.\label{10.1.5.5.2.7-End}
\end{enumerate}
\stepcounter{ParCounter} 

\theparagraph.\theParCounter.\label{10.1.5.5.2.8} In addition to
the method as discussed above, the OPM can be used to back solve for
the value of total equity value when there is a known price for an
individual security. The inputs to a back solve analysis are the same
as above. Valuers will then solve for the price of the known security
by changing the value of total equity. The back solve method will
also provide a value for all other equity securities.\label{10.1.5.5.2.8-End}\label{par:10.1.5.5.2_Option_pricing-End}

\paragraph{Probability-Weighted Expected Return Method (PWERM)\label{par:10.1.5.5.3_Expected_Return}}

\stepcounter{ParCounter} 

\theparagraph.\theParCounter.\label{10.1.5.5.3.1} Under a PWERM,
the value of the various equity securities are estimated based upon
an analysis of future values for the asset, assuming various future
outcomes. Share value is based upon the probability-weighted present
value of expected future investment returns, considering each of the
possible future outcomes available to the asset, as well as the rights
and preferences of the share classes.\label{10.1.5.5.3.1-End}

\stepcounter{ParCounter} 

\theparagraph.\theParCounter.\label{10.1.5.5.3.2} Typically, the
PWERM is used when the company is close to exit and does not plan
on raising additional capital.\label{10.1.5.5.3.2-End}

\stepcounter{ParCounter} 

\theparagraph.\theParCounter.\label{10.1.5.5.3.3} When applying
the PWERM, a non-exhaustive list of the steps valuers should perform
includes:
\begin{enumerate}
\item Determine the possible future outcomes available to the asset;
\item Estimate the future value of the asset under each outcome;
\item Allocate the estimated future value of the asset to each class of
debt and equity under each possible outcome;
\item Discount the expected value allocated to each class of debt and equity
to present value using a risk-adjusted discount rate;
\item Weight each possible outcome by its respective probability to estimate
the expected future probability-weighted cash flows to each class
of debt and equity;
\item Consider additional adjustments to the share classes as necessary,
consistent with the basis of value. For example, it may be appropriate
to apply discounts or premiums.\label{10.5.5.3.3-End}
\end{enumerate}
\stepcounter{ParCounter} 

\theparagraph.\theParCounter.\label{10.1.5.5.3.4} Valuers should
reconcile the probability-weighted present values of the future exit
values to the overall asset value to make sure that the overall valuation
of the enterprise is reasonable.\label{10.1.5.5.3.4-End}

\stepcounter{ParCounter} 

\theparagraph.\theParCounter.\label{10.1.5.5.3.5} Valuers can combine
elements of the OPM with the PWERM to create a hybrid methodology
by using the OPM to estimate the allocation of value within one or
more of the probability-weighted scenarios.\label{10.1.5.5.3.5-End}\label{par:10.1.5.5.3_Expected_Return-End}\label{subsubsec:10.1.5.5_Capital_Structure-End}

\label{subsec:10.1.5_Special_Considerations-End}\label{sec:10.1_IVS-200_Businesses-End}

\section{IVS 210. Intangible Assets\label{sec:10.2_IVS-210_Intangible_Assets}}

\subsection{Overview\label{subsec:10.2.1_Overview}}

\stepcounter{SubSecCounter} 

\thesubsection.\theSubSecCounter.\label{10.2.1.1} The principles
contained in the General Standards apply to valuations of intangible
assets and valuations with an intangible assets component. This standard
contains additional requirements that apply to valuations of intangible
assets.\label{10.2.1.1-End}\label{subsec:10.2.1_Overview-End}

\subsection{Introduction\label{subsec:10.2.2_Introduction}}

\stepcounter{SubSecCounter} 

\thesubsection.\theSubSecCounter.\label{10.2.2.1} An intangible
asset is a non-monetary asset that manifests itself by its economic
properties. It does not have physical substance but grants rights
and/or economic benefits to its owner.\label{10.2.2.1-End}

\stepcounter{SubSecCounter} 

\thesubsection.\theSubSecCounter.\label{10.2.2.2} Specific intangible
assets are defined and described by characteristics such as their
ownership, function, market position and image. These characteristics
differentiate intangible assets from one another.\label{10.2.2.2-End-1}

\stepcounter{SubSecCounter} 

\thesubsection.\theSubSecCounter.\label{10.2.2.3} There are many
types of intangible assets, but they are often considered to fall
into one or more of the following categories (or goodwill): 
\begin{enumerate}
\item Marketing-related: Marketing-related intangible assets are used primarily
in the marketing or promotion of products or services. Examples include
trademarks, trade names, unique trade design and internet domain names.
\item Customer-related: Customer-related intangible assets include customer
lists, backlog, customer contracts, and contractual and non-contractual
customer relationships.
\item Artistic-related: Artistic-related intangible assets arise from the
right to benefits from artistic works such as plays, books, films
and music, and from non-contractual copyright protection.
\item Contract-related: Contract-related intangible assets represent the
value of rights that arise from contractual agreements. Examples include
licensing and royalty agreements, service or supply contracts, lease
agreements, permits, broadcast rights, servicing contracts, employment
contracts and non-competition agreements and natural resource rights.
\item Technology-based: Technology-related intangible assets arise from
contractual or non-contractual rights to use patented technology,
unpatented technology, databases, formulae, designs, software, processes
or recipes.\label{10.2.2.3-End}
\end{enumerate}
\stepcounter{SubSecCounter} 

\thesubsection.\theSubSecCounter.\label{10.2.2.4} Although similar
intangible assets within the same class will share some characteristics
with one another, they will also have differentiating characteristics
that will vary according to the type of intangible asset. In addition,
certain intangible assets, such as brands, may represent a combination
of categories in para 20.3.\label{10.2.2.4-End}

\stepcounter{SubSecCounter} 

\thesubsection.\theSubSecCounter.\label{10.2.2.5} Particularly in
valuing an intangible asset, valuers must understand specifically
what needs to be valued and the purpose of the valuation. For example,
customer data (names, addresses, etc) typically has a very different
value from customer contracts (those contracts in place on the valuation
date) and customer relationships (the value of the ongoing customer
relationship including existing and future contracts). What intangible
assets need to be valued and how those intangible assets are defined
may differ depending on the purpose of the valuation, and the differences
in how intangible assets are defined can lead to significant differences
in value.\label{10.2.2.5-End}

\stepcounter{SubSecCounter} 

\thesubsection.\theSubSecCounter.\label{10.2.2.6} Generally, goodwill
is any future economic benefit arising from a business, an interest
in a business or from the use of a group of assets which has not been
separately recognised in another asset. The value of goodwill is typically
measured as the residual amount remaining after the values of all
identifiable tangible, intangible and monetary assets, adjusted for
actual or potential liabilities, have been deducted from the value
of a business. It is often represented as the excess of the price
paid in a real or hypothetical acquisition of a company over the value
of the company’s other identified assets and liabilities. For some
purposes, goodwill may need to be further divided into transferable
goodwill (that which can be transferred to third parties) and non-transferable
or “personal” goodwill.\label{10.2.2.6-End}

\stepcounter{SubSecCounter} 

\thesubsection.\theSubSecCounter.\label{10.2.2.7} As the amount
of goodwill is dependent on which other tangible and intangible assets
are recognized, its value can be different when calculated for different
purposes. For example, in a business combination accounted for under
IFRS or US GAAP, an intangible asset is only recognized to the extent
that it:
\begin{enumerate}
\item is separable, ie, capable of being separated or divided from the entity
and sold, transferred, licensed, rented or exchanged, either individually
or together with a related contract, identifiable asset or liability,
regardless of whether the entity intends to do so;
\item arises from contractual or other legal rights, regardless of whether
those rights are transferable or separable from the entity or from
other rights and obligations.\label{10.2.2.7-End}
\end{enumerate}
\stepcounter{SubSecCounter} 

\thesubsection.\theSubSecCounter.\label{10.2.2.8} While the aspects
of goodwill can vary depending on the purpose of the valuation, goodwill
frequently includes elements such as:
\begin{enumerate}
\item company-specific synergies arising from a combination of two or more
businesses (eg, reductions in operating costs, economies of scale
or product mix dynamics);
\item opportunities to expand the business into new and different markets;
\item the benefit of an assembled workforce (but generally not any intellectual
property developed by members of that workforce);
\item the benefit to be derived from future assets, such as new customers
and future technologies;
\item assemblage and going concern value.\label{10.2.2.8-End}
\end{enumerate}
\stepcounter{SubSecCounter} 

\thesubsection.\theSubSecCounter.\label{10.2.2.9} Valuers may perform
direct valuations of intangible assets where the value of the intangible
assets is the purpose of the analysis or one part of the analysis.
However, when valuing businesses, business interests, real property,
and machinery and equipment, valuers should consider whether there
are intangible assets associated with those assets and whether those
directly or indirectly impact the asset being valued. For example,
when valuing a hotel based on an income approach, the contribution
to value of the hotel’s brand may already be reflected in the profit
generated by the hotel.\label{10.2.2.9-End}

\stepcounter{SubSecCounter} 

\thesubsection.\theSubSecCounter.\label{10.2.2.10} Intangible asset
valuations are performed for a variety of purposes. It is the valuer’s
responsibility to understand the purpose of a valuation and whether
intangible assets should be valued, whether separately or grouped
with other assets. A non-exhaustive list of examples of circumstances
that commonly include an intangible asset valuation component is provided
below:
\begin{enumerate}
\item For financial reporting purposes, valuations of intangible assets
are often required in connection with accounting for business combinations,
asset acquisitions and sales, and impairment analysis.
\item For tax reporting purposes, intangible asset valuations are frequently
needed for transfer pricing analyses, estate and gift tax planning
and reporting, and ad valorem taxation analyses.
\item Intangible assets may be the subject of litigation, requiring valuation
analysis in circumstances such as shareholder disputes, damage calculations
and marital dissolutions (divorce).
\item Other statutory or legal events may require the valuation of intangible
assets such as compulsory purchases/eminent domain proceedings.
\item Valuers are often asked to value intangible assets as part of general
consulting, collateral lending and transactional support engagements.\label{10.2.2.10-End}\label{subsec:10.2.2_Introduction-End}
\end{enumerate}

\subsection{Bases of Value\label{subsec:10.2.3_Bases_of_Value}}

\stepcounter{SubSecCounter} 

\thesubsection.\theSubSecCounter.\label{10.2.3.1} In accordance
with IVS 104 Bases of Value, a valuer must select the appropriate
basis(es) of value when valuing intangible assets.\label{10.2.3.1-End}

\stepcounter{SubSecCounter} 

\thesubsection.\theSubSecCounter.\label{10.2.3.2} Often, intangible
asset valuations are performed using bases of value defined by entities/organisations
other than the IVSC (some examples of which are mentioned in IVS 104
Bases of Value) and the valuer must understand and follow the regulation,
case law, and other interpretive guidance related to those bases of
value as of the valuation date.\label{10.2.3.2-End}\label{subsec:10.2.3_Bases_of_Value-End}

\subsection{Valuation Approaches and Methods\label{subsec:10.2.4_Valuation_Approaches}}

\stepcounter{SubSubSecCounter} 

\thesubsubsection.\theSubSubSecCounter.\label{10.2.4.0.1} The three
valuation approaches described in IVS 105 Valuation Approaches can
all be applied to the valuation of intangible assets.\label{10.2.4.0.1-End}

\stepcounter{SubSubSecCounter} 

\thesubsubsection.\theSubSubSecCounter.\label{10.2.4.0.2} When selecting
an approach and method, in addition to the requirements of this standard,
a valuer must follow the requirements of IVS 105 Valuation Approaches,
including para 10.3.\label{10.2.4.0.2-End}

\subsubsection{Market Approach\label{subsubsec:10.2.4.1_Market_Approach}}

\stepcounter{SubSubSecCounter} 

\thesubsubsection.\theSubSubSecCounter.\label{10.2.4.1.1} Under
the market approach, the value of an intangible asset is determined
by reference to market activity (for example, transactions involving
identical or similar assets).\label{10.2.4.1.1-End}

\stepcounter{SubSubSecCounter} 

\thesubsubsection.\theSubSubSecCounter.\label{10.2.4.1.2} Transactions
involving intangible assets frequently also include other assets,
such as a business combination that includes intangible assets.\label{10.2.4.1.2-End}

\stepcounter{SubSubSecCounter} 

\thesubsubsection.\theSubSubSecCounter.\label{10.2.4.1.3} Valuers
must comply with paras 20.2 and 20.3 of IVS 105 when determining whether
to apply the market approach to the valuation of intangible assets.
In addition, valuers should only apply the market approach to value
intangible assets if both of the following criteria are met:
\begin{enumerate}
\item information is available on arm’s length transactions involving identical
or similar intangible assets on or near the valuation date;
\item sufficient information is available to allow the valuer to adjust
for all significant differences between the subject intangible asset
and those involved in the transactions.\label{10.2.4.1.3-End}
\end{enumerate}
\stepcounter{SubSubSecCounter} 

\thesubsubsection.\theSubSubSecCounter.\label{10.2.4.1.4} The heterogeneous
nature of intangible assets and the fact that intangible assets seldom
transact separately from other assets means that it is rarely possible
to find market evidence of transactions involving identical assets.
If there is market evidence at all, it is usually in respect of assets
that are similar, but not identical.\label{10.2.4.1.4-End}

\stepcounter{SubSubSecCounter} 

\thesubsubsection.\theSubSubSecCounter.\label{10.2.4.1.5} Where
evidence of either prices or valuation multiples is available, valuers
should make adjustments to these to reflect differences between the
subject asset and those involved in the transactions. These adjustments
are necessary to reflect the differentiating characteristics of the
subject intangible asset and the assets involved in the transactions.
Such adjustments may only be determinable at a qualitative, rather
than quantitative, level. However, the need for significant qualitative
adjustments may indicate that another approach would be more appropriate
for the valuation.\label{10.2.4.1.5-End}

\stepcounter{SubSubSecCounter} 

\thesubsubsection.\theSubSubSecCounter.\label{10.2.4.1.6} Consistent
with the above, examples of intangible assets for which the market
approach is sometimes used include:
\begin{enumerate}
\item broadcast spectrum;
\item internet domain names;
\item taxi medallions.\label{10.2.4.1.6-End}
\end{enumerate}
\stepcounter{SubSubSecCounter} 

\thesubsubsection.\theSubSubSecCounter.\label{10.2.4.1.7} The guideline
transactions method is generally the only market approach method that
can be applied to intangible assets.\label{10.2.4.1.7-End}

\stepcounter{SubSubSecCounter} 

\thesubsubsection.\theSubSubSecCounter.\label{10.2.4.1.8} In rare
circumstances, a security sufficiently similar to a subject intangible
asset may be publicly traded, allowing the use of the guideline public
company method. One example of such securities is contingent value
rights (CVRs) that are tied to the performance of a particular product
or technology.\label{10.2.4.1.8-End}\label{subsubsec:10.2.4.1_Market_Approach-End}

\subsubsection{Income Approach\label{subsubsec:10.2.4.2_Income_Approach}}

\stepcounter{SubSubSecCounter} 

\thesubsubsection.\theSubSubSecCounter.\label{10.2.4.2.1}Under the
income approach, the value of an intangible asset is determined by
reference to the present value of income, cash flows or cost savings
attributable to the intangible asset over its economic life.\label{10.2.4.2.1-End}

\stepcounter{SubSubSecCounter} 

\thesubsubsection.\theSubSubSecCounter.\label{10.2.4.2.2} Valuers
must comply with paras 40.2 and 40.3 of IVS 105 Valuation Approaches
and Methods when determining whether to apply the income approach
to the valuation of intangible assets.\label{10.2.4.2.2-End}

\stepcounter{SubSubSecCounter} 

\thesubsubsection.\theSubSubSecCounter.\label{10.2.4.2.3} Income
related to intangible assets is frequently included in the price paid
for goods or a service. It may be challenging to separate the income
related to the intangible asset from income related to other tangible
and intangible assets. Many of the income approach methods are designed
to separate the economic benefits associated with a subject intangible
asset.\label{10.2.4.2.3-End}

\stepcounter{SubSubSecCounter} 

\thesubsubsection.\theSubSubSecCounter.\label{10.2.4.2.4} The income
approach is the most common method applied to the valuation of intangible
assets and is frequently used to value intangible assets including
the following:
\begin{enumerate}
\item technology;
\item customer-related intangibles (eg, backlog, contracts, relationships);
\item tradenames/trademarks/brands;
\item operating licenses (eg, franchise agreements, gaming licenses, broadcast
spectrum);
\item non-competition agreements.\label{10.2.4.2.4-End}
\end{enumerate}

\paragraph{Income Approach Methods\label{par:10.2.4.2.1_Income_Methods}}

\stepcounter{SubParCounter} 

\thesubparagraph.\theSubParCounter.\label{10.2.4.2.1.0.1}There are
many income approach methods. The following methods are discussed
in this standard in more detail:
\begin{enumerate}
\item excess earnings method;
\item relief-from-royalty method;
\item premium profit method or with-and-without method;
\item greenfield method;
\item distributor method.\label{10.2.4.2.1.0.1-End}
\end{enumerate}

\subparagraph{Excess Earnings Method\label{subpar:10.2.4.2.1.1_Excess_Earnings}}

\stepcounter{SubParCounter} 

\thesubparagraph.\theSubParCounter.\label{10.2.4.2.1.1.1} The excess
earnings method estimates the value of an intangible asset as the
present value of the cash flows attributable to the subject intangible
asset after excluding the proportion of the cash flows that are attributable
to other assets required to generate the cash flows (“contributory
assets”). It is often used for valuations where there is a requirement
for the acquirer to allocate the overall price paid for a business
between tangible assets, identifiable intangible assets and goodwill.\label{10.2.4.2.1.1.1-End}

\stepcounter{SubParCounter} 

\thesubparagraph.\theSubParCounter.\label{10.2.4.2.1.1.2} Contributory
assets are assets that are used in conjunction with the subject intangible
asset in the realization of prospective cash flows associated with
the subject intangible asset. Assets that do not contribute to the
prospective cash flows associated with the subject intangible asset
are not contributory assets.\label{10.2.4.2.1.1.2-End}

\stepcounter{SubParCounter} 

\thesubparagraph.\theSubParCounter.\label{10.2.4.2.1.1.3} The excess
earnings method can be applied using several periods of forecasted
cash flows (“multi-period excess earnings method” or “MPEEM”), a single
period of forecasted cash flows (“single-period excess earnings method”)
or by capitalising a single period of forecasted cash flows (“capitalized
excess earnings method” or the “formula method”).\label{10.2.4.2.1.1.3-End}

\stepcounter{SubParCounter} 

\thesubparagraph.\theSubParCounter.\label{10.2.4.2.1.1.4} The capitalized
excess earnings method or formula method is generally only appropriate
if the intangible asset is operating in a steady state with stable
growth/decay rates, constant profit margins and consistent contributory
asset levels/charges.\label{10.2.4.2.1.1.4-End}

\stepcounter{SubParCounter} 

\thesubparagraph.\theSubParCounter.\label{10.2.4.2.1.1.5} As most
intangible assets have economic lives exceeding one period, frequently
follow non-linear growth/decay patterns and may require different
levels of contributory assets over time, the MPEEM is the most commonly
used excess earnings method as it offers the most flexibility and
allows valuers to explicitly forecast changes in such inputs.\label{10.2.4.2.1.1.5-End}

\stepcounter{SubParCounter} 

\thesubparagraph.\theSubParCounter.\label{10.2.4.2.1.1.6} Whether
applied in a single-period, multi-period or capitalized manner, the
key steps in applying an excess earnings method are to:
\begin{enumerate}
\item forecast the amount and timing of future revenues driven by the subject
intangible asset and related contributory assets;
\item forecast the amount and timing of expenses that are required to generate
the revenue from the subject intangible asset and related contributory
assets;
\item adjust the expenses to exclude those related to creation of new intangible
assets that are not required to generate the forecasted revenue and
expenses. Profit margins in the excess earnings method may be higher
than profit margins for the overall business because the excess earnings
method excludes investment in certain new intangible assets. For example:
\begin{enumerate}
\item research and development expenditures related to development of new
technology would not be required when valuing only existing technology;
\item marketing expenses related to obtaining new customers would not be
required when valuing existing customer-related intangible assets.
\end{enumerate}
\item identify the contributory assets that are needed to achieve the forecasted
revenue and expenses. Contributory assets often include working capital,
fixed assets, assembled workforce and identified intangible assets
other than the subject intangible asset;
\item determine the appropriate rate of return on each contributory asset
based on an assessment of the risk associated with that asset. For
example, low-risk assets like working capital will typically have
a relatively lower required return. Contributory intangible assets
and highly specialized machinery and equipment often require relatively
higher rates of return;
\item in each forecast period, deduct the required returns on contributory
assets from the forecast profit to arrive at the excess earnings attributable
to only the subject intangible asset;
\item determine the appropriate discount rate for the subject intangible
asset and present value or capitalize the excess earnings;
\item if appropriate for the purpose of the valuation (see paras 110.1-110.4),
calculate and add the tax amortization benefit (TAB) for the subject
intangible asset.\label{10.2.4.2.1.1.6-End}
\end{enumerate}
\stepcounter{SubParCounter} 

\thesubparagraph.\theSubParCounter.\label{10.2.4.2.1.1.7} Contributory
asset charges (CACs) should be made for all the current and future
tangible, intangible and financial assets that contribute to the generation
of the cash flow, and if an asset for which a CAC is required is involved
in more than one line of business, its CAC should be allocated to
the different lines of business involved.\label{10.2.4.2.1.1.7-End}

\stepcounter{SubParCounter} 

\thesubparagraph.\theSubParCounter.\label{10.2.4.2.1.1.8} The determination
of whether a CAC for elements of goodwill is appropriate should be
based on an assessment of the relevant facts and circumstances of
the situation, and the valuer should not mechanically apply CACs or
alternative adjustments for elements of goodwill if the circumstances
do not warrant such a charge. Assembled workforce, as it is quantifiable,
is typically the only element of goodwill for which a CAC should be
taken. Accordingly, valuers must ensure they have a strong basis for
applying CACs for any elements of goodwill other than assembled workforce.\label{10.2.4.2.1.1.8-End}

\stepcounter{SubParCounter} 

\thesubparagraph.\theSubParCounter.\label{10.2.4.2.1.1.9} CACs are
generally computed on an after-tax basis as a fair return on the value
of the contributory asset, and in some cases a return of the contributory
asset is also deducted. The appropriate return on a contributory asset
is the investment return a typical participant would require on the
asset. The return of a contributory asset is a recovery of the initial
investment in the asset. There should be no difference in value regardless
of whether CACs are computed on a pre-tax or after-tax basis.\label{10.2.4.2.1.1.9-End}

\stepcounter{SubParCounter} 

\thesubparagraph.\theSubParCounter.\label{10.2.4.2.1.1.10} If the
contributory asset is not wasting in nature, like working capital,
only a fair return on the asset is required.\label{10.2.4.2.1.1.10-End}

\stepcounter{SubParCounter} 

\thesubparagraph.\theSubParCounter.\label{10.2.4.2.1.1.11} For contributory
intangible assets that were valued under a relief-from- royalty method,
the CAC should be equal to the royalty (generally adjusted to an after-tax
royalty rate).\label{10.2.4.2.1.1.11-End}

\stepcounter{SubParCounter} 

\thesubparagraph.\theSubParCounter.\label{10.2.4.2.1.1.12} The excess
earnings method should be applied only to a single intangible asset
for any given stream of revenue and income (generally the primary
or most important intangible asset). For example, in valuing the intangible
assets of a company utilizing both technology and a tradename in delivering
a product or service (ie, the revenue associated with the technology
and the tradename is the same), the excess earnings method should
only be used to value one of the intangible assets and an alternative
method should be used for the other asset. However, if the company
had multiple product lines, each using a different technology and
each generating distinct revenue and profit, the excess earnings method
may be applied in the valuation of the multiple different technologies.\label{10.2.4.2.1.1.12-End}\label{subpar:10.2.4.2.1.1_Excess_Earnings-End}

\subparagraph{Relief-from-Royalty Method\label{subpar:10.2.4.2.1.2_Royalty_Method}}

\stepcounter{SubParCounter} 

\thesubparagraph.\theSubParCounter.\label{10.2.4.2.1.2.1} Under
the relief-from-royalty method, the value of an intangible asset is
determined by reference to the value of the hypothetical royalty payments
that would be saved through owning the asset, as compared with licensing
the intangible asset from a third party. Conceptually, the method
may also be viewed as a discounted cash flow method applied to the
cash flow that the owner of the intangible asset could receive through
licensing the intangible asset to third parties.\label{10.2.4.2.1.2.1-End}

\stepcounter{SubParCounter} 

\thesubparagraph.\theSubParCounter.\label{10.2.4.2.1.2.2} The key
steps in applying a relief-from-royalty method are to:
\begin{enumerate}
\item develop projections associated with the intangible asset being valued
for the life of the subject intangible asset. The most common metric
projected is revenue, as most royalties are paid as a percentage of
revenue. However, other metrics such as a per-unit royalty may be
appropriate in certain valuations;
\item develop a royalty rate for the subject intangible asset. Two methods
can be used to derive a hypothetical royalty rate. The first is based
on market royalty rates for comparable or similar transactions. A
prerequisite for this method is the existence of comparable intangible
assets that are licensed at arm’s length on a regular basis. The second
method is based on a split of profits that would hypothetically be
paid in an arm’s length transaction by a willing licensee to a willing
licensor for the rights to use the subject intangible asset;
\item apply the selected royalty rate to the projections to calculate the
royalty payments avoided by owning the intangible asset;
\item estimate any additional expenses for which a licensee of the subject
asset would be responsible. This can include upfront payments required
by some licensors. A royalty rate should be analyzed to determine
whether it assumes expenses (such as maintenance, marketing and advertising)
are the responsibility of the licensor or the licensee. A royalty
rate that is “gross” would consider all responsibilities and expenses
associated with ownership of a licensed asset to reside with the licensor,
while a royalty that is “net” would consider some or all responsibilities
and expenses associated with the licensed asset to reside with the
licensee. Depending on whether the royalty is “gross” or “net”, the
valuation should exclude or include, respectively, a deduction for
expenses such as maintenance, marketing or advertising expenses related
to the hypothetically licensed asset;
\item if the hypothetical costs and royalty payments would be tax deductible,
it may be appropriate to apply the appropriate tax rate to determine
the after-tax savings associated with ownership of the intangible
asset. However, for certain purposes (such as transfer pricing), the
effects of taxes are generally not considered in the valuation and
this step should be skipped;
\item determine the appropriate discount rate for the subject intangible
asset and present value or capitalise the savings associated with
ownership of the intangible asset;
\item if appropriate for the purpose of the valuation (see paras 110.1-110.4),
calculate and add the TAB for the subject intangible asset.\label{10.2.4.2.1.2.2-End}
\end{enumerate}
\stepcounter{SubParCounter} 

\thesubparagraph.\theSubParCounter.\label{10.2.4.2.1.2.3} Whether
a royalty rate is based on market transactions or a profit split method
(or both), its selection should consider the characteristics of the
subject intangible asset and the environment in which it is utilised.
The consideration of those characteristics form the basis for selection
of a royalty rate within a range of observed transactions and/or the
range of profit available to the subject intangible asset in a profit
split. Factors that should be considered include the following:
\begin{enumerate}
\item Competitive environment: The size of the market for the intangible
asset, the availability of realistic alternatives, the number of competitors,
barriers to entry and presence (or absence) of switching costs.
\item Importance of the subject intangible to the owner: Whether the subject
asset is a key factor of differentiation from competitors, the importance
it plays in the owner’s marketing strategy, its relative importance
compared with other tangible and intangible assets, and the amount
the owner spends on creation, upkeep and improvement of the subject
asset.
\item Life cycle of the subject intangible: The expected economic life of
the subject asset and any risks of the subject intangible becoming
obsolete.\label{10.2.4.2.1.2.3-End}
\end{enumerate}
\stepcounter{SubParCounter} 

\thesubparagraph.\theSubParCounter.\label{10.2.4.2.1.2.4} When selecting
a royalty rate, a valuer should also consider the following:
\begin{enumerate}
\item When entering a license arrangement, the royalty rate participants
would be willing to pay depends on their profit levels and the relative
contribution of the licensed intangible asset to that profit. For
example, a manufacturer of consumer products would not license a tradename
at a royalty rate that leads to the manufacturer realizing a lower
profit selling branded products compared with selling generic products.
\item When considering observed royalty transactions, a valuer should understand
the specific rights transferred to the licensee and any limitations.
For example, royalty agreements may include significant restrictions
on the use of a licensed intangible asset such as a restriction to
a particular geographic area or for a product. In addition, the valuer
should understand how the payments under the licensing agreement are
structured, including whether there are upfront payments, milestone
payments, puts/calls to acquire the licensed property outright, etc.\label{10.2.4.2.1.2.4-End}\label{subpar:10.2.4.2.1.2_Royalty_Method-End}
\end{enumerate}

\subparagraph{With-and-Without Method\label{subpar:10.2.4.2.1.3_With=000026Without}}

\stepcounter{SubParCounter} 

\thesubparagraph.\theSubParCounter.\label{10.2.4.2.1.3.1} The with-and-without
method indicates the value of an intangible asset by comparing two
scenarios: one in which the business uses the subject intangible asset
and one in which the business does not use the subject intangible
asset (but all other factors are kept constant).\label{10.2.4.2.1.3.1-End}

\stepcounter{SubParCounter} 

\thesubparagraph.\theSubParCounter.\label{10.2.4.2.1.3.2} The comparison
of the two scenarios can be done in two ways:
\begin{enumerate}
\item calculating the value of the business under each scenario with the
difference in the business values being the value of the subject intangible
asset;
\item calculating, for each future period, the difference between the profits
in the two scenarios. The present value of those amounts is then used
to reach the value of the subject intangible asset.\label{10.2.4.2.1.3.2-End}
\end{enumerate}
\stepcounter{SubParCounter} 

\thesubparagraph.\theSubParCounter.\label{10.2.4.2.1.3.3} In theory,
either method should reach a similar value for the intangible asset
provided the valuer considers not only the impact on the entity’s
profit, but additional factors such as differences between the two
scenarios in working capital needs and capital expenditures.\label{10.2.4.2.1.3.3-End}

\stepcounter{SubParCounter} 

\thesubparagraph.\theSubParCounter.\label{10.2.4.2.1.3.4} The with-and-without
method is frequently used in the valuation of non-competition agreements
but may be appropriate in the valuation of other intangible assets
in certain circumstances.\label{10.2.4.2.1.3.4-End}

\stepcounter{SubParCounter} 

\thesubparagraph.\theSubParCounter.\label{10.2.4.2.1.3.5} The key
steps in applying the with-and-without method are to:
\begin{enumerate}
\item prepare projections of revenue, expenses, capital expenditures and
working capital needs for the business assuming the use of all of
the assets of the business including the subject intangible asset.
These are the cash flows in the “with” scenario;
\item use an appropriate discount rate to present value the future cash
flows in the “with” scenario, and/or calculate the value of the business
in the “with” scenario;
\item prepare projections of revenue, expenses, capital expenditures and
working capital needs for the business assuming the use of all of
the assets of the business except the subject intangible asset. These
are the cash flows in the “without” scenario;
\item use an appropriate discount rate for the business, present value the
future cash flows in the “with” scenario and/or calculate the value
of the business in the “with” scenario;
\item deduct the present value of cash flows or the value of the business
in the “without” scenario from the present value of cash flows or
value of the business in the “with” scenario;
\item if appropriate for the purpose of the valuation (see paras 110.1-110.4),
calculate and add the TAB for the subject intangible asset.\label{10.2.4.2.1.3.5-End}
\end{enumerate}
\stepcounter{SubParCounter} 

\thesubparagraph.\theSubParCounter.\label{10.2.4.2.1.3.6} As an
additional step, the difference between the two scenarios may need
to be probability-weighted. For example, when valuing a non-competition
agreement, the individual or business subject to the agreement may
choose not to compete, even if the agreement were not in place.\label{10.2.4.2.1.3.6-End}

\stepcounter{SubParCounter} 

\thesubparagraph.\theSubParCounter.\label{10.2.4.2.1.3.7} The differences
in value between the two scenarios should be reflected solely in the
cash flow projections rather than by using different discount rates
in the two scenarios.\label{10.2.4.2.1.3.7-End}\label{subpar:10.2.4.2.1.3_With=000026Without-End}

\subparagraph{Greenfield Method\label{subpar:10.2.4.2.1.4_Greenfield_Method}}

\stepcounter{SubParCounter} 

\thesubparagraph.\theSubParCounter. \label{10.2.4.2.1.4.1} Under
the greenfield method, the value of the subject intangible is determined
using cash flow projections that assume the only asset of the business
at the valuation date is the subject intangible. All other tangible
and intangible assets must be bought, built or rented.\label{10.2.4.2.1.4.1-End}

\stepcounter{SubParCounter} 

\thesubparagraph.\theSubParCounter.\label{10.2.4.2.1.4.2} The greenfield
method is conceptually similar to the excess earnings method. However,
instead of subtracting contributory asset charges from the cash flow
to reflect the contribution of contributory assets, the greenfield
method assumes that the owner of the subject asset would have to build,
buy or rent the contributory assets. When building or buying the contributory
assets, the cost of a replacement asset of equivalent utility is used
rather than a reproduction cost.\label{10.2.4.2.1.4.2-End}

\stepcounter{SubParCounter} 

\thesubparagraph.\theSubParCounter.\label{10.2.4.2.1.4.3} The greenfield
method is often used to estimate the value of ”enabling” intangible
assets such as franchise agreements and broadcast spectrum.\label{10.2.4.2.1.4.3-End}

\stepcounter{SubParCounter} 

\thesubparagraph.\theSubParCounter.\label{10.2.4.2.1.4.4} The key
steps in applying the greenfield method are to:
\begin{enumerate}
\item prepare projections of revenue, expenses, capital expenditures and
working capital needs for the business assuming the subject intangible
asset is the only asset owned by the subject business at the valuation
date, including the time period needed to “ramp up” to stabilized
levels;
\item estimate the timing and amount of expenditures related to the acquisition,
creation or rental of all other assets needed to operate the subject
business;
\item using an appropriate discount rate for the business, present value
the future cash flows to determine the value of the subject business
with only the subject intangible in place;
\item if appropriate for the purpose of the valuation (see paras 110.1-110.4),
calculate and add the TAB for the subject intangible asset.\label{10.2.4.2.1.4.4-End}\label{subpar:10.2.4.2.1.4_Greenfield_Method-End}
\end{enumerate}

\subparagraph{Distributor Method\label{subpar:10.2.4.2.1.5_Distributor_Method}}

\stepcounter{SubParCounter} 

\thesubparagraph.\theSubParCounter.\label{10.2.4.2.1.5.1} The distributor
method, sometimes referred to as the disaggregated method, is a variation
of the multi-period excess earnings method sometimes used to value
customer-related intangible assets. The underlying theory of the distributor
method is that businesses that are comprised of various functions
are expected to generate profits associated with each function. As
distributors generally only perform functions related to distribution
of products to customers rather than development of intellectual property
or manufacturing, information on profit margins earned by distributors
is used to estimate the excess earnings attributable to customer-related
intangible assets.\label{10.2.4.2.1.5.1-End}

\stepcounter{SubParCounter} 

\thesubparagraph.\theSubParCounter.\label{10.2.4.2.1.5.2} The distributor
method is appropriate to value customer-related intangible assets
when another intangible asset (for example, technology or a brand)
is deemed to be the primary or most significant intangible asset and
is valued under a multi-period excess earnings method.\label{10.2.4.2.1.5.2-End}

\stepcounter{SubParCounter} 

\thesubparagraph.\theSubParCounter.\label{10.2.4.2.1.5.3} The key
steps in applying the distributor method are to:
\begin{enumerate}
\item prepare projections of revenue associated with existing customer relationships.
This should reflect expected growth in revenue from existing customers
as well as the effects of customer attrition;
\item identify comparable distributors that have customer relationships
similar to the subject business and calculate the profit margins achieved
by those distributors;
\item apply the distributor profit margin to the projected revenue;
\item identify the contributory assets related to performing a distribution
function that are needed to achieve the forecast revenue and expenses.
Generally distributor contributory assets include working capital,
fixed assets and workforce. However, distributors seldom require other
assets such as trademarks or technology. The level of required contributory
assets should also be consistent with participants performing only
a distribution function;
\item determine the appropriate rate of return on each contributory asset
based on an assessment of the risk associated with that asset;
\item in each forecast period, deduct the required returns on contributory
assets from the forecast distributor profit to arrive at the excess
earnings attributable to only the subject intangible asset;
\item determine the appropriate discount rate for the subject intangible
asset and present value the excess earnings;
\item if appropriate for the purpose of the valuation (see paras 110.1-110.4),
calculate and add the TAB for the subject intangible asset.\label{10.2.4.2.1.5.3_End}\label{subpar:10.2.4.2.1.5_Distributor_Method-End}\label{par:10.2.4.2.1_Income_Methods-End}\label{subsubsec:10.2.4.2_Income_Approach-End}
\end{enumerate}

\subsubsection{Cost Approach\label{subsubsec:10.2.4.3_Cost_Approach}}

\stepcounter{SubSubSecCounter} 

\thesubsubsection.\theSubSubSecCounter.\label{10.2.4.3.1} Under
the cost approach, the value of an intangible asset is determined
based on the replacement cost of a similar asset or an asset providing
similar service potential or utility.\label{10.2.4.3.1-End}

\stepcounter{SubSubSecCounter} 

\thesubsubsection.\theSubSubSecCounter.\label{10.2.4.3.2} Valuers
must comply with paras 60.2 and 60.3 of IVS 105 Valuation Approaches
and Methods when determining whether to apply the cost approach to
the valuation of intangible assets.\label{10.2.4.3.2-End}

\stepcounter{SubSubSecCounter} 

\thesubsubsection.\theSubSubSecCounter.\label{10.2.4.3.3} Consistent
with these criteria, the cost approach is commonly used for intangible
assets such as the following:
\begin{enumerate}
\item acquired third-party software;
\item internally-developed and internally-used, non-marketable software;
\item assembled workforce.\label{10.2.4.3.3-End}
\end{enumerate}
\stepcounter{SubSubSecCounter} 

\thesubsubsection.\theSubSubSecCounter.\label{10.2.4.3.4} The cost
approach may be used when no other approach is able to be applied;
however, a valuer should attempt to identify an alternative method
before applying the cost approach in situations where the subject
asset does not meet the criteria in paras 60.2 and 60.3 of IVS 105
Valuation Approaches and Methods.\label{10.2.4.3.4-End}

\stepcounter{SubSubSecCounter} 

\thesubsubsection.\theSubSubSecCounter.\label{10.2.4.3.5} There
are broadly two main methods that fall under the cost approach: replacement
cost and reproduction cost. However, many intangible assets do not
have physical form that can be reproduced and assets such as software,
which can be reproduced, generally derive value from their function/utility
rather than their exact lines of code. As such, the replacement cost
is most commonly applied to the valuation of intangible assets.\label{10.2.4.3.5-End}

\stepcounter{SubSubSecCounter} 

\thesubsubsection.\theSubSubSecCounter.\label{10.2.4.3.6} The replacement
cost method assumes that a participant would pay no more for the asset
than the cost that would be incurred to replace the asset with a substitute
of comparable utility or functionality.\label{10.2.4.3.6-End}

\stepcounter{SubSubSecCounter} 

\thesubsubsection.\theSubSubSecCounter.\label{10.2.4.3.7} Valuers
should consider the following when applying the replacement cost method:
\begin{enumerate}
\item the direct and indirect costs of replacing the utility of the asset,
including labor, materials and overhead;
\item whether the subject intangible asset is subject to obsolescence. While
intangible assets do not become functionally or physically obsolete,
they can be subject to economic obsolescence;
\item whether it is appropriate to include a profit mark-up on the included
costs. An asset acquired from a third party would presumably reflect
their costs associated with creating the asset as well as some form
of profit to provide a return on investment. As such, under bases
of value (see IVS 104 Bases of Value) that assume a hypothetical transaction,
it may be appropriate to include an assumed profit mark-up on costs.
As noted in IVS 105 Valuation Approaches and Methods, costs developed
based on estimates from third parties would be presumed to already
reflect a profit mark-up;
\item opportunity costs may also be included, which reflect costs associated
with not having the subject intangible asset in place for some period
of time during its creation.\label{10.2.4.3.7-End}\label{subsubsec:10.2.4.3_Cost_Approach-End}\label{subsec:10.2.4_Valuation_Approaches-End}
\end{enumerate}

\subsection{Special Considerations for Intangible Assets\label{subsec:10.2.5_Special_Considerations}}

\stepcounter{SubSubSecCounter} 

\thesubsubsection.\theSubSubSecCounter.\label{10.2.5.0.1} The following
sections address a non-exhaustive list of topics relevant to the valuation
of intangible assets.
\begin{enumerate}
\item Discount Rates/Rates of Return for Intangible Assets (section 90).
\item Intangible Asset Economic Lives (section 100).
\item Tax Amortization Benefit (section 110).\label{10.2.5.0.1-End}
\end{enumerate}

\subsubsection{Discount Rates/Rates of Return for Intangible Assets\label{subsubsec:10.2.5.1_Discount_Rates}}

\stepcounter{SubSubSecCounter} 

\thesubsubsection.\theSubSubSecCounter.\label{10.2.5.1.1} Selecting
discount rates for intangible assets can be challenging as observable
market evidence of discount rates for intangible assets is rare. The
selection of a discount rate for an intangible asset generally requires
significant professional judgment.\label{10.2.5.1.1-End}

\stepcounter{SubSubSecCounter} 

\thesubsubsection.\theSubSubSecCounter.\label{10.2.5.1.2} In selecting
a discount rate for an intangible asset, valuers should perform an
assessment of the risks associated with the subject intangible asset
and consider observable discount rate benchmarks.\label{10.2.5.1.2-End}

\stepcounter{SubSubSecCounter} 

\thesubsubsection.\theSubSubSecCounter.\label{10.2.5.1.3} When assessing
the risks associated with an intangible asset, a valuer should consider
factors including the following:
\begin{enumerate}
\item intangible assets often have higher risk than tangible assets;
\item if an intangible asset is highly specialized to its current use, it
may have higher risk than assets with multiple potential uses;
\item single intangible assets may have more risk than groups of assets
(or businesses);
\item intangible assets used in risky (sometimes referred to as non-routine)
functions may have higher risk than intangible assets used in more
low- risk or routine activities. For example, intangible assets used
in research and development activities may be higher risk than those
used in delivering existing products or services;
\item the life of the asset. Similar to other investments, intangible assets
with longer lives are often considered to have higher risk, all else
being equal;
\item intangible assets with more readily estimable cash flow streams, such
as backlog, may have lower risk than similar intangible assets with
less estimable cash flows, such as customer relationships.\label{10.2.5.1.3-End}
\end{enumerate}
\stepcounter{SubSubSecCounter} 

\thesubsubsection.\theSubSubSecCounter.\label{10.2.5.1.4} Discount
rate benchmarks are rates that are observable based on market evidence
or observed transactions. The following are some of the benchmark
rates that a valuer should consider:
\begin{enumerate}
\item risk-free rates with similar maturities to the life of the subject
intangible asset;
\item cost of debt or borrowing rates with maturities similar to the life
of the subject intangible asset;
\item cost of equity or equity rates or return for participants for the
subject intangible asset;
\item weighted average cost of capital (WACC) of participants for the subject
intangible asset or of the company owning/using the subject intangible
asset;
\item in contexts involving a recent business acquisition including the
subject intangible asset, the Internal Rate of Return (IRR) for the
transaction should be considered;
\item in contexts involving a valuation of all assets of a business, the
valuer should perform a weighted average return on assets (WARA) analysis
to confirm reasonableness of selected discount rates.\label{10.2.5.1.4-End}\label{subsubsec:10.2.5.1_Discount_Rates-End}
\end{enumerate}

\subsubsection{Intangible Asset Economic Lives\label{subsubsec:10.2.5.2_Economic_lives}}

\stepcounter{SubSubSecCounter} 

\thesubsubsection.\theSubSubSecCounter.\label{10.2.5.2.1} An important
consideration in the valuation of an intangible asset, particularly
under the income approach, is the economic life of the asset. This
may be a finite period limited by legal, technological, functional
or economic factors; other assets may have an indefinite life. The
economic life of an intangible asset is a different concept than the
remaining useful life for accounting or tax purposes.\label{10.2.5.2.1-End}

\stepcounter{SubSubSecCounter} 

\thesubsubsection.\theSubSubSecCounter.\label{10.2.5.2.2} Legal,
technological, functional and economic factors must be considered
individually and together in making an assessment of the economic
life. For example, a pharmaceutical technology protected by a patent
may have a remaining legal life of five years before expiry of the
patent, but a competitor drug with improved efficacy may be expected
to reach the market in three years. This might cause the economic
life of the patent to be assessed as only three years. In contrast,
the expected economic life of the technology could extend beyond the
life of the patent if the know-how associated with the technology
would have value in production of a generic drug beyond the expiration
of the patent.\label{10.2.5.2.2-End}

\stepcounter{SubSubSecCounter} 

\thesubsubsection.\theSubSubSecCounter.\label{10.2.5.2.3} In estimating
the economic life of an intangible asset, a valuer should also consider
the pattern of use or replacement. Certain intangible assets may be
abruptly replaced when a new, better or cheaper alternative becomes
available, while others may be replaced slowly over time, such as
when a software developer releases a new version of software every
year but only replaces a portion of the existing code with each new
release.\label{10.2.5.2.3-End}

\stepcounter{SubSubSecCounter} 

\thesubsubsection.\theSubSubSecCounter.\label{10.2.5.2.4} For customer-related
intangibles, attrition is a key factor in estimating an economic life
as well as the cash flows used to value the customer- related intangibles.
Attrition applied in the valuation of intangible assets is a quantification
of expectations regarding future losses of customers. While it is
a forward-looking estimate, attrition is often based on historical
observations of attrition.\label{10.2.5.2.4-End}

\stepcounter{SubSubSecCounter} 

\thesubsubsection.\theSubSubSecCounter.\label{10.2.5.2.5} There
are a number of ways to measure and apply historical attrition:
\begin{enumerate}
\item a constant rate of loss (as a percentage of prior year balance) over
the life of the customer relationships may be assumed if customer
loss does not appear to be dependent on age of the customer relationship;
\item a variable rate of loss may be used over the life of the customer
relationships if customer loss is dependent on age of the customer
relationship. In such circumstances, generally younger/new customers
are lost at a higher rate than older, more established customer relationships;
\item attrition may be measured based on either revenue or number of customers/customer
count as appropriate, based on the characteristics of the customer
group;
\item customers may need to be segregated into different groups. For example,
a company that sells products to distributors and retailers may experience
different attrition rates for each group. Customers may also be segregated
based on other factors such as geography, size of customer and type
of product or service purchased;
\item the period used to measure attrition may vary depending on circumstances.
For example, for a business with monthly subscribers, one month without
revenue from a particular customer would indicate a loss of that customer.
In contrast, for larger industrial products, a customer might not
be considered “lost” unless there have been no sales to that customer
for a year or more.\label{10.2.5.2.5-End}
\end{enumerate}
\stepcounter{SubSubSecCounter} 

\thesubsubsection.\theSubSubSecCounter.\label{10.2.5.2.6} The application
of any attrition factor should be consistent with the way attrition
was measured. Correct application of attrition factor in first projection
year (and therefore all subsequent years) must be consistent with
form of measurement. 
\begin{enumerate}
\item If attrition is measured based on the number of customers at the beginning-of-period
versus end-of-period (typically a year), the attrition factor should
be applied using a “mid-period” convention for the first projection
year (as it is usually assumed that customers were lost throughout
the year). For example, if attrition is measured by looking at the
number of customers at the beginning of the year (100) versus the
number remaining at the end of the year (90), on average the company
had 95 customers during that year, assuming they were lost evenly
throughout the year. Although the attrition rate could be described
as 10\%, only half of that should be applied in the first year.
\item If attrition is measured by analyzing year-over-year revenue or customer
count, the resulting attrition factor should generally be applied
without a mid-period adjustment. For example, if attrition is measured
by looking at the number of customers that generated revenue in Year
1 (100) versus the number of those same customers that had revenue
in Year 2 (90), application would be different even though the attrition
rate could again be described as 10\%.\label{10.2.5.2.6-End}
\end{enumerate}
\stepcounter{SubSubSecCounter} 

\thesubsubsection.\theSubSubSecCounter.\label{10.2.5.2.7} Revenue-based
attrition may include growth in revenue from existing customers unless
adjustments are made. It is generally a best practice to make adjustments
to separate growth and attrition in measurement and application.\label{10.2.5.2.7-End}

\stepcounter{SubSubSecCounter} 

\thesubsubsection.\theSubSubSecCounter.\label{10.2.5.2.8} It is
a best practice for valuers to input historical revenue into the model
being used and check how closely it predicts actual revenue from existing
customers in subsequent years. If attrition has been measured and
applied appropriately, the model should be reasonably accurate. For
example, if estimates of future attrition were developed based on
historical attrition observed from 20X0 through 20X5, a valuer should
input the 20X0 customer revenue into the model and check whether it
accurately predicts the revenue achieved from existing customers in
20X1, 20X2, etc.\label{10.2.5.2.8-End}\label{subsubsec:10.2.5.2_Economic_lives-End}

\subsubsection{Tax Amortization Benefit (TAB)\label{subsubsec:10.2.5.3_Tax_Amortisation}}

\stepcounter{SubSubSecCounter} 

\thesubsubsection.\theSubSubSecCounter.\label{10.2.5.3.1} In many
tax jurisdictions, intangible assets can be amortized for tax purposes,
reducing a taxpayer’s tax burden and effectively increasing cash flows.
Depending on the purpose of a valuation and the valuation method used,
it may be appropriate to include the value of TAB in the value of
the intangible.\label{10.2.5.3.1-End}

\stepcounter{SubSubSecCounter} 

\thesubsubsection.\theSubSubSecCounter.\label{10.2.5.3.2} If the
market or cost approach is used to value an intangible asset, the
price paid to create or purchase the asset would already reflect the
ability to amortize the asset. However, in the income approach, a
TAB needs to be explicitly calculated and included, if appropriate.\label{10.2.5.3.2-End}

\stepcounter{SubSubSecCounter} 

\thesubsubsection.\theSubSubSecCounter.\label{10.2.5.3.3} For some
valuation purposes, such as financial reporting, the appropriate basis
of value assumes a hypothetical sale of the subject intangible asset.
Generally, for those purposes, a TAB should be included when the income
approach is used because a typical participant would be able to amortize
an intangible asset acquired in such a hypothetical transaction. For
other valuation purposes, the assumed transaction might be of a business
or group of assets. For those bases of value, it may be appropriate
to include a TAB only if the transaction would result in a step-up
in basis for the intangible assets.\label{10.2.5.3.3-End}

\stepcounter{SubSubSecCounter} 

\thesubsubsection.\theSubSubSecCounter.\label{10.2.5.3.4} There
is some diversity in practice related to the appropriate discount
rate to be used in calculating a TAB. Valuers may use either of the
following:
\begin{enumerate}
\item a discount rate appropriate for a business utilizing the subject asset,
such as a weighted average cost of capital. Proponents of this view
believe that, since amortization can be used to offset the taxes on
any income produced by the business, a discount rate appropriate for
the business as a whole should be used;
\item a discount rate appropriate for the subject asset (i.\,e, the one
used in the valuation of the asset). Proponents of this view believe
that the valuation should not assume the owner of the subject asset
has operations and income separate from the subject asset and that
the discount rate used in the TAB calculation should be the same as
that used in the valuation of the subject asset.\label{10.2.5.3.4-End}\label{subsubsec:10.2.5.3_Tax_Amortisation-End}\label{subsec:10.2.5_Special_Considerations-End}\label{sec:10.2_IVS-210_Intangible_Assets-End}
\end{enumerate}

\section{IVS 220. Non-Financial Liabilities\label{sec:10.3_IVS-220_Non-Financial_Liabilities}}

\subsection{Overview\label{subsec:10.3.1_Overview}}

\stepcounter{SubSecCounter} 

\thesubsection.\theSubSecCounter.\label{10.3.1.1} The principles
contained in the General Standards apply to valuations of non-financial
liabilities and valuations with a non-financial liability component.
This standard contains additional requirements that apply to valuations
of non-financial liabilities.\label{10.3.1.1-End}

\stepcounter{SubSecCounter} 

\thesubsection.\theSubSecCounter.\label{10.3.1.2} With regard to
the determination of discount rates and risk margins, in circumstances
in which IVS 105 Valuation Approaches and Methods (see paras 50.29-50.31)
conflicts with IVS 220 Non-Financial Liabilities, valuers must apply
the principles in sections 90 and 100 of this Standard in valuations
of non-financial liabilities.\label{10.3.1.3-End}\label{subsec:10.3.1_Overview-End}

\subsection{Introduction\label{subsec:10.3.2_Introduction}}

\stepcounter{SubSecCounter} 

\thesubsection.\theSubSecCounter.\label{10.3.2.1} For purposes of
IVS 220 Non-Financial Liabilities, non-financial liabilities are defined
as those liabilities requiring a non-cash performance obligation to
provide goods or services.\label{10.3.2.1-End}

\stepcounter{SubSecCounter} 

\thesubsection.\theSubSecCounter.\label{10.3.2.2} A non-exhaustive
list of liabilities that may in part or in full require a non- cash
fulfillment and be subject to IVS 220 Non-Financial Liabilities includes:
deferred revenue or contract liabilities, warranties, environmental
liabilities, asset retirement obligations, certain contingent consideration
obligations, loyalty programs, power purchase agreements, certain
litigation reserves and contingencies, and certain indemnifications
and guarantees.\label{10.3.2.2-End}

\stepcounter{SubSecCounter} 

\thesubsection.\theSubSecCounter.\label{10.3.2.3} Although certain
contingent consideration liabilities may require a non-cash performance
obligation, such liabilities are not included in the scope of IVS
220 Non-Financial Liabilities.\label{10.3.2.3-End}

\stepcounter{SubSecCounter} 

\thesubsection.\theSubSecCounter.\label{10.3.2.4} The party assuming
a non-financial liability typically requires a profit margin on the
fulfillment effort to compensate for the effort incurred and risk
borne for the delivery of goods or services.\label{10.3.2.4-End}

\stepcounter{SubSecCounter} 

\thesubsection.\theSubSecCounter.\label{10.3.2.5} For financial
liabilities, cash fulfillment is typically the only performance obligation
and no additional compensation is needed for the fulfillment effort.
Given that cash fulfillment is the only performance obligation for
financial liabilities, asset-liability symmetry most often enables
valuers to assess the subject liability using an asset framework.\label{10.3.2.5-End}

\stepcounter{SubSecCounter} 

\thesubsection.\theSubSecCounter.\label{10.3.2.6} Asset-liability
symmetry typically does not exist for non-financial liabilities due
to the performance obligation to provide goods and services to satisfy
the liability and additional compensation for such effort. As such,
non- financial liabilities will most often be valued using a liability
framework.\label{10.3.2.6-End}

\stepcounter{SubSecCounter} 

\thesubsection.\theSubSecCounter.\label{10.3.2.7} In instances in
which a corresponding asset is recognized by the counterparty, the
valuer must assess if the values would reflect asset-liability symmetry
under circumstances consistent with the basis of value. Certain bases
of value issued by entities/organizations other than the IVSC require
the specific consideration and reconciliation to a corresponding asset
under certain circumstances. The valuer must understand and follow
the regulation, case law, and other interpretive guidance related
to those bases of value as of the valuation date (see IVS 200 Businesses
and Business Interests, para 30.2). Instances in which the valuer
should reconcile to a corresponding asset value will be rare, reasons
include:
\begin{enumerate}
\item Non-financial liabilities often do not have a recorded corresponding
asset recognized by the counterparty (e.\,g, environmental liability),
or can only be transferred in conjunction with another asset (e.\,g,
an automobile and related warranty are only transferred together).
\item The corresponding asset of a non-financial liability may be held by
numerous parties for which it is impractical to identify and reconcile
the asset values.
\item The market for the non-financial asset and liability is often highly
illiquid, thus resulting in asymmetric information, high bid ask spreads,
and asset-liability asymmetry.\label{10.3.2.7-End}
\end{enumerate}
\stepcounter{SubSecCounter} 

\thesubsection.\theSubSecCounter.\label{10.3.2.8} Participants that
most often transact in the subject non-financial liability may not
be the comparable companies and competitors of the entity holding
the subject non-financial liability. Examples include insurance companies,
third party warranty issuers, and more. The valuer should consider
if a market, or participants, exist outside the immediate industry
in which the entity holding the subject non-financial liability operates.\label{10.3.2.8-End}

\stepcounter{SubSecCounter} 

\thesubsection.\theSubSecCounter.\label{10.3.2.9} Non-financial
liability valuations are performed for a variety of purposes. It is
the valuer’s responsibility to understand the purpose of a valuation
and whether the non-financial liabilities should be valued, whether
separately or grouped with other assets. A non-exhaustive list of
examples of circumstances that commonly include a non-financial liability
valuation component is provided below:
\begin{enumerate}
\item For financial reporting purposes, valuations of non-financial liabilities
are often required in connection with accounting for business combinations,
asset acquisitions and sales, and impairment analysis.
\item For tax reporting purposes, non-financial liability valuations are
often needed for transfer pricing analyses, estate and gift tax planning
and reporting, and ad valorem taxation analyses.
\item Non-financial liabilities may be the subject of litigation, requiring
valuation analysis in certain circumstances. 
\item Valuers are sometimes asked to value non-financial liabilities as
part of general consulting, collateral lending and transactional support
engagements.\label{10.3.2.9-End}\label{subsec:10.3.2_Introduction-End}
\end{enumerate}

\subsection{Bases of Value\label{subsec:10.3.3_Bases_of_Value}}

\stepcounter{SubSecCounter} 

\thesubsection.\theSubSecCounter.\label{10.3.3.1}In accordance with
IVS 104 Bases of Value, a valuer must select the appropriate basis(es)
of value when valuing non-financial liabilities.\label{10.3.3.1-End}

\stepcounter{SubSecCounter} 

\thesubsection.\theSubSecCounter.\label{10.3.3.2}Often, non-financial
liability valuations are performed using bases of value defined by
entities/organizations other than the IVSC (some examples of which
are mentioned in IVS 104 Bases of Value) and the valuer must understand
and follow the regulation, case law, and other interpretive guidance
related to those bases of value as of the valuation date (see IVS
200 Businesses and Business Interests, para 30.2).\label{10.3.3.2-End}\label{subsec:10.3.3_Bases_of_Value-End}

\subsection{Valuation Approaches and Methods\label{subsec:10.3.4_Valuation_Approaches}}

\stepcounter{SubSubSecCounter} 

\thesubsubsection.\theSubSubSecCounter.\label{10.3.4.0.1}Elements
of the three valuation approaches described in IVS 105 Valuation Approaches
(market, income and cost approach) can all be applied to the valuation
of non-financial liabilities. The methods described below may exhibit
elements of more than one approach. If it is necessary for the valuer
to classify a method under one of the three approaches, the valuer
should use judgment in making the determination and not necessarily
rely on the classification below.\label{10.3.4.0.1-End}

\stepcounter{SubSubSecCounter} 

\thesubsubsection.\theSubSubSecCounter.\label{10.3.4.0.2} When selecting
an approach and method, in addition to the requirements of this standard,
a valuer must follow the requirements of IVS 105 Valuation Approaches,
including para 10.3.\label{10.3.4.0.2-End}

\subsubsection{Market Approach\label{subsubsec:10.3.4.1_Market_Approach}}

\stepcounter{SubSubSecCounter} 

\thesubsubsection.\theSubSubSecCounter.\label{10.3.4.1.1} Under
the market approach, the value of a non-financial liability is determined
by reference to market activity (for example, transactions involving
identical or similar non-financial liabilities).\label{10.3.4.1.1-End}

\stepcounter{SubSubSecCounter} 

\thesubsubsection.\theSubSubSecCounter.\label{10.3.4.1.2} Transactions
involving non-financial liabilities frequently also include other
assets, such as a business combinations that include tangible and
intangible assets.\label{10.3.4.1.2-End}

\stepcounter{SubSubSecCounter} 

\thesubsubsection.\theSubSubSecCounter.\label{10.3.4.1.3} Transactions
involving standalone non-financial liabilities are infrequent as compared
with transactions for businesses and assets.\label{10.3.4.1.3-End}

\stepcounter{SubSubSecCounter} 

\thesubsubsection.\theSubSubSecCounter.\label{10.3.4.1.4} While
standalone transactions of non-financial liabilities are infrequent,
valuers should consider relevant market-based indications of value.
Although such market-based indications may not provide sufficient
information with which to apply the market approach, the use of market-
based inputs should be maximized in the application of other approaches.\label{10.3.4.1.4-End}

\stepcounter{SubSubSecCounter} 

\thesubsubsection.\theSubSubSecCounter.\label{10.3.4.1.5} A non-exhaustive
list of such market indications of value includes:
\begin{enumerate}
\item Pricing from third parties to provide identical or similar products
as the subject non-financial liability (e.\,g, deferred revenue);
\item Pricing for warranty policies issued by third parties for identical
or similar obligations;
\item The prescribed monetary conversion amount as published by participants
for certain loyalty reward obligations;
\item The traded price for contingent value rights (CVRs) with similarities
to the subject non-financial liability (e.\,g, contingent consideration);
\item Observed rates of return for investment funds that invest in non-financial
liabilities (eg, litigation finance).\label{10.3.4.1.5-End}
\end{enumerate}
\stepcounter{SubSubSecCounter} 

\thesubsubsection.\theSubSubSecCounter.\label{10.3.4.1.6} Valuers
must comply with paras 20.2 and 20.3 of IVS 105 Valuation Approaches
and Methods when determining whether to apply the market approach
to the valuation of non-financial liabilities.\label{10.3.4.1.6-End}

\stepcounter{SubSubSecCounter} 

\thesubsubsection.\theSubSubSecCounter.\label{10.3.4.1.7} The diverse
nature of many non-financial liabilities and the fact that non-financial
liabilities seldom transact separately from other assets means that
it is rarely possible to find market evidence of transactions involving
similar non-financial liabilities.\label{10.3.4.1.7-End}

\stepcounter{SubSubSecCounter} 

\thesubsubsection.\theSubSubSecCounter.\label{10.3.4.1.8} Where
evidence of market prices is available, valuers should consider adjustments
to these to reflect differences between the subject non-financial
liability and those involved in the transactions. These adjustments
are necessary to reflect the differentiating characteristics of the
subject non- financial liability and those involved in the transactions.
Such adjustments may only be determinable at a qualitative, rather
than quantitative, level. However, the need for significant qualitative
adjustments could indicate that another approach would be more appropriate
for the valuation.\label{10.3.4.1.8-End}

\stepcounter{SubSubSecCounter} 

\thesubsubsection.\theSubSubSecCounter.\label{10.3.4.1.9} In certain
instances a valuer may rely on market prices or evidence for an asset
corresponding to the subject non-financial liability. In such instances,
the valuer should consider an entity’s ability to transfer the subject
non- financial liability, whether the asset and related price of the
asset reflect those same restrictions, and whether adjustments to
reflect the restrictions should be included. The valuer should take
care to determine if the transfer restrictions are characteristics
of the subject non-financial liability (for example, an illiquid market)
or restrictions that are characteristics of the entity (for example,
financial distress).\label{10.3.4.1.9-End}

\stepcounter{SubSubSecCounter} 

\thesubsubsection.\theSubSubSecCounter.\label{10.3.4.1.10} The comparable
transaction method, also known as the guideline transactions method,
is generally the only market approach method that can be applied to
value non-financial liabilities.\label{10.3.4.1.10-End}

\stepcounter{SubSubSecCounter} 

\thesubsubsection.\theSubSubSecCounter.\label{10.3.4.1.11} In rare
circumstances, a security sufficiently similar to a subject non- financial
liability could be publicly traded, allowing the use of the guideline
public company method. One example of such securities is contingent
value rights that are tied to the performance of a particular product
or technology.\label{10.3.4.1.11-End}\label{subsubsec:10.3.4.1_Market_Approach-End}

\subsubsection{Market Approach Methods\label{subsubsec:10.3.4.2_Market_Methods}}

\stepcounter{ParCounter} 

\theparagraph.\theParCounter.\label{10.3.4.2.0.1} A method to value
non-financial liabilities under the Market Approach is often referred
to as the Top-Down Method.\label{10.3.4.2.0.1-End}

\paragraph{Top-Down Method\label{par:10.3.4.2.1_Top-Down}}

\stepcounter{ParCounter} 

\theparagraph.\theParCounter.\label{10.3.4.2.1.1} Under the Top-Down
Method, valuing non-financial liabilities is based on the premise
that reliable market-based indications of pricing are available for
the performance obligation.\label{10.3.4.2.1.1-End}

\stepcounter{ParCounter} 

\theparagraph.\theParCounter.\label{10.3.4.2.1.2} A participant
fulfilling the obligation to deliver the product or services associated
with the non-financial liability could theoretically price the liability
by deducting costs already incurred toward the fulfillment obligation,
plus a mark-up on those costs, from the market price of services.\label{10.3.4.2.1.2-End}

\stepcounter{ParCounter} 

\theparagraph.\theParCounter.\label{10.3.4.2.1.3} When market information
is used to determine the value of the subject non- financial liability,
discounting is typically not necessary because the effects of discounting
are incorporated into observed market prices.\label{10.3.4.2.1.3-End}

\stepcounter{ParCounter} 

\theparagraph.\theParCounter.\label{10.3.4.2.1.4} The key steps
in applying a Top-Down Method are to: 
\begin{enumerate}
\item Determine the market price of the non-cash fulfillment.
\item Determine the costs already incurred and assets utilized by the transferor.
The nature of such costs will differ depending on the subject non-financial
liability. For example, for deferred revenue the costs will primarily
consist of sales and marketing costs that have already been incurred
in generating the non-financial liability.
\item Determine a reasonable profit margin on the costs already incurred. 
\item Subtract costs incurred and profit from the market price.\label{10.3.4.2.1.4-End}\label{par:10.3.4.2.1_Top-Down-End}\label{subsubsec:10.3.4.2_Market_Methods-End}
\end{enumerate}

\subsubsection{Income Approach\label{subsubsec:10.3.4.3_Income_Approach}}

\stepcounter{SubSubSecCounter} 

\thesubsubsection.\theSubSubSecCounter.\label{10.3.4.3.1} Under
the income approach, the value of a non-financial liability is often
determined by reference to the present value of the costs to fulfill
the obligation plus a profit margin that would be required to assume
the liability.\label{10.3.4.3.1-End}

\stepcounter{SubSubSecCounter} 

\thesubsubsection.\theSubSubSecCounter.\label{10.3.4.3.2} Valuers
must comply with paras 40.2 and 40.3 of IVS 105 Valuation Approaches
and Methods when determining whether to apply the income approach
to the valuation of non-financial liabilities.\label{10.3.4.3.2-End}\label{subsubsec:10.3.4.3_Income_Approach-End}

\subsubsection{Income Approach Methods\label{subsubsec:10.3.4.4_Income_Methods}}

\stepcounter{ParCounter} 

\theparagraph.\theParCounter.\label{10.3.4.4.0.1} The primary method
to value non-financial liabilities under the Income Approach is often
referred to as the Bottom-Up Method.\label{10.3.4.4.0.1-End}

\paragraph{Bottom-Up Method\label{par:10.3.4.4.1-Bottom_Method}}

\stepcounter{ParCounter} 

\theparagraph.\theParCounter.\label{10.3.4.4.1.1} Under the Bottom-Up
Method, the non-financial liability is measured as the costs (which
may or may not include certain overhead items) required to fulfill
the performance obligation, plus a reasonable mark-up on those costs,
discounted to present value.\label{10.3.3.4.4.1.1-End}

\stepcounter{ParCounter} 

\theparagraph.\theParCounter.\label{10.3.4.4.1.2} The key steps
in applying a Bottom-Up Method are to:
\begin{enumerate}
\item Determine the costs required to fulfil the performance obligation.
Such costs will include the direct costs to fulfil the performance
obligation, but may also include indirect costs such as charges for
the use of contributory assets. Fulfilment costs represent those costs
that are related to fulfilling the performance obligation that generates
the non-financial liability. Costs incurred as part of the selling
activities before the acquisition date should be excluded from the
fulfilment effort.
\begin{enumerate}
\item Contributory asset charges should be included in the fulfilment costs
when such assets would be required to fulfil the obligation and the
related cost is not otherwise captured in the income statement. 
\item In limited instances, in addition to direct and indirect costs, it
may be appropriate to include opportunity costs. For example, in the
licensing of symbolic intellectual property, the direct and indirect
costs of fulfilment may be nominal. However, if the obligation reduces
the ability to monetise the underlying asset (in an exclusive licensing
arrangement for example), then the valuer should consider how participants
would account for the potential opportunity costs associated with
the non-financial liability. 
\end{enumerate}
\item Determine a reasonable mark-up on the fulfilment effort. In most cases
it may be appropriate to include an assumed profit margin on certain
costs which can be expressed as a target profit, either a lump sum
or a percentage return on cost or value. An initial starting point
may be to utilise the operating profit of the entity holding the subject
non- financial liability. However, this methodology assumes the profit
margin would be proportional to the costs incurred. In many circumstances
there is rationale to assume profit margins which are not proportional
to costs. In such cases the risks assumed, value added, or intangibles
contributed to the fulfilment effort are not the same as those contributed
pre-measurement date. When costs are derived from actual, quoted or
estimated prices by third party suppliers or contractors, these costs
will already include a third party’s desired level of profit. 
\item Determine timing of fulfilment and discount to present value. The
discount rate should account for the time value of money and non-
performance risk. Typically it is preferable to reflect the impact
of uncertainty such as changes in anticipated fulfilment costs and
fulfilment margin through the cash flows, rather than in the discount
rate. 
\item When fulfilment costs are derived through a percent of revenue, valuers
should consider whether the fulfilment costs already implicitly include
the impact of discounting. For example, prepayment for services may
result in a discount as one would expect to pay less for the same
service as compared with paying throughout the contract term. As a
result, the derived costs may also contain an implicit discount and
further discounting may not be necessary.\label{10.3.4.4.1.2-End}\label{par:10.3.4.4.1-Bottom_Method-End}\label{subsubsec:10.3.4.4_Income_Methods-End}
\end{enumerate}

\subsubsection{Cost Approach\label{subsubsec:10.3.4.5_Cost_Approach}}

\stepcounter{SubSubSecCounter} 

\thesubsubsection.\theSubSubSecCounter.\label{10.3.4.5.1} The cost
approach has limited application for non-financial liabilities as
participants typically expect a return on the fulfilment effort.\label{10.3.4.5.1-End}

\stepcounter{SubSubSecCounter} 

\thesubsubsection.\theSubSubSecCounter.\label{10.3.4.5.2} Valuers
must comply with paras 60.2 and 60.3 of IVS 105 Valuation Approaches
and Methods when determining whether to apply the cost approach to
the valuation of non-financial liabilities.\label{10.3.4.5.2-End}\label{subsubsec:10.3.4.5_Cost_Approach-End}\label{subsec:10.3.4_Valuation_Approaches-End}

\subsection{Special Considerations for Non-Financial\label{subsec:10.3.5_Special_Considerations}}

\stepcounter{SubSubSecCounter} 

\thesubsubsection.\theSubSubSecCounter.\label{10.3.5.0.1} The following
sections address a non-exhaustive list of topics relevant to the valuation
of non-financial liabilities. 
\begin{enumerate}
\item Discount Rates for Non-Financial Liabilities (section 90);
\item Estimating Cash Flows and Risk Margins (section 100);
\item Restrictions on Transfer (section 110);
\item Taxes (section 120).\label{10.3.5.0.1-End}
\end{enumerate}

\subsubsection{Discount rates for Non-Financial Liabilities\label{subsubsec:10.3.5.1_Discount_rates}}

\stepcounter{SubSubSecCounter} 

\thesubsubsection.\theSubSubSecCounter.\label{10.3.5.1.1} A fundamental
basis for the income approach is that investors expect to receive
a return on their investments and that such a return should reflect
the perceived level of risk in the investment.\label{10.3.5.1.1-End}

\stepcounter{SubSubSecCounter} 

\thesubsubsection.\theSubSubSecCounter.\label{10.3.5.1.2} The discount
rate should account for the time value of money and non-performance
risk. Non-performance risk is typically a function of counterparty
risk (ie, credit risk of the entity obligated to fulfil the liability)
(see para 60.5c of this Standard).\label{10.3.5.1.2-End}

\stepcounter{SubSubSecCounter} 

\thesubsubsection.\theSubSubSecCounter.\label{10.3.5.1.3} Certain
bases of value issued by entities/organisations other than the IVSC
may require the discount rate to specifically account for liability
specific risks. The valuer must understand and follow the regulation,
case law, and other interpretive guidance related to those bases of
value as of the valuation date (see IVS 200 Businesses and Business
Interests, para 30.2).\label{10.3.5.1.3-End}

\stepcounter{SubSubSecCounter} 

\thesubsubsection.\theSubSubSecCounter.\label{10.3.5.1.4} Valuers
should consider the term of the subject non-financial liability when
determining the appropriate inputs for the time value of money and
non- performance risk.\label{10.3.5.1.4-End}

\stepcounter{SubSubSecCounter} 

\thesubsubsection.\theSubSubSecCounter.\label{10.3.5.1.5} In certain
circumstances, the valuer may explicitly adjust the cash flows for
non-performance risk.\label{10.3.5.1.5-End}

\stepcounter{SubSubSecCounter} 

\thesubsubsection.\theSubSubSecCounter.\label{10.3.5.1.6} What a
participant would have to pay to borrow the funds necessary to satisfy
the obligation may provide insights to help quantify the non- performance
risk.\label{10.3.5.1.6-End}

\stepcounter{SubSubSecCounter} 

\thesubsubsection.\theSubSubSecCounter.\label{10.3.5.1.7} Given
the long-term nature of certain non-financial liabilities, the valuer
must consider if inflation has been incorporated into the estimated
cash flows, and must ensure that the discount rate and cash flow estimates
are prepared on a consistent basis.\label{10.3.5.1.7-End}\label{subsubsec:10.3.5.1_Discount_rates-End}

\subsubsection{Estimating Cash Flows and Risk Margins\label{subsubsec:10.3.5.2_Estimating_Cash_Flows}}

\stepcounter{SubSubSecCounter} 

\thesubsubsection.\theSubSubSecCounter.\label{10.3.5.2.1} The principles
contained in IVS 105 Valuation Approaches and Methods may not apply
to valuations of non-financial liabilities and valuations with a non-financial
liability component (see IVS 105 Valuation Approaches and Methods,
paras 50.12-50.19). Valuers must apply the principles in sections
90 and 100 of this Standard in valuations of non-financial liabilities.\label{10.3.5.2.1-End}

\stepcounter{SubSubSecCounter} 

\thesubsubsection.\theSubSubSecCounter.\label{10.3.5.2.2} Non-financial
liability cash flow forecasts often involve the explicit modelling
of multiple scenarios of possible future cash flow to derive a probability-
weighted expected cash flow forecast. This method is often referred
to as the Scenario-Based Method (SBM). The SBM also includes certain
simulation techniques such as the Monte Carlo simulation. The SBM
is commonly used when future payments are not contractually defined
but rather vary depending upon future events. When the non-financial
liability cash flows are a function of systematic risk factors, the
valuer should consider the appropriateness of the SBM, and may need
to utilise other methods such as option pricing models (OPMs).\label{10.3.5.2.2-End}

\stepcounter{SubSubSecCounter} 

\thesubsubsection.\theSubSubSecCounter.\label{10.3.5.2.3} Considerations
in estimating cash flows include developing and incorporating explicit
assumptions, to the extent possible. A non-exhaustive list of such
assumptions may include: 
\begin{enumerate}
\item The costs that a third party would incur in performing the tasks necessary
to fulfil the obligation;
\item Other amounts that a third party would include in determining the
price of the transfer, including, for example, inflation, overhead,
equipment charges, profit margin, and advances in technology;
\item The extent to which the amount of a third party’s costs or the timing
of its costs would vary under different future scenarios and the relative
probabilities of those scenarios;
\item The price that a third party would demand and could expect to receive
for bearing the uncertainties and unforeseeable circumstances inherent
in the obligation.\label{10.3.5.2.3-End}
\end{enumerate}
\stepcounter{SubSubSecCounter} 

\thesubsubsection.\theSubSubSecCounter.\label{10.3.5.2.4} While
expected cash flows (ie, the probability-weighted average of possible
future cash flows) incorporate the variable expected outcomes of the
asset’s cash flows, they do not account for the compensation that
participants demand for bearing the uncertainty of the cash flows.
For non-financial liabilities, forecast risk may include uncertainty
such as changes in anticipated fulfilment costs and fulfilment margin.
The compensation for bearing such risk should be incorporated into
the expected payoff through a cash flow risk margin or the discount
rate.\label{10.3.5.2.4-End}

\stepcounter{SubSubSecCounter} 

\thesubsubsection.\theSubSubSecCounter.\label{10.3.5.2.5} Given
the inverse relationship between the discount rate and value, the
discount rate should be decreased to reflect the impact of forecast
risk (ie, the compensation for bearing risk due to uncertainty about
the amount and timing of cash flows).\label{10.3.5.2.5-End}

\stepcounter{SubSubSecCounter} 

\thesubsubsection.\theSubSubSecCounter.\label{10.3.5.2.6} While
possible to account for forecast risk by reducing the discount rate,
given its limited practical application, the valuer must explain the
rationale for reducing the discount rate rather than incorporating
a risk margin, or specifically note the regulation, case law, or other
interpretive guidance that requires the accounting for forecast risk
of non-financial liabilities through the discount rate rather than
a risk margin (see IVS 200 Businesses and Business Interests, para
30.2).\label{10.3.5.2.6-End}

\stepcounter{SubSubSecCounter} 

\thesubsubsection.\theSubSubSecCounter.\label{10.3.5.2.7} In developing
a risk margin, a valuer must: 
\begin{enumerate}
\item document the method used for developing the risk margin, including
support for its use;
\item provide evidence for the derivation of the risk margin, including
the identification of the significant inputs and support for their
derivation or source.\label{10.3.5.2.7-End}
\end{enumerate}
\stepcounter{SubSubSecCounter} 

\thesubsubsection.\theSubSubSecCounter.\label{10.3.5.2.8} In developing
a cash flow risk margin, a valuer must consider:
\begin{enumerate}
\item the life/term and/or maturity of the asset and the consistency of
inputs;
\item the geographic location of the asset and/or the location of the markets
in which it would trade;
\item the currency denomination of the projected cash flows;
\item the type of cash flow contained in the forecast, for example, a cash
flow forecast may represent expected cash flows (ie, probability-weighted
scenarios), most likely cash flows, contractual cash flows, etc.\label{10.3.5.2.8-End}
\end{enumerate}
\stepcounter{SubSubSecCounter} 

\thesubsubsection.\theSubSubSecCounter.\label{10.3.5.2.9} In developing
a cash flow risk margin, a valuer should consider:
\begin{enumerate}
\item the less certainty there is in the anticipated fulfilment costs and
fulfilment margin, the higher the risk margin should be;
\item given the finite term of most non-financial liabilities, as opposed
to indefinite for many business and asset valuations, to the extent
that emerging experience reduces uncertainty, risk margins should
decrease, and vice versa;
\item the expected distribution of outcomes, and the potential for certain
non-financial liabilities to have high ‘tail risk’ or severity. Non-financial
liabilities with wide distributions and high severity should have
higher risk margins;
\item the respective rights and preferences of the non-financial liability,
and/or related asset, in the event of a liquidation and its relative
position within the liquidation waterfall.\label{10.3.5.2.9-End}
\end{enumerate}
\stepcounter{SubSubSecCounter} 

\thesubsubsection.\theSubSubSecCounter.\label{10.3.5.2.10} The cash
flow risk margin should be the compensation that would be required
for a party to be indifferent between fulfilling a liability that
has a range of possible outcomes, and one that will generate fixed
cash outflows.\label{10.3.5.2.10-End}

\stepcounter{SubSubSecCounter} 

\thesubsubsection.\theSubSubSecCounter.\label{10.3.5.2.11} A valuer
need not conduct an exhaustive quantitative process, but should take
into account all the information that is reasonably available.\label{10.3.5.2.11-End}\label{subsubsec:10.3.5.2_Estimating_Cash_Flows-End}

\subsubsection{Restrictions on transfers\label{subsubsec:10.3.5.3_Restrictions_on_transfers}}

\stepcounter{SubSubSecCounter} 

\thesubsubsection.\theSubSubSecCounter.\label{10.3.5.3.1} Non-financial
liabilities often have restrictions on the ability to transfer. Such
restrictions can be either contractual in nature, or a function of
an illiquid market for the subject non-financial liability.\label{10.3.5.3.1-End}

\stepcounter{SubSubSecCounter} 

\thesubsubsection.\theSubSubSecCounter.\label{10.3.5.3.2} When relying
on market evidence, a valuer should consider an entity’s ability to
transfer such non-financial liabilities and whether adjustments to
reflect the restrictions should be included. The valuer may need to
determine if the transfer restrictions are characteristics of the
non-financial liability or restrictions that are characteristics of
an entity, as certain basis of value may specify one or the other
be considered (see IVS 220 Non-Financial Liabilities, para 50.9).\label{10.3.5.3.2-End}

\stepcounter{SubSubSecCounter} 

\thesubsubsection.\theSubSubSecCounter.\label{10.3.5.3.3} When relying
on an income approach in which the non-financial liability value is
estimated through a fulfilment approach, the valuer should determine
if an investor would require an additional risk margin to account
for the limitations on transfer.\label{10.3.5.3.3-End}\label{subsubsec:10.3.5.3_Restrictions_on_transfers-End}

\subsubsection{Taxes\label{subsubsec:10.3.5.4_Taxes}}

\stepcounter{SubSubSecCounter} 

\thesubsubsection.\theSubSubSecCounter.\label{10.3.5.4.1} Valuers
should use pre-tax cash flows and a pre-tax discount rate for the
valuation of non-financial liabilities.\label{10.3.5.4.1-End}

\stepcounter{SubSubSecCounter} 

\thesubsubsection.\theSubSubSecCounter.\label{10.3.5.4.2} In certain
circumstances, it may be appropriate to perform the analysis with
after tax cash flows and discount rates. In such instances, the valuer
must explain the rationale for use of after tax inputs, or specifically
note the regulation, case law, or other interpretive guidance that
requires the use of after tax inputs (see IVS 200 Businesses and Business
Interests, para 30.2).\label{10.3.5.4.2-End}

\stepcounter{SubSubSecCounter} 

\thesubsubsection.\theSubSubSecCounter.\label{10.3.5.4.3} If after
tax inputs are used, it may be appropriate to include the tax benefit
created by the projected cash outflow associated with the non-financial
liability.\label{10.3.5.4.3-End}\label{subsubsec:10.3.5.4_Taxes-End}\label{subsec:10.3.5_Special_Considerations-End}\label{sec:10.3_IVS-220_Non-Financial_Liabilities-End}

\section{IVS 300. Plant and Equipment\label{sec:10.4_IVS-300_Plant_and_Equipment}}

\subsection{Overview\label{subsec:10.4.1_Overview}}

\stepcounter{SubSecCounter} 

\thesubsection.\theSubSecCounter.\label{10.4.1.1} The principles
contained in the General Standards apply to valuations of plant and
equipment. This standard only includes modifications, additional principles
or specific examples of how the General Standards apply for valuations
to which this standard applies.\label{10.4.1.1-End}\label{subsec:10.4.1_Overview-End}

\subsection{Introduction\label{subsec:10.4.2_Introduction}}

\stepcounter{SubSecCounter} 

\thesubsection.\theSubSecCounter.\label{10.4.2.1} Items of plant
and equipment (which may sometimes be categorised as a type of personal
property) are tangible assets that are usually held by an entity for
use in the manufacturing/production or supply of goods or services,
for rental by others or for administrative purposes and that are expected
to be used over a period of time.\label{10.4.2.1-End}

\stepcounter{SubSecCounter} 

\thesubsection.\theSubSecCounter.\label{10.4.2.2} For lease of machinery
and equipment, the right to use an item of machinery and equipment
(such as a right arising from a lease) would also follow the guidance
of this standard. It must also be noted that the “right to use” an
asset could have a different life span than the service life (that
takes into consideration of both preventive and predictive maintenance)
of the underlying machinery and equipment itself and, in such circumstances,
the service life span must be stated.\label{10.4.2.2-End}

\stepcounter{SubSecCounter} 

\thesubsection.\theSubSecCounter.\label{10.4.2.3} Assets for which
the highest and best use is “in use” as part of a group of assets
must be valued using consistent assumptions. Unless the assets belonging
to the sub-systems may reasonably be separated independently from
its main system, then the sub-systems may be valued separately, having
consistent assumptions within the sub-systems. This will also cascade
down to sub-sub-systems and so on.\label{10.4.2.3-End}

\stepcounter{SubSecCounter} 

\thesubsection.\theSubSecCounter.\label{10.4.2.4} Intangible assets
fall outside the classification of plant and equipment assets. However,
an intangible asset may have an impact on the value of plant and equipment
assets. For example, the value of patterns and dies is often inextricably
linked to associated intellectual property rights. Operating software,
technical data, production records and patents are further examples
of intangible assets that can have an impact on the value of plant
and equipment assets, depending on whether or not they are included
in the valuation. In such cases, the valuation process will involve
consideration of the inclusion of intangible assets and their impact
on the valuation of the plant and equipment assets. When there is
an intangible asset component, the valuer should also follow IVS 210
Intangible Assets.\label{10.4.2.4-End}

\stepcounter{SubSecCounter} 

\thesubsection.\theSubSecCounter.\label{10.4.2.5} A valuation of
plant and equipment will normally require consideration of a range
of factors relating to the asset itself, its environment and physical,
functional and economic potential. Therefore, all plant and equipment
valuers should normally inspect the subject assets to ascertain the
condition of the plant and also to determine if the information provided
to them is usable and related to the subject assets being valued.
Examples of factors that may need to be considered under each of these
headings include the following:
\begin{enumerate}
\item Asset-related: 
\begin{enumerate}
\item the asset’s technical specification;
\item the remaining useful, economic or effective life, considering both
preventive and predictive maintenance;
\item the asset’s condition, including maintenance history;
\item any functional, physical and technological obsolescence;
\item if the asset is not valued in its current location, the costs of decommissioning
and removal, and any costs associated with the asset’s existing in-place
location, such as installation and re-commissioning of assets to its
optimum status;
\item for machinery and equipment that are used for rental purposes, the
lease renewal options and other end-of-lease possibilities;
\item any potential loss of a complementary asset, e.\,g, the operational
life of a machine may be curtailed by the length of lease on the building
in which it is located;
\item additional costs associated with additional equipment, transport,
installation and commissioning, etc;
\item in cases where the historical costs are not available for the machinery
and equipment that may reside within a plant during a construction,
the valuer may take references from the Engineering, Procurement,
Construction (“EPC”) contract.
\end{enumerate}
\item Environment-related: 
\begin{enumerate}
\item the location in relation to the source of raw material and market
for the product. The suitability of a location may also have a limited
life, eg, where raw materials are finite or where demand is transitory;
\item the impact of any environmental or other legislation that either restricts
utilisation or imposes additional operating or decommissioning costs;
\item radioactive substances that may be in certain machinery and equipment
have a severe impact if not used or disposed of appropriately. This
will have a major impact on expense consideration and the environment;
\item toxic wastes which may be chemical in the form of a solid, liquid
or gaseous state must be professionally stored or disposed of. This
is critical for all industrial manufacturing;
\item licenses to operate certain machines in certain countries may be restricted.
\end{enumerate}
\item Economic-related: 
\begin{enumerate}
\item the actual or potential profitability of the asset based on comparison
of operating costs with earnings or potential earnings (see IVS 200
Business and Business Interests);
\item the demand for the product manufactured by the plant with regard to
both macro- and micro-economic factors could impact on demand;
\item the potential for the asset to be put to a more valuable use than
the current use (ie, highest and best use).\label{10.4.2.5-End}
\end{enumerate}
\end{enumerate}
\stepcounter{SubSecCounter} 

\thesubsection.\theSubSecCounter.\label{10.4.2.6} Valuations of
plant and equipment should reflect the impact of all forms of obsolescence
on value.\label{10.4.2.6-End}

\stepcounter{SubSecCounter} 

\thesubsection.\theSubSecCounter.\label{10.4.2.7} To comply with
the requirement to identify the asset or liability to be valued in
IVS 101 Scope of Work, para 20.3.(d) to the extent it impacts on value,
consideration must be given to the degree to which the asset is attached
to, or integrated with, other assets. For example: 
\begin{enumerate}
\item assets may be permanently attached to the land and could not be removed
without substantial demolition of either the asset or any surrounding
structure or building;
\item an individual machine may be part of an integrated production line
where its functionality is dependent upon other assets;
\item an asset may be considered to be classified as a component of the
real property (eg, a Heating, Ventilation and Air Conditioning System
(HVAC)). 
\end{enumerate}
In such cases, it will be necessary to clearly define what is to be
included or excluded from the valuation. Any special assumptions relating
to the availability of any complementary assets must also be stated
(see also para 20.8).\label{10.4.2.7-End}

\stepcounter{SubSecCounter} 

\thesubsection.\theSubSecCounter.\label{10.4.2.8} Plant and equipment
connected with the supply or provision of services to a building are
often integrated within the building and, once installed, are not
separable from it. These items will normally form part of the real
property interest. Examples include plant and equipment with the primary
function of supplying electricity, gas, heating, cooling or ventilation
to a building and equipment such as elevators. If the purpose of the
valuation requires these items to be valued separately, the scope
of work must include a statement to the effect that the value of these
items would normally be included in the real property interest and
may not be separately realisable. When different valuation assignments
are undertaken to carry out valuations of the real property interest
and plant and equipment assets at the same location, care is necessary
to avoid either omissions or double counting.\label{10.4.2.8-End}

\stepcounter{SubSecCounter} 

\thesubsection.\theSubSecCounter.\label{10.4.2.9} Because of the
diverse nature and transportability of many items of plant and equipment,
additional assumptions will normally be required to describe the situation
and circumstances in which the assets are valued. In order to comply
with IVS 101 Scope of Work, para 20.3.(k) these must be considered
and included in the scope of work. Examples of assumptions that may
be appropriate in different circumstances include: 
\begin{enumerate}
\item that the plant and equipment assets are valued as a whole, in place
and as part of an operating business;
\item that the plant and equipment assets are valued as a whole, in place
but on the assumption that the business is not yet in production;
\item that the plant and equipment assets are valued as a whole, in place
but on the assumption that the business is closed;
\item that the plant and equipment assets are valued as a whole, in place
but on the assumption that it is a forced sale (See IVS 104 Bases
of Value);
\item that the plant and equipment assets are valued as individual items
for removal from their current location.\label{10.4.2.9-End}
\end{enumerate}
\stepcounter{SubSecCounter} 

\thesubsection.\theSubSecCounter.\label{10.4.2.10}In some circumstances,
it may be appropriate to report on more than one set of assumptions,
e.\,g, in order to illustrate the effect of business closure or cessation
of operations on the value of plant and equipment.\label{10.4.2.10-End}

\stepcounter{SubSecCounter} 

\thesubsection.\theSubSecCounter.\label{10.4.2.11} In addition to
the minimum requirements in IVS 103 Reporting, a valuation report
on plant and equipment must include appropriate references to matters
addressed in the scope of work. The report must also include comment
on the effect on the reported value of any associated tangible or
intangible assets excluded from the actual or assumed transaction
scenario, e.\,g, operating software for a machine or a continued
right to occupy the land on which the item is situated.\label{10.4.2.11-End}

\stepcounter{SubSecCounter} 

\thesubsection.\theSubSecCounter.\label{10.4.2.12} Valuations of
plant and equipment are often required for different purposes including
financial reporting, leasing, secured lending, disposal, taxation,
litigation and insolvency proceedings.\label{10.4.2.12-End}\label{subsec:10.4.2_Introduction-End}

\subsection{Bases of Value\label{subsec:10.4.3_Bases_of_Value}}

\stepcounter{SubSecCounter} 

\thesubsection.\theSubSecCounter.\label{10.4.3.1} In accordance
with IVS 104 Bases of Value, a valuer must select the appropriate
basis(es) of value when valuing plant and equipment.\label{10.4.3.1-End}

\stepcounter{SubSecCounter} 

\thesubsection.\theSubSecCounter.\label{10.4.3.2} Using the appropriate
basis(es) of value and associated premise of value (see IVS 104 Bases
of Value, sections 140-170) is particularly crucial in the valuation
of plant and equipment because differences in value can be pronounced,
depending on whether an item of plant and equipment is valued under
an “in use” premise, orderly liquidation or forced liquidation (see
IVS 104 Bases of Value, para 80.1). The value of most plant and equipment
is particularly sensitive to different premises of value.\label{10.4.3.2-End}

\stepcounter{SubSecCounter} 

\thesubsection.\theSubSecCounter.\label{10.4.3.3} An example of
forced liquidation conditions is where the assets have to be removed
from a property in a timeframe that precludes proper marketing because
a lease of the property is being terminated. The impact of such circumstances
on value needs careful consideration. In order to advise on the value
likely to be realised, it will be necessary to consider any alternatives
to a sale from the current location, such as the practicality and
cost of removing the items to another location for disposal within
the available time limit and any diminution in value due to moving
the item from its working location.\label{10.4.3.3-End}\label{subsec:10.4.3_Bases_of_Value-End}

\subsection{Valuation Approaches and Methods\label{subsec:10.4.4_Valuation_Approaches}}

\stepcounter{SubSubSecCounter} 

\thesubsubsection.\theSubSubSecCounter.\label{10.4.4.0.1} The three
principal valuation approaches described in the IVS may all be applied
to the valuation of plant and equipment assets depending on the nature
of the assets, the information available, and the facts and circumstances
surrounding the valuation.\label{10.4.4.0.1-End}

\subsubsection{Market Approach\label{subsubsec:10.4.4.1_Market_Approach}}

\stepcounter{SubSubSecCounter} 

\thesubsubsection.\theSubSubSecCounter.\label{10.4.4.1.1} For classes
of plant and equipment that are homogeneous, e.\,g, motor vehicles
and certain types of office equipment or industrial machinery, the
market approach is commonly used as there may be sufficient data of
recent sales of similar assets. However, many types of plant and equipment
are specialized and where direct sales evidence for such items will
not be available, care must be exercised in offering an income or
cost approach opinion of value when available market data is poor
or non-existent. In such circumstances it may be appropriate to adopt
either the income approach or the cost approach to the valuation.\label{10.4.4.1.1-End}\label{subsubsec:10.4.4.1_Market_Approach-End}

\subsubsection{Income Approach\label{subsubsec:10.4.4.2_Income_Approach}}

\stepcounter{SubSubSecCounter} 

\thesubsubsection.\theSubSubSecCounter.\label{10.4.4.2.1} The income
approach to the valuation of plant and equipment can be used where
specific cash flows can be identified for the asset or a group of
complementary assets, eg, where a group of assets forming a process
plant is operating to produce a marketable product. However, some
of the cash flows may be attributable to intangible assets and difficult
to separate from the cash flow contribution of the plant and equipment.
Use of the income approach is not normally practical for many individual
items of plant or equipment; however, it can be utilized in assessing
the existence and quantum of economic obsolescence for an asset or
asset group.\label{10.4.4.2.1-End}

\stepcounter{SubSubSecCounter} 

\thesubsubsection.\theSubSubSecCounter.\label{10.4.4.2.2} When an
income approach is used to value plant and equipment, the valuation
must consider the cash flows expected to be generated over the life
of the asset(s) as well as the value of the asset at the end of its
life. Care must be exercised when plant and equipment is valued on
an income approach to ensure that elements of value relating to intangible
assets, goodwill and other contributory assets is excluded (see IVS
210 Intangible Assets).\label{10.4.4.2.2-End}\label{subsubsec:10.4.4.2_Income_Approach-End}

\subsubsection{Cost Approach\label{subsubsec:10.4.4.3_Cost_Approach}}

\stepcounter{SubSubSecCounter} 

\thesubsubsection.\theSubSubSecCounter.\label{10.4.4.3.1} The cost
approach is commonly adopted for plant and equipment, particularly
in the case of individual assets that are specialized or special-use
facilities. The first step is to estimate the cost to a market participant
of replacing the subject asset by reference to the lower of either
reproduction or replacement cost. The replacement cost is the cost
of obtaining an alternative asset of equivalent utility; this can
either be a modern equivalent providing the same functionality or
the cost of reproducing an exact replica of the subject asset. After
concluding on a replacement cost, the value should be adjusted to
reflect the impact on value of physical, functional, technological
and economic obsolescence on value. In any event, adjustments made
to any particular replacement cost should be designed to produce the
same cost as the modern equivalent asset from an output and utility
point of view.\label{10.4.4.3.1-End}

\stepcounter{SubSubSecCounter} 

\thesubsubsection.\theSubSubSecCounter.\label{10.4.4.3.2} An entity’s
actual costs incurred in the acquisition or construction of an asset
may be appropriate for use as the replacement cost of an asset under
certain circumstances. However, prior to using such historical cost
information, the valuer should consider the following:
\begin{enumerate}
\item Timing of the historical expenditures: An entity’s actual costs may
not be relevant, or may need to be adjusted for inflation/indexation
to an equivalent as of the valuation date, if they were not incurred
recently due to changes in market prices, inflation/deflation or other
factors. 
\item The basis of value: Care must be taken when adopting a particular
market participant’s own costings or profit margins, as they may not
represent what typical market participants might have paid. The valuer
must also consider the possibility that the entity’s costs incurred
may not be historical in nature due to prior purchase accounting or
the purchase of used plant and equipment assets. In any case, historical
costs must be trended using appropriate indices. 
\item Specific costs included: A valuer must consider all significant costs
that have been included and whether those costs contribute to the
value of the asset and for some bases of value, some amount of profit
margin on costs incurred may be appropriate. 
\item Non-market components: Any costs, discounts or rebates that would
not be incurred by, or available to, typical market participants should
be excluded.\label{10.4.4.3.2-End}
\end{enumerate}
\stepcounter{SubSubSecCounter} 

\thesubsubsection.\theSubSubSecCounter.\label{10.4.4.3.3} Having
established the replacement cost, deductions must be made to reflect
the physical, functional, technological and economic obsolescence
as applicable (see IVS 105 Valuation Approaches and Methods, section
80).\label{10.4.4.3.3-End}\label{subsubsec:10.4.4.3_Cost_Approach-End}

\subsubsection{Cost-to-Capacity Method\label{subsubsec:10.4.4.4_Capacity_Method}}

\stepcounter{SubSubSecCounter} 

\thesubsubsection.\theSubSubSecCounter.\label{10.4.4.4.1} Under
the cost-to-capacity method, the replacement cost of an asset with
an actual or required capacity can be determined by reference to the
cost of a similar asset with a different capacity.\label{10.4.4.4.1-End}

\stepcounter{SubSubSecCounter} 

\thesubsubsection.\theSubSubSecCounter.\label{10.4.4.4.2} The cost-to-capacity
method is generally used in one of two ways: 
\begin{enumerate}
\item to estimate the replacement cost for an asset or assets with one capacity
where the replacement costs of an asset or assets with a different
capacity are known (such as when the capacity of two subject assets
could be replaced by a single asset with a known cost);
\item to estimate the replacement cost for a modern equivalent asset with
capacity that matches foreseeable demand where the subject asset has
excess capacity (as a means of measuring the penalty for the lack
of utility to be applied as part of an economic obsolescence adjustment).\label{10.4.4.4.2-End}
\end{enumerate}
\stepcounter{SubSubSecCounter} 

\thesubsubsection.\theSubSubSecCounter.\label{10.4.4.4.3} This method
may only be used as a check method unless there is an existence of
an exact comparison plant of the same designed capacity that resides
within the same geographical area.\label{10.4.4.4.3-End}

\stepcounter{SubSubSecCounter} 

\thesubsubsection.\theSubSubSecCounter.\label{10.4.4.4.4} It is
noted that the relationship between cost and capacity is often not
linear, so some form of exponential adjustment may also be required.\label{10.4.4.4.4-End}\label{subsubsec:10.4.4.4_Capacity_Method-End}\label{subsec:10.4.4_Valuation_Approaches-End}

\subsection{Special Considerations for Plant and Equipment\label{subsec:10.4.5_Special_Considerations}}

\stepcounter{SubSubSecCounter} 

\thesubsubsection.\theSubSubSecCounter.\label{10.4.5.0.1} The following
section Financing Arrangements addresses a non-exhaustive list of
topics relevant to the valuation of plant and equipment.\label{10.4.5.0.1-End}

\subsubsection{Financing Arrangements\label{subsubsec:10.4.5.1_Financing_Arrangements}}

\stepcounter{SubSubSecCounter} 

\thesubsubsection.\theSubSubSecCounter. \label{10.4.5.1.1} Generally,
the value of an asset is independent of how it is financed. However,
in some circumstances the way items of plant and equipment are financed
and the stability of that financing may need to be considered in valuation.\label{10.4.5.1.1-End}

\stepcounter{SubSubSecCounter} 

\thesubsubsection.\theSubSubSecCounter.\label{10.4.5.1.2} An item
of plant and equipment may be subject to a leasing or financing arrangement.
Accordingly, the asset cannot be sold without the lender or lessor
being paid any balance outstanding under the financing arrangement.
This payment may or may not exceed the unencumbered value of the item
to the extent unusual/excessive for the industry. Depending upon the
purpose of the valuation, it may be appropriate to identify any encumbered
assets and to report their values separately from the unencumbered
assets.\label{10.4.5.1.2-End}

\stepcounter{SubSubSecCounter} 

\thesubsubsection.\theSubSubSecCounter.\label{10.4.5.1.3} Items
of plant and equipment that are subject to operating leases are the
property of third parties and are therefore not included in a valuation
of the assets of the lessee, subject to the lease meeting certain
conditions. However, such assets may need to be recorded as their
presence may impact on the value of owned assets used in association.
In any event, prior to undertaking a valuation, the valuer should
establish (in conjunction with Client and/or advisors) whether assets
are subject to operating lease, finance lease or loan, or other secured
lending. The conclusion on this regard and wider purpose of the valuation
will then dictate the appropriate basis and valuation methodology.\label{10.4.5.1.3-End}\label{subsubsec:10.4.5.1_Financing_Arrangements-End}\label{subsec:10.4.5_Special_Considerations-End}\label{sec:10.4_IVS-300_Plant_and_Equipment-End}

\section{IVS 400. Real Property Interests\label{sec:10.5_IVS-400_Real_Property}}

\subsection{Overview\label{subsec:10.5.1_Overview}}

\stepcounter{SubSecCounter} 

\thesubsection.\theSubSecCounter.\label{10.5.1.1} The principles
contained in the General Standards apply to valuations of real property
interests. This standard contains additional requirements for valuations
of real property interests.\label{10.5.1.1-End}\label{subsec:10.5.1_Overview-End}

\subsection{Introduction\label{subsec:10.5.2_Introduction}}

\stepcounter{SubSecCounter} 

\thesubsection.\theSubSecCounter.\label{10.5.2.1} Property interests
are normally defined by state or the law of individual jurisdictions
and are often regulated by national or local legislation. Before undertaking
a valuation of a real property interest, a valuer must understand
the relevant legal framework that affects the interest being valued.\label{10.5.2.1-End}

\stepcounter{SubSecCounter} 

\thesubsection.\theSubSecCounter.\label{10.5.2.2} A real property
interest is a right of ownership, control, use or occupation of land
and buildings. There are three main types of interest:
\begin{enumerate}
\item the superior interest in any defined area of land. The owner of this
interest has an absolute right of possession and control of the land
and any buildings upon it in perpetuity, subject only to any subordinate
interests and any statutory or other legally enforceable constraints;
\item a subordinate interest that normally gives the holder rights of exclusive
possession and control of a defined area of land or buildings for
a defined period, eg, under the terms of a lease contract;
\item a right to use land or buildings but without a right of exclusive
possession or control, eg, a right to pass over land or to use it
only for a specified activity.\label{10.5.2.2-End}
\end{enumerate}
\stepcounter{SubSecCounter} 

\thesubsection.\theSubSecCounter.\label{10.5.2.3} Intangible assets
fall outside the classification of real property assets. However,
an intangible asset may be associated with, and have a material impact
on, the value of real property assets. It is therefore essential to
be clear in the scope of work precisely what the valuation assignment
is to include or exclude. For example, the valuation of a hotel can
be inextricably linked to the hotel brand. In such cases, the valuation
process will involve consideration of the inclusion of intangible
assets and their impact on the valuation of the real property and
plant and equipment assets. When there is an intangible asset component,
the valuer should also follow IVS 210 Intangible Assets.\label{10.5.2.3-End}

\stepcounter{SubSecCounter} 

\thesubsection.\theSubSecCounter.\label{10.5.2.4} Although different
words and terms are used to describe these types of real property
interest in different jurisdictions, the concepts of an unlimited
absolute right of ownership, an exclusive interest for a limited period
or a non-exclusive right for a specified purpose are common to most.
The immovability of land and buildings means that it is the right
that a party holds that is transferred in an exchange, not the physical
land and buildings. The value, therefore, attaches to the legal interest
rather than to the physical land and buildings.\label{10.5.2.4-End}

\stepcounter{SubSecCounter} 

\thesubsection.\theSubSecCounter.\label{10.5.2.5} To comply with
the requirement to identify the asset to be valued in IVS 101 Scope
of Work, para 20.3.(d) the following matters must be included: 
\begin{enumerate}
\item a description of the real property interest to be valued;
\item identification of any superior or subordinate interests that affect
the interest to be valued.\label{10.5.2.5-End}
\end{enumerate}
\stepcounter{SubSecCounter} 

\thesubsection.\theSubSecCounter.\label{10.5.2.6} To comply with
the requirements to state the extent of the investigation and the
nature and source of the information to be relied upon in IVS 101
Scope of Work, para 20.3.(j) and IVS 102 Investigations and Compliance,
the following matters must be considered:
\begin{enumerate}
\item the evidence required to verify the real property interest and any
relevant related interests;
\item the extent of any inspection;
\item responsibility for information on the site area and any building floor
areas;
\item responsibility for confirming the specification and condition of any
building;
\item the extent of investigation into the nature, specification and adequacy
of services;
\item the existence of any information on ground and foundation conditions;
\item responsibility for the identification of actual or potential environmental
risks;
\item legal permissions or restrictions on the use of the property and any
buildings, as well as any expected or potential changes to legal permissions
and restrictions.\label{10.5.2.6-End}
\end{enumerate}
\stepcounter{SubSecCounter} 

\thesubsection.\theSubSecCounter.\label{10.5.2.7} Typical examples
of special assumptions that may need to be agreed and confirmed in
order to comply with IVS 101 Scope of Work, para 20.3. (k) include: 
\begin{enumerate}
\item that a defined physical change had occurred, e.\,g, a proposed building
is valued as if complete at the valuation date;
\item that there had been a change in the status of the property, e.\,g,
a vacant building had been leased or a leased building had become
vacant at the valuation date;
\item that the interest is being valued without taking into account other
existing interests;
\item that the property is free from contamination or other environmental
risks.\label{10.5.2.7-End}
\end{enumerate}
\stepcounter{SubSecCounter} 

\thesubsection.\theSubSecCounter.\label{10.5.2.8} Valuations of
real property interests are often required for different purposes
including secured lending, sales and purchases, taxation, litigation,
compensation, insolvency proceedings and financial reporting.\label{10.5.2.8-End}\label{subsec:10.5.2_Introduction-End}

\subsection{Bases of Value\label{subsec:10.5.3_Bases_of_Value}}

\stepcounter{SubSecCounter} 

\thesubsection.\theSubSecCounter.\label{10.5.3.1} In accordance
with IVS 104 Bases of Value, a valuer must select the appropriate
basis(es) of value when valuing real property interests.\label{10.5.3.1-End}

\stepcounter{SubSecCounter} 

\thesubsection.\theSubSecCounter.\label{10.5.3.2} Under most bases
of value, a valuer must consider the highest and best use of the real
property, which may differ from its current use (see IVS 104 Bases
of Value, para 30.3). This assessment is particularly important to
real property interests which can be changed from one use to another
or that have development potential.\label{10.5.3.2-End}\label{subsec:10.5.3_Bases_of_Value-End}

\subsection{Valuation Approaches and Methods\label{subsec:10.5.4_Valuation_Approaches}}

\stepcounter{SubSubSecCounter} 

\thesubsubsection.\theSubSubSecCounter.\label{10.5.4.0.1} The three
valuation approaches described in the IVS 105 Valuation Approaches
and Methods can all be applicable for the valuation of a real property
interest.\label{10.5.4.0.1-End}

\stepcounter{SubSubSecCounter} 

\thesubsubsection.\theSubSubSecCounter.\label{10.5.4.0.2} When selecting
an approach and method, in addition to the requirements of this standard,
a valuer must follow the requirements of IVS 105 Valuation Approaches
and Methods, including para 10.3 and 10.4.\label{10.5.4.0.2-End}

\subsubsection{Market Approach\label{subsubsec:10.5.4.1_Market_Approach}}

\stepcounter{SubSubSecCounter} 

\thesubsubsection.\theSubSubSecCounter.\label{10.5.4.1.1}Property
interests are generally heterogeneous (ie, with different characteristics).
Even if the land and buildings have identical physical characteristics
to others being exchanged in the market, the location will be different.
Notwithstanding these dissimilarities, the market approach is commonly
applied for the valuation of real property interests.\label{10.5.4.1.1-End}

\stepcounter{SubSubSecCounter} 

\thesubsubsection.\theSubSubSecCounter.\label{10.5.4.1.2} In order
to compare the subject of the valuation with the price of other real
property interests, valuers should adopt generally accepted and appropriate
units of comparison that are considered by participants, dependent
upon the type of asset being valued. Units of comparison that are
commonly used include: 
\begin{enumerate}
\item price per square meter (or per square foot) of a building or per hectare
for land;
\item price per room;
\item price per unit of output, e.\,g, crop yields.\label{10.5.4.1.2-End}
\end{enumerate}
\stepcounter{SubSubSecCounter} 

\thesubsubsection.\theSubSubSecCounter.\label{10.5.4.1.3} A unit
of comparison is only useful when it is consistently selected and
applied to the subject property and the comparable properties in each
analysis. To the extent possible, any unit of comparison used should
be one commonly used by participants in the relevant market.\label{10.5.4.1.3-End}

\stepcounter{SubSubSecCounter} 

\thesubsubsection.\theSubSubSecCounter.\label{10.5.4.1.4} The reliance
that can be applied to any comparable price data in the valuation
process is determined by comparing various characteristics of the
property and transaction from which the data was derived with the
property being valued. Differences between the following should be
considered in accordance with IVS 105 Valuation Approaches and Methods,
para 30.8. Specific differences that should be considered in valuing
real property interests include, but are not limited to:
\begin{enumerate}
\item the type of interest providing the price evidence and the type of
interest being valued;
\item the respective locations;
\item the respective quality of the land or the age and specification of
the buildings;
\item the permitted use or zoning at each property;
\item the circumstances under which the price was determined and the basis
of value required;
\item the effective date of the price evidence and the valuation date;
\item market conditions at the time of the relevant transactions and how
they differ from conditions at the valuation date.\label{10.5.4.1.4-End}
\end{enumerate}
End\label{subsubsec:10.5.4.1_Market_Approach-End}

\subsubsection{Income Approach\label{subsubsec:10.5.4.2_Income_Approach}}

\stepcounter{SubSubSecCounter} 

\thesubsubsection.\theSubSubSecCounter.\label{10.5.4.2.1} Various
methods are used to indicate value under the general heading of the
income approach, all of which share the common characteristic that
the value is based upon an actual or estimated income that either
is, or could be, generated by an owner of the interest. In the case
of an investment property, that income could be in the form of rent
(see paras 90.1-90.3); in an owner-occupied building, it could be
an assumed rent (or rent saved) based on what it would cost the owner
to lease equivalent space.\label{10.5.4.2.1-End}

\stepcounter{SubSubSecCounter} 

\thesubsubsection.\theSubSubSecCounter.\label{10.5.4.2.2} For some
real property interests, the income-generating ability of the property
is closely tied to a particular use or business/trading activity (for
example, hotels, golf courses, etc). Where a building is suitable
for only a particular type of trading activity, the income is often
related to the actual or potential cash flows that would accrue to
the owner of that building from the trading activity. The use of a
property’s trading potential to indicate its value is often referred
to as the “profits method”.\label{10.5.4.2.2-End}

\stepcounter{SubSubSecCounter} 

\thesubsubsection.\theSubSubSecCounter.\label{10.5.4.2.3} When the
income used in the income approach represents cash flow from a business/trading
activity (rather than cash flow related to rent, maintenance and other
real property-specific costs), the valuer should also comply as appropriate
with the requirements of IVS 200 Business and Business Interests and,
where applicable, IVS 210 Intangible Assets.\label{10.5.4.2.3-End}

\stepcounter{SubSubSecCounter} 

\thesubsubsection.\theSubSubSecCounter.\label{10.5.4.2.4} For real
property interests, various forms of discounted cash flow models may
be used. These vary in detail but share the basic characteristic that
the cash flow for a defined future period is adjusted to a present
value using a discount rate. The sum of the present day values for
the individual periods represents an estimate of the capital value.
The discount rate in a discounted cash flow model will be based on
the time cost of money and the risks and rewards of the income stream
in question.\label{10.5.4.2.4-End}

\stepcounter{SubSubSecCounter} 

\thesubsubsection.\theSubSubSecCounter.\label{10.5.4.2.5} Further
information on the derivation of discount rates is included in IVS
105 Valuation Approaches and Methods, paras 50.29-50.31. The development
of a yield or discount rate should be influenced by the objective
of the valuation. For example: 
\begin{enumerate}
\item if the objective of the valuation is to establish the value to a particular
owner or potential owner based on their own investment criteria, the
rate used may reflect their required rate of return or their weighted
average cost of capital;
\item if the objective of the valuation is to establish the market value,
the discount rate may be derived from observation of the returns implicit
in the price paid for real property interests traded in the market
between participants or from hypothetical participants’ required rates
or return. When a discount rate is based on an analysis of market
transactions, valuers should also follow the guidance contained in
IVS 105 Valuation Approaches and Methods, paras 30.7 and 30.8.\label{10.5.4.2.5-End}
\end{enumerate}
\stepcounter{SubSubSecCounter} 

\thesubsubsection.\theSubSubSecCounter.\label{10.5.4.2.6} An appropriate
discount rate may also be built up from a typical “risk-free“ return
adjusted for the additional risks and opportunities specific to the
particular real property interest.\label{10.5.4.2.6-End}\label{subsubsec:10.5.4.2_Income_Approach-End}

\subsubsection{Cost Approach\label{subsubsec:10.5.4.3_Cost_Approach}}

\stepcounter{SubSubSecCounter} 

\thesubsubsection.\theSubSubSecCounter.\label{10.5.4.3.1} In applying
the cost approach, valuers must follow the guidance contained in IVS
105 Valuation Approaches and Methods, paras 70.1-70.14.\label{10.5.4.3.1-End}

\stepcounter{SubSubSecCounter} 

\thesubsubsection.\theSubSubSecCounter.\label{10.5.4.3.2} This approach
is generally applied to the valuation of real property interests through
the depreciated replacement cost method.\label{10.5.4.3.2-End}

\stepcounter{SubSubSecCounter} 

\thesubsubsection.\theSubSubSecCounter.\label{10.5.4.3.3} It may
be used as the primary approach when there is either no evidence of
transaction prices for similar property or no identifiable actual
or notional income stream that would accrue to the owner of the relevant
interest.\label{10.5.4.3.3-End}

\stepcounter{SubSubSecCounter} 

\thesubsubsection.\theSubSubSecCounter.\label{10.5.4.3.4} In some
cases, even when evidence of market transaction prices or an identifiable
income stream is available, the cost approach may be used as a secondary
or corroborating approach.\label{10.5.4.3.4-End}

\stepcounter{SubSubSecCounter} 

\thesubsubsection.\theSubSubSecCounter.\label{10.5.4.3.5} The first
step requires a replacement cost to be calculated. This is normally
the cost of replacing the property with a modern equivalent at the
relevant valuation date. An exception is where an equivalent property
would need to be a replica of the subject property in order to provide
a participant with the same utility, in which case the replacement
cost would be that of reproducing or replicating the subject building
rather than replacing it with a modern equivalent. The replacement
cost must reflect all incidental costs, as appropriate, such as the
value of the land, infrastructure, design fees, finance costs and
developer profit that would be incurred by a participant in creating
an equivalent asset.\label{10.5.4.3.5-End}

\stepcounter{SubSubSecCounter} 

\thesubsubsection.\theSubSubSecCounter.\label{10.5.4.3.6} The cost
of the modern equivalent must then, as appropriate, be subject to
adjustment for physical, functional, technological and economic obsolescence
(see IVS 105 Valuation Approaches and Methods, section 80). The objective
of an adjustment for obsolescence is to estimate how much less valuable
the subject property might, or would be, to a potential buyer than
the modern equivalent. Obsolescence considers the physical condition,
functionality and economic utility of the subject property compared
to the modern equivalent.\label{10.5.4.3.6-End}\label{subsubsec:10.5.4.3_Cost_Approach-End}\label{subsec:10.5.4_Valuation_Approaches-End}

\subsection{Special Considerations for Real Property Interests\label{subsec:10.5.5_Special_Considerations}}

\stepcounter{SubSubSecCounter} 

\thesubsubsection.\theSubSubSecCounter.\label{10.5.5.0.1} The following
sections address a non-exhaustive list of topics relevant to the valuation
of real property interests.
\begin{enumerate}
\item Hierarchy of Interests (section 90). 
\item Rent (section 100).\label{10.5.5.0.1-End}
\end{enumerate}

\subsubsection{Hierarchy of Interests\label{subsubsec:10.5.5.1_Hierarchy_of_Interests}}

\stepcounter{SubSubSecCounter} 

\thesubsubsection.\theSubSubSecCounter.\label{10.5.5.1.1} The different
types of real property interests are not mutually exclusive. For example,
a superior interest may be subject to one or more subordinate interests.
The owner of the absolute interest may grant a lease interest in respect
of part or all of his interest. Lease interests granted directly by
the owner of the absolute interest are “head lease” interests. Unless
prohibited by the terms of the lease contract, the holder of a head
lease interest can grant a lease of part or all of that interest to
a third party, which is known as a sub-lease interest. A sub-lease
interest will always be shorter than, or coterminous with, the head
lease out of which it is created.\label{10.5.5.1.1-End}

\stepcounter{SubSubSecCounter} 

\thesubsubsection.\theSubSubSecCounter.\label{10.5.5.1.2} These
property interests will have their own characteristics, as illustrated
in the following examples: 
\begin{enumerate}
\item Although an absolute interest provides outright ownership in perpetuity,
it may be subject to the effect of subordinate interests. These subordinate
interests could include leases, restrictions imposed by a previous
owner or restrictions imposed by statute.
\item A lease interest will be for a defined period, at the end of which
the property reverts to the holder of the superior interest out of
which it was created. The lease contract will normally impose obligations
on the lessee, e.\,g, the payment of rent and other expenses. It
may also impose conditions or restrictions, such as in the way the
property may be used or on any transfer of the interest to a third
party.
\item A right of use may be held in perpetuity or may be for a defined period.
The right may be dependent on the holder making payments or complying
with certain other conditions.\label{10.5.5.1.2-End}
\end{enumerate}
\stepcounter{SubSubSecCounter} 

\thesubsubsection.\theSubSubSecCounter.\label{10.5.5.1.3} When valuing
a real property interest it is therefore necessary to identify the
nature of the rights accruing to the holder of that interest and reflect
any constraints or encumbrances imposed by the existence of other
interests in the same property. The sum of the individual values of
various different interests in the same property will frequently differ
from the value of the unencumbered superior interest.\label{10.5.5.1.3-End}\label{subsubsec:10.5.5.1_Hierarchy_of_Interests-End}

\subsubsection{Rent\label{subsubsec:10.5.5.2_Rent}}

\stepcounter{SubSubSecCounter} 

\thesubsubsection.\theSubSubSecCounter.\label{10.5.5.2.1} Market
rent is addressed as a basis of value in IVS 104 Bases of Value.\label{10.5.5.2.1-End}

\stepcounter{SubSubSecCounter} 

\thesubsubsection.\theSubSubSecCounter.\label{10.5.5.2.2} When valuing
either a superior interest that is subject to a lease or an interest
created by a lease, valuers must consider the contract rent and, in
cases where it is different, the market rent.\label{10.5.5.2.2-End}

\stepcounter{SubSubSecCounter} 

\thesubsubsection.\theSubSubSecCounter.\label{10.5.5.2.3}The contract
rent is the rent payable under the terms of an actual lease. It may
be fixed for the duration of the lease or variable. The frequency
and basis of calculating variations in the rent will be set out in
the lease and must be identified and understood in order to establish
the total benefits accruing to the lessor and the liability of the
lessee. \label{10.5.5.2.3-End}\label{subsubsec:10.5.5.2_Rent-End}\label{subsec:10.5.5_Special_Considerations-End}\label{sec:10.5_IVS-400_Real_Property-End}

\section{IVS 410. Development Property\label{sec:10.6_IVS-410_Development_Property}}

\subsection{Overview\label{subsec:10.6.1_Overview}}

\stepcounter{SubSecCounter} 

\thesubsection.\theSubSecCounter.\label{10.6.1.1} The principles
contained in the General Standards IVS 101 to IVS 105 apply to valuations
of development property. This standard only includes modifications,
additional requirements or specific examples of how the General Standards
apply for valuations to which this standard applies. Valuations of
development property must also follow IVS 400 Real Property Interests.\label{10.6.1.1-End}\label{subsec:10.6.1_Overview-End}

\subsection{Introduction\label{subsec:10.6.2_Introduction}}

\stepcounter{SubSecCounter} 

\thesubsection.\theSubSecCounter.\label{10.6.2.1} In the context
of this standard, development properties are defined as interests
where redevelopment is required to achieve the highest and best use,
or where improvements are either being contemplated or are in progress
at the valuation date and include:
\begin{enumerate}
\item the construction of buildings;
\item previously undeveloped land which is being provided with infrastructure;
\item the redevelopment of previously developed land;
\item the improvement or alteration of existing buildings or structures;
\item land allocated for development in a statutory plan;
\item land allocated for a higher value uses or higher density in a statutory
plan.\label{10.6.2.1-End}
\end{enumerate}
\stepcounter{SubSecCounter} 

\thesubsection.\theSubSecCounter.\label{10.6.2.2} Valuations of
development property may be required for different purposes. It is
the valuer’s responsibility to understand the purpose of a valuation.
A non-exhaustive list of examples of circumstances that may require
a development valuation is provided below:
\begin{enumerate}
\item when establishing whether proposed projects are financially feasible;
\item as part of general consulting and transactional support engagements
for acquisition and loan security;
\item for tax reporting purposes, development valuations are frequently
needed for ad valorem taxation analyses;
\item for litigation requiring valuation analysis in circumstances such
as shareholder disputes and damage calculations;
\item for financial reporting purposes, valuation of a development property
is often required in connection with accounting for business combinations,
asset acquisitions and sales, and impairment analysis;
\item for other statutory or legal events that may require the valuation
of development property such as compulsory purchases.\label{10.6.2.2-End}
\end{enumerate}
\stepcounter{SubSecCounter} 

\thesubsection.\theSubSecCounter.\label{10.6.2.3} When valuing development
property, valuers must follow the applicable standard for that type
of asset or liability (for example, IVS 400 Real Property Interests).\label{10.6.2.3-End}

\stepcounter{SubSecCounter} 

\thesubsection.\theSubSecCounter.\label{10.6.2.4} The residual value
or land value of a development property can be very sensitive to changes
in assumptions or projections concerning the income or revenue to
be derived from the completed project or any of the development costs
that will be incurred. This remains the case regardless of the method
or methods used or however diligently the various inputs are researched
in relation to the valuation date.\label{10.6.2.4-End}

\stepcounter{SubSecCounter} 

\thesubsection.\theSubSecCounter.\label{10.6.2.5} This sensitivity
also applies to the impact of significant changes in either the costs
of the project or the value on completion. If the valuation is required
for a purpose where significant changes in value over the duration
of a construction project may be of concern to the user (eg, where
the valuation is for loan security or to establish a project’s viability),
the valuer must highlight the potentially disproportionate effect
of possible changes in either the construction costs or end value
on the profitability of the project and the value of the partially
completed property. A sensitivity analysis may be useful for this
purpose provided it is accompanied by a suitable explanation.\label{10.6.2.5-End}\label{subsec:10.6.2_Introduction-End}

\subsection{Bases of Value\label{subsec:10.6.3_Bases_of_Value}}

\stepcounter{SubSecCounter} 

\thesubsection.\theSubSecCounter.\label{10.6.3.1} In accordance
with IVS 104 Bases of Value, a valuer must select the appropriate
basis(es) of value when valuing development property.\label{10.6.3.1-End}

\stepcounter{SubSecCounter} 

\thesubsection.\theSubSecCounter.\label{10.6.3.2} The valuation
of development property often includes a significant number of assumptions
and special assumptions regarding the condition or status of the project
when complete. For example, special assumptions may be made that the
development has been completed or that the property is fully leased.
As required by IVS 101 Scope of Work, significant assumptions and
special assumptions used in a valuation must be communicated to all
parties to the valuation engagement and must be agreed and confirmed
in the scope of work. Particular care may also be required where reliance
may be placed by third parties on the valuation outcome.\label{10.6.3.2-End}

\stepcounter{SubSecCounter} 

\thesubsection.\theSubSecCounter.\label{10.6.3.3} Frequently it
will be either impracticable or impossible to verify every feature
of a development property which could have an impact on potential
future development, such as where ground conditions have yet to be
investigated. When this is the case, it may be appropriate to make
assumptions (eg, that there are no abnormal ground conditions that
would result in significantly increased costs). If this was an assumption
that a participant would not make, it would need to be presented as
a special assumption.\label{10.6.3.3-End}

\stepcounter{SubSecCounter} 

\thesubsection.\theSubSecCounter.\label{10.6.3.4} In situations
where there has been a change in the market since a project was originally
conceived, a project under construction may no longer represent the
highest and best use of the land. In such cases, the costs to complete
the project originally proposed may be irrelevant as a buyer in the
market would either demolish any partially completed structures or
adapt them for an alternative project. The value of the development
property under construction would need to reflect the current value
of the alternative project and the costs and risks associated with
completing that project.\label{10.6.3.4-End}

\stepcounter{SubSecCounter} 

\thesubsection.\theSubSecCounter.\label{10.6.3.5} For some development
properties, the property is closely tied to a particular use or business/trading
activity or a special assumption is made that the completed property
will trade at specified and sustainable levels. In such cases, the
valuer must, as appropriate, also comply with the requirements of
IVS 200 Business and Business Interests and, where applicable, IVS
210 Intangible Assets.\label{10.6.3.5-End}\label{subsec:10.6.3_Bases_of_Value-End}

\subsection{Valuation Approaches and Methods\label{subsec:10.6.4_Valuation_Approaches}}

\stepcounter{SubSubSecCounter} 

\thesubsubsection.\theSubSubSecCounter.\label{10.6.4.0.1} The three
principal valuation approaches described in IVS 105 Valuation Approaches
and Methods may all be applicable for the valuation of a real property
interest. There are two main approaches in relation to the valuation
of the development property. These are: 
\begin{enumerate}
\item the market approach (see section 50);
\item the residual method, which is a hybrid of the market approach, the
income approach and the cost approach (see sections 40-70). This is
based on the completed “gross development value” and the deduction
of development costs and the developer’s return to arrive at the residual
value of the development property (see section 90).\label{10.6.4.0.1-End}
\end{enumerate}
\stepcounter{SubSubSecCounter} 

\thesubsubsection.\theSubSubSecCounter.\label{10.6.4.0.2} When selecting
an approach and method, in addition to the requirements of this standard,
a valuer must follow the requirements of IVS 105 Valuation Approaches
and Methods, including para 10.3.\label{10.6.4.0.2-End}

\stepcounter{SubSubSecCounter} 

\thesubsubsection.\theSubSubSecCounter.\label{10.6.4.0.3} The valuation
approach to be used will depend on the required basis of value as
well as specific facts and circumstances, eg, the level of recent
transactions, the stage of development of the project and movements
in property markets since the project started, and should always be
that which is most appropriate to those circumstances. Therefore,
the exercise of judgment in the selection of the most suitable approach
is critical.\label{10.6.4.0.3-End}

\subsubsection{Market Approach\label{subsubsec:10.6.4.1_Market_Approach}}

\stepcounter{SubSubSecCounter} 

\thesubsubsection.\theSubSubSecCounter.\label{10.6.4.1.1} Some types
of development property can be sufficiently homogeneous and frequently
exchanged in a market for there to be sufficient data from recent
sales to use as a direct comparison where a valuation is required.\label{10.6.4.1.1-End}

\stepcounter{SubSubSecCounter} 

\thesubsubsection.\theSubSubSecCounter.\label{10.6.4.1.2} In most
markets, the market approach may have limitations for larger or more
complex development property, or smaller properties where the proposed
improvements are heterogeneous. This is because the number and extent
of the variables between different properties make direct comparisons
of all variables inapplicable though correctly adjusted market evidence
(See IVS 105 Valuation Approaches and Methods, section 20.5) may be
used as the basis for a number of variables within the valuation.\label{10.6.4.1.2-End}

\stepcounter{SubSubSecCounter} 

\thesubsubsection.\theSubSubSecCounter.\label{10.6.4.1.3} For development
property where work on the improvements has commenced but is incomplete,
the application of the market approach is even more problematic. Such
properties are rarely transferred between participants in their partially-completed
state, except as either part of a transfer of the owning entity or
where the seller is either insolvent or facing insolvency and therefore
unable to complete the project. Even in the unlikely event of there
being evidence of a transfer of another partially-completed development
property close to the valuation date, the degree to which work has
been completed would almost certainly differ, even if the properties
were otherwise similar.\label{10.6.4.1.3-End}

\stepcounter{SubSubSecCounter} 

\thesubsubsection.\theSubSubSecCounter.\label{10.6.4.1.4} The market
approach may also be appropriate for establishing the value of a completed
property as one of the inputs required under the residual method,
which is explained more fully in the section on the residual method
(section 90).\label{10.6.4.1.4-End}\label{subsubsec:10.6.4.1_Market_Approach-End}

\subsubsection{Income Approach\label{subsubsec:10.6.4.2_Income_Approach}}

\stepcounter{SubSubSecCounter} 

\thesubsubsection.\theSubSubSecCounter.\label{10.6.4.2.1} Establishing
the residual value of a development property may involve the use of
a cash flow model in some markets.\label{10.6.4.2.1-End}

\stepcounter{SubSubSecCounter} 

\thesubsubsection.\theSubSubSecCounter.\label{10.6.4.2.2} The income
approach may also be appropriate for establishing the value of a completed
property as one of the inputs required under the residual method,
which is explained more fully in the section on the residual method
(see section 90).\label{10.6.4.2.2-End}\label{subsubsec:10.6.4.2_Income_Approach-End}

\subsubsection{Cost Approach\label{subsubsec:10.6.4.3_Cost_Approach}}

\stepcounter{SubSubSecCounter} 

\thesubsubsection.\theSubSubSecCounter.\label{10.6.4.3.1} Establishing
the development costs is a key component of the residual approach
(see para 90.5).\label{10.6.4.3.1-End}

\stepcounter{SubSubSecCounter} 

\thesubsubsection.\theSubSubSecCounter.\label{10.6.4.3.2} The cost
approach may also exclusively be used as a means of indicating the
value of development property such as a proposed development of a
building or other structure for which there is no active market on
completion.\label{10.6.4.3.2-End}

\stepcounter{SubSubSecCounter} 

\thesubsubsection.\theSubSubSecCounter.\label{10.6.4.3.3} The cost
approach is based on the economic principle that a buyer will pay
no more for an asset than the amount to create an asset of equal utility.
To apply this principle to development property, the valuer must consider
the cost that a prospective buyer would incur in acquiring a similar
asset with the potential to earn a similar profit from development
as could be obtained from development of the subject property. However,
unless there are unusual circumstances affecting the subject development
property, the process of analysing a proposed development and determining
the anticipated costs for a hypothetical alternative would effectively
replicate either the market approach or the residual method as described
above, which can be applied directly to the subject property.\label{10.6.4.3.3-End}

\stepcounter{SubSubSecCounter} 

\thesubsubsection.\theSubSubSecCounter.\label{10.6.4.3.4} Another
difficulty in applying the cost approach to development property is
in determining the profit level, which is its “utility” to a prospective
buyer. Although a developer may have a target profit at the commencement
of a project, the actual profit is normally determined by the value
of the property at completion. Moreover, as the property approaches
completion, some of the risks associated with development are likely
to reduce, which may impact on the required return of a buyer. Unless
a fixed price has been agreed, profit is not determined by the costs
incurred in acquiring the land and undertaking the improvements.\label{10.6.4.3.4-End}\label{subsubsec:10.6.4.3_Cost_Approach-End}\label{subsec:10.6.4_Valuation_Approaches-End}

\subsection{Special Considerations for a Development Property\label{subsec:10.6.5_Special_Considerations}}

\stepcounter{SubSubSecCounter} 

\thesubsubsection.\theSubSubSecCounter.\label{10.6.5.0.1} The following
sections address a non-exhaustive list of topics relevant to the valuation
of development property:
\begin{enumerate}
\item Residual Method (section 90);
\item Existing Asset (section 100);
\item Special Considerations for Financial Reporting (section 110);
\item Special Considerations for Secured Lending (section 120).\label{10.6.5.0.1-End}
\end{enumerate}

\subsubsection{Residual Method\label{subsubsec:10.6.5.1_Residual_Method}}

\stepcounter{ParCounter} 

\theparagraph.\theParCounter.\label{10.6.5.1.0.1} The residual method
is so called because it indicates the residual amount after deducting
all known or anticipated costs required to complete the development
from the anticipated value of the project when completed after consideration
of the risks associated with completion of the project. This is known
as the residual value.\label{10.6.5.1.0.1-End}

\stepcounter{ParCounter} 

\theparagraph.\theParCounter.\label{10.6.5.1.0.2} The residual value
can be highly sensitive to relatively small changes in the forecast
cash flows and the practitioner should provide separate sensitivity
analyses for each significant factor.\label{10.6.5.1.0.2-End}

\stepcounter{ParCounter} 

\theparagraph.\theParCounter.\label{10.6.5.1.0.3} Caution is required
in the use of this method because of the sensitivity of the result
to changes in many of the inputs, which may not be precisely known
on the valuation date, and therefore have to be estimated with the
use of assumptions.\label{10.6.5.1.0.3-End}

\stepcounter{ParCounter} 

\theparagraph.\theParCounter.\label{10.6.5.1.0.4} The models used
to apply the residual method vary considerably in complexity and sophistication,
with the more complex models allowing for greater granularity of inputs,
multiple development phases and sophisticated analytical tools. The
most suitable model will depend on the size, duration and complexity
of the proposed development.\label{10.6.5.1.0.4-End}

\stepcounter{ParCounter} 

\theparagraph.\theParCounter.\label{10.6.5.1.0.5} In applying the
residual method, a valuer should consider and evaluate the reasonableness
and reliability of the following:
\begin{enumerate}
\item the source of information on any proposed building or structure, eg,
any plans and specification that are to be relied on in the valuation;
\item any source of information on the construction and other costs that
will be incurred in completing the project and which will be used
in the valuation.\label{10.6.5.1.0.5-End}
\end{enumerate}
\stepcounter{ParCounter} 

\theparagraph.\theParCounter.\label{10.6.5.1.0.6} The following
basic elements require consideration in any application of the method
to estimate the market value of development property and if another
basis is required, alternative inputs may be required.
\begin{enumerate}
\item Completed property value;
\item Construction costs;
\item Consultants fees;
\item Marketing costs;
\item Timetable;
\item Finance costs;
\item Development profit;
\item Discount rate.\label{10.6.5.1.0.6-End}
\end{enumerate}

\paragraph{Value of Completed Property\label{par:10.6.5.1.1_Completed_Property}}

\stepcounter{ParCounter} 

\theparagraph.\theParCounter.\label{10.6.5.1.1.1} The first step
requires an estimate of the value of the relevant interest in the
real property following notional completion of the development project,
which should be developed in accordance with IVS 105 Valuation Methods
and Approaches.\label{10.6.5.1.1.1-End}

\stepcounter{ParCounter} 

\theparagraph.\theParCounter.\label{10.6.5.1.1.2} Regardless of
the methods adopted under either the market or income approach, the
valuer must adopt one of the two basic underlying assumptions:
\begin{enumerate}
\item the estimated market value on completion is based on values that are
current on the valuation date on the special assumption the project
had already been completed in accordance with the defined plans and
specification;
\item the estimated value on completion is based on the special assumption
that the project is completed in accordance with the defined plans
and specification on the anticipated date of completion.\label{10.6.5.1.1.2-End}
\end{enumerate}
\stepcounter{ParCounter} 

\theparagraph.\theParCounter.\label{10.6.5.1.1.3} Market practice
and availability of relevant data should determine which of these
assumptions is more appropriate. However, it is important that there
is clarity as to whether current or projected values are being used.\label{10.6.5.1.1.3-End}

\stepcounter{ParCounter} 

\theparagraph.\theParCounter.\label{10.6.5.1.1.4} If estimated gross
development value is used, it should be made clear that these are
based on special assumptions that a participant would make based on
information available on the valuation date.\label{10.6.5.1.1.4-End}

\stepcounter{ParCounter} 

\theparagraph.\theParCounter.\label{10.6.5.1.1.5} It is also important
that care is taken to ensure that consistent assumptions are used
throughout the residual value calculation, ie, if current values are
used then the costs should also be current and discount rates derived
from analysis of current prices.\label{10.6.5.1.1.5-End}

\stepcounter{ParCounter} 

\theparagraph.\theParCounter.\label{10.6.5.1.1.6} If there is a
pre-sale or pre-lease agreement in place that is conditional on the
project, or a relevant part, being completed, this will be reflected
in the valuation of the completed property. Care should be taken to
establish whether the price in a pre-sale agreement or the rent and
other terms in a pre-lease agreement reflect those that would be agreed
between participants on the valuation date.\label{10.6.5.1.1.6-End}

\stepcounter{ParCounter} 

\theparagraph.\theParCounter.\label{10.6.5.1.1.7} If the terms are
not reflective of the market, adjustments may need to be made to the
valuation.\label{10.6.5.1.1.7-End}

\stepcounter{ParCounter} 

\theparagraph.\theParCounter.\label{10.6.5.1.1.8} It would also
be appropriate to establish if these agreements would be assignable
to a purchaser of the relevant interest in the development property
prior to the completion of the project.\label{10.6.5.1.1.8-End}\label{par:10.6.5.1.1_Completed_Property-End}

\paragraph{Construction Costs\label{par:10.6.5.1.2_Construction_Cost}}

\stepcounter{ParCounter} 

\theparagraph.\theParCounter.\label{10.6.5.1.2.1} The costs of all
work required at the valuation date to complete the project to the
defined specification need to be identified. Where no work has started,
this will include any preparatory work required prior to the main
building contract, such as the costs of obtaining statutory permissions,
demolition or off-site enabling work.\label{10.6.5.1.2.1-End}

\stepcounter{ParCounter} 

\theparagraph.\theParCounter.\label{10.6.5.1.2.2} Where work has
commenced, or is about to commence, there will normally be a contract
or contracts in place that can provide the independent confirmation
of cost. However, if there are no contracts in place, or if the actual
contract costs are not typical of those that would be agreed in the
market on the valuation date, then it may be necessary to estimate
these costs reflecting the reasonable expectation of participants
on the valuation date of the probable costs.\label{10.6.5.1.2.2-End}

\stepcounter{ParCounter} 

\theparagraph.\theParCounter.\label{10.6.5.1.2.3} The benefit of
any work carried out prior to the valuation date will be reflected
in the value, but will not determine that value. Similarly, previous
payments under the actual building contract for work completed prior
to the valuation date are not relevant to current value.\label{10.6.5.1.2.3-End}

\stepcounter{ParCounter} 

\theparagraph.\theParCounter.\label{10.6.5.1.2.4} The benefit of
any work carried out prior to the valuation date will be reflected
in the value, but will not determine that value. Similarly, previous
payments under the actual building contract for work completed prior
to the valuation date are not relevant to current value.\label{10.6.5.1.2.4-End}

\stepcounter{ParCounter} 

\theparagraph.\theParCounter.\label{10.6.5.1.2.5} In contrast, if
payments under a building contract are geared to the work completed,
the sums remaining to be paid for work not yet undertaken at the valuation
date may be the best evidence of the construction costs required to
complete the work. \label{10.6.5.1.2.5-End}

\stepcounter{ParCounter} 

\theparagraph.\theParCounter.\label{10.6.5.1.2.6} However, contractual
costs may include special requirements of a specific end user and
therefore may not reflect the general requirements of participants.\label{10.6.5.1.2.6-End}

\stepcounter{ParCounter} 

\theparagraph.\theParCounter.\label{10.6.5.1.2.7} Moreover, if there
is a material risk that the contract may not be fulfilled, (e.\,g,
due to a dispute or insolvency of one of the parties), it may be more
appropriate to reflect the cost of engaging a new contractor to complete
the outstanding work.\label{10.6.5.1.2.7-End}

\stepcounter{ParCounter} 

\theparagraph.\theParCounter.\label{10.6.5.1.2.8} When valuing a
partly completed development property, it is not appropriate to rely
solely on projected costs and income contained in any project plan
or feasibility study produced at the commencement of the project.\label{10.6.5.1.2.8-End}

\stepcounter{ParCounter} 

\theparagraph.\theParCounter.\label{10.6.5.1.2.9} Once the project
has commenced, this is not a reliable tool for measuring value as
the inputs will be historic. Likewise, an approach based on estimating
the percentage of the project that has been completed prior to the
valuation date is unlikely to be relevant in determining the current
market value.\label{10.6.5.1.2.9-End}\label{par:10.6.5.1.2_Construction_Cost-End}

\paragraph{Consultants’ Fees\label{par:10.6.5.1.3_Consultants_fee}}

\stepcounter{ParCounter} 

\theparagraph.\theParCounter.\label{10.6.5.1.3.1} These include
legal and professional costs that would be reasonably incurred by
a participant at various stages through the completion of the project.\label{10.6.5.1.3.1-End}\label{par:10.6.5.1.3_Consultants_fee-End}

\paragraph{Marketing Costs\label{par:10.6.5.1.4_Markting_Costs}}

\stepcounter{ParCounter}

\theparagraph.\theParCounter.\label{10.6.5.1.4.1} If there is no
identified buyer or lessee for the completed project, it will normally
be appropriate to allow for the costs associated with appropriate
marketing, and for any leasing commissions and consultants’ fees incurred
for marketing not included under para 90.23.\label{10.6.5.1.4.1-End}\label{par:10.6.5.1.4_Markting_Costs-End}

\paragraph{Timetable\label{par:10.6.5.1.5_Timetable}}

\stepcounter{ParCounter} 

\theparagraph.\theParCounter.\label{10.6.5.1.5.1} The duration of
the project from the valuation date to the expected date of physical
completion of the project needs to be considered, together with the
phasing of all cash outflows for construction costs, consultants’
fees, etc.\label{10.6.5.1.5.1-End}

\stepcounter{ParCounter} 

\theparagraph.\theParCounter.\label{10.6.5.1.5.2} If there is no
sale agreement in place for the relevant interest in the development
property following practical completion, an estimate should be made
of the marketing period that might typically be required following
completion of construction until a sale is achieved.\label{10.6.5.1.5.2-End}

\stepcounter{ParCounter} 

\theparagraph.\theParCounter.\label{10.6.5.1.5.3} If the property
is to be held for investment after completion and if there are no
pre-leasing agreements, the time required to reach stabilised occupancy
needs to be considered (ie, the period required to reach a realistic
long-term occupancy level). For a project where there will be individual
letting units, the stabilised occupancy levels may be less than 100
percent if market experience indicates that a number of units may
be expected to always be vacant, and allowance should be considered
for costs incurred by the owner during this period such as additional
marketing costs, incentives, maintenance and/or unrecoverable service
charges.\label{10.6.5.1.5.3-End}\label{par:10.6.5.1.5_Timetable-End}

\paragraph{Finance Costs\label{par:10.6.5.1.6_Finance_Cost}}

\stepcounter{ParCounter} 

\theparagraph.\theParCounter. \label{10.6.5.1.6.1} These represent
the cost of finance for the project from the valuation date through
to the completion of the project, including any period required after
physical completion to either sell the interest or achieve stabilised
occupancy. As a lender may perceive the risks during construction
to differ substantially from the risks following completion of construction,
the finance cost during each period may also need to be considered
separately. Even if an entity is intending to self-fund the project,
an allowance should be made for interest at a rate which would be
obtainable by a participant for borrowing to fund the completion of
the project on the valuation date.\label{10.6.5.1.6.1-End}\label{par:10.6.5.1.6_Finance_Cost-End}

\paragraph{Development Profit\label{par:10.6.5.1.7_Development_Profit}}

\stepcounter{ParCounter} 

\theparagraph.\theParCounter.\label{10.6.5.1.7.1} Allowance should
be made for development profit, or the return that would be required
by a buyer of the development property in the market place for taking
on the risks associated with completion of the project on the valuation
date. This will include the risks involved in achieving the anticipated
income or capital value following physical completion of the project.\label{10.6.5.1.7.1-End}

\stepcounter{ParCounter} 

\theparagraph.\theParCounter.\label{10.6.5.1.7.2} This target profit
can be expressed as a lump sum, a percentage return on the costs incurred
or a percentage of the anticipated value of the project on completion
or a rate of return. Market practice for the type of property in question
will normally indicate the most appropriate option. The amount of
profit that would be required will reflect the level of risk that
would be perceived by a prospective buyer on the valuation date and
will vary according to factors such as:
\begin{enumerate}
\item the stage which the project has reached on the valuation date. A project
which is nearing completion will normally be viewed as being less
risky than one at an early stage, with the exception of situations
where a party to the development is insolvent;
\item whether a buyer or lessee has been secured for the completed project;
\item the size and anticipated remaining duration of the project. The longer
the project, the greater the risk caused by exposure to fluctuations
in future costs and receipts and changing economic conditions generally.\label{10.6.5.1.7.2-End}
\end{enumerate}
\stepcounter{ParCounter} 

\theparagraph.\theParCounter.\label{10.6.5.1.7.3} The following
are examples of factors that may typically need to be considered in
an assessment of the relative risks associated with the completion
of a development project:
\begin{enumerate}
\item unforeseen complications that increase construction costs;
\item potential for contract delays caused by adverse weather or other matters
outside of developer’s control;
\item delays in obtaining statutory consents;
\item supplier failures;
\item entitlement risk and changes in entitlements over the development
period;
\item regulatory changes;
\item delays in finding a buyer or lessee for the completed project.\label{10.6.5.1.7.3-End}
\end{enumerate}
\stepcounter{ParCounter} 

\theparagraph.\theParCounter.\label{10.6.5.1.7.4} Whilst all of
the above factors will impact the perceived risk of a project and
the profit that a buyer or the development property would require,
care must be taken to avoid double counting, either where contingencies
are already reflected in the residual valuation model or risks in
the discount rate used to bring future cash flows to present value.\label{10.6.5.1.7.4-End}

\stepcounter{ParCounter} 

\theparagraph.\theParCounter.\label{10.6.5.1.7.5} The risk of the
estimated value of the completed development project changing due
to changed market conditions over the duration of the project will
normally be reflected in the discount rate or capitalization rate
used to value the completed project.\label{10.6.5.1.7.5-End}

\stepcounter{ParCounter} 

\theparagraph.\theParCounter.\label{10.6.5.1.7.6} The profit anticipated
by the owner of an interest in development property at the commencement
of a development project will vary according to the valuation of its
interest in the project once construction has commenced. The valuation
should reflect those risks remaining at the valuation date and the
discount or return that a buyer of the partially completed project
would require for bringing it to a successful conclusion.\label{10.6.5.1.7.6-End}\label{par:10.6.5.1.7_Development_Profit-End}

\paragraph{Discount Rate\label{par:10.6.5.1.8_Discount_Rate}}

\stepcounter{ParCounter} 

\theparagraph.\theParCounter.\label{10.6.5.1.8.1} In order to arrive
at an indication of the value of the development property on the valuation
date, the residual method requires the application of a discount rate
to all future cash flows in order to arrive at a net present value.
This discount rate may be derived using a variety of methods (see
IVS 105 Valuation Approaches and Methods, paras 50.30-50.39.\label{10.6.5.1.8.1-End}

\stepcounter{ParCounter} 

\theparagraph.\theParCounter.\label{10.6.5.1.8.2} If the cash flows
are based on values and costs that are current on the valuation date,
the risk of these changing between the valuation date and the anticipated
completion date should be considered and reflected in the discount
rate used to determine the present value. If the cash flows are based
on prospective values and costs, the risk of those projections proving
to be inaccurate should be considered and reflected in the discount
rate.\label{10.6.5.1.8.2-End}\label{par:10.6.5.1.8_Discount_Rate-End}\label{subsec:10.6.5.1_Residual_Method-End}

\subsubsection{Existing Asset\label{subsubsec:10.6.5.2_Existing_Asset}}

\stepcounter{SubSubSecCounter} 

\thesubsubsection.\theSubSubSecCounter.\label{10.6.5.2.1} In the
valuation of development property, it is necessary to establish the
suitability of the real property in question for the proposed development.
Some matters may be within the valuer’s knowledge and experience but
some may require information or reports from other specialists. Matters
that typically need to be considered for specific investigation when
undertaking a valuation of a development property before a project
commences include:
\begin{enumerate}
\item whether or not there is a market for the proposed development;
\item is the proposed development the highest and best use of the property
in the current market;
\item whether there are other non-financial obligations that need to be
considered (political or social criteria);
\item legal permissions or zoning, including any conditions or constraints
on permitted development;
\item limitations, encumbrances or conditions imposed on the relevant interest
by private contract;
\item rights of access to public highways or other public areas;
\item geotechnical conditions, including potential for contamination or
other environmental risks;
\item the availability of, and requirements to, provide or improve necessary
services, eg, water, drainage and power;
\item the need for any off-site infrastructure improvements and the rights
required to undertake this work;
\item any archaeological constraints or the need for archaeological investigations;
\item sustainability and any client requirements in relation to green buildings;
\item economic conditions and trends and their potential impact on costs
and receipts during the development period;
\item current and projected supply and demand for the proposed future uses;
\item the availability and cost of funding;
\item the expected time required to deal with preparatory matters prior
to starting work, for the completion of the work and, if appropriate,
to rent or sell the completed property;
\item any other risks associated with the proposed development.\label{10.6.5.2.1-End}
\end{enumerate}
\stepcounter{SubSubSecCounter} 

\thesubsubsection.\theSubSubSecCounter.\label{10.6.5.2.2} Where
a project is in progress, additional enquires or investigations will
typically be needed into the contracts in place for the design of
the project, for its construction and for supervision of the construction.\label{10.6.5.2.2-End}\label{subsubsec:10.6.5.2_Existing_Asset-End}

\subsubsection{Special Considerations for Financial Reporting\label{subsubsec:10.6.5.3_Financial_Reporting}}

\stepcounter{SubSubSecCounter} 

\thesubsubsection.\theSubSubSecCounter.\label{10.6.5.3.1} The accounting
treatment of development property can vary depending on how it is
classified by the reporting entity (eg, whether it is being held for
sale, for owner occupation or as investment property). This may affect
the valuation requirements and therefore the classification and the
relevant accounting requirements need to be determined before selecting
an appropriate valuation method.\label{10.6.5.3.1-End}

\stepcounter{SubSubSecCounter} 

\thesubsubsection.\theSubSubSecCounter.\label{10.6.5.3.2} Financial
statements are normally produced on the assumption that the entity
is a going concern. It is therefore normally appropriate to assume
that any contracts (eg, for the construction of a development property
or for its sale or leasing on completion), would pass to the buyer
in the hypothetical exchange, even if those contracts may not be assignable
in an actual exchange. An exception would be if there was evidence
of an abnormal risk of default by a contracted party on the valuation
date.\label{10.6.5.3.2-End}\label{subsubsec:10.6.5.3_Financial_Reporting-End}

\subsubsection{Special Considerations for Secured Lending\label{subsubsec:10.6.5.4_Secure_Lending}}

\stepcounter{SubSubSecCounter} 

\thesubsubsection.\theSubSubSecCounter.\label{10.6.5.4.1}The appropriate
basis of valuation for secured lending is normally market value. However,
in considering the value of a development property, regard should
be given to the probability that any contracts in place, eg, for construction
or for the sale or leasing of the completed project may, become void
or voidable in the event of one of the parties being the subject of
formal insolvency proceedings. Further regard should be given to any
contractual obligations that may have a material impact on market
value. Therefore, it may be appropriate to highlight the risk to a
lender caused by a prospective buyer of the property not having the
benefit of existing building contracts and/or pre-leases, and pre-sales
and any associated warrantees and guarantees in the event of a default
by the borrower.\label{10.6.5.4.1-End}

\stepcounter{SubSubSecCounter} 

\thesubsubsection.\theSubSubSecCounter.\label{10.6.5.4.2} To demonstrate
an appreciation of the risks involved in valuing development property
for secured lending or other purposes, the valuer should apply a minimum
of two appropriate and recognised methods to valuing development property
for each valuation project, as this is an area where there is often
“insufficient factual or observable inputs for a single method to
produce a reliable conclusion” (see IVS 105 Valuation Approaches and
Methods, para 10.4).\label{10.6.5.4.2-End}

\stepcounter{SubSubSecCounter} 

\thesubsubsection.\theSubSubSecCounter.\label{10.6.5.4.3} The valuer
must be able to justify the selection of the valuation approach(es)
reported and should provide an “As Is” (existing stage of development)
and an “As Proposed” (completed development) value for the development
property and record the process undertaken and a rationale for the
reported value (see IVS 103 Reporting, paras 30.1-30.2).\label{10.6.5.4.3-End}\label{subsubsec:10.6.5.4_Secure_Lending-End}\label{subsec:10.6.5_Special_Considerations-End}\label{sec:10.6_IVS-410_Development_Property-End}

\section{IVS 500. Financial Instruments\label{sec:10.7_IVS-500_Financial_Instruments}}

\subsection{Overview\label{subsec:10.7.1_Overview}}

\stepcounter{SubSecCounter} 

\thesubsection.\theSubSecCounter.\label{10.7.1.1} The principles
contained in the General Standards apply to valuations of financial
instruments. This standard only includes modifications, additional
requirements or specific examples of how the General Standards apply
for valuations to which this standard applies.\label{10.7.1.1-End}\label{subsec:10.7.1_Overview-End}

\subsection{Introduction\label{subsec:10.7.2_Introduction}}

\stepcounter{SubSecCounter} 

\thesubsection.\theSubSecCounter.\label{10.7.2.1} A financial instrument
is a contract that creates rights or obligations between specified
parties to receive or pay cash or other financial consideration. Such
instruments include but are not limited to, derivatives or other contingent
instruments, hybrid instruments, fixed income, structured products
and equity instruments. A financial instrument can also be created
through the combination of other financial instruments in a portfolio
to achieve a specific net financial outcome.\label{10.7.2.1-End}

\stepcounter{SubSecCounter} 

\thesubsection.\theSubSecCounter.\label{10.7.2.2} Valuations of
financial instruments conducted under IVS 500 Financial Instruments
can be performed for many different purposes including, but not limited
to: 
\begin{enumerate}
\item acquisitions, mergers and sales of businesses or parts of businesses;
\item purchase and sale;
\item financial reporting;
\item legal or regulatory requirements (subject to any specific requirements
set by the relevant authority);
\item internal risk and compliance procedures;
\item tax;
\item litigation.\label{10.7.2.2-End}
\end{enumerate}
\stepcounter{SubSecCounter} 

\thesubsection.\theSubSecCounter.\label{10.7.2.3} A thorough understanding
of the instrument being valued is required to identify and evaluate
the relevant market information available for identical or comparable
instruments. Such information includes prices from recent transactions
in the same or a similar instrument, quotes from brokers or pricing
services, credit ratings, yields, volatility, indices or any other
inputs relevant to the valuation process.\label{10.7.2.3-End}

\stepcounter{SubSecCounter} 

\thesubsection.\theSubSecCounter. \label{10.7.2.4}When valuations
are being undertaken by the holding entity that are intended for use
by external investors, regulatory authorities or other entities, to
comply with the requirement to confirm the identity and status of
the valuer in IVS 101 Scope of Work, para 20.3.(a), reference must
be made to the control environment in place, as required by IVS 105
Valuation Approaches and Methods and IVS 500 Financial Instruments
paras 120.1-120.3 regarding control environment.\label{10.7.2.4-End}

\stepcounter{SubSecCounter} 

\thesubsection.\theSubSecCounter.\label{10.7.2.5} To comply with
the requirement to identify the asset or liability to be valued as
in IVS 101 Scope of Work, para 20.3.(d), the following matters must
be addressed:
\begin{enumerate}
\item the class or classes of instrument to be valued;
\item whether the valuation is to be of individual instruments or a portfolio;
\item the unit of account.\label{10.7.2.5-End}
\end{enumerate}
\stepcounter{SubSecCounter} 

\thesubsection.\theSubSecCounter.\label{10.7.2.6} IVS 102 Investigations
and Compliance, paras 20.2-20.4 provide that the investigations required
to support the valuation must be adequate having regard to the purpose
of the assignment. To support these investigations, sufficient evidence
supplied by the valuer and/or a credible and reliable third party
must be assembled. To comply with these requirements, the following
are to be considered:
\begin{enumerate}
\item All market data used or considered as an input into the valuation
process must be understood and, as necessary, validated.
\item Any model used to estimate the value of a financial instrument shall
be selected to appropriately capture the contractual terms and economics
of the financial instrument.
\item Where observable prices of, or market inputs from, similar financial
instruments are available, those imputed inputs from comparable price(s)
and/or observable inputs should be adjusted to reflect the contractual
and economic terms of the financial instrument being valued.
\item Where possible, multiple valuation approaches are preferred. If differences
in value occur between the valuation approaches, the valuer must explain
and document the differences in value.\label{10.7.2.6-End}
\end{enumerate}
\stepcounter{SubSecCounter} 

\thesubsection.\theSubSecCounter.\label{10.7.2.7} To comply with
the requirement to disclose the valuation approach(es) and reasoning
in IVS 103 Reporting, para 20.1, consideration must be given to the
appropriate degree of reporting detail. The requirement to disclose
this information in the valuation report will differ for different
categories of financial instruments. Sufficient information should
be provided to allow users to understand the nature of each class
of instrument valued and the primary factors influencing the values.
Information that adds little to a users’ understanding as to the nature
of the asset or liability, or that obscures the primary factors influencing
value, must be avoided. In determining the level of disclosure that
is appropriate, regard must be had to the following:
\begin{enumerate}
\item Materiality: The value of an instrument or class of instruments in
relation to the total value of the holding entity’s assets and liabilities
or the portfolio that is valued. 
\item Uncertainty: The value of the instrument may be subject to significant
uncertainty on the valuation date due to the nature of the instrument,
the model or inputs used or to market abnormalities. Disclosure of
the cause and nature of any material uncertainty should be made. 
\item Complexity: The greater the complexity of the instrument, the greater
the appropriate level of detail to ensure that the assumptions and
inputs affecting value are identified and explained. 
\item Comparability: The instruments that are of particular interest to
users may differ with the passage of time. The usefulness of the valuation
report, or any other reference to the valuation, is enhanced if it
reflects the information demands of users as market conditions change,
although, to be meaningful, the information presented should allow
comparison with previous periods. 
\item Underlying instruments: If the cash flows of a financial instrument
are generated from or secured by identifiable underlying assets or
liabilities, the relevant factors that influence the underlying value
must be provided in order to help users understand how the underlying
value impacts the estimated value of the financial instrument.\label{10.7.2.7-End}\label{subsec:10.7.2_Introduction-End}
\end{enumerate}

\subsection{Bases of Value\label{subsec:10.7.3_Bases_of_Value}}

\stepcounter{SubSecCounter} 

\thesubsection.\theSubSecCounter.\label{10.7.3.1} In accordance
with IVS 104 Bases of Value, a valuer must select the appropriate
basis(es) of value when valuing financial instruments.\label{10.7.3.1-End}

\stepcounter{SubSecCounter} 

\thesubsection.\theSubSecCounter.\label{10.7.3.2} Often, financial
instrument valuations are performed using bases of value defined by
entities/organizations other than the IVSC (some examples of which
are mentioned in IVS 104 Bases of Value) and it is the valuer’s responsibility
to understand and follow the regulation, case law, tax law and other
interpretive guidance related to those bases of value as of the valuation
date.\label{10.7.3.2-End}\label{subsec:10.7.3_Bases_of_Value-End}

\subsection{Valuation Approaches and Methods\label{subsec:10.7.4_Valuation_Approaches}}

\stepcounter{SubSubSecCounter} 

\thesubsubsection.\theSubSubSecCounter.\label{10.7.4.0.1} When selecting
an approach and method, in addition to the requirements of this chapter,
a valuer must follow the requirements of IVS 105 Valuation Approaches
and Methods.\label{10.7.4.0.1-End}

\stepcounter{SubSubSecCounter} 

\thesubsubsection.\theSubSubSecCounter.\label{10.7.4.0.2} The three
valuation approaches described in IVS 105 Valuation Approaches and
Methods may be applied to the valuation of financial instruments.\label{10.7.4.0.2-End}

\stepcounter{SubSubSecCounter} 

\thesubsubsection.\theSubSubSecCounter.\label{10.7.4.0.3} The various
valuation methods used in financial markets are based on variations
of the market approach, the income approach or the cost approach as
described in the IVS 105 Valuation Approaches and Methods. This standard
describes the commonly used methods and matters that need to be considered
or the inputs needed when applying these methods.\label{10.7.4.0.3-End}

\stepcounter{SubSubSecCounter} 

\thesubsubsection.\theSubSubSecCounter.\label{10.7.4.0.4} When using
a particular valuation method or model, it is important to ensure
that it is calibrated with observable market information, where available,
on a regular basis to ensure that the model reflects current market
conditions. As market conditions change, it may become necessary to
change to a more suitable model(s) or to modify the existing model
and recalibrate and/ or make additional adjustments to the valuation
inputs. Those adjustments should be made to ensure consistency with
the required valuation basis, which in turn is determined by the purpose
for which the valuation is required; see the IVS Framework.\label{10.7.4.0.4-End}

\subsubsection{Market Approach\label{subsubsec:10.7.4.1_Market_Approach}}

\stepcounter{SubSubSecCounter} 

\thesubsubsection.\theSubSubSecCounter.\label{10.7.4.1.1} A price
obtained from trading on a liquid exchange on, or very close to, the
time or date of valuation is normally the best indication of the market
value of a holding of the identical instrument. In cases where there
have not been recent relevant transactions, the evidence of quoted
or consensus prices, or private transactions may also be relevant.\label{10.7.4.1.1-End}

\stepcounter{SubSubSecCounter} 

\thesubsubsection.\theSubSubSecCounter.\label{10.7.4.1.2} It may
be necessary to make adjustments to the price information if the observed
instrument is dissimilar to that being valued or if the information
is not recent enough to be relevant. For example, if an observable
price is available for similar instruments with one or more different
characteristics to the instrument being valued, then the implied inputs
from the comparable observable price are to be adjusted to reflect
the specific terms of the financial instrument being valued.\label{10.7.4.1.2-End}

\stepcounter{SubSubSecCounter} 

\thesubsubsection.\theSubSubSecCounter.\label{10.7.4.1.3} When relying
on a price from a pricing service, the valuer must understand how
the price was derived.\label{10.7.4.1.3-End}

End\label{subsubsec:10.7.4.1_Market_Approach-End}

\subsubsection{Income Approach\label{subsubsec:10.7.4.2_Income_Approach}}

\stepcounter{SubSubSecCounter} 

\thesubsubsection.\theSubSubSecCounter.\label{10.7.4.2.1} The value
of financial instruments may be determined using a discounted cash
flow method. The terms of an instrument determine, or allow estimation
of, the undiscounted cash flows. The terms of a financial instrument
typically set out:
\begin{enumerate}
\item the timing of the cash flows, ie, when the entity expects to realise
the cash flows related to the instrument;
\item the calculation of the cash flows, eg, for a debt instrument, the
interest rate that applies, or for a derivative instrument, how the
cash flows are calculated in relation to the underlying instrument
or index (or indices);
\item the timing and conditions for any options in the contract, eg, put
or call, prepayment, extension or conversion options;
\item protection of the rights of the parties to the instrument, eg, terms
relating to credit risk in debt instruments or the priority over,
or subordination to, other instruments held.\label{10.7.4.2.1-End}
\end{enumerate}
\stepcounter{SubSubSecCounter} 

\thesubsubsection.\theSubSubSecCounter.\label{10.7.4.2.2} In establishing
the appropriate discount rate, it is necessary to assess the return
that would be required on the instrument to compensate for the time
value of money and potential additional risks from, but not limited
to the following:
\begin{enumerate}
\item the terms and conditions of the instrument, eg, subordination;
\item the credit risk, ie, uncertainty about the ability of the counterparty
to make payments when due;
\item the liquidity and marketability of the instrument;
\item the risk of changes to the regulatory or legal environment;
\item the tax status of the instrument.\label{10.7.4.2.2-End}
\end{enumerate}
\stepcounter{SubSubSecCounter} 

\thesubsubsection.\theSubSubSecCounter.\label{10.7.4.2.3} Where
future cash flows are not based on fixed contracted amounts, estimates
of the expected cash flows will need to be made in order to determine
the necessary inputs. The determination of the discount rate must
reflect the risks of, and be consistent with, the cash flows. For
example, if the expected cash flows are measured net of credit losses
then the discount rate must be reduced by the credit risk component.
Depending upon the purpose of the valuation, the inputs and assumptions
made into the cash flow model will need to reflect either those that
would be made by participants, or those that would be based on the
holder’s current expectations or targets. For example, if the purpose
of the valuation is to determine market value, or fair value as defined
in IFRS, the assumptions should reflect those of participants. If
the purpose is to measure performance of an asset against management
determined benchmarks, eg, a target internal rate of return, then
alternative assumptions may be appropriate.\label{10.7.4.2.3-End}\label{subsubsec:10.7.4.2_Income_Approach-End}

\subsubsection{Cost Approach\label{subsubsec:10.7.4.3_Cost_Approach}}

\stepcounter{SubSubSecCounter} 

\thesubsubsection.\theSubSubSecCounter.\label{10.7.4.3.1} In applying
the cost approach, valuers must follow the guidance contained in IVS
105 Valuation Approaches and Methods, paras 70.1-70.14.\label{10.7.4.3.1-End}\label{subsubsec:10.7.4.3_Cost_Approach-End}\label{subsec:10.7.4_Valuation_Approaches-End}

\subsection{Special Considerations for Financial Instruments\label{subsec:10.7.5_Special_Considerations}}

\stepcounter{SubSubSecCounter} 

\thesubsubsection.\theSubSubSecCounter.\label{10.7.5.0.1} The following
sections address a non-exhaustive list of topics relevant to the valuation
of financial instruments:
\begin{enumerate}
\item Valuation Inputs (section 90).
\item Credit Risk (section 100).
\item Liquidity and Market Activity (section 110).
\item Control Environment (section 120).\label{10.7.5.0.1-End}
\end{enumerate}

\subsubsection{Valuation Inputs\label{subsubsec:10.7.5.1_Valuation_Inputs}}

\stepcounter{SubSubSecCounter} 

\thesubsubsection.\theSubSubSecCounter.\label{10.7.5.1.1} As per
IVS 105 Valuation Approaches and Methods, para 10.7, any data set
used as a valuation input, understanding the sources and how inputs
are adjusted by the provider, if any, is essential to understanding
the reliance that should be given to the use of the valuation input.\label{10.7.5.1.1-End}

\stepcounter{SubSubSecCounter} 

\thesubsubsection.\theSubSubSecCounter.\label{10.7.5.1.2} Valuation
inputs may come from a variety of sources. Commonly used valuation
input sources are broker quotations, consensus pricing services, the
prices of comparable instruments from third parties and market data
pricing services. Implied inputs can often be derived from such observable
prices such as volatility and yields. \label{10.7.5.1.2-End}

\stepcounter{SubSubSecCounter} 

\thesubsubsection.\theSubSubSecCounter.\label{10.7.5.1.3} When assessing
the validity of broker quotations, as evidence of how participants
would price an asset, the valuer should consider the following:
\begin{enumerate}
\item Brokers generally make markets and provide bids in respect of more
popular instruments and may not extend coverage to less liquid instruments.
Because liquidity often reduces with time, quotations may be harder
to find for older instruments.
\item A broker is concerned with trading, not supporting valuation, and
they have little incentive to research an indicative quotation as
thoroughly as they would an executable quotation. A valuer is required
to understand whether the broker quote is a binding, executable quote
or a non-binding, theoretical quote. In the case of a non-binding
quote, the valuer is required to gather additional information to
understand if the quote should be adjusted or omitted from the valuation.
\item There is an inherent conflict of interest where the broker is the
counterparty to an instrument.
\item Brokers have an incentive to encourage trading.\label{10.7.5.1.3-End}
\end{enumerate}
\stepcounter{SubSubSecCounter} 

\thesubsubsection.\theSubSubSecCounter.\label{10.7.5.1.4} Consensus
pricing services operate by collecting price or valuation input information
about an instrument from several participating subscribers. They reflect
a pool of quotations from different sources, sometimes with adjustment
to compensate for any sampling bias. This overcomes the conflict of
interest problems associated with single brokers. However, as with
a broker quotation, it may not be possible to find a suitable input
for all instruments in all markets. Additionally, despite its name,
a consensus price may not necessarily constitute a true market “consensus”,
but rather is more of a statistical estimate of recent market transactions
or quoted prices. Therefore, the valuer needs to understand how the
consensus pricing was estimated and if such estimates are reasonable,
given the instrument being valued. Information and inputs relevant
to the valuation of an illiquid instrument can often be gleaned through
comparable transactions (see section 110 for further details).\label{10.7.5.1.4-End}\label{subsubsec:10.7.5.1_Valuation_Inputs-End}

\subsubsection{Credit Risk Adjustments\label{subsubsec:10.7.5.2_Credit_Risk}}

\stepcounter{SubSubSecCounter} 

\thesubsubsection.\theSubSubSecCounter.\label{10.7.5.2.1} Understanding
the credit risk is often an important aspect of valuing a financial
instrument and most importantly the issuer. Some of the common factors
that need to be considered in establishing and measuring credit risk
include the following:
\begin{enumerate}
\item Own credit and counterparty risk: Assessing the financial strength
of the issuer or any credit support providers will involve consideration
of not only historical and projected financial performance of the
relevant entity or entities but also consideration of performance
and prospects for the industry sector in which the business operates.
In addition to issuer credit, the valuer must also consider the credit
exposure of any counterparties to the asset or liability being valued.
In the case of a clearing house settlement process, many jurisdictions
now require certain derivatives to be transacted through a central
counterparty which can mitigate risk, however residual counterparty
risk needs to be considered.
\item The valuer also needs to be able to differentiate between the credit
risk of the instrument and the credit risk of the issuer and/or counterparty.
Generally, the credit risk of the issuer or counterparty does not
consider specific collateral related to the instrument.
\item Subordination: Establishing the priority of an instrument is critical
in assessing the default risk. Other instruments may have priority
over an issuer’s assets or the cash flows that support the instrument. 
\item Leverage: The amount of debt used to fund the assets from which an
instrument’s return is derived can affect the volatility of returns
to the issuer and credit risk. 
\item Netting agreements: Where derivative instruments are held between
counterparties, credit risk may be reduced by a netting or offset
agreement that limits the obligations to the net value of the transactions,
ie, if one party becomes insolvent, the other party has the right
to offset sums owed to the insolvent party against sums due under
other instruments. 
\item Default protection: Many instruments contain some form of protection
to reduce the risk of non-payment to the holder. Protection might
take the form of a guarantee by a third party, an insurance contract,
a credit default swap or more assets to support the instrument than
are needed to make the payments. Credit exposure is also reduced if
subordinated instruments take the first losses on the underlying assets
and therefore reduce the risk to more senior instruments. When protection
is in the form of a guarantee, an insurance contract or a credit default
swap, it is necessary to identify the party providing the protection
and assess that party’s creditworthiness. Considering the credit worthiness
of a third party involves not only the current position but also the
possible effect of any other guarantees or insurance contracts the
entity has written. If the provider of a guarantee has also guaranteed
other correlated debt securities, the risk of its non-performance
will likely increase.\label{10.7.5.2.1-End}
\end{enumerate}
\stepcounter{SubSubSecCounter} 

\thesubsubsection.\theSubSubSecCounter.\label{10.7.5.2.2} For parties
for which limited information is available, if secondary trading in
a financial instrument exists, there may be sufficient market data
to provide evidence of the appropriate risk adjustment. If not, it
might be necessary to look to credit indices, information available
for entities with similar risk characteristics, or estimate a credit
rating for the party using its own financial information. The varying
sensitivities of different liabilities to credit risk, such as collateral
and/or maturity differences, should be taken into account in evaluating
which source of credit data provides the most relevant information.
The risk adjustment or credit spread applied is based on the amount
a participant would require for the particular instrument being valued.\label{10.7.5.2.2-End}

\stepcounter{SubSubSecCounter} 

\thesubsubsection.\theSubSubSecCounter.\label{10.7.5.2.3} The own
credit risk associated with a liability is important to its value
as the credit risk of the issuer is relevant to the value in any transfer
of that liability. Where it is necessary to assume a transfer of the
liability regardless of any actual constraints on the ability of the
counterparties to do so, eg, in order to comply with financial reporting
requirements, there are various potential sources for reflecting own
credit risk in the valuation of liabilities. These include the yield
curve for the entity’s own bonds or other debt issued, credit default
swap spreads, or by reference to the value of the corresponding asset.
However, in many cases the issuer of a liability will not have the
ability to transfer it and can only settle the liability with the
counterparty.\label{10.7.5.2.3-End}

\stepcounter{SubSubSecCounter} 

\thesubsubsection.\theSubSubSecCounter.\label{10.7.5.2.4} Collateral:
The assets to which the holder of an instrument has recourse in the
event of default need to be considered. In particular, the valuer
needs to be understand whether recourse is to all the assets of the
issuer or only to specified asset(s). The greater the value and liquidity
of the asset(s) to which an entity has recourse in the event of default,
the lower the overall risk of the instrument due to increased recovery.
In order not to double count, the valuer also needs to consider if
the collateral is already accounted for in another area of the balance
sheet.\label{10.7.5.2.4-End}

\stepcounter{SubSubSecCounter} 

\thesubsubsection.\theSubSubSecCounter.\label{10.7.5.2.5} When adjusting
for own credit risk of the instrument, it is also important to consider
the nature of the collateral available for the liabilities being valued.
Collateral that is legally separated from the issuer normally reduces
the credit exposure. If liabilities are subject to a frequent collateralisation
process, there might not be a material own credit risk adjustment
because the counterparty is mostly protected from loss in the event
of default.\label{10.7.5.2.5-End}\label{subsubsec:10.7.5.2_Credit_Risk-End}

\subsubsection{Liquidity and Market Activity\label{subsubsec:10.7.5.3_Liquidity}}

\stepcounter{SubSubSecCounter} 

\thesubsubsection.\theSubSubSecCounter.\label{10.7.5.3.1} The liquidity
of financial instruments range from those that are standardised and
regularly transacted in high volumes to those that are agreed between
counterparties that are incapable of assignment to a third party.
This range means that consideration of the liquidity of an instrument
or the current level of market activity is important in determining
the most appropriate valuation approach.\label{10.7.5.3.1-End}

\stepcounter{SubSubSecCounter} 

\thesubsubsection.\theSubSubSecCounter.\label{10.7.5.3.2} Liquidity
and market activity are distinct. The liquidity of an asset is a measure
of how easily and quickly it can be transferred in return for cash
or a cash equivalent. Market activity is a measure of the volume of
trading at any given time, and is a relative rather than an absolute
measure. Low market activity for an instrument does not necessarily
imply the instrument is illiquid.\label{10.7.5.3.2-End}

\stepcounter{SubSubSecCounter} 

\thesubsubsection.\theSubSubSecCounter.\label{10.7.5.3.3} Although
separate concepts, illiquidity or low levels of market activity pose
similar valuation challenges through a lack of relevant market data,
ie, data that is either current at the valuation date or that relates
to a sufficiently similar asset to be reliable. The lower the liquidity
or market activity, the greater the reliance that will be needed on
valuation approaches that use techniques to adjust or weight the inputs
based on the evidence of other comparable transactions to reflect
either market changes or differing characteristics of the asset.\label{10.7.5.3.3-End}\label{subsubsec:10.7.5.3_Liquidity-End}

\subsubsection{Valuation Control and Objectivity\label{subsubsec:10.7.5.4_Valuation_Control}}

\stepcounter{SubSubSecCounter} 

\thesubsubsection.\theSubSubSecCounter.\label{10.7.5.4.1} The control
environment consists of the internal governance and control procedures
that are in place with the objective of increasing the confidence
of those who may rely on the valuation in the valuation process and
conclusion. Where an external valuer is placing reliance upon an internally
performed valuation, the external valuer must consider the adequacy
and independence of the valuation control environment.\label{10.7.5.4.1-End}

\stepcounter{SubSubSecCounter} 

\thesubsubsection.\theSubSubSecCounter.\label{10.7.5.4.2} In comparison
with other asset classes, financial instruments are more commonly
valued internally by the same entity that creates and trades them.
Internal valuations bring into question the independence of the valuer
and hence this creates risk to the perceived objectivity of valuations.
Please reference 40.1 and 40.2 of the IVS Framework regarding valuation
performed by internal valuers and the need for procedures to be in
place to ensure the objectivity of the valuation and steps that should
be taken to ensure that an adequate control environment exists to
minimise threats to the independence of the valuation. Many entities
which deal with the valuation of financial instruments are registered
and regulated by statutory financial regulators. Most financial regulators
require banks or other regulated entities that deal with financial
instruments to have independent price verification procedures. These
operate separately from trading desks to produce valuations required
for financial reporting or the calculation of regulatory capital guidance
on the specific valuation controls required by different regulatory
regimes. This is outside the scope of this standard. However, as a
general principle, valuations produced by one department of an entity
that are to be included in financial statements or otherwise relied
on by third parties should be subject to scrutiny and approval by
an independent department of the entity. Ultimate authority for such
valuations should be separate from, and fully independent of, the
risk-taking functions. The practical means of achieving a separation
of the function will vary according to the nature of the entity, the
type of instrument being valued and the materiality of the value of
the particular class of instrument to the overall objective. The appropriate
protocols and controls should be determined by careful consideration
of the threats to objectivity that would be perceived by a third party
relying on the valuation.\label{10.7.5.4.2-End}

\stepcounter{SubSubSecCounter} 

\thesubsubsection.\theSubSubSecCounter.\label{10.7.5.4.3} When accessing
your valuation controls, the following include items you should consider
in the valuation process:
\begin{enumerate}
\item establishing a governance group responsible for valuation policies
and procedures and for oversight of the entity’s valuation process,
including some members external to the entity;
\item systems for regulatory compliance if applicable;
\item a protocol for the frequency and methods for calibration and testing
of valuation models;
\item criteria for verification of certain valuations by different internal
or external experts;
\item periodic independent validation of the valuation model(s);
\item identifying thresholds or events that trigger more thorough investigation
or secondary approval requirements;
\item identifying procedures for establishing significant inputs that are
not directly observable in the market, eg, by establishing pricing
or audit committees.\label{subsubsec:10.7.5.4_Valuation_Control-End}
\end{enumerate}
\label{subsec:10.7.5_Special_Considerations-End}\label{sec:10.7_IVS-500_Financial_Instruments-End}\label{part:II_IVS-End}\label{End-of-all.}\selectlanguage{russian}%

\end{document}
